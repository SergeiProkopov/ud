\upaper{33}{АДМИНИСТРАЦИЯ ЛОКАЛЬНОЙ ВСЕЛЕННОЙ}
\uminitoc{МИХАИЛ НЕБАДОНА}
\uminitoc{СУВЕРЕН НЕБАДОНА}
\uminitoc{ВСЕЛЕНСКИЕ СЫН И ДУХ}
\uminitoc{ГАВРИИЛ --- ГЛАВА ИСПОЛНИТЕЛЬНОЙ ВЛАСТИ}
\uminitoc{ПОСЛЫ ТРОИЦЫ}
\uminitoc{ОБЩЕЕ УПРАВЛЕНИЕ}
\uminitoc{СУДЫ НЕБАДОНА}
\uminitoc{ЗАКОНОДАТЕЛЬНЫЕ И ИСПОЛНИТЕЛЬНЫЕ ФУНКЦИИ}
\author{Глава Архангелов}
\vs p033 0:1 В то время как Всеобщий Отец, несомненно, правит своим огромным творением, в управлении локальной вселенной он действует в лице Сына Создателя. В остальном Отец лично не участвует в административных делах локальной вселенной. Эти вопросы вверены Сыну Создателю и Материнскому Духу локальной вселенной, а также их многочисленным детям. Планы, политика и административные действия в локальной вселенной формируются и исполняются этим Сыном, который, вместе со своим партнёром Духом, передаёт исполнительную власть Гавриилу, а судебную власть~--- Отцам Созвездий, Властелинам Систем и Планетарным Принцам.
\usection{МИХАИЛ НЕБАДОНА}
\vs p033 1:1 Наш Сын Создатель является олицетворением 611\,121-й оригинальной концепции бесконечного отождествления, берущего одновременное начало во Всеобщем Отце и Вечном Сыне. Михаил Небадона~--- <<единородный Сын>>, олицетворяющий эту 611\,121-ю всеобщую концепцию божественности и бесконечности. Его центр находится в тройной обители света на Спасограде. И жилище это устроено именно таким образом потому, что Михаил испытал жизнь во всех трёх фазах существования разумного создания: духовной, моронтийной и материальной. Из-за имени, связанного с его седьмым и последним посвящением на Урантии, о нём иногда говорят как о Христе Михаиле.
\vs p033 1:2 Наш Сын Создатель не есть Вечный Сын, экзистенциальный Райский партнёр Всеобщего Отца и Бесконечного Духа. Михаил Небадона не является членом Райской Троицы. Тем не менее наш Сын Властелин обладает в своих владениях всеми божественными атрибутами и полномочиями, которые проявил бы сам Вечный Сын, если бы он реально находился на Спасограде и действовал в Небадоне. Михаил обладает даже дополнительными могуществом и властью, так как он не только олицетворяет Вечного Сына, но также полностью представляет и фактически воплощает в себе личностное присутствие Всеобщего Отца в этой локальной вселенной и для неё. Он даже представляет Отца\hyp{}Сына. Эти отношения делают Сына Создателя самым могущественным, разносторонним и влиятельным из всех божественных существ, способных непосредственно управлять эволюционными вселенными и личностно контактировать с незрелыми созданными существами.
\vs p033 1:3 Наш Сын Создатель проявляет ту же духовную притягательную силу, гравитацию духа, из столицы локальной вселенной, которую проявлял бы Вечный Сын Рая, если бы лично присутствовал на Спасограде, и \bibemph{более того:} этот Вселенский Сын также является олицетворением Всеобщего Отца для вселенной Небадон. Сыны Создатели представляют собой личностные центры духовных сил Райского Отца\hyp{}Сына. Сыны Создатели являются завершающими средоточиями мощи\hyp{}личности могущественных время\hyp{}пространственных атрибутов Бога Семичастного.
\vs p033 1:4 Сын Создатель есть персонализация Всеобщего Отца в качестве наместника; в божественности он равен Вечному Сыну и является созидательным партнёром Бесконечного Духа. Для нашей вселенной и всех её обитаемых миров Суверенный Сын практически является Богом. Он олицетворяет все Райские Божества, доступные осмысленному постижению развивающихся смертных. Этот Сын и его партнёр Дух \bibemph{суть} ваши создатели\hyp{}родители. Для вас Михаил, Сын Создатель, является верховной личностью; Вечный Сын для вас~--- сверхверховная~--- бесконечная личность Божества.
\vs p033 1:5 \pc В лице Сына Создателя у нас есть правитель и божественный родитель, столь же могущественный, действенный и благотворный, какими были бы Всеобщий Отец и Вечный Сын, если бы они оба присутствовали на Спасограде и занимались управлением вселенной Небадон.
\usection{СУВЕРЕН НЕБАДОНА}
\vs p033 2:1 Наблюдение за Сынами Создателями показывает, что некоторые из них больше напоминают Отца, какие\hyp{}то~--- Сына, тогда как другие представляют собой сочетание обоих своих бесконечных родителей. Наш Сын Создатель определённо проявляет черты и атрибуты, которые больше напоминают Вечного Сына.
\vs p033 2:2 Михаил решил организовать эту локальную вселенную, и здесь он сейчас правит безраздельно. Его личная власть ограничена изначально существующими контурами гравитации, сходящимися в Рае, и тем, что От Века Древние правительства сверхвселенной сохраняют за собой право вынесения всех окончательных исполнительных судебных решений, касающихся прекращения существования личности. Личность является даром одного лишь Отца, но Сыны Создатели, с одобрения Вечного Сына, порождают новые типы созданий и, в сотрудничестве со своими партнёрами Духами, могут предпринимать попытки новых преобразований энергии\hyp{}материи.
\vs p033 2:3 \pc Михаил является олицетворением Райского Отца\hyp{}Сына в локальной вселенной Небадон и для неё; поэтому, когда Созидательный Материнский Дух, представитель Бесконечного Духа в локальной вселенной, подчинила себя Христу Михаилу по возвращении его из завершающего посвящения на Урантии, Сын Властелин благодаря этому приобрёл право на <<всю власть на небе и на земле>>.
\vs p033 2:4 Это подчинение Божественных Служительниц Сынам Создателям локальных вселенных утверждает этих Сынов Властелинов в качестве личных вместилищ выражаемой на конечном уровне божественности Отца, Сына и Духа, в то время как опыт посвящений Михаилов в облике созданий даёт им право выражать эмпирическую божественность Верховного Существа. Никакие другие существа во вселенных не исчерпали так лично потенциалы существующего конечного опыта, и никакие другие существа во вселенных не обладают такой квалификацией для единоличного суверенитета.
\vs p033 2:5 \pc Хотя центр Михаила официально расположен на Спасограде, столице Небадона, значительную часть своего времени он проводит посещая столицы созвездий и систем и даже отдельные планеты. Периодически он путешествует в Рай и часто на Уверсу, где советуется с От Века Древними. Когда он отсутствует на Спасограде, его место занимает Гавриил, который в таких случаях действует как регент вселенной Небадон.
\usection{ВСЕЛЕНСКИЕ СЫН И ДУХ}
\vs p033 3:1 В то время как Бесконечный Дух пронизывает все вселенные времени и пространства, из столицы каждой локальной вселенной он действует как специализированная фокализация, приобретающая полные качества личности через созидательное сотрудничество с Сыном Создателем. В отношении локальной вселенной, Сын Создатель обладает верховной административной властью; Бесконечный Дух, в качестве Божественной Служительницы, действует в полном с ним согласии, несмотря на совершенно равное положение.
\vs p033 3:2 \pc Вселенский Материнский Дух Спасограда, партнёр Михаила в управлении и руководстве Небадоном, принадлежит к шестой группе Верховных Духов, являясь 611\,121-й в этой категории. Она добровольно вызвалась сопровождать Михаила по случаю его освобождения от Райских обязанностей и с тех пор действует вместе с ним в создании и управлении его вселенной.
\vs p033 3:3 \pc Полновластный Сын Создатель~--- единоличный суверен своей вселенной, но во всех тонкостях управления ею Сын руководит совместно со Вселенским Духом. В то время как Дух всегда признаёт Сына сувереном и правителем, Сын всегда предоставляет Духу равное положение и равенство полномочий во всех делах, имеющих отношение к области их управления. Во всём своём деле любви и дарования жизни Сын Создатель всегда и вечно получает совершенную поддержку и умелую помощь от премудрого и вечно верного Вселенского Духа и всей её свиты разнообразных ангельских личностей. Такая Божественная Служительница на самом деле является матерью духов и духовных личностей, вездесущим и премудрым советчиком Сына Создателя, верным и истинным проявлением Райского Бесконечного Духа.
\vs p033 3:4 \pc Сын действует как отец в своей локальной вселенной. Дух же, в доступном для смертных созданий понимании, исполняет роль матери, неизменно помогающей Сыну и навечно незаменимой в управлении вселенной. В случае восстания только Сын и связанные с ним Сыны могут действовать как избавители. Дух никогда не может противостать мятежу или защитить власть, но всегда поддерживает Сына во всём, с чем бы тот ни сталкивался в своих усилиях стабилизировать управление и поддержание власти на мирах, заражённых злом или порабощённых грехом. Только Сын может восстановить работу их совместного творения, но ни один Сын не может надеяться на окончательный успех без непрестанной поддержки Божественной Служительницы и её обширного собрания духовных помощниц, дочерей Бога, которые столь преданно и доблестно борются за благополучие смертных людей и во славу своих божественных родителей.
\vs p033 3:5 После завершения седьмого и последнего посвящения Сына Создателя в облике создания, неопределённость, связанная с периодической изоляцией Божественной Служительницы, заканчивается, и вселенская помощница Сына навечно и надёжно утверждается в управлении. И в юбилей юбилеев в честь возведения Сына Создателя на престол в качестве Сына Властелина, Вселенский Дух перед собравшимися воинствами впервые делает публичное и всеобщее заявление о подчинении Сыну, принося клятву верности и послушания. Это событие произошло в Небадоне по возвращении Михаила на Спасоград после его посвящения на Урантии. Никогда до этого знаменательного события Вселенский Дух не признавала своего подчинения Вселенскому Сыну, и только после этого добровольного отказа Духа от могущества и власти можно было правдиво провозгласить о Сыне, что <<в его руки отдана вся власть на небе и на земле>>.
\vs p033 3:6 После этой клятвы подчинения со стороны Созидательного Материнского Духа Михаил Небадона благородно признал свою вечную зависимость от своей спутницы Духа, назначив Дух соправителем его вселенских владений и потребовав от всех своих созданий дать такую же клятву верности Духу, какую они дали Сыну; и затем было издано и распространено окончательное <<Провозглашение равенства>>. Несмотря на полновластие в своей локальной вселенной, Сын объявил мирам факт равенства Духа с ним во всех личностных качествах и атрибутах божественного характера. И это стало трансцендентным образцом семейной организации и управления даже для низших созданий миров пространства. Это действительно и истинно высокий идеал семьи и человеческого института добровольного брака.
\vs p033 3:7 Сегодня Сын и Дух руководят вселенной во многом так же, как отец и мать заботятся о своей семье сыновей и дочерей и помогают ей. Не будет целиком неуместным назвать Вселенский Дух созидательной спутницей Сына Создателя, а создания миров считать их сыновьями и дочерьми~--- великой и славной семьёй, но семьёй, за которую несут несказанную ответственность и окружают бесконечной заботой.
\vs p033 3:8 \pc Сын инициирует создание определённых вселенских детей, в то время как Дух несёт единоличную ответственность за создание многочисленных категорий духовных личностей, которые помогают и служат под управлением и руководством того же Материнского Духа. В создании других типов вселенских личностей и Сын, и Дух действуют вместе, при этом ни один из них не приступает ни к одному из созидательных актов, не получив совета и одобрения другого.
\usection{ГАВРИИЛ --- ГЛАВА ИСПОЛНИТЕЛЬНОЙ ВЛАСТИ}
\vs p033 4:1 Яркая Утренняя Звезда~--- это персонализация первой концепции индивидуального существования [identity] и идеала личности, задуманной Сыном Создателем и проявлением Бесконечного Духа в локальной вселенной. Возвращаясь к ранним дням локальной вселенной, до союза Сына Создателя и Материнского Духа в узах творческого объединения, к временам до начала создания их разносторонней семьи сыновей и дочерей, первый совместный акт раннего и свободного объединения этих двух божественных личностей привёл к созданию высочайшей духовной личности Сына и Духа~--- Яркой Утренней Звезды.
\vs p033 4:2 В каждой локальной вселенной рождается только одно столь мудрое и величественное существо. Всеобщий Отец и Вечный Сын могут создавать, и фактически создают, неограниченное число Сынов, равных себе в божественности; но такие Сыны, в союзе с Дочерьми Бесконечного Духа, могут создать только одну Яркую Утреннюю Звезду в каждой вселенной, существо, подобное им самим и щедро наследующее их объединённую природу, но не их созидательные прерогативы. Гавриил Спасограда подобен Вселенскому Сыну природной божественностью, хотя и значительно ограничен в атрибутах Божества.
\vs p033 4:3 Этот первенец родителей новой вселенной~--- уникальная личность, обладающая многими замечательными чертами, которые не были ярко выражены ни в одном из родителей, существо беспрецедентной разносторонности и невообразимого великолепия. Эта небесная личность заключает в себе божественную волю Сына в сочетании с творческим воображением Духа. Мысли и действия Яркой Утренней Звезды всегда будут полностью представлять как Сына Создателя, так и Созидательного Духа. Кроме того, такое существо способно глубоко понимать и сочувственно общаться как с духовными серафическими воинствами, так и с материальными эволюционными волевыми созданиями.
\vs p033 4:4 \pc Яркая Утренняя Звезда не создатель, но он изумительный руководитель~--- личный административный представитель Сына Создателя. Кроме созидания и наделения жизнью, Сын и Дух никогда не совещаются о важных вселенских вопросах в отсутствие Гавриила.
\vs p033 4:5 Гавриил Спасограда является главой исполнительной власти вселенной Небадон и арбитром всех административных апелляций, касающихся её управления. Этот руководитель вселенной был создан полностью подготовленным для выполнения своей работы, но он приобрёл опыт с ростом и эволюцией нашего локального творения.
\vs p033 4:6 Гавриил~--- главное должностное лицо, ответственное за исполнение мандатов сверхвселенной, касающихся неличностных дел в локальной вселенной. Большинство вопросов, связанных с массовыми судами и диспенсационными воскресениями, по которым принимают решения От Века Древние, также делегируются для исполнения Гавриилу и его персоналу. Таким образом, Гавриил исполняет объединённые обязанности главного руководителя правителей как сверх-, так и локальной вселенной. В его распоряжении квалифицированный корпус нераскрытых эволюционным смертным административных помощников, созданных для выполнения своей специальной работы. В дополнение к этим помощникам Гавриил может использовать любые и все категории небесных существ, действующих в Небадоне, а также он является главнокомандующим <<армий небес>>~--- небесных воинств.
\vs p033 4:7 \pc Гавриил и его сотрудники не являются учителями; они~--- администраторы. Известно, что они никогда не отклонялись от своей постоянной работы, за исключением случаев, когда Михаил воплощался для посвящений в качестве создания. В течение таких посвящений Гавриил всегда служил в соответствии с волей воплощённого Сына, и при содействии От Века Единого он стал фактическим руководителем вселенских дел во время более поздних посвящений. Со времени посвящения Михаила в облике смертного Гавриил тесно связан с историей и развитием Урантии.
\vs p033 4:8 Кроме встреч с Гавриилом на мирах посвящений и во время общих и особых перекличек по случаю воскресения, смертные редко встречаются с ним на протяжении своего восхождения через локальную вселенную, пока не будут вовлечены в административную работу локального творения. В качестве администраторов любой категории или уровня вы будете находиться под руководством Гавриила.
\usection{ПОСЛЫ ТРОИЦЫ}
\vs p033 5:1 Управление личностями троичного происхождения завершается на уровне правительства сверхвселенных. Для локальных вселенных характерно двойное руководство, начало концепции отца\hyp{}матери. Отец вселенной~--- это Сын Создатель; а вселенская мать~--- Божественная Служительница, Созидательный Дух локальной вселенной. Однако каждая локальная вселенная благословлена присутствием определённых личностей из центральной вселенной и Рая. Возглавляет эту Райскую группу в Небадоне посол Райской Троицы~--- Иммануил Спасограда, От Века Единый, назначенный в локальную вселенную Небадон. В определённом смысле этот высокий Сын Троицы является также личным представителем Всеобщего Отца при дворе Сына Создателя; отсюда и его имя~--- Иммануил\fnst{В переводе с древнееврейского слово \textheb{עִמָּנוּאֵל} означает <<с нами Бог>>. (См.\,Исайя~7:14, 8:8,10, Матфея~1:23).}.
\vs p033 5:2 Иммануил Спасограда, номер 611\,121 из шестой категории Верховных Троичных Личностей, является существом столь высочайшего достоинства и такой благородной снисходительности в обращении, что отказывается от почитания и поклонения всех живых созданий. Он отличается тем, что является единственной личностью во всём Небадоне, которая никогда не признавала подчинения своему брату Михаилу. Он действует как советник Суверенного Сына, но даёт советы только по просьбе. В отсутствие Сына Создателя он может председательствовать в любом высоком вселенском совете, однако в остальных административных делах вселенной принимает участие только тогда, когда его об этом просят.
\vs p033 5:3 Этот посол Рая в Небадоне не подчинён юрисдикции правительства локальной вселенной. Не применяет он и властных полномочий в исполнительных делах развивающейся локальной вселенной, за исключением надзора за своими, выполняющими роль связных, собратьями, От Века Верными, служащими на столичных мирах созвездий.
\vs p033 5:4 От Века Верные, как и От Века Единый, никогда не дают совета и не предлагают помощи правителям созвездий, если их об этом не просят. Эти Райские послы в созвездиях представляют последнее личное присутствие Стационарных Сынов Троицы, действующих в роли консультантов в локальных вселенных. Созвездия более тесно связаны с администрацией сверхвселенной, чем локальные системы, которые управляются исключительно исконными личностями локальной вселенной.
\usection{ОБЩЕЕ УПРАВЛЕНИЕ}
\vs p033 6:1 Гавриил является главой исполнительной власти и действительным администратором Небадона. Отсутствие Михаила на Спасограде никоим образом не мешает упорядоченному ведению вселенских дел. В отсутствие Михаила, как во время недавней миссии воссоединения Сынов Властелинов Орвонтона на Рае, Гавриил становится регентом вселенной. В таких случаях по всем важным вопросам Гавриил всегда обращается за советом к Иммануилу Спасограда.
\vs p033 6:2 Отец Мелхиседек~--- первый помощник Гавриила. Когда Яркая Утренняя Звезда отсутствует на Спасограде, его обязанности берёт на себя этот изначальный Сын Мелхиседек.
\vs p033 6:3 \pc Различным подчинённым администрациям вселенной поручены определённые особые сферы ответственности. Правительство системы, заботясь в целом о благополучии своих планет, в первую очередь занимается физическим статусом живых существ, биологическими проблемами. В свою очередь, правители созвездия уделяют особое внимание социальным вопросам и условиям управления, преобладающим на различных планетах и системах. Правительство созвездия в основном занимается объединением и стабилизацией. Вселенские правители более высокого уровня больше заняты духовным статусом миров.
\vs p033 6:4 \pc Послы назначаются судейским решением и представляют одни вселенные другим вселенным. Консулы являются представителями созвездий друг для друга и для столицы вселенной; они назначаются законодательным актом и действуют только в пределах локальной вселенной. Наблюдатели уполномочены исполнительным распоряжением Властелина Системы представлять данную систему остальным системам и столице созвездия, и они также функционируют только в пределах локальной вселенной.
\vs p033 6:5 \pc Передачи из Спасограда транслируются одновременно столицам созвездий и систем, а также отдельным планетам. Все высшие категории небесных существ способны использовать эту службу для связи со своими собратьями, разбросанными по вселенной. Вселенское вещание распространяется на все обитаемые миры, независимо от их духовного статуса. В межпланетной связи отказано только тем мирам, которые находятся в духовном карантине.
\vs p033 6:6 Трансляции созвездия периодически посылаются из столицы созвездия главой Отцов Созвездия.
\vs p033 6:7 \pc Хронология поддерживается, вычисляется и корректируется специальной группой существ на Спасограде. Стандартный день Небадона равен 18 дням, 6 часам и 2\bibfrac{1}{2} минутам времени Урантии. Год Небадона состоит из определённой части времени вселенского оборота по отношению к контуру Уверсы и равен 100 дням стандартного вселенского времени~--- примерно 5 годам времени Урантии.
\vs p033 6:8 Время Небадона, транслируемое из Спасограда, является стандартом для всех созвездий и систем этой локальной вселенной. Каждое созвездие ведёт свои дела по времени Небадона, однако системы, как и отдельные планеты, поддерживают собственную хронологию.
\vs p033 6:9 День Сатании, согласно исчислению Иерусема, немного короче (на 1 час, 4 минуты и 15 секунд) трёх дней времени Урантии. Эти системы отсчёта времени обычно известны как время Спасограда, или вселенское время, и время Сатании, или системное время, соответственно. Стандартным является вселенское время.
\usection{СУДЫ НЕБАДОНА}
\vs p033 7:1 Сын Властелин Михаил наивысшее внимание уделяет трём направлениям: созиданию, поддержке и служению. Он лично не участвует в судебной деятельности вселенной. Создатели никогда не судят свои создания; эта функция принадлежит исключительно созданиям с высокой подготовкой и реальным опытом созданных существ.
\vs p033 7:2 Весь судебный механизм Небадона находится под надзором Гавриила. Высокие суды, расположенные на Спасограде, занимаются проблемами, имеющими значение для всей вселенной, и апелляционными делами, поступающими из трибуналов систем. Существует 70 отделений этих вселенских судов, и они функционируют в семи подразделениях по десять секций в каждом. Во всех судебных разбирательствах преобладает двойная магистратура, состоящая из одного судьи, обладающего изначальным совершенством, и одного мирового судьи, прошедшего опыт восхождения.
\vs p033 7:3 Что касается юрисдикции, то суды локальных вселенных ограничены в следующих вопросах:
\vs p033 7:4 \li{1.}Администрация локальной вселенной занимается созиданием, эволюцией, поддержкой и служением. Поэтому вселенским трибуналам отказано в праве рассматривать дела, связанные с вопросами вечной жизни и смерти. Это не имеет отношения к естественной смерти в том виде, в каком она существует на Урантии, но если возникает необходимость вынесения решения по вопросу о праве на продолжение существования, вечную жизнь, то такое решение должно быть передано в трибуналы Орвонтона, и если решение принято не в пользу индивидуума, то приговоры о прекращении существования приводятся в исполнение по приказу и средствами правителей сверхправительства.
\vs p033 7:5 \li{2.}Невыполнение обязательств или отступничество любого из Сынов Бога Локальной Вселенной, которые ставят под угрозу их статус и авторитет как Сынов, никогда не рассматриваются судами Сына; такое недоразумение немедленно передаётся в суды сверхвселенной.
\vs p033 7:6 \li{3.}После духовной изоляции вопрос реадмиссии любой составной части локальной вселенной~--- такой как локальная система~--- в содружество с полным духовным статусом в локальном творении должен получить одобрение высшей ассамблеи сверхвселенной.
\vs p033 7:7 \pc Во всех остальных вопросах суды Спасограда окончательны и верховны. Их решения и постановления неизбежны и обжалованию не подлежат.
\vs p033 7:8 Какими бы несправедливыми ни казались порой решения, принимаемые человеческими судами на Урантии, во вселенной торжествуют справедливость и божественная беспристрастность. Ты живёшь в хорошо организованной вселенной, и рано или поздно сможешь положиться на справедливое и даже милосердное обхождение.
\usection{ЗАКОНОДАТЕЛЬНЫЕ И ИСПОЛНИТЕЛЬНЫЕ ФУНКЦИИ}
\vs p033 8:1 На Спасограде, столице Небадона, нет настоящих законодательных органов. Столичные миры вселенных в основном занимаются вынесением решений. Законодательные ассамблеи локальной вселенной расположены на столицах 100 созвездий. Системы главным образом связаны с исполнительной и административной работой локальных творений. Властелины Систем и их помощники обеспечивают соблюдение законодательных полномочий правителей созвездий и исполняют судебные постановления высоких судов вселенной.
\vs p033 8:2 Хотя в столице вселенной собственно законодательство не задействовано, на Спасограде действуют разнообразные консультативные и исследовательские ассамблеи, которые по\hyp{}разному формируются и проводятся в соответствии с их масштабами и целями. Некоторые из них являются постоянными, другие распускаются после достижения своей цели.
\vs p033 8:3 \pc \bibemph{Верховный совет} локальной вселенной состоит из трёх членов от каждой системы и семи представителей от каждого созвездия. Изолированные системы не имеют представительства в этой ассамблее, но им разрешено посылать наблюдателей, которые присутствуют и изучают все её дискуссии.
\vs p033 8:4 \pc \bibemph{Сто советов верховных санкций} также находятся на Спасограде. Президенты этих советов образуют непосредственный рабочий кабинет Гавриила.
\vs p033 8:5 \pc Все заключения высоких консультативных советов вселенной передаются либо судебным органам Спасограда, либо законодательным ассамблеям созвездий. Эти высокие советы не обладают властью или полномочиями обеспечить исполнение своих рекомендаций. Если их советы основаны на фундаментальных законах вселенной, тогда суды Небадона выносят постановление об исполнении; но если их рекомендации связаны с локальными или чрезвычайными условиями, то они должны передаваться вниз, в законодательные ассамблеи созвездия для обсуждения и введения закона в силу, а затем~--- властям системы для исполнения. В действительности эти высокие советы представляют собой сверхзаконодательные органы вселенной, но они действуют без права введения в силу законов и без власти обеспечивать их исполнение.
\vs p033 8:6 Хотя мы говорим о вселенской администрации, пользуясь терминами <<суды>> и <<ассамблеи>>, следует понимать, что эти духовные процессы очень отличаются от более примитивных материальных видов деятельности, носящих соответствующие названия на Урантии.
\vsetoff
\vs p033 8:7 [Представлено Главой Архангелов Небадона.]
\quizlink
