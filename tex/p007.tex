\upaper{7}{СВЯЗЬ ВЕЧНОГО СЫНА СО ВСЕЛЕННОЙ}
\uminitoc{КОНТУР ГРАВИТАЦИИ ДУХА}
\uminitoc{АДМИНИСТРАЦИЯ ВЕЧНОГО СЫНА}
\uminitoc{СВЯЗЬ ВЕЧНОГО СЫНА С ИНДИВИДУУМОМ}
\uminitoc{БОЖЕСТВЕННЫЕ ПЛАНЫ СОВЕРШЕНСТВОВАНИЯ}
\uminitoc{ДУХ ПОСВЯЩЕНИЯ}
\uminitoc{РАЙСКИЕ СЫНЫ БОГА}
\uminitoc{ВЕРХОВНОЕ ОТКРОВЕНИЕ ОТЦА}
\author{Божественный Советник}
\vs p007 0:1 Изначальный Сын постоянно занят исполнением духовных аспектов вечного замысла Отца по мере постепенного раскрытия этого замысла в феноменах эволюционирующих вселенных с их многочисленными группами живых существ. Мы не полностью понимаем этот вечный план, но Райский Сын, несомненно, понимает.
\vs p007 0:2 Сын подобен Отцу в том, что он стремится даровать всё возможное от себя своим равным Сынам и подчинённым им Сынам. И Сын разделяет самораспределительную природу Отца в безграничном посвящении себя Бесконечному Духу, их совместному исполнителю.
\vs p007 0:3 \pc Как хранитель реальностей духа, Второй Источник и Центр является вечным противовесом Острова Рай, который так великолепно поддерживает всё материальное. Таким образом, Первый Источник и Центр навеки раскрыт в материальной красоте изысканных образцов центрального Острова и в духовных ценностях небесной личности Вечного Сына.
\vs p007 0:4 \pc Вечный Сын --- действительный хранитель огромного творения реальностей духа и духовных существ. Мир духа --- это привычка, личное поведение Сына, и безличностные реальности духовной природы всегда реагируют на волю и цель совершенной личности Абсолютного Сына.
\vs p007 0:5 Однако Сын не несёт личной ответственности за поведение всех духов\hyp{}личностей. Воля личностного создания относительно свободна и, следовательно, определяет действия таких волевых существ. Поэтому мир свободно\hyp{}волевых духов не всегда правильно представляет характер Вечного Сына, так же как природа на Урантии не является подлинным откровением совершенства и неизменности Рая и Божества. Но независимо от того, что может характеризовать добровольное действие человека или ангела, вечное объятие Сыном универсального гравитационного контроля всех реальностей духа остаётся абсолютным.
\usection{КОНТУР ГРАВИТАЦИИ ДУХА}
\vs p007 1:1 Всё, чему учат об имманентности Бога, его вездесущности, всемогуществе и всеведении, одинаково верно и в отношении Сына в духовных областях. Чистая и всеобщая гравитация духа во всём творении, этот исключительно духовный контур, ведёт прямо к личности Второго Источника и Центра в Раю. Он возглавляет контроль и функционирование этого постоянно присутствующего и безошибочного духовного охвата всех истинных ценностей духа. Так Вечный Сын осуществляет абсолютное духовное владычество. Он буквально держит все духовные реальности и все одухотворённые ценности как бы на ладони своей руки. Контроль над всеобщей духовной гравитацией \bibemph{и есть} всеобщее духовное владычество.
\vs p007 1:2 Этот гравитационный контроль всего духовного действует независимо от времени и пространства; поэтому энергия духа не уменьшается при передаче. Гравитация духа не подвержена временн\'ым задержкам или пространственному ослаблению. Она не убывает в соответствии с квадратом расстояния, на которое передаётся; контуры мощи чистого духа не замедляются массой материального творения. И эта трансцендентность энергий чистого духа над временем и пространством присуща абсолютности Сына; она не обязана вмешательству сил антигравитации Третьего Источника и Центра.
\vs p007 1:3 Реальности духа реагируют на притягивающую мощь центра духовной гравитации в соответствии с их качественной ценностью, их действительной степенью природы духа. Субстанция духа (качество) так же реагирует на гравитацию духа, как организованная энергия физической материи (количество) реагирует на физическую гравитацию. Духовные ценности и силы духа \bibemph{реальны}. С точки зрения личности, дух --- это душа творения; материя --- призрачное физическое тело.
\vs p007 1:4 Реакции и флуктуации гравитации духа всегда соответствуют содержанию духовных ценностей, качественному духовному статусу индивидуума или мира. Эта сила притяжения мгновенно реагирует на меж\hyp{} и внутридуховные ценности любой вселенской ситуации или планетарного состояния. Каждый раз, когда духовная реальность актуализируется во вселенных, это изменение требует непосредственной и мгновенной перенастройки гравитации духа. Такой новый дух на самом деле является частью Второго Источника и Центра; и как смертный человек становится одухотворённым существом, так же несомненно он достигает духовного Сына, центра и источника гравитации духа.
\vs p007 1:5 \pc Духовная притягательная мощь Сына присуща, в меньшей степени, многим Райским категориям сыновства. Ибо внутри контура абсолютной гравитации духа действительно существуют те локальные системы духовного притяжения, которые функционируют в меньших единицах творения. Такие субабсолютные фокализации\fnst{То есть <<сосредоточения>>.} гравитации духа являются частью божественности личностей Создателей времени и пространства и коррелируют с выявляющимся эмпирическим сверхконтролем Верховного Существа.
\vs p007 1:6 Притяжение гравитации духа и реакция на него действуют не только на вселенную как целое, но и между индивидуумами и группами людей. Духовная сплочённость возникает между духовными и одухотворёнными личностями любого мира, расы, нации или группы верующих индивидуумов. Между духовно мыслящими людьми со схожими вкусами и стремлениями существует прямое притяжение духовной природы. Термин \bibemph{<<родственные духи>>}\fnst{Русским аналогом английского выражения \bibemph{kindred spirits} является \bibemph{родственные души,} но, поскольку здесь речь идёт именно о \bibemph{духах,} а не о \bibemph{душах,} я предпочёл дословный перевод.} --- не просто фигура речи.
\vs p007 1:7 \pc Духовная гравитация Вечного Сына --- подобно материальной гравитации Рая --- абсолютна. Грех и восстание могут помешать работе контуров локальной вселенной, но ничто не может приостановить гравитацию духа Вечного Сына. Восстание Люцифера привело ко многим изменениям в вашей системе обитаемых миров и на Урантии, но мы не замечаем, чтобы возникший в результате духовный карантин вашей планеты хоть сколько\hyp{}нибудь повлиял на присутствие и функцию вездесущего духа Вечного Сына или связанного с ним контура гравитации духа.
\vs p007 1:8 \pc Все реакции контура гравитации духа большой вселенной предсказуемы. Мы распознаём все действия и реакции вездесущего духа Вечного Сына и считаем их надёжными. В соответствии с хорошо известными законами мы можем измерить и измеряем духовную гравитацию точно так же, как человек пытается рассчитать действия конечной физической гравитации. Дух Сына неизменно откликается на все духовные предметы, существа и личности, и этот отклик всегда соответствует степени актуальности (качественной степени реальности) всех таких духовных ценностей.
\vs p007 1:9 Но, наряду с этой очень надёжной и предсказуемой функцией духовного присутствия Вечного Сына, встречаются явления, которые не так предсказуемы в своих реакциях. Подобные явления, вероятно, указывают на согласованное действие Божества Абсолюта в сферах выявляющихся духовных потенциалов. Мы знаем, что присутствие духа Вечного Сына --- это влияние величественной и бесконечной личности, но едва ли мы можем считать личностными реакции, связанные с предполагаемыми действиями Божества Абсолюта.
\vs p007 1:10 \pc Если рассмотреть с точки зрения личности и лицами\fnst{Англ. Viewed from the personality standpoint and by persons.}, Вечный Сын и Божество Абсолют кажутся связанными следующим образом: Вечный Сын доминирует в сфере актуальных духовных ценностей, тогда как Божество Абсолют, по\hyp{}видимому, пронизывает обширную область потенциальных ценностей духа. Всякая актуальная ценность природы духа находит пристанище в гравитационном охвате Вечного Сына, но если она потенциальна, то, по\hyp{}видимому, находится в присутствии Божества Абсолюта.
\vs p007 1:11 Надо полагать, что Дух исходит из потенциалов Божества Абсолюта; развивающийся дух находит корреляцию в эмпирическом и незавершённом охватах Верховного и Предельного; в итоге дух обретает окончательное предназначение в абсолютном охвате духовной гравитации Вечного Сына. Таким представляется цикл эмпирического духа, но экзистенциальный дух присущ бесконечности Второго Источника и Центра.
\usection{АДМИНИСТРАЦИЯ ВЕЧНОГО СЫНА}
\vs p007 2:1 В Раю присутствие и личная деятельность Изначального Сына являются полными, абсолютными в духовном смысле. По мере того как мы продвигаемся из Рая через Хавону в области семи сверхвселенных, мы обнаруживаем всё меньше и меньше личной активности Вечного Сына. В пост\hyp{}Хавонских вселенных присутствие Вечного Сына персонализуется в Райских Сынах, обуславливается эмпирическими реальностями Верховного и Предельного и координируется с неограниченным потенциалом духа Божества Абсолюта.
\vs p007 2:2 В центральной вселенной личная деятельность Изначального Сына проявляется в изысканной духовной гармонии этого вечного творения. Хавона так изумительно совершенна, что духовный статус и энергетические состояния этой образцовой вселенной находятся в совершенном и вечном равновесии.
\vs p007 2:3 В сверхвселенных Сын не присутствует и не обитает лично; в этих творениях он поддерживает только сверхличностное представительство. Эти духовные проявления Сына не являются личностными; они не входят в личностный контур Всеобщего Отца. Мы не знаем лучшего термина для их обозначения, чем \bibemph{сверхличности;} и это конечные существа; они не являются ни абсонитными, ни абсолютными.
\vs p007 2:4 Aдминистрация Вечного Сына в сверхвселенных, будучи исключительно духовной и сверхличностной, не различима личностными созданиями. Тем не менее всепроникающее духовное побуждение личного влияния Сына встречается в каждой фазе деятельности всех секторов владений Ветхих Днями. Однако в локальных вселенных мы видим Вечного Сына, лично присутствующим в лицах Райских Сынов. Здесь Бесконечный Сын духовно и созидательно функционирует в лицах величественного корпуса равных Сынов Создателей.
\usection{СВЯЗЬ ВЕЧНОГО СЫНА С ИНДИВИДУУМОМ}
\vs p007 3:1 При восхождении в локальной вселенной смертные времени смотрят на Сына Создателя как на личного представителя Вечного Сына. Но начиная восхождение в режиме обучения сверхвселенной, пилигримы времени всё чаще обнаруживают возвышенное присутствие вдохновляющего духа Вечного Сына, и они могут извлечь пользу принятием этой помощи духовного возбуждения. В Хавоне восходящие создания начинают ещё больше осознавать объятия любви всепроникающего духа Изначального Сына. Ни на одном этапе всего восхождения смертных дух Вечного Сына не обитает в разуме или душе пилигрима времени, но его милосердие всегда рядом и всегда заботится о благополучии и духовной безопасности продвигающихся детей времени.
\vs p007 3:2 Притяжение духовной гравитации Вечного Сына составляет тайну, присущую восхождению к Раю выживающих человеческих душ. Все истинные духовные ценности и все по\hyp{}настоящему одухотворённые личности удерживаются в пределах неизменного охвата духовной гравитации Вечного Сына. Например, смертный разум начинает свою карьеру как материальный механизм и в итоге принимается в Корпус Завершения как почти совершенное духовное существо\fnst{Дословно <<существование>> (англ. existence).}, постепенно становясь всё менее подверженным материальной гравитации и, соответственно, более реагирующим на влекущую вовнутрь гравитацию духа, в течение всего этого опыта. Контур гравитации духа буквально тянет душу человека в направлении Рая.
\vs p007 3:3 \pc
\vs p007 3:4 \pc
\vs p007 3:5
\vs p007 3:6
\vs p007 3:7
\usection{БОЖЕСТВЕННЫЕ ПЛАНЫ СОВЕРШЕНСТВОВАНИЯ}
\vs p007 4:1
\vs p007 4:2
\vs p007 4:3 \pc
\vs p007 4:4
\vs p007 4:5
\vs p007 4:6
\vs p007 4:7 \pc
\usection{ДУХ ПОСВЯЩЕНИЯ}
\vs p007 5:1
\vs p007 5:2
\vs p007 5:3
\vs p007 5:4
\vs p007 5:5 \pc
\vs p007 5:6
\vs p007 5:7 \pc
\vs p007 5:8 \pc
\vs p007 5:9
\vs p007 5:10 \pc
\vs p007 5:11
\usection{РАЙСКИЕ СЫНЫ БОГА}
\vs p007 6:1
\vs p007 6:2
\vs p007 6:3 \pc
\vs p007 6:4
\vs p007 6:5 \pc
\vs p007 6:6 \pc
\vs p007 6:7 \pc
\vs p007 6:8
\usection{ВЕРХОВНОЕ ОТКРОВЕНИЕ ОТЦА}
\vs p007 7:1
\vs p007 7:2
\vs p007 7:3
\vs p007 7:4
\vs p007 7:5 \pc
\vs p007 7:6
\vsetoff
\vs p007 7:7
\quizlink
