\upaper{7}{СВЯЗЬ ВЕЧНОГО СЫНА СО ВСЕЛЕННОЙ}
\uminitoc{КОНТУР ГРАВИТАЦИИ ДУХА}
\uminitoc{АДМИНИСТРАЦИЯ ВЕЧНОГО СЫНА}
\uminitoc{СВЯЗЬ ВЕЧНОГО СЫНА С ИНДИВИДУУМОМ}
\uminitoc{БОЖЕСТВЕННЫЕ ПЛАНЫ СОВЕРШЕНСТВОВАНИЯ}
\uminitoc{ДУХ ПОСВЯЩЕНИЯ}
\uminitoc{РАЙСКИЕ СЫНЫ БОГА}
\uminitoc{ВЕРХОВНОЕ ОТКРОВЕНИЕ ОТЦА}
\author{Божественный Советник}
\vs p007 0:1 Изначальный Сын постоянно занят исполнением духовных аспектов вечного замысла Отца по мере постепенного раскрытия этого замысла в феноменах эволюционирующих вселенных с их многочисленными группами живых существ. Мы не полностью понимаем этот вечный план, но Райский Сын, несомненно, понимает.
\vs p007 0:2 Сын подобен Отцу в том, что он стремится даровать всё возможное от себя своим равным Сынам и подчинённым им Сынам. И Сын разделяет самораспределительную природу Отца в безграничном посвящении себя Бесконечному Духу, их совместному исполнителю.
\vs p007 0:3 \pc Как хранитель реальностей духа, Второй Источник и Центр является вечным противовесом Острова Рай, который так великолепно поддерживает всё материальное. Таким образом, Первый Источник и Центр навеки раскрыт в материальной красоте изысканных образцов центрального Острова и в духовных ценностях небесной личности Вечного Сына.
\vs p007 0:4 \pc Вечный Сын --- действительный хранитель огромного творения реальностей духа и духовных существ. Мир духа --- это привычка, личное поведение Сына, и безличностные реальности духовной природы всегда реагируют на волю и цель совершенной личности Абсолютного Сына.
\vs p007 0:5 Однако Сын не несёт личной ответственности за поведение всех духов\hyp{}личностей. Воля личностного создания относительно свободна и, следовательно, определяет действия таких волевых существ. Поэтому мир свободно\hyp{}волевых духов не всегда правильно представляет характер Вечного Сына, так же как природа на Урантии не является подлинным откровением совершенства и неизменности Рая и Божества. Но независимо от того, что может характеризовать добровольное действие человека или ангела, вечное объятие Сыном универсального гравитационного контроля всех реальностей духа остаётся абсолютным.
\usection{КОНТУР ГРАВИТАЦИИ ДУХА}
\vs p007 1:1 Всё, чему учат об имманентности Бога, его вездесущности, всемогуществе и всеведении, одинаково верно и в отношении Сына в духовных областях. Чистая и всеобщая гравитация духа во всём творении, этот исключительно духовный контур, ведёт прямо к личности Второго Источника и Центра в Раю. Он возглавляет контроль и функционирование этого постоянно присутствующего и безошибочного духовного охвата всех истинных ценностей духа. Так Вечный Сын осуществляет абсолютное духовное владычество. Он буквально держит все духовные реальности и все одухотворённые ценности как бы на ладони своей руки. Контроль над всеобщей духовной гравитацией \bibemph{и есть} всеобщее духовное владычество.
\vs p007 1:2 Этот гравитационный контроль всего духовного действует независимо от времени и пространства; поэтому энергия духа не уменьшается при передаче. Гравитация духа не подвержена временн\'ым задержкам или пространственному ослаблению. Она не убывает в соответствии с квадратом расстояния, на которое передаётся; контуры мощи чистого духа не замедляются массой материального творения. И эта трансцендентность энергий чистого духа над временем и пространством присуща абсолютности Сына; она не обязана вмешательству сил антигравитации Третьего Источника и Центра.
\vs p007 1:3 Реальности духа реагируют на притягивающую мощь центра духовной гравитации в соответствии с их качественной ценностью, их действительной степенью природы духа. Субстанция духа (качество) так же реагирует на гравитацию духа, как организованная энергия физической материи (количество) реагирует на физическую гравитацию. Духовные ценности и силы духа \bibemph{реальны}. С точки зрения личности, дух --- это душа творения; материя --- призрачное физическое тело.
\vs p007 1:4 Реакции и флуктуации гравитации духа всегда соответствуют содержанию духовных ценностей, качественному духовному статусу индивидуума или мира. Эта сила притяжения мгновенно реагирует на меж\hyp{} и внутридуховные ценности любой вселенской ситуации или планетарного состояния. Каждый раз, когда духовная реальность актуализируется во вселенных, это изменение требует непосредственной и мгновенной перенастройки гравитации духа. Такой новый дух на самом деле является частью Второго Источника и Центра; и как смертный человек становится одухотворённым существом, так же несомненно он достигает духовного Сына, центра и источника гравитации духа.
\vs p007 1:5 \pc Духовная притягательная мощь Сына присуща, в меньшей степени, многим Райским категориям сыновства. Ибо внутри контура абсолютной гравитации духа действительно существуют те локальные системы духовного притяжения, которые функционируют в меньших единицах творения. Такие субабсолютные фокализации\fnst{То есть <<сосредоточения>>.} гравитации духа являются частью божественности личностей Создателей времени и пространства и коррелируют с выявляющимся эмпирическим сверхконтролем Верховного Существа.
\vs p007 1:6 Притяжение гравитации духа и реакция на него действуют не только на вселенную как целое, но и между индивидуумами и группами людей. Духовная сплочённость возникает между духовными и одухотворёнными личностями любого мира, расы, нации или группы верующих индивидуумов. Между духовно мыслящими людьми со схожими вкусами и стремлениями существует прямое притяжение духовной природы. Термин \bibemph{<<родственные духи>>}\fnst{Русским аналогом английского выражения \bibemph{kindred spirits} является \bibemph{родственные души,} но, поскольку здесь речь идёт именно о \bibemph{духах,} а не о \bibemph{душах,} я предпочёл дословный перевод.} --- не просто фигура речи.
\vs p007 1:7 \pc Духовная гравитация Вечного Сына --- подобно материальной гравитации Рая --- абсолютна. Грех и восстание могут помешать работе контуров локальной вселенной, но ничто не может приостановить гравитацию духа Вечного Сына. Восстание Люцифера привело ко многим изменениям в вашей системе обитаемых миров и на Урантии, но мы не замечаем, чтобы возникший в результате духовный карантин вашей планеты хоть сколько\hyp{}нибудь повлиял на присутствие и функцию вездесущего духа Вечного Сына или связанного с ним контура гравитации духа.
\vs p007 1:8 \pc Все реакции контура гравитации духа большой вселенной предсказуемы. Мы распознаём все действия и реакции вездесущего духа Вечного Сына и считаем их надёжными. В соответствии с хорошо известными законами мы можем измерить и измеряем духовную гравитацию точно так же, как человек пытается рассчитать действия конечной физической гравитации. Дух Сына неизменно откликается на все духовные предметы, существа и личности, и этот отклик всегда соответствует степени актуальности (качественной степени реальности) всех таких духовных ценностей.
\vs p007 1:9 Но, наряду с этой очень надёжной и предсказуемой функцией духовного присутствия Вечного Сына, встречаются явления, которые не так предсказуемы в своих реакциях. Подобные явления, вероятно, указывают на согласованное действие Божества Абсолюта в сферах выявляющихся духовных потенциалов. Мы знаем, что присутствие духа Вечного Сына --- это влияние величественной и бесконечной личности, но едва ли мы можем считать личностными реакции, связанные с предполагаемыми действиями Божества Абсолюта.
\vs p007 1:10 \pc Если рассмотреть с точки зрения личности и лицами\fnst{Англ. Viewed from the personality standpoint and by persons.}, Вечный Сын и Божество Абсолют кажутся связанными следующим образом: Вечный Сын доминирует в сфере актуальных духовных ценностей, тогда как Божество Абсолют, по\hyp{}видимому, пронизывает обширную область потенциальных ценностей духа. Всякая актуальная ценность природы духа находит пристанище в гравитационном охвате Вечного Сына, но если она потенциальна, то, по\hyp{}видимому, находится в присутствии Божества Абсолюта.
\vs p007 1:11 Надо полагать, что Дух исходит из потенциалов Божества Абсолюта; развивающийся дух находит корреляцию в эмпирическом и незавершённом охватах Верховного и Предельного; в итоге дух обретает окончательное предназначение в абсолютном охвате духовной гравитации Вечного Сына. Таким представляется цикл эмпирического духа, но экзистенциальный дух присущ бесконечности Второго Источника и Центра.
\usection{АДМИНИСТРАЦИЯ ВЕЧНОГО СЫНА}
\vs p007 2:1 В Раю присутствие и личная деятельность Изначального Сына являются полными, абсолютными в духовном смысле. По мере того как мы продвигаемся из Рая через Хавону в области семи сверхвселенных, мы обнаруживаем всё меньше и меньше личной активности Вечного Сына. В пост\hyp{}Хавонских вселенных присутствие Вечного Сына персонализуется в Райских Сынах, обуславливается эмпирическими реальностями Верховного и Предельного и координируется с неограниченным потенциалом духа Божества Абсолюта.
\vs p007 2:2 В центральной вселенной личная деятельность Изначального Сына проявляется в изысканной духовной гармонии этого вечного творения. Хавона так изумительно совершенна, что духовный статус и энергетические состояния этой образцовой вселенной находятся в совершенном и вечном равновесии.
\vs p007 2:3 В сверхвселенных Сын не присутствует и не обитает лично; в этих творениях он поддерживает только сверхличностное представительство. Эти духовные проявления Сына не являются личностными; они не входят в личностный контур Всеобщего Отца. Мы не знаем лучшего термина для их обозначения, чем \bibemph{сверхличности;} и это конечные существа; они не являются ни абсонитными, ни абсолютными.
\vs p007 2:4 Aдминистрация Вечного Сына в сверхвселенных, будучи исключительно духовной и сверхличностной, не различима личностными созданиями. Тем не менее всепроникающее духовное побуждение личного влияния Сына встречается в каждой фазе деятельности всех секторов владений Ветхих Днями. Однако в локальных вселенных мы видим Вечного Сына, лично присутствующим в лицах Райских Сынов. Здесь Бесконечный Сын духовно и созидательно функционирует в лицах величественного корпуса равных Сынов Создателей.
\usection{СВЯЗЬ ВЕЧНОГО СЫНА С ИНДИВИДУУМОМ}
\vs p007 3:1 При восхождении в локальной вселенной смертные времени смотрят на Сына Создателя как на личного представителя Вечного Сына. Но начиная восхождение в режиме обучения сверхвселенной, пилигримы времени всё чаще обнаруживают возвышенное присутствие вдохновляющего духа Вечного Сына, и они могут извлечь пользу от принятия этой помощи духовного возбуждения. В Хавоне восходящие создания начинают ещё больше осознавать объятия любви всепроникающего духа Изначального Сына. Ни на одном этапе всего восхождения смертных дух Вечного Сына не обитает в разуме или душе пилигрима времени, но его милосердие всегда рядом и всегда заботится о благополучии и духовной безопасности продвигающихся детей времени.
\vs p007 3:2 Притяжение духовной гравитации Вечного Сына составляет тайну, присущую восхождению к Раю выживающих человеческих душ. Все истинные духовные ценности и все по\hyp{}настоящему одухотворённые личности удерживаются в пределах неизменного охвата духовной гравитации Вечного Сына. Например, смертный разум начинает свою карьеру как материальный механизм и в итоге принимается в Корпус Завершения как почти совершенное духовное существо\fnst{Дословно <<существование>> (англ. existence).}, постепенно становясь всё менее подверженным материальной гравитации и, соответственно, более реагирующим на влекущую вовнутрь гравитацию духа, в течение всего этого опыта. Контур гравитации духа буквально тянет душу человека в направлении Рая.
\vs p007 3:3 \pc Контур гравитации духа является основным каналом для передачи искренних молитв верующего человеческого сердца с уровня человеческого сознания к самом\'у сознанию Божества. То, что представляет истинную духовную ценность в твоих прошениях, будет захвачено вселенским контуром гравитации духа и будет передано немедленно и одновременно\fnst{Если эти божественные личности находятся в состоянии движения друг относительно друга, то неясно, что имеется в виду под словом <<одновременно>>, ибо, согласно теории относительности Пуанкаре\hyp{}Лоренца\hyp{}Эйнштейна, понятие одновременности не является абсолютным.} ко всем заинтересованным божественным личностям. Каждая из них будет заниматься тем, что принадлежит её личной компетенции. Следовательно, в твоём практическом религиозном опыте при отправлении прошений не имеет значения, представляешь ли ты мысленно Сына Создателя твоей локальной вселенной или Вечного Сына в центре всего сущего.
\vs p007 3:4 \pc Различительное действие контура гравитации духа можно сравнить с функциями нервных контуров в материальном человеческом теле: ощущения путешествуют внутрь по нервным путям; некоторые задерживаются нижними автоматическими спинномозговыми центрами, реагирующими на них; другие передаются менее автоматическим, но обученным привычками, центрам нижнего мозга, в то время как самые важные и жизненно необходимые поступающие сигналы минуют эти подчинённые центры и немедленно регистрируются в самых высших уровнях человеческого сознания.
\vs p007 3:5 Но насколько более совершенны возвышенные методы духовного мира! Если что\hyp{}нибудь, наделённое высшей духовной ценностью, родится в твоём сознании, то, как только ты выразишь это, никакая сила во вселенной не сможет предотвратить его молниеносную передачу непосредственно к Абсолютной Духовной Личности всего творения.
\vs p007 3:6 И наоборот, если твои мольбы чисто материальны и полностью эгоцентричны, не существует метода, посредством которого такие недостойные молитвы могли бы найти пристанище в контуре духа Вечного Сына. Для любого прошения, содержание которого не является <<продиктованным духом>>, не может быть места во вселенском духовном контуре; такие чисто эгоистичные и материальные запросы бесполезны; они не поднимаются по контурам истинных ценностей духа. Такие слова --- лишь <<медь звенящая и кимвал звучащий>>\fnst{Цитата из 1~Кор~13:1: <<Если я говорю языками человеческими и ангельскими, а любви не имею, то я --- медь звенящая или кимвал звучащий>>.}.
\vs p007 3:7 Именно побуждающая мысль --- духовное содержание --- делает реальным мольбу смертного. Слова ничего не стоят.
\usection{БОЖЕСТВЕННЫЕ ПЛАНЫ СОВЕРШЕНСТВОВАНИЯ}
\vs p007 4:1 Вечный Сын находится в постоянном взаимодействии с Отцом для успешного осуществления \bibemph{божественного плана развития:} вселенского плана создания, эволюции, восхождения и совершенствования волевых созданий. И в божественной преданности Сын вечно равен Отцу.
\vs p007 4:2 Отец и его Сын едины в формулировании и реализации этого гигантского плана достижений для продвижения материальных существ времени к совершенству вечности. Этот проект духовного возвышения восходящих душ пространства является совместным творением Отца и Сына, и они, в сотрудничестве с Бесконечным Духом, занимаются совместным исполнением своего божественного замысла.
\vs p007 4:3 \pc Этот божественный план достижения совершенства включает три уникальные, но изумительно согласованные инициативы вселенского приключения:
\vs p007 4:4 \ublistelem{1.}\bibnobreakspace \bibemph{План постепенного достижения.} Это --- план эволюционного восхождения, принадлежащий Всеобщему Отцу, план, безоговорочно принятый Вечным Сыном, когда он согласился с предложением Отца: <<Сотворим смертных созданий по образу нашему>>. Это обеспечение возможности возвышения созданий времени включает в себя дарование Отцом Настройщиков Мыслей и наделение материальных созданий прерогативами личности.
\vs p007 4:5 \ublistelem{2.}\bibnobreakspace \bibemph{План посвящений.} Следующий вселенский план --- это великая инициатива Вечного Сына и его равных Сынов по раскрытию Отца. Это --- предложение Вечного Сына, и состоит оно в посвящении им Сынов Бога эволюционным творениям, чтобы там персонализовать и сделать фактической, воплотить и реализовать любовь Отца и милосердие Сына к созданиям всех вселенных. В качестве неотъемлемой части плана посвящений и являясь предусмотрительной чертой этого служения любви, Райские Сыны действуют как восстановители того, что воля заблуждающихся созданий подвергла духовной опасности. Когда бы и где бы ни произошла задержка в функционировании плана достижения, если случается, что восстание искажает или осложняет это начинание, то немедленно вводятся в действие экстренные меры плана посвящений. Райские Сыны дали клятву и готовы действовать в качестве спасателей, идти в охваченные восстанием миры и там восстановить их духовный статус. Именно такое героическое служение совершил равный Сын Создатель на Урантии в связи с его эмпирической посвященческой задачей обретения верховной власти.
\vs p007 4:6 \ublistelem{3.}\bibnobreakspace \bibemph{План служения милосердия.} После того, как план достижения и план посвящений были сформулированы и провозглашены, Бесконечный Дух --- один и сам по себе -- спроектировал и ввёл в действие грандиозную вселенскую инициативу служения милосердия. Это служение весьма существенно для практического и эффективного выполнения обоих планов --- как достижения, так и посвящений, --- и все духовные личности Третьего Источника и Центра причастны духу служения милосердия, который во многом является частью природы Третьего Лица Божества. Не только в творении, но и в управлении Бесконечный Дух действует истинно и буквально как совместный исполнитель Отца и Сына.
\vs p007 4:7 \pc Вечный Сын --- доверенное лицо, божественный хранитель всеобщего плана Отца восхождения созданий. Провозгласив всеобщий мандат: <<Будьте совершенны, как я совершенен>>, Отец доверил выполнение этого грандиозного начинания Вечному Сыну; и Вечный Сын делит заботу об этом небесном предприятии со своим божественным соратником --- Бесконечным Духом. Так Божества эффективно сотрудничают в работе по созиданию, контролю, эволюции, откровению и служению, и, если понадобится, в восстановлении и реабилитации.
\usection{ДУХ ПОСВЯЩЕНИЯ}
\vs p007 5:1 Вечный Сын безоговорочно присоединился к Всеобщему Отцу, провозгласив это потрясающее предписание всему творению: <<Будьте совершенны, как совершенен Отец ваш в Хавоне>>. И с тех пор это приглашение\hyp{}приказ служит побуждением для всех планов выживания и проектов посвящения Вечного Сына и его огромной семьи равных и ассоциированных Сынов. Именно в этих посвящениях Сыны Бога стали для всех эволюционных созданий <<путём, истиной и жизнью>>.
\vs p007 5:2 Вечный Сын не может напрямую контактировать с человеческими существами, как это делает Отец через дар доличностных Настройщиков Мыслей, но Вечный Сын приближается к сотворённым личностям посредством серии нисходящих ступеней божественного сыновства, пока он не оказывается стоящим в присутствии человека и --- временами --- как сам человек.
\vs p007 5:3 Чисто личная природа Вечного Сына не способна к фрагментации. Вечный Сын служит либо как духовное влияние, либо как личность, и никак иначе. Для Сына невозможно стать частью опыта создания в том смысле, в каком в нём участвует Отец\hyp{}Настройщик, но Вечный Сын компенсирует это ограничение посредством посвящения. То, что опыт фрагментированных сущностей означает для Всеобщего Отца, опыт воплощений Райских Сынов значит для Вечного Сына.
\vs p007 5:4 Вечный Сын не приходит к смертному человеку как божественная воля, Настройщик Мыслей, обитающий в человеческом разуме, но Вечный Сын действительно явился смертному человеку на Урантии, когда божественная \bibemph{личность} его Сына, Михаила Небадона, воплотилась в человеческой природе Иисуса из Назарета. Чтобы разделить опыт сотворённых личностей, Райские Сыны Бога должны принять саму природу таких созданий и воплотить свои божественные личности в качестве настоящих созданий. Воплощение --- секрет Сонарингтона --- это метод избавления Сына от всеохватывающих оков личностного абсолютизма.
\vs p007 5:5 \pc Давным\hyp{}давно Вечный Сын посвятил себя каждому из контуров центрального творения для просвещения и продвижения всех обитателей и пилигримов Хавоны, включая восходящих пилигримов времени. Ни в одном из этих семи посвящений он не действовал ни как восходящий, ни как хавонец. Он был самим собой. Его опыт был уникальным --- приобретённым не \bibemph{вместе с} человеком или \bibemph{подобно} ему или другому пилигриму, а, некоторым образом, ассоциативно в сверхличностном смысле.
\vs p007 5:6 Не прошёл он и через покой, лежащий между внутренним контуром Хавоны и берегами Рая. Для него, абсолютного существа, невозможно приостановить сознание личности, поскольку в нём сосредоточены все линии духовной гравитации. И во время этих посвящений центральная Райская обитель духовного сияния не была затемнена, и охват Сына всеобщей гравитации духа не ослаблялся.
\vs p007 5:7 \pc Посвящения Вечного Сына в Хавоне не вмещаются в рамки человеческого воображения; они были трансцендентными. Он добавил к опыту всей Хавоны тогда и впоследствии, но мы не знаем, добавил ли он к предполагаемой эмпирической способности своей экзистенциальной природы. Это относится к тайне посвящения Райских Сынов. Однако мы верим, что Вечный Сын с тех пор сохраняет всё, что он приобрёл в ходе этих миссий посвящения; но что это --- мы не знаем.
\vs p007 5:8 \pc Какими бы ни были наши трудности в понимании посвящений Второго Лица Божества, мы вполне понимаем посвящение Хавоне одного из Сынов Вечного Сына, который буквально прошёл через контуры центральной вселенной и действительно разделил опыт, составляющий подготовку восходящего к достижению Божества. Это был изначальный Михаил, первородный Сын Создатель, и он прошёл через жизненный опыт восходящих пилигримов от контура к контуру, лично путешествуя с ними по этапу каждого круга во времена Грандфанды, первого из всех смертных, достигших Хавоны.
\vs p007 5:9 Кроме всего остального, что раскрыл этот изначальный Михаил, он сделал трансцендентное посвящение Изначального Сына\hyp{}Матери реальным для созданий Хавоны. Настолько реальным, что навеки каждый пилигрим времени, который трудится над подвигами прохождения контуров Хавоны, ободряется и укрепляется твёрдым знанием того, что Вечный Сын Бога семь раз отрекался от силы и славы Рая, чтобы участвовать в опыте время\hyp{}пространственных пилигримов на семи контурах постепенного освоения Хавоны.
\vs p007 5:10 \pc Вечный Сын --- образец вдохновения для всех Сынов Бога в их службе посвящения во вселенных времени и пространства. Равные Сыны Создатели и ассоциированные Сыны Арбитры, вместе с другими нераскрытыми категориями сыновства, разделяют эту замечательную готовность посвятить себя различным категориям созданной жизни и в качестве самих созданий. Следовательно, по духу и из\hyp{}за родства природы, а также факта происхождения, оказывается истинным, что в посвящении каждого Сына Бога мирам пространства, в этих посвящениях и через них, Вечный Сын посвятил себя разумным волевым созданиям вселенных.
\vs p007 5:11 По духу и природе, если не во всех атрибутах, каждый Райский Сын является божественно совершенным изображением Изначального Сына. Буквальна истина: всякий, кто видел Райского Сына, видел Вечного Сына Бога.
\usection{РАЙСКИЕ СЫНЫ БОГА}
\vs p007 6:1 Недостаток знания о многочисленных Сынах Бога --- источник большой путаницы на Урантии. И это невежество сохраняется, несмотря на наличие таких утверждений, как запись о собрании этих божественных личностей: <<Когда Сыны Божьи провозгласили радость, и все Утренние Звёзды пели вместе>>\fnst{Ср. Иов~38:7: <<при общем ликовании утренних звёзд, когда все сыны Божии восклицали от радости?>>}. Каждое тысячелетие стандартного времени сектора различные категории божественных Сынов собираются на свои периодические собрания.
\vs p007 6:2 Вечный Сын --- это личный источник восхитительных атрибутов милосердия и служения, которые так полно характеризуют все категории нисходящих Сынов Бога, функционирующих по всему творению. Всю божественную природу, если не всю бесконечность атрибутов, Вечный Сын неизменно передаёт Райским Сынам, отправляющимся с Вечного Острова, чтобы раскрыть его божественный характер вселенной вселенных.
\vs p007 6:3 \pc
\vs p007 6:4
\vs p007 6:5 \pc
\vs p007 6:6 \pc
\vs p007 6:7 \pc
\vs p007 6:8
\usection{ВЕРХОВНОЕ ОТКРОВЕНИЕ ОТЦА}
\vs p007 7:1
\vs p007 7:2
\vs p007 7:3
\vs p007 7:4
\vs p007 7:5 \pc
\vs p007 7:6
\vsetoff
\vs p007 7:7
\quizlink
