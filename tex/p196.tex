\upaper{196}{ВЕРА ИИСУСА}
\uminitoc{ИИСУС --- ЧЕЛОВЕК}
\uminitoc{РЕЛИГИЯ ИИСУСА}
\uminitoc{ВЕРХОВЕНСТВО РЕЛИГИИ}
\author{Промежуточные создания}
\vs p196 0:1 Иисус обладал возвышенной и искренней верой в Бога. Он испытывал обычные взлёты и падения смертного существования, но никогда в религиозном смысле не сомневался в бесспорности заботы и водительства Бога. Его вера была результатом проницательности, порождаемой деятельностью божественного присутствия, обитающего в нём Настройщика. Его вера не была ни традиционной, ни чисто интеллектуальной; она была полностью личной и чисто духовной.
\vs p196 0:2 Иисус как человек видел Бога святым, справедливым и великим, а также истинным, прекрасным и добрым. Все эти атрибуты божественности он сосредоточил в своём уме как <<волю Отца Небесного>>. Бог Иисуса был одновременно <<Святым Израиля>> и <<Живым и любящим Отцом небесным>>. Концепция Бога как Отца не исходила от Иисуса, но он возвеличил и поднял эту идею до уровня возвышенного опыта, достигнув нового откровения Бога и провозгласив, что каждое смертное создание есть дитя этого Отца любви, сын Бога.
\vs p196 0:3 Иисус не цеплялся за веру в Бога так, как это делала бы борющаяся душа в войне со вселенной и в смертельной схватке с враждебным и грешным миром; он не прибегал к вере только как к утешению среди трудностей или как к успокоению при надвигающемся отчаянии; вера не была лишь иллюзорной компенсацией за неприятные реалии и печали жизни. Перед лицом всех естественных трудностей и временн\'ых противоречий смертного существования, он испытывал спокойствие высочайшего и бесспорного доверия к Богу и ощущал огромную увлекательность жизни, проживаемой, благодаря вере, в сам\'ом присутствии небесного Отца. И эта торжествующая вера была живым опытом обретения настоящего духа. Великий вклад Иисуса в ценности человеческого опыта заключался не в том, что он раскрыл так много новых идей об Отце небесном, а в том, что он так великолепно и по\hyp{}человечески продемонстрировал новый, более высокий тип \bibemph{живой веры в Бога}. Никогда на всех мирах этой вселенной, в жизни какого\hyp{}либо одного смертного, Бог не становился такой \bibemph{живой реальностью,} как в человеческом опыте Иисуса из Назарета.
\vs p196 0:4 В жизни Учителя на Урантии этот и все другие миры локального творения открывают для себя новый, более высокий тип религии, основанной на личных духовных отношениях с Всеобщим Отцом и полностью подтверждаемой высшим авторитетом подлинного личного опыта. Эта живая вера в Иисуса была больше, чем интеллектуальное размышление, не была она и мистической медитацией.
\vs p196 0:5 
\vs p196 0:6 
\vs p196 0:7 
\vs p196 0:8 
\vs p196 0:9 
\vs p196 0:10 
\vs p196 0:11 
\vs p196 0:12 
\vs p196 0:13 
\vs p196 0:14 
\usection{ИИСУС --- ЧЕЛОВЕК}
\vs p196 1:1 
\vs p196 1:2 
\vs p196 1:3 \pc 
\vs p196 1:4 
\vs p196 1:5 
\vs p196 1:6 
\vs p196 1:7 
\vs p196 1:8 
\vs p196 1:9 
\vs p196 1:10 
\vs p196 1:11 
\vs p196 1:12 
\vs p196 1:13 
\usection{РЕЛИГИЯ ИИСУСА}
\vs p196 2:1 
\vs p196 2:2 
\vs p196 2:3 
\vs p196 2:4 
\vs p196 2:5 
\vs p196 2:6 \pc 
\vs p196 2:7 
\vs p196 2:8 
\vs p196 2:9 
\vs p196 2:10 
\vs p196 2:11 
\usection{ВЕРХОВЕНСТВО РЕЛИГИИ}
\vs p196 3:1 
\vs p196 3:2 
\vs p196 3:3 
\vs p196 3:4 
\vs p196 3:5 
\vs p196 3:6 
\vs p196 3:7 
\vs p196 3:8 
\vs p196 3:9 
\vs p196 3:10 \pc 
\vs p196 3:11 
\vs p196 3:12 
\vs p196 3:13 
\vs p196 3:14 
\vs p196 3:15 \pc 
\vs p196 3:16 
\vs p196 3:17 
\vs p196 3:18 
\vs p196 3:19 
\vs p196 3:20 
\vs p196 3:21 
\vs p196 3:22 
\vs p196 3:23 
\vs p196 3:24 
\vs p196 3:25 
\vs p196 3:26 
\vs p196 3:27 
\vs p196 3:28 
\vs p196 3:29 \pc 
\vs p196 3:30 \pc 
\vs p196 3:31 \pc 
\vs p196 3:32 
\vs p196 3:33 
\vs p196 3:34 
\vs p196 3:35 
\quizlink
