\upaper{11}{ВЕЧНЫЙ ОСТРОВ РАЙ}
\uminitoc{БОЖЕСТВЕННАЯ РЕЗИДЕНЦИЯ}
\uminitoc{ПРИРОДА ВЕЧНОГО ОСТРОВА}
\uminitoc{ВЕРХНИЙ РАЙ}
\uminitoc{ПЕРИФЕРИЙНЫЙ РАЙ}
\uminitoc{НИЖНИЙ РАЙ}
\uminitoc{ДЫХАНИЕ ПРОСТРАНСТВА}
\uminitoc{ПРОСТРАНСТВЕННЫЕ ФУНКЦИИ РАЯ}
\uminitoc{ГРАВИТАЦИЯ РАЯ}
\uminitoc{УНИКАЛЬНОСТЬ РАЯ}
\author{Совершенствователь Мудрости}
\vs p011 0:1 Рай --- вечный центр вселенной вселенных и место пребывания Всеобщего Отца, Вечного Сына, Бесконечного Духа и их божественных равных и партнёров. Этот центральный Остров представляет собой самое гигантское организованное тело космической реальности во всей главной вселенной. Рай --- это как материальная сфера, так и духовная обитель. Всё разумное творение Всеобщего Отца расселено на материальных местах обитания; поэтому и абсолютный управляющий центр должен быть также материальным, буквальным. И вновь необходимо повторить, что предметы духа и духовные существа \bibemph{реальны}.
\vs p011 0:2 Материальная красота Рая состоит в великолепии его физического совершенства; величие Острова Бога представлено в возвышенных интеллектуальных достижениях и развитии разума его обитателей; слава центрального Острова явлена в бесконечном даровании божественной личности духа --- свете жизни. Но глуб\'ины духовной красоты и чудеса этого великолепного ансамбля совершенно за пределами понимания конечного разума материальных созданий. Слава и духовное великолепие божественной обители недоступны смертному пониманию. И Рай --- от вечности; нет ни записей, ни преданий относительно происхождения этого центрального ядра --- Острова Света и Жизни.
\usection{БОЖЕСТВЕННАЯ РЕЗИДЕНЦИЯ}
\vs p011 1:1 Рай служит многим целям в управлении вселенскими сферами, но для созданий он существует прежде всего как место обитания Божества. Личное присутствие Всеобщего Отца расположено в самом центре верхней поверхности этой почти круглой, но не сферической обители Божеств. Это Райское присутствие Всеобщего Отца непосредственно окружено личным присутствием Вечного Сына, в то время как оба они окутаны несказ\'анной славой Бесконечного Духа.
\vs p011 1:2 Бог обитает, обитал и всегда будет обитать в этой центральной и вечной обители. Мы всегда находили и всегда будем находить его там. Всеобщий Отец космически фокализован, духовно персонализован и географически постоянно присутствует в этом центре вселенной вселенных.
\vs p011 1:3 \pc Мы все знаем прямой путь, по которому надо следовать, чтобы найти Всеобщего Отца. Ты не способен постичь многого относительно божественной резиденции из\hyp{}за её удалённости от тебя и необъятности пролегающего пространства, но те, кто способен понять значение этих огромных расстояний, знают положение и место жительства Бога так же определённо и буквально, как ты знаешь местоположение Нью\hyp{}Йорка, Лондона, Рима или Сингапура --- городов, определённо и географически расположенных на Урантии. Если ты знающий штурман, оснащённый кораблём, картами и компасом, ты сможешь легко найти эти города. Точно так же, если бы ты располагал временем и средствами передвижения, имел достаточную духовную подготовку и необходимое указание направления, ты мог бы проходить от вселенной к вселенной и от контура к контуру, постоянно продвигаясь внутрь через звёздные миры, пока, наконец, не предстал бы перед центральным сиянием духовной славы Всеобщего Отца. Предусмотрев всё необходимое для путешествия, так же возможно найти личное присутствие Бога в центре всего сущего, как найти далёкие города на твоей собственной планете. То, что ты не посещал эти места, нисколько не опровергает их реальность или актуальность их существования. То, что так мало вселенских созданий нашли Бога на Рае, ни в коей мере не опровергает ни реальности его существования, ни актуальности его духовного лица в центре всего сущего.
\vs p011 1:4 Отца всегда можно найти в этом центральном месте. Если бы он переместился, возник бы всеобщий хаос, ибо в нём, в центре его обитания, сходятся всеобщие линии гравитации со всех концов творения. Отслеживаем ли мы личностный контур обратно к вселенным или следуем за восходящими личностями, когда они путешествуют внутрь к Отцу; отслеживаем ли мы линии материальной гравитации к нижнему Раю или следуем за циклическими всплесками космической силы; отслеживаем ли мы линии духовной гравитации к Вечному Сыну или следуем за продвигающейся вовнутрь процессией Райских Сынов Бога; прослеживаем ли мы контуры разума или следуем за триллионами и триллионами небесных существ, которые происходят от Бесконечного Духа, --- любое из этих наблюдений или все они приводят нас непосредственно к присутствию Отца, к его центральной обители. Здесь Бог присутствует лично, буквально и актуально. И от его бесконечного существа изливаются реки\hyp{}потоки жизни, энергии и личности для всех вселенных.
\usection{ПРИРОДА ВЕЧНОГО ОСТРОВА}
\vs p011 2:1 Так как ты уже начинаешь осознавать громадность материальной вселенной, различимую даже из вашего астрономического местонахождения, вашего положения в пространстве в звёздных системах, тебе должно становиться очевидным, что такая огромная материальная вселенная должна иметь соответствующую и достойную столицу, центр, соразмерный достоинству и бесконечности всеобщего Правителя всего этого громадного и далеко раскинувшегося творения материальных миров и живых существ.
\vs p011 2:2 \pc По форме Рай отличается от обитаемых космических тел: он не сферический. Это определённо эллипсоид, на одну шестую длиннее в диаметре в направлении север\hyp{}юг, чем в диаметре восток\hyp{}запад. Центральный Остров практически плоский: расстояние от верхней поверхности до нижней поверхности составляет одну десятую диаметра в направлении восток\hyp{}запад.
\vs p011 2:3 Эти различия в размерах, взятые в связи с его стационарным статусом и б\'ольшим исходящим давлением силы\hyp{}энергии на северном конце Острова, позволяют установить абсолютное направление в главной вселенной.
\vs p011 2:4 \pc Центральный Остров географически разделён на три области активности:
\vs p011 2:5 \li{1.}Верхний Рай.
\vs p011 2:6 \li{2.}Периферийный Рай.
\vs p011 2:7 \li{3.}Нижний Рай.
\vs p011 2:8 \pc Мы называем поверхность Рая, занятую личностной деятельностью, --- верхней стороной, а противоположную поверхность --- нижней стороной. Периферия Рая предназначена для деятельности, которая не является в строгом смысле личностной или неличностной. Троица, по\hyp{}видимому, доминирует на личностной, или верхней, плоскости; Безусловный Абсолют --- на нижней, или неличностной, плоскости. Мы вряд ли представляем себе Безусловный Абсолют как личность, но мы действительно думаем, что функциональное пространственное присутствие этого Абсолюта сосредоточено на нижнем Рае.
\vs p011 2:9 \pc Вечный Остров состоит из единственной формы материализации --- стационарных систем реальности. Эта буквальная субстанция Рая представляет собой однородную организацию потенции пространства, которую нельзя найти где\hyp{}либо ещё во всей обширной вселенной вселенных. Она получила многочисленные названия в различных вселенных, и Мелхиседеки Небадона давно назвали её \bibemph{абсолютум}. Этот исходный материал Рая не является ни мёртвым, ни живым; это --- изначальное недуховное выражение Первого Источника и Центра; это --- Рай, а Рай не имеет дубликатов.
\vs p011 2:10 Нам представляется, что Первый Источник и Центр сконцентрировал весь абсолютный потенциал для космической реальности в Раю как часть своего метода самоосвобождения от ограничений бесконечности, как средство создания возможности суббесконечного, даже время\hyp{}пространственного, творения. Но из этого не следует, что Рай ограничен во времени и пространстве только потому, что вселенная вселенных раскрывает эти качества. Рай существует вне времени и не имеет положения в пространстве.
\vs p011 2:11 Ориентировочно: пространство, по\hyp{}видимому, возникает сразу же под нижним Раем; время --- сразу же над верхним Раем. Время, как вы его понимаете, не является характерной чертой Райского существования, хотя жители центрального Острова полностью осознают вневременн\'ую последовательность событий. Движение не является необходимостью на Рае, оно --- следствие волеизъявления. Но концепция расстояния, даже абсолютного расстояния, имеет очень большое значение, ибо оно может применяться к относительным местоположениям на Рае. Рай внепространственен, поэтому его области абсолютны и, следовательно, могут использоваться многими способами, выходящими за рамки концепций смертного разума.
\usection{ВЕРХНИЙ РАЙ}
\vs p011 3:1 На верхнем Рае расположены три большие области деятельности: \bibemph{присутствие Божества,} \bibemph{Святейшая Сфера} и \bibemph{Святая Область}. Обширный регион, непосредственно окружающий присутствие Божеств, выделен как Святейшая Сфера и отведён для функций поклонения, тринитизации и высокого духовного достижения. В этой зоне нет ни материальных структур, ни чисто интеллектуальных творений; они не могли бы там существовать. Мне бесполезно пытаться описать человеческому разуму божественную природу и прекрасное величие Святейшей Сферы Рая. Эта область целиком духовна, а ты почти целиком материален. Для чисто материального существа чисто духовная реальность как будто не существует.
\vs p011 3:2 Хотя в Святейшей Сфере отсутствуют физические материализации, в секторах Святой Земли есть изобилие памятных свидетельств ваших материальных дней, и ещё больше в вызывающих воспоминания исторических областях периферийного Рая.
\vs p011 3:3 Святая Область --- внешний, или жилой, район --- разделена на семь концентрических зон. Рай иногда называют <<Домом Отца>>, ибо это его вечная резиденция, а эти семь зон часто называют <<Райскими обителями Отца>>. Внутреннюю, или первую, зону занимают Граждане Рая и уроженцы Хавоны, которым случается жить на Рае. Следующая, или вторая, зона --- это область обитания уроженцев семи сверхвселенных времени и пространства. Эта вторая зона частично поделена ещё на семь огромных разделов --- Райский дом духовных существ и восходящих созданий, происходящих из вселенных эволюционного развития. Каждый из этих секторов специально посвящён благополучию и развитию личностей отдельной сверхвселенной, но эти возможности почти бесконечно превышают потребности нынешних семи сверхвселенных.
\vs p011 3:4 Каждый из семи секторов Рая разделён на жилые единицы, предназначенные в качестве жилых центров для миллиарда рабочих групп прославленных индивидуумов. \begin{itemize}\item 1000 таких единиц составляет отделение. \item 100\,000 отделений равны одной конгрегации. \item 10\,000\,000 конгрегаций составляет ассамблею. \item 1\,000\,000\,000 ассамблей образует одну большую единицу.\end{itemize} И эта восходящая серия продолжается через вторую большую единицу, через третью и так далее --- до седьмой большой единицы. И семь больших единиц составляют главные единицы, и семь главных единиц образуют высшую единицу; и так --- кратно семи --- восходящий ряд расширяется через высшую, сверхвысшую, небесную, сверхнебесную до верховных единиц. Но даже и при этом не используется всё имеющееся в распоряжении пространство. Это ошеломляющее число жилых помещений на Рае --- число, выходящее за пределы твоего представления, --- занимает значительно меньше 1\% отведённой области Святой Земли. Ещё остаётся предостаточно места для тех, кто находится на своём пути внутрь, и даже для тех, кто не начнёт восхождение к Раю до наступления времён вечного будущего.
\usection{ПЕРИФЕРИЙНЫЙ РАЙ}
\vs p011 4:1 Центральный Остров резко обрывается на периферии, но его размер настолько огромен, что этот крайний угол относительно неразличим в пределах любой ограниченной области. Периферийная поверхность Рая частично занята взлётно\hyp{}посадочными площадками для различных групп духовных личностей. Поскольку зоны ненасыщенного пространства почти соприкасаются с периферией, весь личностный транспорт, направляющийся к Раю, производит посадку в этих регионах. Ни верхний, ни нижний Рай не доступны для транспортных супернафимов или других типов путешественников космоса.
\vs p011 4:2 Семь Главных Духов восседают каждый на своём троне могущества и власти на семи сферах Духа, обращающихся вокруг Рая в пространстве между сияющими сферами Сына и внутренним контуром миров Хавоны, но они сохраняют центры фокализации силы на периферии Рая. Здесь медленно циркулирующие присутствия Семи Верховных Управляющих Мощью указывают расположение семи станций, вспышками отправляющих в семь сверхвселенных определённые виды энергии Рая.
\vs p011 4:3 Здесь же, на периферии Рая, находятся колоссальные исторические и пророческие выставочные области, предназначенные для Сынов Создателей, посвящённые локальным вселенным времени и пространства. Существует ровно семь триллионов\fnst{Это максимальное число обитаемых планет в семи сверхвселенных.} таких исторических заповедников, ныне действующих или находящихся в резерве, но всё это устройство в совокупности занимает лишь около 4\% отведённой для этой цели части периферийной области. Мы предполагаем, что эти огромные резервы соответствуют творениям, которые когда\hyp{}нибудь появятся за пределами известных в настоящее время и обитаемых семи сверхвселенных.
\vs p011 4:4 Часть Рая, предназначенная для нужд существующих вселенных, занимает всего 1\%--4\%, в то время как отведённая для этой деятельности область по крайней мере в миллион раз превышает требуемую для подобных целей. Рай достаточно велик, чтобы вместить деятельность почти бесконечного творения.
\vs p011 4:5 Но дальнейшая попытка помочь тебе представить величие Рая будет тщетной. Ты должен подождать, и пока ожидаешь, продолжать своё восхождение, ибо истинно сказано: <<Глаз не видел, ухо не слышало, и не приходило на ум смертному человеку, чт\'о Всеобщий Отец приготовил для тех, кто переживёт жизнь во плоти на мирах времени и пространства>>.
\usection{НИЖНИЙ РАЙ}
\vs p011 5:1 О нижнем Рае мы знаем только то, что раскрыто; личности там не обитают. Он не имеет никакого отношения к делам разумных духов, не функционирует там и Божество Абсолют. Нам сообщают, что все контуры физической энергии и космической силы берут своё начало на нижнем Рае, и что он устроен следующим образом:
\vs p011 5:2 \li{1.}Прямо под месторасположением Троицы, в центральной части нижнего Рая, находится неизвестная и нераскрытая Зона Бесконечности.
\vs p011 5:3 \li{2.}Эта Зона непосредственно окружена безымянной областью.
\vs p011 5:4 \li{3.}Внешние границы нижней поверхности занимает область, в основном имеющая отношение к потенции пространства и силе\hyp{}энергии. Действие этого огромного эллиптического силового центра не идентифицируется с известными функциями какого\hyp{}либо триединства, но изначальный сила\hyp{}заряд пространства, по\hyp{}видимому, сосредоточен в этой области. Этот центр состоит из трёх концентрических эллиптических зон: самая внутренняя представляет собой фокальную точка энерго\hyp{}силовой деятельности самог\'о Рая; самая внешняя может быть идентифицирована с функциями Безусловного Абсолюта, но мы не знаем ничего определённого относительно пространственных функций средней зоны.
\vs p011 5:5 \pc Действие \bibemph{внутренней зоны} этого силового центра похоже на гигантское сердце, чьи пульсации направляют потоки энергии к самым внешним границам физического пространства. Она направляет и видоизменяет силы\hyp{}энергии, но вряд ли управляет их движением. Реальность давления\hyp{}присутствия этой изначальной силы определённо больше на северном конце Райского центра, чем в южных регионах; это различие регистрируется равномерно. Материнская сила пространства, по\hyp{}видимому, втекает на юге и вытекает на севере благодаря действию какой\hyp{}то неизвестной системы циркуляции, связанной с распространением этой основной формы силы\hyp{}энергии. Время от времени отмечаются также различия в давлениях по направлению восток\hyp{}запад. Силы, исходящие из этой зоны, не реагируют на наблюдаемую физическую гравитацию, но всегда подчиняются Райской гравитации.
\vs p011 5:6 \pc \bibemph{Средняя зона} силового центра непосредственно окружает эту область. Эта средняя зона кажется статичной, за исключением того, что она расширяется и сжимается, проходя три цикла активности. Наименьшая из этих пульсаций происходит в направлении восток\hyp{}запад , следующая --- в направлении север\hyp{}юг, тогда как наибольшие колебания --- общее расширение и сжатие --- происходят по всем направлениям. Функция этой средней области никогда не была полностью определена, но она должна иметь какое\hyp{}то отношение к взаимному регулированию между внутренней и внешней зонами силового центра. Многие считают, что средняя зона --- это механизм управления промежуточным пространством или зонами покоя, которые разделяют последовательные пространственные уровни главной вселенной, но никакие доказательства или откровения не подтверждают это. Этот вывод основан на знании того, что эта средняя область каким\hyp{}то образом связана с функционированием механизма ненасыщенного пространства главной вселенной.
\vs p011 5:7 \pc \bibemph{Внешняя зона} --- самый большой и наиболее активный из трёх концентрических эллиптических поясов неидентифицированного потенциала пространства. Эта область --- арена невообразимой активности, центральная точка контуров, излучения которых отправляются в пространство по всем направлениям к самым внешним границам семи сверхвселенных и дальше, охватывая огромные и непостижимые области всего внешнего пространства. Это пространственное присутствие полностью безличностно, несмотря на то, что каким\hyp{}то нераскрытым образом оно, по\hyp{}видимому, косвенно реагирует на волю и распоряжения бесконечных Божеств, действующих в качестве Троицы. Считается, что здесь сфокусирован Райский центр пространственного присутствия Безусловного Абсолюта.
\vs p011 5:8 Все формы силы и все фазы энергии, по\hyp{}видимому, замкнуты в контуры; они циркулируют по вселенным и возвращаются определёнными путями. Но излучения активированной зоны Безусловного Абсолюта, по\hyp{}видимому, либо исходящие, либо входящие, --- никогда оба действия не происходят одновременно. Эта внешняя зона пульсирует циклами гигантских размеров эпохальной протяжённости. В течение чуть более миллиарда лет Урантии пространство\hyp{}сила этого центра исходит наружу; затем в течение аналогичного промежутка времени она возвращается вовнутрь. И проявления пространства\hyp{}силы этого центра универсальны; они распространяются по всему насыщенному пространству.
\vs p011 5:9 \pc Вся физическая сила, энергия и материя едины. Вся сила\hyp{}энергия первоначально исходит из нижнего Рая и в конечном итоге вернётся туда после прохождения своего пространственного контура. Но не все виды энергии и материальной организации вселенной вселенных вышли из нижнего Рая в их современных видимых состояниях; пространство --- это лоно нескольких форм материи и предматерии. Хотя внешняя зона силового центра Рая является источником энергий пространства, само пространство там не возникает. Пространство --- это не сила, энергия или мощь. И пульсации этой зоны не являются причиной дыхания пространства, но чередование входящих и исходящих фаз этой зоны синхронизировано с циклами расширения\hyp{}сжатия пространства длительностью в два миллиарда лет.
\usection{ДЫХАНИЕ ПРОСТРАНСТВА}
\vs p011 6:1 Мы не знаем действительного механизма дыхания пространства; мы просто наблюдаем, что всё пространство попеременно сжимается и расширяется. Это дыхание влияет как на горизонтальную протяженность насыщенного пространства, так и на вертикальную протяженность ненасыщенного пространства, существующих в огромных резервуарах пространства выше и ниже Рая. Чтобы представить объёмные очертания этих резервуаров пространства вспомни песочные часы.
\vs p011 6:2 В то время как вселенные горизонтального протяжения насыщенного пространства расширяются, резервуары вертикального протяжения ненасыщенного пространства сжимаются --- и наоборот. Непосредственно под нижним Раем происходит слияние насыщенного и ненасыщенного пространства. Оба типа пространства протекают там через преобразующие каналы регулирования, где при циклических сжатиях и расширениях космоса происходят изменения, делающие насыщаемое пространство ненасыщяемым, и наоборот.
\vs p011 6:3 \pc <<Ненасыщенным>>, называется пространство, не пронизанное теми силами, энергиями, мощью и присутствиями, которые, как известно, существуют в насыщенном пространстве. Мы не знаем, предназначено ли вертикальное (резервуарное) пространство всегда служить противовесом горизонтальному (вселенскому) пространству; мы не знаем, существует ли творческий замысел в отношении ненасыщенного пространства; мы действительно очень мало знаем о резервуарах пространства --- только то, что они существуют, и, что они, по\hyp{}видимому, уравновешивают циклы расширения\hyp{}сжатия пространства вселенной вселенных.
\vs p011 6:4 \pc Каждая фаза цикла дыхания пространства длится немногим более одного миллиарда лет Урантии. В течение одной фазы вселенные расширяются, в течение следующей --- сжимаются. В настоящее время насыщенное пространство подходит к средней точке фазы расширения, в то время как ненасыщенное пространство подходит к средней точке фазы сжатия, и нам известно, что наиболее удалённые границы обоих протяжений пространства, теоретически, сейчас приблизительно равноудалены от Рая. В настоящее время резервуары ненасыщенного пространства простираются по вертикали над верхним Раем и под нижним Раем на такое же расстояние, на какое насыщенное пространство вселенной простирается горизонтально наружу от периферийного Рая до четвертого внешнего пространственного уровня и даже за его пределы.
\vs p011 6:5 В течение миллиарда лет времени Урантии резервуары пространства сжимаются, в то время как главная вселенная и силовая активность всего горизонтального пространства расширяются. Таким образом, для завершения полного цикла расширения\hyp{}сжатия требуется немногим более двух миллиардов лет Урантии.
\usection{ПРОСТРАНСТВЕННЫЕ ФУНКЦИИ РАЯ}
\vs p011 7:1 
\vs p011 7:2 
\vs p011 7:3 
\vs p011 7:4 \pc 
\vs p011 7:5 
\vs p011 7:6 
\vs p011 7:7 \pc 
\vs p011 7:8 
\vs p011 7:9 
\usection{ГРАВИТАЦИЯ РАЯ}
\vs p011 8:1 
\vs p011 8:2 
\vs p011 8:3 
\vs p011 8:4 \pc 
\vs p011 8:5 
\vs p011 8:6 
\vs p011 8:7 
\vs p011 8:8 \pc 
\vs p011 8:9 
\usection{УНИКАЛЬНОСТЬ РАЯ}
\vs p011 9:1 
\vs p011 9:2 \pc 
\vs p011 9:3 \pc 
\vs p011 9:4 
\vs p011 9:5 \pc 
\vs p011 9:6 \pc 
\vs p011 9:7 
\vs p011 9:8 \pc 
\vsetoff
\vs p011 9:9 
\quizlink
