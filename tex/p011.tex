\upaper{11}{ВЕЧНЫЙ ОСТРОВ РАЙ}
\uminitoc{БОЖЕСТВЕННАЯ РЕЗИДЕНЦИЯ}
\uminitoc{ПРИРОДА ВЕЧНОГО ОСТРОВА}
\uminitoc{ВЕРХНИЙ РАЙ}
\uminitoc{ПЕРИФЕРИЙНЫЙ РАЙ}
\uminitoc{НИЖНИЙ РАЙ}
\uminitoc{ДЫХАНИЕ ПРОСТРАНСТВА}
\uminitoc{ПРОСТРАНСТВЕННЫЕ ФУНКЦИИ РАЯ}
\uminitoc{ГРАВИТАЦИЯ РАЯ}
\uminitoc{УНИКАЛЬНОСТЬ РАЯ}
\author{Совершенствователь Мудрости}
\vs p011 0:1 Рай --- вечный центр вселенной вселенных и место пребывания Всеобщего Отца, Вечного Сына, Бесконечного Духа и их божественных равных и партнёров. Этот центральный Остров представляет собой самое гигантское организованное тело космической реальности во всей главной вселенной. Рай --- это как материальная сфера, так и духовная обитель. Всё разумное творение Всеобщего Отца расселено на материальных местах обитания; поэтому и абсолютный управляющий центр должен быть также материальным, буквальным. И вновь необходимо повторить, что предметы духа и духовные существа \bibemph{реальны}.
\vs p011 0:2 Материальная красота Рая состоит в великолепии его физического совершенства; величие Острова Бога представлено в возвышенных интеллектуальных достижениях и развитии разума его обитателей; слава центрального Острова явлена в бесконечном даровании божественной личности духа --- свете жизни. Но глуб\'ины духовной красоты и чудеса этого великолепного ансамбля совершенно за пределами понимания конечного разума материальных созданий. Слава и духовное великолепие божественной обители недоступны смертному пониманию. И Рай --- от вечности; нет ни записей, ни преданий относительно происхождения этого центрального ядра --- Острова Света и Жизни.
\usection{БОЖЕСТВЕННАЯ РЕЗИДЕНЦИЯ}
\vs p011 1:1 Рай служит многим целям в управлении вселенскими сферами, но для созданий он существует прежде всего как место обитания Божества. Личное присутствие Всеобщего Отца расположено в самом центре верхней поверхности этой почти круглой, но не сферической обители Божеств. Это Райское присутствие Всеобщего Отца непосредственно окружено личным присутствием Вечного Сына, в то время как оба они окутаны несказ\'анной славой Бесконечного Духа.
\vs p011 1:2 Бог обитает, обитал, и всегда будет обитать в этой центральной и вечной обители. Мы всегда находили и всегда будем находить его там. Всеобщий Отец космически фокализован, духовно персонализован и географически постоянно присутствует в этом центре вселенной вселенных.
\vs p011 1:3 \pc Мы все знаем прямой путь, по которому надо следовать, чтобы найти Всеобщего Отца. Ты не способен постичь многого, относительно божественной резиденции, из\hyp{}за её удалённости от тебя и необъятности пролегающего пространства, но те, кто способен понять значение этих огромных расстояний, знают положение и место жительства Бога так же определённо и буквально, как ты знаешь местоположение Нью\hyp{}Йорка, Лондона, Рима или Сингапура --- городов, определённо и географически расположенных на Урантии. Если ты знающий штурман, оснащённый кораблём, картами и компасом, ты сможешь легко найти эти города. Точно так же, если бы ты располагал временем и средствами передвижения, имел достаточную духовную подготовку и необходимое указание направления, ты мог бы проходить от вселенной к вселенной и от контура к контуру, постоянно продвигаясь внутрь через звёздные миры, пока, наконец, не предстал бы перед центральным сиянием духовной славы Всеобщего Отца. Предусмотрев всё необходимое для путешествия, так же возможно найти личное присутствие Бога в центре всего сущего, как найти далёкие города на твоей собственной планете. То, что ты не посещал эти места, нисколько не опровергает их реальность или актуальность их существования. То, что так мало вселенских созданий нашли Бога на Рае, ни в коей мере не опровергает ни реальности его существования, ни актуальности его духовного лица в центре всего сущего.
\vs p011 1:4 Отца всегда можно найти в этом центральном месте. Если бы он переместился, возник бы всеобщий хаос, ибо в нём, в центре его обитания, сходятся всеобщие линии гравитации со всех концов творения. Отслеживаем ли мы личностный контур обратно к вселенным или следуем за восходящими личностями, когда они путешествуют внутрь к Отцу; отслеживаем ли мы линии материальной гравитации к нижнему Раю или следуем за циклическими всплесками космической силы; отслеживаем ли мы линии духовной гравитации к Вечному Сыну или следуем за продвигающейся вовнутрь процессией Райских Сынов Бога; прослеживаем ли мы контуры разума или следуем за триллионами и триллионами небесных существ, которые происходят от Бесконечного Духа, --- любое из этих наблюдений или все они приводят нас непосредственно к присутствию Отца, к его центральной обители. Здесь Бог присутствует лично, буквально и актуально. И от его бесконечного существа изливаются реки\hyp{}потоки жизни, энергии и личности для всех вселенных.
\usection{ПРИРОДА ВЕЧНОГО ОСТРОВА}
\vs p011 2:1 Так как ты уже начинаешь осознавать громадность материальной вселенной, различимую даже из вашего астрономического местонахождения, вашего положения в пространстве в звёздных системах, тебе должно становиться очевидным, что такая огромная материальная вселенная должна иметь соответствующую и достойную столицу, центр, соразмерный достоинству и бесконечности всеобщего Правителя всего этого громадного и далеко раскинувшегося творения материальных миров и живых существ.
\vs p011 2:2 \pc По форме Рай отличается от обитаемых космических тел: он не сферический. Это определённо эллипсоид, на одну шестую длиннее в диаметре в направлении север\hyp{}юг, чем в диаметре восток\hyp{}запад. Центральный Остров практически плоский: расстояние от верхней поверхности до нижней поверхности составляет одну десятую диаметра в направлении восток\hyp{}запад.
\vs p011 2:3 Эти различия в размерах, взятые в связи с его стационарным статусом и б\'ольшим исходящим давлением силы\hyp{}энергии на северном конце Острова, позволяют установить абсолютное направление в главной вселенной.
\vs p011 2:4 \pc Центральный Остров географически разделён на три области активности:
\vs p011 2:5 \li{1.}Верхний Рай.
\vs p011 2:6 \li{2.}Периферийный Рай.
\vs p011 2:7 \li{3.}Нижний Рай.
\vs p011 2:8 \pc Мы называем поверхность Рая, занятую личностной деятельностью --- верхней стороной, а противоположную поверхность --- нижней стороной. Периферия Рая предназначена для деятельности, которая не является в строгом смысле личностной или неличностной. Троица, по\hyp{}видимому, доминирует на личностной, или верхней, плоскости, Безусловный Абсолют --- на нижней, или неличностной плоскости. Мы вряд ли представляем себе Безусловный Абсолют как личность, но мы действительно думаем, что функциональное пространственное присутствие этого Абсолюта сосредоточено на нижнем Рае.
\vs p011 2:9 \pc Вечный Остров состоит из единственной формы материализации --- стационарных систем реальности. Эта буквальная субстанция Рая представляет собой однородную организацию потенциала пространства, которую нельзя найти где\hyp{}либо ещё во всей обширной вселенной вселенных. Она получила многочисленные названия в различных вселенных, и Мелхиседеки Небадона давно назвали её \bibemph{абсолютум}. Этот исходный материал Рая не является ни мёртвым, ни живым; это --- изначальное недуховное выражение Первого Источника и Центра; это --- Рай, а Рай не имеет дубликатов.
\vs p011 2:10 Нам представляется, что Первый Источник и Центр сконцентрировал весь абсолютный потенциал для космической реальности в Раю как часть своего метода самоосвобождения от ограничений бесконечности, как средство создания возможности суббесконечного, даже время\hyp{}пространственного, творения. Но из этого не следует, что Рай ограничен во времени и пространстве только потому, что вселенная вселенных раскрывает эти качества. Рай существует вне времени и не имеет положения в пространстве.
\vs p011 2:11 Ориентировочно: пространство, по\hyp{}видимому, возникает сразу же под нижним Раем; время --- сразу же над верхним Раем. Время, как вы его понимаете, не является характерной чертой Райского существования, хотя жители центрального Острова полностью осознают вневременн\'ую последовательность событий. Движение не присуще на Рае, оно --- следствие волеизъявления. Но концепция расстояния, даже абсолютного расстояния, имеет очень большое значение, ибо оно может применяться к относительным местоположениям на Рае. Рай внепространственнен, поэтому его области абсолютны и, следовательно, могут использоваться многими способами, выходящими за рамки концепций смертного разума.
\usection{ВЕРХНИЙ РАЙ}
\vs p011 3:1 
\vs p011 3:2 
\vs p011 3:3 
\vs p011 3:4 
\usection{ПЕРИФЕРИЙНЫЙ РАЙ}
\vs p011 4:1 
\vs p011 4:2 
\vs p011 4:3 
\vs p011 4:4 
\vs p011 4:5 
\usection{НИЖНИЙ РАЙ}
\vs p011 5:1 
\vs p011 5:2 
\vs p011 5:3 
\vs p011 5:4 
\vs p011 5:5 \pc 
\vs p011 5:6 \pc 
\vs p011 5:7 \pc 
\vs p011 5:8 
\vs p011 5:9 \pc 
\usection{ДЫХАНИЕ ПРОСТРАНСТВА}
\vs p011 6:1 
\vs p011 6:2 
\vs p011 6:3 \pc 
\vs p011 6:4 \pc 
\vs p011 6:5 
\usection{ПРОСТРАНСТВЕННЫЕ ФУНКЦИИ РАЯ}
\vs p011 7:1 
\vs p011 7:2 
\vs p011 7:3 
\vs p011 7:4 \pc 
\vs p011 7:5 
\vs p011 7:6 
\vs p011 7:7 \pc 
\vs p011 7:8 
\vs p011 7:9 
\usection{ГРАВИТАЦИЯ РАЯ}
\vs p011 8:1 
\vs p011 8:2 
\vs p011 8:3 
\vs p011 8:4 \pc 
\vs p011 8:5 
\vs p011 8:6 
\vs p011 8:7 
\vs p011 8:8 \pc 
\vs p011 8:9 
\usection{УНИКАЛЬНОСТЬ РАЯ}
\vs p011 9:1 
\vs p011 9:2 \pc 
\vs p011 9:3 \pc 
\vs p011 9:4 
\vs p011 9:5 \pc 
\vs p011 9:6 \pc 
\vs p011 9:7 
\vs p011 9:8 \pc 
\vsetoff
\vs p011 9:9 
\quizlink
