\upaper{35}{СЫНЫ БОГА ЛОКАЛЬНОЙ ВСЕЛЕННОЙ}
\uminitoc{ОТЕЦ МЕЛХИСЕДЕК}
\uminitoc{СЫНЫ МЕЛХИСЕДЕКИ}
\uminitoc{МИРЫ МЕЛХИСЕДЕКОВ}
\uminitoc{ОСОБАЯ РАБОТА МЕЛХИСЕДЕКОВ}
\uminitoc{СЫНЫ ВОРОНДАДЕКИ}
\uminitoc{ОТЦЫ СОЗВЕЗДИЙ}
\uminitoc{МИРЫ ВОРОНДАДЕКОВ}
\uminitoc{СЫНЫ ЛАНОНАНДЕКИ}
\uminitoc{ЛАНОНАНДЕКИ ПРАВИТЕЛИ}
\uminitoc{МИРЫ ЛАНОНАНДЕКОВ}
\author{Глава Архангелов}
\vs p035 0:1 Представленные ранее Сыны Бога имели Райское происхождение. Они~--- потомки божественных Правителей вселенских владений. Из первой Райской категории сыновства, Сынов Создателей, в Небадоне имеется только один~--- Михаил, вселенский отец и властелин. Из второй категории Райского сыновства, Авоналов, или Сынов Повелителей, Небадон обладает полной квотой, в количестве~--- 1\,062. И эти <<меньшие Христы>> так же эффективны и всемогущи в своих планетарных посвящениях, каким был Сын Создатель и Властелин на Урантии. Третья категория, Троичного происхождения, не регистрируется в локальной вселенной, но, по моим оценкам, в Небадоне насчитывается от 15\,000 до 20\,000 Троичных Сынов Учителей, не считая 9\,642 зарегистрированных помощников, тринитизованных созданиями. Эти райские Дайналы не являются ни судьями, ни администраторами; они~--- сверхучителя.
\vs p035 0:2 Типы Сынов, которые будут рассмотрены далее, происходят из локальной вселенной; это~--- потомство Райского Сына Создателя в разнообразных ассоциациях с дополняющим Вселенским Материнским Духом. В этих повествованиях упоминаются следующие категории сыновства локальной вселенной:
\vs p035 0:3 \li{1.}Сыны Мелхиседеки.
\vs p035 0:4 \li{2.}Сыны Ворондадеки.
\vs p035 0:5 \li{3.}Сыны Ланонандеки.
\vs p035 0:6 \li{4.}Сыны Носители Жизни.
\vs p035 0:7 \pc Действием Триединого Райского Божества создаются три категории сыновства: Михаилы, Авоналы и Дайналы. Двойственное Божество в локальной вселенной, Сын и Дух, также функционирует в создании трёх высоких категорий Сынов: Мелхиседеков, Ворондадеков и Ланонандеков; и, достигнув этого троичного выражения, они сотрудничают со следующим уровнем Бога Семичастного в создании разносторонней категории Носителей Жизни. Эти существа классифицируются вместе с нисходящими Сынами Бога, но они представляют собой уникальную и оригинальную форму вселенской жизни, рассмотрению которой посвящён весь следующий документ.
\usection{ОТЕЦ МЕЛХИСЕДЕК}
\vs p035 1:1 После создания существ, являющихся личными помощниками, таких как Яркая Утренняя Звезда, и других управляющих личностей, в соответствии с божественным замыслом и созидательными планами данной вселенной, возникает новая форма созидательного союза между Сыном Создателем и Созидательным Духом, Дочерью Бесконечного Духа в локальной вселенной. Личностный потомок, появившийся в результате этого творческого партнёрства, это изначальный Мелхиседек~--- Отец Мелхиседек~--- то уникальное существо, которое впоследствии сотрудничает с Сыном Создателем и Созидательным Духом, производя на свет всю одноимённую группу.
\vs p035 1:2 Во вселенной Небадон Отец Мелхиседек действует как первый исполнительный помощник Яркой Утренней Звезды. Гавриил больше занят вселенским планированием. Мелхиседек~--- практическими процедурами. Гавриил возглавляет регулярно собираемые суды и советы Небадона, Мелхиседек~--- специальные, внеочередные и чрезвычайные комиссии и совещательные органы. Гавриил и Отец Мелхиседек никогда не покидают Спасоград одновременно\fnst{Иногда они всё же покидают Спасоград одновременно (см.\,\bibref[158:1.6--7]{p158 1:6}).}, так как в отсутствие Гавриила Отец Мелхиседек выполняет функции главы Небадона.
\vs p035 1:3 Все Мелхиседеки нашей вселенной были созданы в течение одного тысячелетия стандартного времени Сыном Создателем и Созидательным Духом в союзе с Отцом Мелхиседеком. Будучи категорией сыновства, в которой один из них действовал как равноправный создатель, по своему строению Мелхиседеки отчасти происходят от самих себя и поэтому являются кандидатами на реализацию высшей формы самоуправления. Они периодически выбирают своего собственного административного руководителя сроком на семь лет стандартного времени и в остальном действуют как саморегулирующаяся категория, хотя изначальный Мелхиседек пользуется некоторыми неотъемлемыми прерогативами совместного родителя. Время от времени этот Отец Мелхиседек назначает определённых индивидуумов своей категории в качестве особых Носителей Жизни на мидсонитные миры~--- обитаемые планеты не раскрытого пока ещё на Урантии типа.
\vs p035 1:4 Мелхиседеки не ведут широкой деятельности за пределами локальной вселенной, за исключением тех случаев, когда их вызывают в качестве свидетелей по делам, рассматриваемым судами сверхвселенной, и когда они назначаются, как это иногда бывает, специальными послами для представления одной вселенной перед другой в пределах своей сверхвселенной. Изначальный, или первородный, Мелхиседек каждой вселенной всегда волен отправиться в соседние вселенные или в Рай с миссиями, связанными с интересами и обязанностями своей категории.
\usection{СЫНЫ МЕЛХИСЕДЕКИ}
\vs p035 2:1 Мелхиседеки~--- это первая категория божественных Сынов, сто\'ящая достаточно близко к низшим формам созданной жизни, чтобы участвовать непосредственно в оказании помощи по подъёму смертных и служении эволюционным расам без необходимости воплощения. Эти Сыны естественным образом находятся на полпути великого личностного нисхождения, по происхождению занимая примерно среднее положение между высочайшей Божественностью и низшим уровнем жизни созданий, наделённых волей. Это делает их естественными посредниками между высшими, божественными, уровнями живого существования и низшими, даже материальными, формами жизни на эволюционных мирах. Серафические категории, ангелы, с удовольствием работают с Мелхиседеками; фактически все формы разумной жизни находят в этих Сынах понимающих друзей, сочувствующих учителей и мудрых советников.
\vs p035 2:2 Мелхиседеки~--- самоуправляющаяся категория. В этой уникальной группе мы сталкиваемся с первой попыткой самоопределения со стороны существ локальной вселенной и наблюдаем высший тип истинного самоуправления. Эти Сыны организуют собственные средства для управления своей группой и планетой обитания, а также для шести связанных сфер и подчинённых им миров. И следует отметить, что они никогда не злоупотребляли своими прерогативами; ни разу во всей сверхвселенной Орвонтон эти Сыны Мелхиседеки не обманули оказанного им доверия. Они~--- надежда каждой вселенской группы, стремящейся к самоуправлению, они являются образцом и учителями самоуправления для всех сфер Небадона. Все категории разумных существ, от вышестоящих руководителей до нижестоящих подчинённых, искренне восхваляют правление Мелхиседеков.
\vs p035 2:3 \pc Категория сыновства Мелхиседек занимает положение и берёт на себя ответственность старшего сына в большой семье. Наибольшая часть их работы регулярна и несколько рутинна, но значительная её часть выполняется совершенно добровольно и по собственной инициативе. Большинство специальных ассамблей, которые время от времени собираются на Спасограде, созываются по предложению Мелхиседеков. По своей собственной инициативе эти Сыны патрулируют свою родную вселенную. Они поддерживают автономную организацию, занимающуюся сбором вселенских данных, периодически отчитываясь перед Сыном Создателем, независимо от всей информации, поступающей в столицу вселенной через регулярные службы повседневного управления миром. По своей природе они являются беспристрастными наблюдателями и пользуются полным доверием всех классов разумных существ.
\vs p035 2:4 Мелхиседеки действуют как мобильные и совещательные кассационные суды сфер; эти вселенские Сыны отправляются небольшими группами в миры, чтобы служить в качестве консультативных комиссий, снимать показания, получать предложения и действовать как советники, тем самым помогая преодолевать основные трудности и улаживать серьёзные разногласия, которые возникают время от времени в делах эволюционных владений.
\vs p035 2:5 Эти старшие Сыны вселенной~--- главные помощники Яркой Утренней Звезды в выполнении мандатов Сына Создателя. Когда Мелхиседек отправляется в отдалённый мир от имени Гавриила, то для целей этой конкретной миссии он может представлять своего отправителя, и в таком случае он появится на планете назначения, наделённый всей полнотой власти Яркой Утренней Звезды. Особенно это относится к тем сферам, где более высокий Сын ещё не воплощался в облике создания данного мира.
\vs p035 2:6 Когда Сын Создатель вступает на путь посвящения эволюционному миру, он отправляется туда один; но когда к посвящению приступает один из его Райских братьев, Сын Авонал, его сопровождают 12 Мелхиседеков поддержки, которые столь эффективно способствуют успеху миссии посвящения. Они также поддерживают Райских Авоналов, посещающих обитаемые миры с миссиями повеления, и при выполнении этих заданий Мелхиседеки видимы глазам смертных, если Сын Авонал проявляет себя таким же образом.
\vs p035 2:7 Нет ни одной фазы планетарной духовной потребности, в которой бы они не служили. Это те учителя, которые так часто побуждают целые миры высокоразвитой жизни прийти к окончательному и полному признанию ими Сына Создателя и его Райского Отца.
\vs p035 2:8 \pc Мелхиседеки почти совершенны в мудрости, но не безошибочны в суждениях. Находясь в изоляции и одиночестве при выполнении планетарной миссии, они иногда ошибались в незначительных вопросах, то есть выбранный ими определённый образ действий впоследствии не одобрялся их руководителями. Такая ошибка суждения временно дисквалифицирует Мелхиседека до тех пор, пока он не отправится на Спасоград и на аудиенции у Сына Создателя не получит наставление, эффективно устраняющее ту дисгармонию, которая привела к разногласиям с его собратьями; а на третий день, после исправительного отдыха, его восстанавливают на службе. Но эти незначительные промахи в действиях Мелхиседеков редко случались в Небадоне.
\vs p035 2:9 Категория этих Сынов не возрастает в количестве; их число постоянно, хотя и варьируется в каждой локальной вселенной. Число Мелхиседеков, зарегистрированных на их столичной планете в Небадоне, превышает 10\,000\,000.
\usection{МИРЫ МЕЛХИСЕДЕКОВ}
\vs p035 3:1 Мелхиседеки занимают свой собственный мир недалеко от Спасограда, столицы вселенной. Эта сфера, именуемая Мелхиседек, является контрольным миром контура Спасограда, состоящего из 70 первичных сфер, каждая из которых окружена шестью подчинёнными сферами специализированной деятельности. Эти удивительные сферы~--- 70 первичных и 420 подчинённых~--- часто называют Университетом Мелхиседеков. Восходящие смертные из всех созвездий Небадона проходят обучение на всех 490 мирах для приобретения статуса постоянного проживания на Спасограде. Но образование восходящих~--- только один из аспектов многообразной деятельности, имеющей место на скоплении архитектурных сфер Спасограда.
\vs p035 3:2 Все 490 сфер контура Спасограда разделены на 10 групп, каждая из которых содержит 7 первичных и 42 подчинённых сферы. Каждая из этих групп находится под общим надзором какой\hyp{}либо одной из основных категорий вселенской жизни. Первая группа, включающая контрольный мир и следующие шесть первичных сфер окружающей планетарной процессии, находится под надзором Мелхиседеков. Эти миры Мелхиседеков следующие:
\vs p035 3:3 \li{1.}Контрольный мир~--- родной мир Сынов Мелхиседеков.
\vs p035 3:4 \li{2.}Мир школ физической жизни и лабораторий живых энергий.
\vs p035 3:5 \li{3.}Мир моронтийной жизни.
\vs p035 3:6 \li{4.}Сфера начальной жизни духа.
\vs p035 3:7 \li{5.}Мир средней жизни духа.
\vs p035 3:8 \li{6.}Сфера развитой жизни духа.
\vs p035 3:9 \li{7.}Область согласованной и верховной самореализации.
\vs p035 3:10 \pc Шесть подчинённых миров каждой из этих сфер Мелхиседеков посвящены деятельности, относящейся к работе соответствующей первичной сферы.
\vs p035 3:11 \pc Контрольный мир, сфера \bibemph{Мелхиседек}~--- это общее место встреч всех существ, занятых обучением и одухотворением восходящих смертных времени и пространства. Для восходящего этот мир, вероятно, самое интересное место во всём Небадоне. Всем эволюционным смертным, которые завершают подготовку в своих созвездиях, предназначено прибыть на Мелхиседек, где их знакомят с расписанием предметов и духовного прогресса образовательной системы Спасограда. И ты никогда не забудешь своих впечатлений от первого дня жизни на этом уникальном мире, даже после достижения своей Райской цели.
\vs p035 3:12 Восходящие смертные сохраняют место жительства на мире Мелхиседек, продолжая свою подготовку на шести окружающих планетах специализированного образования. И этого же метода придерживаются на протяжении всего их пребывания на 70 мирах культуры, первичных сферах контура Спасограда.
\vs p035 3:13 \pc Время многочисленных существ, обитающих на шести подчинённых мирах сферы Мелхиседек, заполнено разнообразными видами деятельности, но в отношении восходящих смертных эти спутники посвящены следующим специальным фазам обучения:
\vs p035 3:14 \li{1.}Сфера номер один посвящена обзору начальной планетарной жизни восходящих смертных. Эта работа проводится в классах, состоящих из смертных уроженцев определённого мира. Выходцы из Урантии проводят такое рассмотрение опыта вместе.
\vs p035 3:15 \li{2.}Специфика работы сферы номер два заключается в аналогичном обзоре опыта, пройденного на обительских мирах, окружающих первый спутник столицы локальной системы.
\vs p035 3:16 \li{3.}Обзор, проводимый на этой сфере, относится к пребыванию на столице локальной системы и охватывает деятельность остальных архитектурных миров столичного скопления системы.
\vs p035 3:17 \li{4.}Четвёртая сфера посвящена обзору опыта 70 подчинённых миров созвездия и связанных с ними сфер.
\vs p035 3:18 \li{5.}На пятой сфере проводится обзор пребывания восходящих на столичном мире созвездия.
\vs p035 3:19 \li{6.}Время на сфере номер шесть посвящено попытке скоррелировать эти пять эпох и таким образом достичь согласования опыта, необходимого для поступления в начальные школы вселенского обучения Мелхиседеков.
\vs p035 3:20 \pc Школы вселенского управления и духовной мудрости расположены на родном мире Мелхиседеков, где также можно найти школы, посвящённые узкому направлению исследований, таких как энергия, материя, организация, коммуникация, архивное дело, этика и сравнительное существование созданий.
\vs p035 3:21 В Мелхиседекском Колледже Духовного Одарения все категории Сынов Бога~--- даже Райские~--- сотрудничают с Мелхиседеками и серафическими учителями в обучении множеств, которые отправляются как евангелисты предназначения, провозглашая духовную свободу и божественное сыновство даже в отдалённых мирах вселенной. Эта специальная школа Университета Мелхиседеков является исключительно учреждением данной вселенной; сюда не принимаются приезжие студенты из других регионов.
\vs p035 3:22 Высший курс обучения управлению вселенной преподаётся Мелхиседеками на их родном мире. Этот Колледж Высокой Этики возглавляет изначальный Отец Мелхиседек. Именно в эти школы различные вселенные посылают студентов по обмену. Несмотря на то что молодая вселенная Небадон стоит невысоко по шкале вселенных с точки зрения духовных достижений и высшего этического развития, тем не менее наши административные трудности настолько превратили всю вселенную в огромную клинику для соседних творений, что колледжи Мелхиседеков переполнены приезжими студентами и наблюдателями из других регионов. Помимо колоссальной группы местных зарегистрированных лиц, школы Мелхиседеков всегда посещают более 100\,000 приезжих студентов, ибо категория Мелхиседеков Небадона знаменита по всему Спландону.
\usection{ОСОБАЯ РАБОТА МЕЛХИСЕДЕКОВ}
\vs p035 4:1 Одно из специализированных направлений деятельности Мелхиседеков связано с наблюдением за продвижением восходящих смертных по моронтийному пути. Значительную часть этого обучения проводят терпеливые и мудрые серафические служители, которым помогают смертные, поднявшиеся на относительно высокие уровни вселенских достижений, но вся эта образовательная работа находится под общим наблюдением Мелхиседеков совместно с Троичными Сынами Учителями.
\vs p035 4:2 \pc Хотя категории Мелхиседеков главным образом посвящены обширной образовательной системе и режиму эмпирического обучения в локальной вселенной, они также выполняют уникальные задания и в необычных обстоятельствах. В развивающейся вселенной, которая в конечном итоге должна охватить приблизительно 10\,000\,000 обитаемых миров, неизбежно происходит множество событий, выходящих за рамки обычного, и именно в таких чрезвычайных обстоятельствах действуют Мелхиседеки. На Эденции, столице вашего созвездия, они известны как чрезвычайные Сыны. Они всегда готовы служить во всех экстренных ситуациях~--- физических, интеллектуальных или духовных~--- будь то на планете, в системе, в созвездии или во вселенной. Когда бы и где бы ни понадобилась особая помощь, там всегда можно найти одного или нескольких Сынов Мелхиседеков.
\vs p035 4:3 Когда возникает угроза провала какой\hyp{}либо части плана Сына Создателя, Мелхиседек немедленно отправляется на помощь. Но не часто случается так, чтобы пришлось вызывать их на помощь в случае греховного восстания, подобного произошедшему в Сатании\fnst{То есть подобные греховные восстания происходят, слава Богу, не часто.}.
\vs p035 4:4 Мелхиседеки первыми действуют во всех чрезвычайных ситуациях любой природы на всех мирах, где живут волевые создания. Иногда они выступают в качестве временных хранителей на сбившихся с пути планетах, служа преемниками несостоятельного планетарного правительства. Во время планетарного кризиса эти Сыны Мелхиседеки служат во многих уникальных качествах. Такому Сыну легко сделать себя видимым для смертных существ, а иногда один из представителей этой категории даже воплощается в подобии смертной плоти. Семь раз в Небадоне Мелхиседек служил на эволюционном мире в облике смертной плоти, и во многих случаях эти Сыны появлялись в подобии других категорий вселенских созданий. Они действительно являются разносторонними и добровольными помощниками в чрезвычайных ситуациях для всех категорий вселенского разума всех миров и систем миров.
\vs p035 4:5 \pc Мелхиседек, живший на Урантии во времена Авраама, был известен в той местности как Принц Салима, так как он возглавлял небольшую колонию искателей истины, живущих в месте, называемом Салим. Он добровольно воплотился в облике смертной плоти и сделал это с одобрения Мелхиседеков\hyp{}преемников планеты, опасавшихся, что свет жизни угаснет в этот период нарастающей духовной тьмы. И он действительно поддержал истину своего времени и бережно передал её Аврааму и его товарищам.
\usection{СЫНЫ ВОРОНДАДЕКИ}
\vs p035 5:1 После создания личных помощников и первой группы разносторонних Мелхиседеков, Сын Создатель и Созидательный Дух локальной вселенной спланировали и создали вторую великую и разнообразную категорию вселенского сыновства~--- Ворондадеков. Они более широко известны как Отцы Созвездий, потому что один из Сынов этой категории неизменно находится во главе правительства каждого созвездия в любой локальной вселенной.
\vs p035 5:2 \pc Число Ворондадеков варьируется в каждой локальной вселенной, и в Небадоне их зарегистрировано ровно 1\,000\,000. Эти Сыны, как и равные им Мелхиседеки, не обладают способностью к воспроизводству. Не существует никакого известного способа, с помощью которого они могли бы увеличить свою численность.
\vs p035 5:3 \pc Во многих отношениях эти Сыны представляют собой самоуправляющуюся группу; как индивидуумы, как группы и даже как единое целое, они обладают значительной степенью самоопределения, как и Мелхиседеки, но функционально диапазон деятельности Ворондадеков не столь широк. Они не равны своим братьям Мелхиседекам по блестящей разносторонности, но при этом они даже более надёжны и эффективны в качестве правителей и дальновидных администраторов. Не во всём равны они как администраторы и своим подчинённым, Властелинам Систем категории Ланонандеков, но превосходят все категории вселенского сыновства в стабильности целей и божественности суждений.
\vs p035 5:4 Хотя решения и постановления этой категории Сынов всегда находятся в соответствии с духом божественного сыновства и в гармонии с планами Сына Создателя, они призывались к Сыну Создателю за допущенные ошибки, а технические детали их решений иногда отменялись по апелляции, подаваемой в верховные суды вселенной. Но эти Сыны редко впадают в ошибку и никогда не восставали; никогда за всю историю Небадона ни один Ворондадек не был замечен в неуважении к правительству вселенной.
\vs p035 5:5 Служба Ворондадеков в локальных вселенных обширна и разнообразна. Они служат как послы в других вселенных и как консулы, представляющие созвездия в пределах своей родной вселенной. Из всех категорий сыновства локальной вселенной им чаще всего вверяют полную передачу верховных полномочий, которые применяются в критических вселенских ситуациях.
\vs p035 5:6 На мирах, изолированных в духовной темноте, на тех сферах, которые из-за восстания и провинности подверглись планетарной изоляции, наблюдатель Ворондадек обычно присутствует вплоть до восстановления нормального статуса. В определённых чрезвычайных ситуациях этот Всевышний наблюдатель может использовать абсолютную и произвольную власть над каждым небесным существом, назначенным на данную планету. На Спасограде зарегистрировано, что Ворондадеки иногда обладали властью Всевышних регентов таких планет. И это верно даже в отношении обитаемых миров, не затронутых восстанием.
\vs p035 5:7 Часто корпус из 12 или более Сынов Ворондадеков заседает в полном составе [en banc] в качестве высокого суда по рассмотрению и обжалованию особых дел, связанных со статусом планеты или системы. Но в большей степени их работа относится к законодательным функциям, присущим правительствам созвездий. В результате выполнения всех этих видов служения Сыны Ворондадеки стали историками локальных вселенных; они лично знакомы со всей политической борьбой и социальными потрясениями обитаемых миров.
\usection{ОТЦЫ СОЗВЕЗДИЙ}
\vs p035 6:1 Как минимум три Ворондадека назначаются для управления каждым из 100 созвездий локальной вселенной. Эти Сыны выбираются Сыном Создателем и назначаются Гавриилом в качестве \bibemph{Всевышних} созвездий для служения сроком на один декамиллениум~--- 10\,000 стандартных лет, около 50\,000 лет по времени Урантии. У правящего Всевышнего, Отца Созвездия, есть два помощника: старший и младший. При каждой смене администрации старший помощник становится главой правительства, младший принимает обязанности старшего, в то время как свободные от задания Ворондадеки, проживающие на мирах Спасограда, выдвигают из своего числа кандидата на должность младшего помощника. Таким образом, в соответствии с нынешними правилами, у каждого из Всевышних правителей период служения на столице созвездия длится три декамиллениума, около 150\,000 лет Урантии.
\vs p035 6:2 Сто Отцов Созвездий~--- действительные главы правительств созвездий~--- составляют верховный совещательный кабинет Сына Создателя. Этот совет часто заседает в столице вселенной и не ограничен определёнными рамками и кругом обсуждаемых тем, но в основном заботится о благополучии созвездий и объединении управления всей локальной вселенной.
\vs p035 6:3 Когда Отец Созвездия исполняет свои обязанности, находясь на столице вселенной, как это часто бывает, старший помощник становится исполняющим обязанности управляющего делами созвездия. Обычная функция старшего помощника~--- надзор за духовными делами, в то время как младший помощник лично занимается физическим благополучием созвездия. Однако ни один значительный план никогда не осуществляется в созвездии, пока все трое Всевышних не придут к согласию относительно всех деталей его исполнения.
\vs p035 6:4 Весь механизм духовного сбора сведений и каналов связи находится в распоряжении Всевышних созвездия. Они поддерживают совершенную связь со своими руководителями на Спасограде и своими прямыми подчинёнными, властелинами локальных систем. Они часто собираются на совет с этими Властелинами Систем, обсуждая состояние созвездия.
\vs p035 6:5 Всевышние окружают себя корпусом советников, численность и состав которого временами варьируется в зависимости от присутствия в столице созвездия различных групп, а также в соответствии с местными потребностями. В особо напряжённые времена они могут запросить, и быстро получат, дополнительное число Сынов категории Ворондадеков для помощи в административной работе. Норлатиадек, ваше собственное созвездие, в настоящее время возглавляют 12 Сынов Ворондадеков.
\usection{МИРЫ ВОРОНДАДЕКОВ}
\vs p035 7:1 Вторая группа из семи миров в контуре 70 первичных сфер, окружающих Спасоград, содержит планеты Ворондадеков. Каждая из этих сфер, вместе с шестью окружающими её спутниками, посвящена определённой фазе деятельности Ворондадеков. На этих 49 сферах восходящие смертные достигают вершины своего образования в области вселенского законодательства.
\vs p035 7:2 Прежде, на столичных мирах созвездий, восходящие смертные наблюдали функционирование законодательных ассамблей, но здесь, на этих мирах Ворондадеков, они участвуют во введении в силу общего действующего законодательства локальной вселенной под опекой старших Ворондадеков. Такое введение законов в силу предназначено для координации различных постановлений автономных законодательных ассамблей 100 созвездий. Образование, полученное в школах Ворондадеков, остаётся непревзойдённым даже на Уверсе. Эта подготовка осуществляется постепенно: начинается на первой сфере с дополнительной работой на её шести спутниках и продолжается на остальных шести первичных сферах и связанных с ними группах спутников.
\vs p035 7:3 На этих мирах учёбы и практической работы восходящие пилигримы знакомятся с многочисленными новыми видами деятельности. Нам не запрещено предпринять попытку откровения об этих новых и невероятных занятиях, но мы отчаялись найти возможность описать их материальному разуму смертных существ. У нас нет слов для передачи смыслов этих небесных видов деятельности, и нет аналогов среди человеческих занятий, способных проиллюстрировать эти новые занятия восходящих смертных, продолжающих своё обучение на данных 49 мирах. На этих мирах Ворондадеков контура Спасограда сосредоточены и многие другие виды деятельности, не являющиеся частью режима восхождения.
\usection{СЫНЫ ЛАНОНАНДЕКИ}
\vs p035 8:1 После создания Ворондадеков Сын Создатель и Вселенский Материнский Дух соединяются с целью создания третьей категории вселенского сыновства~--- Ланонандеков. Хотя Ланонандеки выполняют разнообразные задачи по управлению системами, более всего они известны как Властелины Систем, правители локальных систем, и как Планетарные Принцы, административные главы обитаемых миров.
\vs p035 8:2 Будучи категорией сыновства, созданной позже и, в отношении уровней божественности, ниже, этим существам потребовалось пройти определённые курсы обучения на мирах Мелхиседеков для подготовки к последующему служению. Они были первыми студентами в Университете Мелхиседеков и были классифицированы и аттестованы своими учителями и экзаменаторами Мелхиседеками в соответствии со способностями, личностью и достижениями.
\vs p035 8:3 С начала своего существования Вселенная Небадон насчитывала ровно 12\,000\,000 Ланонандеков, и после прохождения подготовки на сфере Мелхиседек, их разделили по итогам окончательных тестов на три класса:
\vs p035 8:4 \li{1.}\bibemph{Первичные Ланонандеки}. Высшая категория составила 709\,841. Это Сыны, назначенные Властелинами Систем и помощниками верховных советов созвездий, а также советниками по высшей административной работе вселенной.
\vs p035 8:5 \li{2.}\bibemph{Вторичные Ланонандеки}. В эту категорию выпускников сферы Мелхиседек вошли 10\,234\,601 Ланонандек. Они назначаются Планетарными Принцами, а также в резервы данной категории.
\vs p035 8:6 \li{3.}\bibemph{Третичные Ланонандеки}. Эта группа составила 1\,055\,558. Эти Сыны действуют в качестве подчинённых помощников, посланников, хранителей, уполномоченных, наблюдателей и выполняют различные обязанности в системе и составляющих её мирах.
\vs p035 8:7 \pc В отличие от эволюционирующих существ, эти Сыны не могут переходить из одной группы в другую. Пройдя обучение у Мелхиседеков, после тестирования и классификации, они служат постоянно в назначенном ранге. Не занимаются эти Сыны и воспроизводством; их число во вселенной постоянно.
\vs p035 8:8 В округлённых числах категория Сынов Ланонандеков классифицируется на Спасограде следующим образом:
\vs p035 8:9 \pc Вселенские Координаторы и Советники Созвездий\bibdf100\,000
\vs p035 8:10 Властелины Систем и Помощники\bibdf600\,000
\vs p035 8:11 Планетарные Принцы и Резервы\bibdf10\,000\,000
\vs p035 8:12 Корпус Посланников\bibdf400\,000
\vs p035 8:13 Хранители и Писцы\bibdf100\,000
\vs p035 8:14 Резервный Корпус\bibdf800\,000
\vs p035 8:15 \pc Так как Ланонандеки несколько ниже по категории сыновства, чем Мелхиседеки и Ворондадеки, они приносят ещё б\'ольшую пользу в подчинённых единицах вселенной, ибо способны к большему сближению с низшими созданиями разумных рас. Но при этом им угрожает б\'ольшая опасность сбиться с пути и отклониться от приемлемого метода управления вселенной. Однако Ланонандеки, особенно их первичная категория, являются наиболее способными и разносторонними из всех администраторов локальной вселенной. В отношении административных способностей, их превосходят только Гавриил и его нераскрытые партнёры.
\usection{ЛАНОНАНДЕКИ ПРАВИТЕЛИ}
\vs p035 9:1 Ланонандеки~--- постоянные правители планет и сменяющиеся властелины систем. Такой Сын сейчас правит на Иерусеме, столице вашей локальной системы обитаемых миров.
\vs p035 9:2 Властелины Систем управляют в составе комиссий по двое или трое на столице каждой системы обитаемых миров. Отец Созвездия назначает одного из этих Ланонандеков главой каждый декамиллениум. Иногда никаких изменений в отношении главы трио не происходит, решение этого вопроса целиком предоставлено на усмотрение правителей созвездия. Внезапные изменения в составе правительств систем возможны только в случае какой\hyp{}либо трагедии.
\vs p035 9:3 Когда Властелины Систем или помощники отзываются, их места занимают те, кто был выбран верховным советом, расположенным на столице созвездия, из резервов этой категории~--- группы, численность которой на Эденции превышает указанный средний уровень.
\vs p035 9:4 Верховные советы Ланонандеков размещаются на столицах различных созвездий. Такой совет возглавляет старший Всевышний помощник Отца Созвездия, в то время как младший помощник наблюдает за резервами вторичной категории.
\vs p035 9:5 \pc Властелины Систем соответствуют своему названию; они практически полновластны в местных делах обитаемых миров. Они почти по\hyp{}отечески руководят Планетарными Принцами, Материальными Сынами и духами\hyp{}помощниками. Личный авторитет властелина почти непререкаем. Эти правители не контролируются Троичными наблюдателями из центральной вселенной. Они являются подразделением исполнительной власти локальной вселенной, и как ответственные за исполнение законодательных мандатов и исполнители судебных вердиктов, они представляют собой единственное место во всей вселенской администрации, где легко и просто может возникнуть и заявить о себе личная нелояльность по отношению к воле Сына Михаила.
\vs p035 9:6 В нашей локальной вселенной, к несчастью, более 700 Сынов категории Ланонандек восстали против вселенского правительства, тем самым ввергнув в хаос несколько систем и множество планет. Из всех сбившихся с пути только трое были Властелинами Систем; практически все эти Сыны принадлежали ко второй и третьей категориям, Планетарным Принцам и третичным Ланонандекам.
\vs p035 9:7 Большое число таких обесчестивших себя Сынов не указывает на какую\hyp{}либо ошибку при их сотворении. Они могли быть созданы божественно совершенными, но они были сотворены такими, чтобы лучше понимать эволюционные существа, обитающие на мирах времени и пространства, и сближаться с ними.
\vs p035 9:8 Из всех локальных вселенных в Орвонтоне, наша вселенная, за исключением Хенселона, потеряла наибольшее число Сынов этой категории. По общему мнению Уверсы, такое обилие административных проблем Небадона связано с тем, что наши Сыны категории Ланонандек наделены при создании столь высокой степенью личной свободы в вопросах выбора и планирования. Я привожу это наблюдение не в качестве критики. Создатель нашей вселенной обладает всей полнотой власти и мощи поступать именно так. С точки зрения наших высоких правителей, хотя свободные в принятии решений Сыны и создают излишние неприятности на ранних этапах существования вселенной, после полного анализа и окончательного урегулирования ситуации, достижения более высокой лояльности и более полного добровольного служения со стороны этих основательно испытанных Сынов с избытком компенсируют сумятицу и невзгоды более ранних времён.
\vs p035 9:9 \pc В случае восстания на столице системы, новый властелин обычно назначается в течение сравнительно короткого времени, но на отдельных планетах это не так. Они являются составными единицами материального творения, и свободная воля созданий~--- существенный фактор при вынесении окончательного решения по всем подобным вопросам. На изолированные миры, планеты, чьи облечённые властью принцы могли сбиться с пути, назначаются Планетарные Принцы преемники, но они не приступают к активному правлению такими мирами до тех пор, пока последствия восстания не преодолены частично и не устранены корректирующими действиями Мелхиседеков и других помогающих личностей. Восстание Планетарного Принца приводит к немедленной изоляции планеты и мгновенному отключению локальных духовных контуров. Восстановить межпланетные линии связи на таком духовно изолированном мире может только Сын посвящения.
\vs p035 9:10 Существует план спасения таких сбившихся с пути и неблагоразумных Сынов, и многие воспользовались этой милосердно предусмотренной возможностью; но никогда они уже не смогут функционировать в том качестве, в котором не выполнили своих обязательств. После реабилитации они назначаются на должности хранителей и в отделы физического управления.
\usection{МИРЫ ЛАНОНАНДЕКОВ}
\vs p035 10:1 Третья группа из семи миров в контуре 70 планет Спасограда, с их 42 спутниками, составляет скопление административных сфер Ланонандеков. На этих мирах опытные Ланонандеки, принадлежащие к корпусу бывших Властелинов Систем, исполняют обязанности учителей восходящих пилигримов и серафических воинств в области управления. Эволюционные смертные наблюдают управляющих системами в работе уже на столицах систем, но здесь они участвуют в непосредственном согласовании административных решений 10\,000 локальных систем.
\vs p035 10:2 Эти административные школы локальной вселенной находятся под наблюдением корпуса Сынов Ланонандеков, имеющих большой опыт работы в качестве Властелинов Систем и советников созвездий. Эти колледжи исполнительной власти уступают только административным школам Энсы.
\vs p035 10:3 Миры Ланонандеков, служа сферами подготовки для восходящих смертных, являются также центрами обширных начинаний, связанных с обычным и рутинным административным управлением вселенной. На протяжении всего пути к Раю восходящие пилигримы продолжают своё обучение в практических школах прикладного знания, где учатся применять полученные знания на практике. Образовательная система Вселенной, организованная Мелхиседеками, практична, прогрессивна, исполнена смысла и основана на опыте. Она включает в себя подготовку в области материального, интеллектуального, моронтийного и духовного.
\vs p035 10:4 \pc Именно в связи с этими административными сферами Ланонандеков большинство спасённых Сынов этой категории служат хранителями и управляющими планетарными делами. И эти не выполнившие обязательств Планетарные Принцы и их соучастники в восстании, решившие принять предложенную реабилитацию, продолжат выполнять эти рутинные обязанности, по крайней мере, до тех пор, пока вселенная Небадон не утвердится в свете и жизни.
\vs p035 10:5 \pc Многие Сыны Ланонандеки в более старых системах, однако, добились замечательных результатов в служении, управлении и духовных достижениях. Они представляют собой благородную, верную и преданную группу, несмотря на их склонность впадать в заблуждения из-за обманчивых представлений о личной свободе и ложных представлений о самоопределении.
\vsetoff
\vs p035 10:6 [При поддержке Главы Архангелов, уполномоченного Гавриилом Спасограда.]
\quizlink
