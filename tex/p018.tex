\upaper{18}{ВЕРХОВНЫЕ ТРОИЧНЫЕ ЛИЧНОСТИ}
\uminitoc{ТРИНИТИЗОВАННЫЕ СЕКРЕТЫ ВЕРХОВНОСТИ}
\uminitoc{ОТ ВЕКА ВЕЧНЫЕ}
\uminitoc{ОТ ВЕКА ДРЕВНИЕ}
\uminitoc{ОТ ВЕКА СОВЕРШЕННЫЕ}
\uminitoc{ОТ ВЕКА НЕДАВНИЕ}
\uminitoc{ОТ ВЕКА ЕДИНЫЕ}
\uminitoc{ОТ ВЕКА ВЕРНЫЕ}
\author{Божественный Советник}
\vs p018 0:1 Все Верховные Троичные Личности созданы для определённого служения. Они задуманы божественной Троицей для выполнения определённых конкретных обязанностей, и они компетентны для служения с совершенством метода и окончательностью преданности. Существует семь категорий Верховных Троичных Личностей:
\vs p018 0:2 \li{1.}Тринитизованные Секреты Верховности.
\vs p018 0:3 \li{2.}От Века Вечные [Eternals of Days].
\vs p018 0:4 \li{3.}От Века Древние [Ancients of Days].
\vs p018 0:5 \li{4.}От Века Совершенные [Perfections of Days].
\vs p018 0:6 \li{5.}От Века Недавние [Recents of Days].
\vs p018 0:7 \li{6.}От Века Единые [Unions of Days].
\vs p018 0:8 \li{7.}От Века Верные [Faithfuls of Days].
\vs p018 0:9 \pc Существует определённое и окончательное число этих существ административного совершенства. Их создание~--- событие прошлого; новых персонализаций больше не происходит.
\vs p018 0:10 По всей большой вселенной эти Верховные Троичные Личности представляют административную политику Райской Троицы; они представляют правосудие и \bibemph{являются} исполнительным судебным решением Райской Троицы. Они образуют взаимосвязанную линию административного совершенства, простирающуюся от Райских сфер Отца до столичных миров локальных вселенных и столиц составляющих их созвездий.
\vs p018 0:11 Все существа Троичного происхождения созданы в Райском совершенстве во всех своих божественных атрибутах. Только в сферах опыта течение времени добавляло к их подготовке к несению космической службы. Среди существ Троичного происхождения исключена опасность проступка или риск восстания. Они по своей сущности божественны и никогда не были замечены в отклонении от божественной и совершенной линии поведения личности.
\usection{ТРИНИТИЗОВАННЫЕ СЕКРЕТЫ ВЕРХОВНОСТИ}
\vs p018 1:1 Во внутреннем контуре Райских спутников находятся семь миров, и каждый из этих возвышенных миров возглавляется корпусом из десяти Тринитизованных Секретов Верховности. Они не создатели, а верховные и предельные администраторы. Управление делами этих семи братских сфер полностью возложено на этот корпус из 70 верховных руководителей. Хотя потомство Троицы контролирует эти семь ближайших к Раю священных сфер, эта группа миров широко известна как личный контур Всеобщего Отца.
\vs p018 1:2 Тринитизованные Секреты Верховности действуют в группах по десять равных и совместных руководителей соответствующих сфер, но они также действуют индивидуально в определённых областях ответственности. Работа каждого из этих особых миров разделена на семь основных отделов, и один из этих равных правителей заведует каждым подобным отделом специализированной деятельности. Остальные три действуют как личные представители триединого Божества по отношению к остальным семи: один представляет Отца, один~--- Сына, и один~--- Духа.
\vs p018 1:3 Хотя существует определённое групповое сходство, которое типично для Тринитизованных Секретов Верховности, они также раскрывают семь различных групповых характерных черт. Десять верховных руководителей дел Божеграда отражают личный характер и природу Всеобщего Отца; и то же самое верно относительно каждой из этих семи сфер: каждая группа из десяти похожа на то Божество или ассоциацию Божеств, которая характерна для их области. Десять управляющих, которые правят Восходоградом, отражают объединённую природу Отца, Сына и Духа.
\vs p018 1:4 \pc Я могу раскрыть очень мало о работе этих высоких личностей семи священных миров Отца, ибо они действительно~--- \bibemph{Секреты} Верховности. Нет никаких произвольных секретов, связанных с приближением к Всеобщему Отцу, Вечному Сыну или Бесконечному Духу. Божества~--- открытая книга для всех, кто достигает божественного совершенства, но не все Секреты Верховности могут быть до конца постигнуты. Мы никогда не сможем полностью проникнуть в сферы, содержащие личностные секреты связи Божества с семичастным группированием созданных существ.
\vs p018 1:5 Поскольку работа этих верховных руководителей связана с близким и личным контактом Божеств с семью основными группами вселенских существ,~--- размещаются ли они на этих семи особых мирах или функционируют по всей большой вселенной,~--- естественно, что эти очень личные отношения и необычные контакты следует хранить в священной тайне. Райские Создатели уважают конфиденциальность и неприкосновенность личности даже своих низших созданий. И это верно как для отдельных лиц, так и для различных категорий личностей.
\vs p018 1:6 Даже для существ высокого вселенского достижения эти секретные миры всегда остаются испытанием на преданность. Нам дано полностью и лично познавать вечных Богов, беспрепятственно познавать их божественность и совершенство, но нам не дано полностью проникнуть во все личные отношения Райских Правителей со всеми их созданиями.
\usection{ОТ ВЕКА ВЕЧНЫЕ}
\vs p018 2:1 Каждым из миллиарда миров Хавоны управляет Верховная Троичная Личность. Эти правители известны как От Века Вечные, и число их ровно один миллиард, по одному на каждую из сфер Хавоны. Они являются потомством Райской Троицы, но, подобно Секретам Верховности, об их происхождении нет никаких записей. Эти две группы премудрых отцов вечно правили своими изысканными мирами системы Рай\hyp{}Хавона, и они действуют без смены или переназначения.
\vs p018 2:2 От Века Вечные видимы всем волевым созданиям, обитающим в их владениях. Они председательствуют на регулярных планетарных конклавах. Периодически и по очереди они посещают столичные сферы семи сверхвселенных. Они близкородственны и божественно равны От Века Древним, которые вершат судьбы семи сверхправительств. Когда От Века Вечный отсутствует на своей сфере, его миром управляет Троичный Сын Учитель.
\vs p018 2:3 За исключением учреждённых типов жизни, таких как уроженцы Хавоны и другие живые существа центральной вселенной, каждый из постоянно проживающих От Века Вечных разработал свою сферу полностью в соответствии со своими личными идеями и идеалами. Они посещают планеты друг друга, но не копируют и не подражают; они всегда и во всём оригинальны.
\vs p018 2:4 Архитектура, крас\'оты природы, моронтийные структуры и творения духа исключительны и уникальны на каждой сфере. Каждый мир~--- это место вечной красоты и совершенно не похож на любой другой мир центральной вселенной. И каждый из вас проведёт более или менее длительное время на каждой из этих уникальных и захватывающих сфер на пути внутрь через Хавону в Рай. В вашем мире принято говорить \bibemph{вверх} о направлении к Раю, но о божественной цели восхождения было бы правильнее говорить \bibemph{внутрь}.
\usection{ОТ ВЕКА ДРЕВНИЕ}
\vs p018 3:1 Когда смертные времени завершают обучение на мирах подготовки, окружающих столицу локальной вселенной, и переходят в сферы образования своей сверхвселенной, они продвигаются в духовном развитии до той точки, где они в состоянии осознавать и общаться с высокими духовными правителями и руководителями этих более высоких сфер, включая От Века Древних.
\vs p018 3:2 Все От Века Древние в основном идентичны; они раскрывают объединённый характер и единую природу Троицы. Они обладают индивидуальностью и личностно разнообразны, но не отличаются друг от друга так, как Семь Главных Духов. Они обеспечивают единообразное руководство во всём остальном различающимися семью сверхвселенными, каждая из которых является отдельным, обособленным и уникальным творением. Семь Главных Духов не похожи друг на друга по своей природе и свойствам, но все От Века Древние~--- личные правители сверхвселенных~--- это единообразные и сверхсовершенные потомки Райской Троицы.
\vs p018 3:3 Семь Главных Духов свыше определяют \bibemph{природу} своих сверхвселенных, а От Века Древние диктуют \bibemph{управление} этими же самыми сверхвселенными. Они накладывают административное единообразие на творческое разнообразие и обеспечивают гармонию целого перед лицом фундаментальных творческих различий семи сегментных группировок большой вселенной.
\vs p018 3:4 \pc Все От Века Древние были тринитизованы в одно и то же время. Они представляют собой начало личностных записей вселенной вселенных, отсюда и их название~--- От Века \bibemph{Древние}. Когда ты достигнешь Рая и исследуешь письменные записи о начале всего, ты обнаружишь, что первая запись в разделе личностей~--- это описание тринитизации этих 21 От Века Древних.
\vs p018 3:5 \pc Эти высокие существа всегда правят группами по трое. Есть много фаз деятельности, в которых они работают как индивидуумы, есть и такие, в которых могут действовать любые двое, но в более высоких сферах своего управления они должны действовать совместно. Они никогда лично не покидают свои сферы\hyp{}резиденции, но им и не надо этого делать, ибо эти миры являются сверхвселенскими фокальными точками в обширной системе отражения.
\vs p018 3:6 Личные обители каждого трио От Века Древних расположены в точке духовной полярности на их центральной сфере. Такая сфера разделена на 70 административных секторов и имеет 70 региональных столиц, в которых время от времени живут От Века Древние.
\vs p018 3:7 По власти, сфере полномочий и ограничению юрисдикции От Века Древние~--- самые могущественные и сильные из всех непосредственных правителей творений времени\hyp{}пространства. Во всей необъятной вселенной вселенных только они наделены высокими полномочиями окончательного исполнительного судебного решения относительно вечного исчезновения волевых созданий. И все трое От Века Древних должны участвовать в вынесении окончательных приговоров верховного трибунала сверхвселенной.
\vs p018 3:8 \pc Помимо Божеств и их Райских партнёров, От Века Древние~--- самые совершенные, самые разносторонние и самые божественно одарённые правители во всём время\hyp{}пространственном существовании. Очевидно, что они являются верховными правителями сверхвселенных; но они не заслужили это право руководить эмпирически, и поэтому им суждено когда\hyp{}нибудь быть смещёнными Верховным Существом~--- эмпирическим властелином, чьими наместниками они, несомненно, станут.
\vs p018 3:9 Верховное Существо достигает полновластия над семью сверхвселенными посредством эмпирического служения, точно так же как Сын Создатель эмпирически заслуживает суверенную власть над своей локальной вселенной. Но в нынешнюю эпоху незавершённой эволюции Верховного От Века Древние обеспечивают координированный и совершенный административный сверхконтроль над развивающимися вселенными времени и пространства. Все указы и постановления От Века Древних характеризуют оригинальная мудрость и индивидуальная инициатива.
\usection{ОТ ВЕКА СОВЕРШЕННЫЕ}
\vs p018 4:1 Существует ровно 210 От Века Совершенных, и они возглавляют правительства десяти больших секторов каждой сверхвселенной. Они были тринитизованы для особой работы по оказанию помощи руководителям сверхвселенной; они правят как непосредственные и личные наместники От Века Древних.
\vs p018 4:2 Трое От Века Совершенных назначаются в столицу каждого большого сектора, но, в отличие от От Века Древних, необязательно, чтобы все трое присутствовали постоянно. Время от времени один из этого трио может отсутствовать, чтобы лично посоветоваться с От Века Древними относительно благополучия своего владения.
\vs p018 4:3 \pc Эти триединые правители больших секторов особенно совершенны в мастерском владении административными тонкостями, отсюда и их название~--- От Века \bibemph{Совершенные}. Записывая имена этих существ духовного мира, мы сталкиваемся с проблемой перевода на ваш язык, и очень часто чрезвычайно трудно дать удовлетворительный перевод. Нам не хотелось бы использовать произвольные обозначения, которые были бы бессмысленны для вас; поэтому нам часто бывает трудно выбрать подходящее имя~--- такое, которое будет вам понятно и в то же время в некоторой степени представлять оригинал.
\vs p018 4:4 \pc От Века Совершенные имеют при своих правительствах небольшой корпус Божественных Советников, Совершенствователей Мудрости и Всеобщих Цензоров. Ещё больше у них Могущественных Посланников, Высокоуполномоченных и Не Имеющих Имени и Номера. Но значительная часть рутинной работы большого сектора выполняется Небесными Хранителями и Помощниками Высоких Сынов. Эти две группы набираются из числа тринитизованного потомства, либо личностей Рая\hyp{}Хавоны, либо прославленных смертных завершителей. Некоторые из этих двух категорий существ, тринитизованных созданиями, повторно тринитизуются Райскими Божествами, а затем их направляют помогать в управлении сверхвселенскими правительствами.
\vs p018 4:5 Большинство Небесных Хранителей и Помощников Высоких Сынов назначаются на службу в больших и малых секторах, но Тринитизованные Хранители (объятые Троицей серафимы и промежуточные создания) являются должностными лицами судов всех трёх подразделений, действующих в трибуналах От Века Древних, От Века Совершенных и От Века Недавних. Тринитизованных Послов (объятых Троицей восходящих смертных из тех, что испытали слияние с Сыном или Духом) можно встретить повсюду в сверхвселенной, но большинство из них находится на службе малых секторов.
\vs p018 4:6 До наступления времени полного развёртывания схемы правительств семи сверхвселенных, практически все администраторы различных подразделений этих правительств, за исключением От Века Древних, служили подмастерьями в течение сроков разной продолжительности у От Века Вечных на различных мирах совершенной вселенной Хавона. Тринитизованные позднее существа также прошли период обучения у От Века Вечных, прежде чем они были прикреплены к службе От Века Древних, От Века Совершенных и От Века Недавних. Все они~--- закалённые, испытанные и опытные управляющие.
\vs p018 4:7 \pc Ты рано повстречаешь От Века Совершенных, как только достигнешь столицы Спландона после пребывания в мирах малого сектора, поскольку эти возвышенные правители тесно связаны с 70 мирами большого сектора,~--- мирами высшей подготовки восходящих созданий времени. От Века Совершенные лично приводят к групповой присяге восходящих выпускников школ большого сектора.
\vs p018 4:8 Работа пилигримов времени на мирах, окружающих столицу большого сектора, носит в основном интеллектуальный характер, в отличие от более физического и материального характера обучения на семи образовательных сферах малого сектора и от духовных занятий на 490 университетских мирах столицы сверхвселенной.
\vs p018 4:9 Хотя тебя занесут только в реестр Спландона~--- большого сектора, включающего локальную вселенную твоего происхождения,~--- тебе придётся пройти через все десять основных подразделений нашей сверхвселенной. Ты увидишь все 30 От Века Совершенных Орвонтона, прежде чем доберёшься до Уверсы.
\usection{ОТ ВЕКА НЕДАВНИЕ}
\vs p018 5:1 От Века Недавние~--- самые молодые из верховных руководителей сверхвселенных; в группах по трое они руководят делами малых секторов. По своей природе они равны От Века Совершенным, но в административной власти они являются подчинёнными. Существует ровно 21\,000 этих в личном отношении восхитительных и божественно эффективных Троичных личностей. Они были созданы одновременно и вместе прошли обучение в Хавоне под руководством От Века Вечных.
\vs p018 5:2 От Века Недавние имеют корпус партнёров и помощников, подобный корпусу От Века Совершенных. Вдобавок к ним приписано огромное количество различных подчинённых категорий небесных существ. К управлению малыми секторами они привлекают большое количество постоянно проживающих восходящих смертных, персонал различных гостящих колоний и различные группы существ, происходящих от Бесконечного Духа.
\vs p018 5:3 Правительства малых секторов в значительной степени, хотя и не исключительно, занимаются крупными физическими проблемами сверхвселенных. Сферы малого сектора являются центрами Главных Физических Регуляторов. На этих мирах восходящие смертные проводят исследования и эксперименты, связанные с изучением деятельности третьей категории Верховных Центров Мощи третьего уровня и всех семи категорий Главных Физических Регуляторов.
\vs p018 5:4 Поскольку режим малого сектора так сильно связан с физическими проблемами, его три От Века Недавних редко бывают вместе на столичной сфере. Б\'ольшую часть времени один из них отсутствует, находясь на совещании с От Века Совершенными вышестоящего большого сектора или представляя От Века Древних на Райских конклавах высоких существ Троичного происхождения. Они чередуются с От Века Совершенными, представляя От Века Древних на верховных советах в Раю. Между тем другой От Века Недавний может уехать на инспекцию столичных миров локальных вселенных, принадлежащих его юрисдикции. Но по крайней мере один из этих правителей всегда остаётся на дежурстве в столице малого сектора.
\vs p018 5:5 Вы все когда\hyp{}нибудь узнаете трёх От Века Недавних, отвечающих за ваш малый сектор Энса, ибо вам предстоит пройти через их руки на пути внутрь к подготовительным мирам больших секторов. Восходя к Уверсе, вы пройдёте только одну группу сфер подготовки малых секторов.
\usection{ОТ ВЕКА ЕДИНЫЕ}
\vs p018 6:1 Троичные личности категории <<От Века>> не выполняют административных функций на уровне ниже сверхвселенских правительств. В эволюционирующих локальных вселенных они действуют только как советники и консультанты. От Века Единые~--- это группа связн\'ых личностей, аккредитованных Райской Троицей для служения с парными правителями\fnst{То есть с Сыном Создателем и Материнским Духом локальной вселенной.} локальных вселенных. Каждой организованной и обитаемой локальной вселенной назначается один из этих Райских советников, который действует как представитель Троицы и, в некоторых отношениях, Всеобщего Отца для локального творения.
\vs p018 6:2 Всего 700\,000 таких существ, хотя не все они ещё получили назначения. Резервный корпус От Века Единых функционирует на Рае как Верховный Совет Вселенского Урегулирования.
\vs p018 6:3 Особым образом эти Троичные наблюдатели координируют административную деятельность всех ветвей вселенского правительства, от правительств локальных вселенных, через правительства секторов и до правительств сверхвселенных, отсюда и их название~--- От Века \bibemph{Единые}. Они представляют тройной отчёт своему начальству: \begin{itemize}\item От Века Недавним своего малого сектора~--- относящиеся к делу данные физического и полуинтеллектуального характера;\item От Века Совершенным своего большого сектора~--- события интеллектуального и квазидуховного характера;\item От Века Древним в столице своей сверхвселенной~--- дела духовного и полурайского уровня.\end{itemize}
\vs p018 6:4 Поскольку они являются существами Троичного происхождения, все Райские контуры доступны им для связи, и таким образом они всегда находятся в контакте друг с другом и со всеми другими необходимыми личностями вплоть до верховных советов Рая.
\vs p018 6:5 \pc От Века Единый не связан органически с правительством локальной вселенной своего назначения. Помимо своих обязанностей наблюдателя, он действует только по запросу местных властей. Он по должности [ex officio] является членом всех первичных советов и всех важных конклавов локального творения, но не участвует в специальном рассмотрении административных проблем.
\vs p018 6:6 Когда локальная вселенная утверждается в свете и жизни, её прославленные существа свободно общаются с От Века Единым, который с этого времени действует в расширенном качестве в таком царстве эволюционного совершенства. Но он по\hyp{}прежнему остаётся в первую очередь послом Троицы и Райским советником.
\vs p018 6:7 Локальной вселенной непосредственно правит божественный Сын, происходящий от двух Божеств, но рядом с ним постоянно находится Райский брат~--- личность Троичного происхождения. В случае временного отсутствия Сына Создателя в столице его локальной вселенной, исполняющие обязанности правителей при принятии важных решений в значительной степени руководствуются советом своего От Века Единого.
\usection{ОТ ВЕКА ВЕРНЫЕ}
\vs p018 7:1 Эти высокие личности Троичного происхождения являются Райскими советниками правителей 100 созвездий в каждой локальной вселенной. Существует 70\,000\,000 От Века Верных, но, как и От Века Единые, не все находятся на службе. Их Райский резервный корпус~--- это Совещательная Комиссия по Межвселенской Этике и Самоуправлению. От Века Верные сменяют друг друга на службе в соответствии с постановлениями верховного совета их резервного корпуса.
\vs p018 7:2 Всем тем, чем От Века Единый является для Сына Создателя локальной вселенной, От Века Верные являются для Сынов Ворондадеков, которые правят созвездиями данного локального творения. Они в высшей степени преданы и божественно верны благополучию созвездия своего назначения, отсюда и название~--- От Века \bibemph{Верные}. Они действуют только как советники и никогда не участвуют в административной деятельности, кроме как по приглашению властей созвездия. Они также не имеют прямого отношения к системе образования пилигримов восхождения на архитектурных сферах обучения, окружающих столицу созвездия. Все подобные начинания находятся под надзором Сынов Ворондадеков.
\vs p018 7:3 Все От Века Верные, действующие в созвездиях локальной вселенной, находятся под юрисдикцией От Века Единого и докладывают непосредственно ему. У них нет обширной системы связи, так как обычно они ограничивают себя взаимодействием в пределах локальной вселенной. Любой От Века Верный, несущий службу в Небадоне, может связаться и связывается со всеми остальными существами его типа, находящимися на службе в этой локальной вселенной.
\vs p018 7:4 Подобно От Века Единому на столице вселенной, От Века Верные содержат свои личные резиденции на столицах созвездий отдельно от резиденций административных руководителей этих миров. Их жилища действительно скромны по сравнению с домами Ворондадеков~--- правителей созвездий.
\vs p018 7:5 От Века Верные~--- последнее звено в длинной административно\hyp{}консультативной цепи, которая простирается от священных сфер Всеобщего Отца, находящихся близко к центру всего, до первичных подразделений локальных вселенных. Режим Троичного происхождения заканчивается на уровне созвездий; такие Райские советники не находятся постоянно в своих составных системах или на обитаемых мирах. Эти последние административные единицы находятся целиком под юрисдикцией существ\hyp{}уроженцев локальных вселенных.
\vsetoff
\vs p018 7:6 [Представлено Божественным Советником Уверсы]
\quizlink
