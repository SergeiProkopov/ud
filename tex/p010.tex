\upaper{10}{РАЙСКАЯ ТРОИЦА}
\uminitoc{САМОРАСПРЕДЕЛЕНИЕ ПЕРВОГО ИСТОЧНИКА И ЦЕНТРА}
\uminitoc{ПЕРСОНАЛИЗАЦИЯ БОЖЕСТВА}
\uminitoc{ТРИ ЛИЦА БОЖЕСТВА}
\uminitoc{СОЮЗ ТРОИЦЫ БОЖЕСТВА}
\uminitoc{ФУНКЦИИ ТРОИЦЫ}
\uminitoc{СТАЦИОНАРНЫЕ СЫНЫ ТРОИЦЫ}
\uminitoc{СВЕРХКОНТРОЛЬ ВЕРХОВНОСТИ}
\uminitoc{ТРОИЦА ЗА ПРЕДЕЛАМИ КОНЕЧНОГО}
\author{Всеобщий Цензор}
\vs p010 0:1 Райская Троица вечных Божеств помогает избавить Отца от абсолютизма личности. Троица в совершенстве связывает безграничное выражение бесконечной личной воли Бога с абсолютностью Божества. Вечный Сын и различные Сыны божественного происхождения вместе с Совместным Вершителем и его вселенскими детьми эффективно обеспечивают освобождение Отца от ограничений, в противном случае присущих первенству, совершенству, неизменности, вечности, универсальности, абсолютности и бесконечности.
\vs p010 0:2 Райская Троица эффективно обеспечивает полное выражение и совершенное откровение вечной природы Божества. Стационарные Сыны Троицы подобным же образом предоставляют полное и совершенное откровение божественной справедливости. Троица --- это единство Божества, и это единство вечно покоится на абсолютных основаниях божественного единства трёх изначальных, согласованных и сосуществующих личностей: Бога Отца, Бога Сына и Бога Духа.
\vs p010 0:3 \pc Из нынешней ситуации на круге вечности, оглядываясь назад, в бесконечное прошлое, мы можем обнаружить только одну неотвратимую неизбежность во вселенских событиях --- Райскую Троицу. Я считаю, что Троица была неизбежна. Когда я обозреваю прошлое, настоящее и будущее время, я считаю, что ничто другое во всей вселенной вселенных не было неизбежным. Нынешняя главная вселенная, рассматриваемая в ретроспективе или в перспективе, немыслима без Троицы. При наличии Райской Троицы мы можем постулировать альтернативные или даже множественные способы свершения чего бы то ни было, но без Троицы Отца, Сына и Духа мы не можем представить себе, как Бесконечное могло бы достичь тройственной и равной персонализации при абсолютном единстве Божества. Никакая другая концепция творения не соответствует стандартам Троицы в смысле полноты и абсолютности, присущих единству Божества, в соединении с полнотой волевого высвобождения, присущего тройственной персонализации Божества.
\usection{САМОРАСПРЕДЕЛЕНИЕ ПЕРВОГО ИСТОЧНИКА И ЦЕНТРА}
\vs p010 1:1 По\hyp{}видимому, некогда в вечности Отец выбрал путь глубокого самораспределения. Бескорыстному, любящему и вызывающему любовь характеру Всеобщего Отца внутренне присуще нечто такое, что заставляет его оставлять за собой только те полномочия и ту власть, которые он, по\hyp{}видимому, находит невозможным делегировать или даровать.
\vs p010 1:2 Всеобщий Отец всё время избавлялся от каждой части себя, которую можно было бы подарить какому\hyp{}либо другому Создателю или созданию. Он делегировал своим божественным Сынам и связанным с ними разумным существам всё могущество и всю власть, которые могли быть делегированы. Он фактически передал своим Суверенным Сынам в их вселенных все прерогативы административной власти, которые можно было передать. В делах локальной вселенной он сделал каждого Суверенного Сына Создателя таким же совершенным, компетентным и наделённым властью, каким является Вечный Сын в изначальной и центральной вселенной. Он отдал, фактически даровал, сохраняя достоинство и святость обладания личностью, всего себя и все свои атрибуты, всё то, от чего он мог отказаться, всеми способами, во все эпохи, повсюду, каждой личности и во всех вселенных, кроме вселенной его центрального обитания.
\vs p010 1:3 \pc Божественная личность не эгоцентрична; самораспределение и способность делиться личностью характеризуют божественную индивидуальность, обладающую свободой воли. Создания жаждут общения с другими личностными созданиями; Создатели побуждаются делиться божественностью со своими вселенскими детьми; личность Бесконечного раскрывается как Всеобщий Отец, который разделяет реальность бытия и равенство себя с двумя равными личностями: Вечным Сыном и Совместным Вершителем.
\vs p010 1:4 \pc В поисках знания относительно личности и божественных атрибутов Отца мы всегда будем зависеть от откровений Вечного Сына, ибо когда был осуществлён совместный акт творения, когда Третье Лицо Божества возникло в личностном существовании и реализовало объединённые концепции своих божественных родителей, Отец перестал существовать как безусловная личность. С появлением Совместного Вершителя и материализацией центрального ядра творения произошли определённые вечные изменения. Бог отдал себя как абсолютную личность своему Вечному Сыну. Так Отец наделяет своего единородного Сына <<личностью бесконечности>>, в то время как они оба даруют <<объединённую личность>> своего вечного союза Бесконечному Духу.
\vs p010 1:5 По этим и другим причинам, выходящим за рамки концепции конечного разума, человеку чрезвычайно трудно постичь бесконечную личность Бога как отца, кроме как в универсальном проявлении в Вечном Сыне и, вместе с Сыном, в универсальной активности в Бесконечном Духе.
\vs p010 1:6 Поскольку Райские Сыны Бога посещают эволюционные миры, а иногда даже живут там в подобии смертной плоти, и поскольку эти посвящения позволяют смертному человеку действительно кое\hyp{}что узнать о природе и характере божественной личности, следовательно, создания планетарных сфер должны обращаться к посвящениям этих Райских Сынов за надёжной и достоверной информацией об Отце, Сыне и Духе.
\usection{ПЕРСОНАЛИЗАЦИЯ БОЖЕСТВА}
\vs p010 2:1 Посредством тринитизации Отец лишает себя той безусловной духовной личности, которой является Сын, но, тем самым он делает себя Отцом этого самого Сына и потому наделяет себя неограниченной способностью стать божественным Отцом всех впоследствии созданных, возникших и  других персонализованных типов разумных волевых созданий. Как \bibemph{абсолютная и безусловная личность} Отец может функционировать только как Сын и вместе с ним, но как \bibemph{личностный Отец} он продолжает наделять личностью сонмы разнообразных разумных волевых созданий разных уровней, и всегда поддерживает личные отношения любящего союза с этой огромной семьёй вселенских детей.
\vs p010 2:2 После того, как Отец даровал личности своего Сына полноту себя, и когда этот акт дарования себя становится завершённым и совершенным, из бесконечной мощи и природы, которые, таким образом, существуют в союзе Отец\hyp{}Сын, вечные партнёры совместно даруют те качества и атрибуты, которые составляют ещё одно существо, подобное им; и эта объединённая личность, Бесконечный Дух, завершает экзистенциальную персонализацию Божества.
\vs p010 2:3 Сын необходим для отцовства Бога. Дух необходим для братства Второго и Третьего Лиц. Три лица составляют минимальную социальную группу, но это самая незначительная из всех многочисленных причин для веры в неизбежность Совместного Вершителя.
\vs p010 2:4 \pc Первый Источник и Центр --- это бесконечная \bibemph{личность\hyp{}отец,} неограниченная личность\hyp{}источник. Вечный Сын --- безусловная \bibemph{личность\hyp{}абсолют,} то божественное существо, которое во все времена и в вечности выступает как совершенное откровение личной природы Бога. Бесконечный Дух --- \bibemph{объединённая личность,} уникальное личностное следствие вечного союза Отец\hyp{}Сын.
\vs p010 2:5 \pc Личность Первого Источника и Центра есть личность бесконечности минус абсолютная личность Вечного Сына. Личность Третьего Источника и Центра есть супераддитивное следствие союза освобожденной личности\hyp{}Отца и абсолютной личности\hyp{}Сына.
\vs p010 2:6 \pc Всеобщий Отец, Вечный Сын и Бесконечный Дух --- уникальные лица; ни одно из них не является дубликатом; каждое оригинально; все они объединены.
\vs p010 2:7 \pc Один лишь Вечный Сын испытывает полноту божественных отношений личности, сознание как сыновства с Отцом, так и отцовства по отношению к Духу, а также божественного равенства как с Отцом\hyp{}прародителем, так и с Духом\hyp{}партнёром. Отец знает опыт обретения Сына, равного ему, но Отец не знает никаких родовых предшественников. Вечный Сын имеет опыт сыновства, признание родословной своей личности, и в то же время Сын сознаёт, что является совместным родителем Бесконечного Духа. Бесконечный Дух сознаёт двойную родословную своей личности, но не является родителем ещё одной равной личности Божества. На Духе экзистенциальный цикл персонализации Божества достигает завершения; первичные личности Третьего Источника и Центра являются эмпирическими, и числом их семь.
\vs p010 2:8 Я происхожу от Райской Троицы. Я знаю Троицу как объединённое Божество; я также знаю, что Отец, Сын и Дух существуют и действуют в своих определённых личных качествах. Я точно знаю, что они не только действуют лично и коллективно, но также координируют свои действия в различных сочетаниях, так что в конечном счёте они функционируют в семи различных одиночных и множественных качествах. И поскольку эти семь ассоциаций исчерпывают возможности такого сочетания божественности, реальности вселенной неизбежно проявляются в семи вариациях ценностей, смыслов и личности.
\usection{ТРИ ЛИЦА БОЖЕСТВА}
\vs p010 3:1 
\vs p010 3:2 \pc 
\vs p010 3:3 
\vs p010 3:4 
\vs p010 3:5 \pc 
\vs p010 3:6 
\vs p010 3:7 
\vs p010 3:8 
\vs p010 3:9 \pc 
\vs p010 3:10 
\vs p010 3:11 
\vs p010 3:12 
\vs p010 3:13 
\vs p010 3:14 
\vs p010 3:15 
\vs p010 3:16 
\vs p010 3:17 \pc 
\vs p010 3:18 \pc 
\vs p010 3:19 \pc 
\usection{СОЮЗ ТРОИЦЫ БОЖЕСТВА}
\vs p010 4:1 
\vs p010 4:2 \pc 
\vs p010 4:3 
\vs p010 4:4 
\vs p010 4:5 \pc 
\vs p010 4:6 \pc 
\vs p010 4:7 
\usection{ФУНКЦИИ ТРОИЦЫ}
\vs p010 5:1 
\vs p010 5:2 
\vs p010 5:3 \pc 
\vs p010 5:4 
\vs p010 5:5 
\vs p010 5:6 
\vs p010 5:7 \pc 
\vs p010 5:8 \pc 
\usection{СТАЦИОНАРНЫЕ СЫНЫ ТРОИЦЫ}
\vs p010 6:1 
\vs p010 6:2 \pc 
\vs p010 6:3 \pc 
\vs p010 6:4 \pc 
\vs p010 6:5 \pc 
\vs p010 6:6 
\vs p010 6:7 
\vs p010 6:8 
\vs p010 6:9 
\vs p010 6:10 
\vs p010 6:11 
\vs p010 6:12 
\vs p010 6:13 
\vs p010 6:14 
\vs p010 6:15 
\vs p010 6:16 \pc 
\vs p010 6:17 
\vs p010 6:18 
\usection{СВЕРХКОНТРОЛЬ ВЕРХОВНОСТИ}
\vs p010 7:1 
\vs p010 7:2 \pc 
\vs p010 7:3 
\vs p010 7:4 \pc 
\vs p010 7:5 
\vs p010 7:6 
\usection{ТРОИЦА ЗА ПРЕДЕЛАМИ КОНЕЧНОГО}
\vs p010 8:1 
\vs p010 8:2 
\vs p010 8:3 \pc 
\vs p010 8:4 
\vs p010 8:5 
\vs p010 8:6 \pc 
\vs p010 8:7 
\vs p010 8:8 
\vs p010 8:9 \pc 
\vsetoff
\vs p010 8:10 [При поддержке Всеобщего Цензора, действующего с разрешения Ветхих Днями, проживающих на Уверсе.]
\quizlink
