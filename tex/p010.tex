\upaper{10}{РАЙСКАЯ ТРОИЦА}
\uminitoc{САМОРАСПРЕДЕЛЕНИЕ ПЕРВОГО ИСТОЧНИКА И ЦЕНТРА}
\uminitoc{ПЕРСОНАЛИЗАЦИЯ БОЖЕСТВА}
\uminitoc{ТРИ ЛИЦА БОЖЕСТВА}
\uminitoc{ТРОИЧНЫЙ СОЮЗ БОЖЕСТВА}
\uminitoc{ФУНКЦИИ ТРОИЦЫ}
\uminitoc{СТАЦИОНАРНЫЕ СЫНЫ ТРОИЦЫ}
\uminitoc{СВЕРХКОНТРОЛЬ ВЕРХОВНОСТИ}
\uminitoc{ТРОИЦА ЗА ПРЕДЕЛАМИ КОНЕЧНОГО}
\author{Всеобщий Цензор}
\vs p010 0:1 Райская Троица вечных Божеств помогает избавить Отца от абсолютизма личности. Троица в совершенстве связывает безграничное выражение бесконечной личной воли Бога с абсолютностью Божества. Вечный Сын и различные Сыны божественного происхождения вместе с Совместным Вершителем и его вселенскими детьми эффективно обеспечивают освобождение Отца от ограничений, в противном случае присущих первенству, совершенству, неизменности, вечности, универсальности, абсолютности и бесконечности.
\vs p010 0:2 Райская Троица эффективно обеспечивает полное выражение и совершенное откровение вечной природы Божества. Стационарные Сыны Троицы подобным же образом предоставляют полное и совершенное откровение божественной справедливости. Троица~--- это единство Божества, и это единство вечно покоится на абсолютных основаниях божественного единства трёх изначальных, согласованных и сосуществующих личностей: Бога Отца, Бога Сына и Бога Духа.
\vs p010 0:3 \pc Из нынешней ситуации на круге вечности, оглядываясь назад в бесконечное прошлое, мы можем обнаружить только одну неотвратимую неизбежность во вселенских событиях~--- Райскую Троицу. Я считаю, что Троица была неизбежна. Когда я обозреваю прошлое, настоящее и будущее время, я считаю, что ничто другое во всей вселенной вселенных не было неизбежным. Нынешняя главная вселенная, рассматриваемая в ретроспективе или в перспективе, немыслима без Троицы. При наличии Райской Троицы мы можем постулировать альтернативные или даже множественные способы свершения чего бы то ни было, но без Троицы Отца, Сына и Духа мы не можем представить себе, как Бесконечное могло бы достичь тройственной и равной персонализации при абсолютном единстве Божества. Никакая другая концепция творения не соответствует стандартам Троицы в смысле полноты и абсолютности, присущих единству Божества, в соединении с полнотой волевого высвобождения, присущего тройственной персонализации Божества.
\usection{САМОРАСПРЕДЕЛЕНИЕ ПЕРВОГО ИСТОЧНИКА И ЦЕНТРА}
\vs p010 1:1 По\hyp{}видимому, некогда в вечности Отец выбрал путь глубокого самораспределения. Бескорыстному, любящему и вызывающему любовь характеру Всеобщего Отца внутренне присуще нечто такое, что заставляет его оставлять за собой только те полномочия и ту власть, которые он, по\hyp{}видимому, находит невозможным делегировать или даровать.
\vs p010 1:2 Всеобщий Отец всё время избавлялся от каждой части себя, которую можно было бы подарить какому\hyp{}либо другому Создателю или созданию. Он делегировал своим божественным Сынам и связанным с ними разумным существам всё могущество и всю власть, которые могли быть делегированы. Он фактически передал своим Суверенным Сынам в их вселенных все прерогативы административной власти, которые можно было передать. В делах локальной вселенной он сделал каждого Суверенного Сына Создателя таким же совершенным, компетентным и наделённым властью, каким является Вечный Сын в изначальной и центральной вселенной. Он отдал, фактически даровал, сохраняя достоинство и святость обладания личностью, всего себя и все свои атрибуты, всё то, от чего он мог отказаться, всеми способами, во все эпохи, повсюду, каждой личности и во всех вселенных, кроме вселенной его центрального обитания.
\vs p010 1:3 \pc Божественная личность не эгоцентрична; самораспределение и способность делиться личностью характеризуют божественную индивидуальность, обладающую свободой воли. Создания жаждут общения с другими личностными созданиями; Создатели побуждаются делиться божественностью со своими вселенскими детьми; личность Бесконечного раскрывается как Всеобщий Отец, который разделяет реальность бытия и равенство себе с двумя равными личностями: Вечным Сыном и Совместным Вершителем.
\vs p010 1:4 \pc В поисках знания относительно личности и божественных атрибутов Отца мы всегда будем зависеть от откровений Вечного Сына, ибо когда был осуществлён совместный акт творения, когда Третье Лицо Божества возникло в личностном существовании и реализовало объединённые концепции своих божественных родителей, Отец перестал существовать как безусловная личность. С появлением Совместного Вершителя и материализацией центрального ядра творения произошли определённые вечные изменения. Бог отдал себя как абсолютную личность своему Вечному Сыну. Так Отец наделяет своего единородного Сына <<личностью бесконечности>>, в то время как они оба даруют <<объединённую личность>> своего вечного союза Бесконечному Духу.
\vs p010 1:5 По этим и другим причинам, выходящим за рамки концепции конечного разума, человеку чрезвычайно трудно постичь бесконечную личность Бога как отца, кроме как в универсальном проявлении в Вечном Сыне и, вместе с Сыном, в универсальной активности в Бесконечном Духе.
\vs p010 1:6 Поскольку Райские Сыны Бога посещают эволюционные миры, а иногда даже живут там в подобии смертной плоти, и поскольку эти посвящения позволяют смертному человеку действительно кое\hyp{}что узнать о природе и характере божественной личности, следовательно, создания планетарных сфер должны обращаться к посвящениям этих Райских Сынов за надёжной и достоверной информацией об Отце, Сыне и Духе.
\usection{ПЕРСОНАЛИЗАЦИЯ БОЖЕСТВА}
\vs p010 2:1 Посредством тринитизации Отец лишает себя той безусловной духовной личности, которой является Сын, но тем самым он делает себя Отцом этого самого Сына и потому наделяет себя неограниченной способностью стать божественным Отцом всех впоследствии созданных, возникших и  других персонализированных типов разумных волевых созданий. Как \bibemph{абсолютная и безусловная личность,} Отец может функционировать только как Сын и вместе с ним, но как \bibemph{личностный Отец} он продолжает наделять личностью сонмы разнообразных разумных волевых созданий разных уровней и всегда поддерживает личные отношения любящего союза с этой огромной семьёй вселенских детей.
\vs p010 2:2 После того как Отец даровал личности своего Сына полноту себя, и когда этот акт дарования себя становится завершённым и совершенным, из бесконечной мощи и природы, которые, таким образом, существуют в союзе Отец\hyp{}Сын, вечные партнёры совместно даруют те качества и атрибуты, которые составляют ещё одно существо, подобное им; и эта объединённая личность, Бесконечный Дух, завершает экзистенциальную персонализацию Божества.
\vs p010 2:3 Сын необходим для отцовства Бога. Дух необходим для братства Второго и Третьего Лиц. Три лица составляют минимальную социальную группу, но это самая незначительная из всех многочисленных причин для веры в неизбежность Совместного Вершителя.
\vs p010 2:4 \pc Первый Источник и Центр~--- это бесконечная \bibemph{личность\hyp{}отец,} неограниченная личность\hyp{}источник. Вечный Сын~--- безусловная \bibemph{личность\hyp{}абсолют,} то божественное существо, которое во все времена и в вечности выступает как совершенное откровение личной природы Бога. Бесконечный Дух~--- \bibemph{объединённая личность,} уникальное личностное следствие вечного союза Отец\hyp{}Сын.
\vs p010 2:5 \pc Личность Первого Источника и Центра есть личность бесконечности минус абсолютная личность Вечного Сына. Личность Третьего Источника и Центра есть супераддитивное следствие союза освобождённой личности\hyp{}Отца и абсолютной личности\hyp{}Сына.
\vs p010 2:6 \pc Всеобщий Отец, Вечный Сын и Бесконечный Дух~--- уникальные лица; ни одно из них не является дубликатом; каждое оригинально; все они объединены.
\vs p010 2:7 \pc Один лишь Вечный Сын испытывает полноту божественных отношений личности, сознание как сыновства с Отцом, так и отцовства по отношению к Духу, а также божественного равенства как с Отцом\hyp{}прародителем, так и с Духом\hyp{}партнёром. Отец знает опыт обретения Сына, равного ему, но Отец не знает никаких родовых предшественников. Вечный Сын имеет опыт сыновства, признание родословной своей личности, и в то же время Сын сознаёт, что является совместным родителем Бесконечного Духа. Бесконечный Дух сознаёт двойную родословную своей личности, но не является родителем ещё одной равной личности Божества. На Духе экзистенциальный цикл персонализации Божества достигает завершения; первичные личности Третьего Источника и Центра являются эмпирическими, и числом их семь.
\vs p010 2:8 Я происхожу от Райской Троицы. Я знаю Троицу как объединённое Божество; я также знаю, что Отец, Сын и Дух существуют и действуют в своих определённых личных качествах. Я точно знаю, что они не только действуют лично и коллективно, но также координируют свои действия в различных сочетаниях, так что в конечном счёте они функционируют в семи различных одиночных и множественных качествах. И поскольку эти семь ассоциаций исчерпывают возможности такого сочетания божественности, реальности вселенной неизбежно проявляются в семи вариациях ценностей, смыслов и личности.
\usection{ТРИ ЛИЦА БОЖЕСТВА}
\vs p010 3:1 Несмотря на то, что есть только одно Божество, есть три несомненных и божественных персонализации Божества. Относительно наделения человека божественными Настройщиками Отец сказал: <<Сотворим смертного человека по образу нашему>>. В урантийских источниках неоднократно встречается эта ссылка на действия и деяния множественного Божества, ясно показывающая признание существования и работы трёх Источников и Центров.
\vs p010 3:2 \pc Нас учат, что Сын и Дух поддерживают одинаковые и равные отношения с Отцом в объединении Троицы. В вечности и как Божества они несомненно являются такими, но во времени и как личности они определённо раскрывают отношения самого разнообразного характера. Глядя из Рая на вселенные, эти отношения кажутся очень похожими, но если смотреть из областей пространства, то они представляются совершенно разными.
\vs p010 3:3 Божественные Сыны~--- это поистине <<Слово Бога>>, а дети Духа воистину являются <<Деяниями Бога>>. Бог говорит через Сына и вместе с Сыном действует через Бесконечного Духа, в то время как во всей вселенской деятельности Сын и Дух демонстрируют изысканно братские отношения, работая как два равных брата с восхищением и любовью относясь к почитаемому и божественно уважаемому общему Отцу.
\vs p010 3:4 Отец, Сын и Дух бесспорно равны по своей природе, равноправны по своему бытию, но существуют несомненные различия в их вселенских действиях, и, когда они действуют в одиночку, каждая личность Божества кажется ограниченной в абсолютности.
\vs p010 3:5 \pc До того, как Всеобщий Отец освободился от личности, сил и атрибутов, составляющих Сына и Духа, он, по\hyp{}видимому, был (с философской точки зрения) безусловным, абсолютным и бесконечным Божеством. Но такой теоретический Первый Источник и Центр без Сына не мог ни в каком смысле этого слова считаться \bibemph{Всеобщим Отцом;} отцовство невозможно без сыновства. Более того, Отец, чтобы быть абсолютным в полном смысле слова, должен был существовать в одиночестве в какой\hyp{}то вечно удалённый момент. Но у него никогда не было такого уединённого существования; Сын и Дух сосуществуют с Отцом в вечности. Первый Источник и Центр всегда был и всегда будет вечным Отцом Изначального Сына и, вместе с Сыном,~--- вечным прародителем Бесконечного Духа.
\vs p010 3:6 Мы замечаем, что Отец отказался от всех непосредственных проявлений абсолютности, кроме абсолютного отцовства и абсолютной воли. Мы не знаем, является ли воля неотъемлемым атрибутом Отца; мы можем только заметить, что он \bibemph{не} отказался от воли. По\hyp{}видимому, такая бесконечность воли была вечно присуща Первому Источнику и Центру.
\vs p010 3:7 Наделяя абсолютностью личности Вечного Сына, Всеобщий Отец освобождается от пут абсолютизма личности, но, поступая таким образом, он делает шаг, который навсегда лишает его возможности действовать в одиночку как личность\hyp{}абсолют. А с окончательной персонализацией сосуществующего Божества~--- Совместного Вершителя~--- возникает критическая тринитарная взаимозависимость трёх божественных личностей в отношении целостности функционирования Божества в абсолюте.
\vs p010 3:8 Бог~--- это Отец\hyp{}Абсолют всех личностей во вселенной вселенных. Отец лично абсолютен в свободе действий, но во вселенных времени и пространства, созданных, создаваемых, и ещё не созданных, Отец не является заметно абсолютным как тотальное Божество, кроме как в Райской Троице.
\vs p010 3:9 \pc Первый Источник и Центр функционирует вне Хавоны, в наблюдаемых\fnst{Англ. phenomenal.} вселенных, следующим образом:
\vs p010 3:10 \li{1.}Как создатель~--- через Сынов Создателей, своих внуков.
\vs p010 3:11 \li{2.}Как регулятор~--- через гравитационный центр Рая.
\vs p010 3:12 \li{3.}Как дух~--- через Вечного Сына.
\vs p010 3:13 \li{4.}Как разум~--- через Совместного Создателя.
\vs p010 3:14 \li{5.}Как Отец, он поддерживает родительский контакт со всеми существами через свой личностный контур.
\vs p010 3:15 \li{6.}Как личность, он действует \bibemph{непосредственно} по всему творению с помощью своих исключительных фрагментов~--- в смертном человеке с помощью Настройщиков Мыслей.
\vs p010 3:16 \li{7.}Как тотальное Божество, он функционирует только в Райской Троице.
\vs p010 3:17 \pc Все эти уступки и делегирование юрисдикции Всеобщим Отцом являются полностью добровольными и возложенными на себя им самим. Всемогущий Отец намеренно принимает эти ограничения вселенской власти.
\vs p010 3:18 \pc Вечный Сын, по\hyp{}видимому, действует как единое целое с Отцом во всех духовных отношениях, за исключением посвящений фрагментов Бога и других доличностных действий. Сын также не отождествляется тесно с интеллектуальной деятельностью материальных созданий или с энергетической деятельностью материальных вселенных. Как абсолют, Сын действует в качестве личности и только в сфере духовной вселенной.
\vs p010 3:19 \pc Бесконечный Дух удивительно универсален и невероятно разносторонен во всех своих действиях. Он действует в сферах разума, материи и духа. Совместный Вершитель представляет ассоциацию Отец\hyp{}Сын, но он также действует и от себя. Он не занимается непосредственно физической гравитацией, духовной гравитацией или личностным контуром, но, в большей или меньшей степени, участвует в любой другой вселенской деятельности. Несмотря на то, что Бесконечный Дух кажется зависящим от трёх экзистенциальных и абсолютных средств контроля гравитации, он, по\hyp{}видимому, осуществляет три сверхконтроля. Эта тройная способность используется многими способами, чтобы превзойти и, по\hyp{}видимому, нейтрализовать даже проявления первичных сил и энергий, вплоть до сверхпредельных границ абсолютности. В определённых ситуациях эти сверхконтроли абсолютно превосходят даже первичные проявления космической реальности.
\usection{ТРОИЧНЫЙ СОЮЗ БОЖЕСТВА}
\vs p010 4:1 Из всех абсолютных объединений Райская Троица (первое триединство) уникальна как исключительное объединение личного Божества. Бог действует как Бог только по отношению к Богу и тем, кто может познать Бога, но как абсолютное Божество только в Райской Троице и по отношению ко вселенской тотальности.
\vs p010 4:2 \pc Вечное Божество является совершенно объединённым; тем не менее, есть три совершенно индивидуализированных лица Божества. Райская Троица делает возможным одновременное выражение всего разнообразия черт характера и бесконечного могущества Первого Источника и Центра и его вечных равных, а также всего божественного единства вселенских функций нераздельного Божества.
\vs p010 4:3 Троица~--- это объединение бесконечных личностей, функционирующее в неличностном качестве, но без противоречия с личностью. Это грубая иллюстрация, но отец, сын и внук могут образовать целостную сущность, которая будет неличностной, но, тем не менее, подчиняться их личной воле.
\vs p010 4:4 Райская Троица \bibemph{реальна}. Она существует как союз Божества, состоящий из Отца, Сына и Духа; тем не менее, Отец, Сын или Дух, или любые два из них, могут функционировать по отношению ко всё той же Райской Троице. Отец, Сын и Дух могут сотрудничать не в составе Троицы, но не как три Божества. Как лица они могут сотрудничать так, как сочтут нужным, но это не Троица.
\vs p010 4:5 \pc Всегда помни: то, что делает Бесконечный Дух, есть функция Совместного Вершителя. И Отец, и Сын действуют в нём, через него и в качестве него. Но было бы тщетно пытаться прояснить тайну Троицы: три как одно и в одном, а один как двое и действующий за двоих.
\vs p010 4:6 \pc Троица настолько связана с делами вселенной в целом, что с ней необходимо считаться в наших попытках объяснить целостность любого изолированного космического события или личностного отношения. Троица действует на всех уровнях космоса, а смертный человек ограничен конечным уровнем; поэтому человек должен довольствоваться конечным представлением о Троице как Троице.
\vs p010 4:7 Как смертный во плоти, ты должен рассматривать Троицу в соответствии с твоей индивидуальной просвещённостью и в гармонии с реакциями твоего разума и души. Ты можешь очень мало узнать об абсолютности Троицы, но по мере того, как ты восходишь к Раю, ты не раз испытаешь изумление от новых откровений и неожиданных открытий верховности и предельности Троицы, если не абсолютности.
\usection{ФУНКЦИИ ТРОИЦЫ}
\vs p010 5:1 Личные Божества имеют атрибуты, но вряд ли можно говорить о Троице как имеющей атрибуты. Это объединение божественных существ более правильно рассматривать как имеющее \bibemph{функции,} такие как отправление правосудия, отношение тотальности, согласованные действия и космический сверхконтроль. Эти функции являются в активном смысле верховными, предельными и (в пределах Божества) абсолютными в том, что касается всех живых реальностей, имеющих личностную ценность.
\vs p010 5:2 Функции Райской Троицы~--- это не просто сумма видимого обладания божественностью Отцом плюс те специфические атрибуты, которые уникальны в личном существовании Сына и Духа. Троичное объединение трёх Райских Божеств приводит к эволюции, возникновению и обожествлению новых смыслов, ценностей, мощи и способностей к вселенскому откровению, действию и управлению. Живые сообщества, человеческие семьи, социальные группы или Райская Троица не увеличиваются простым арифметическим суммированием. Групповой потенциал всегда намного превосходит простую сумму атрибутов составляющих индивидуумов.
\vs p010 5:3 \pc Троица поддерживает уникальное отношение как Троица ко всей вселенной прошлого, настоящего и будущего. И функции Троицы лучше всего можно рассматривать в связи со вселенскими отношениями Троицы. Такие отношения одновременны и могут быть множественными в отношении любой изолированной ситуации или события:
\vs p010 5:4 \li{1.}\bibemph{Отношение к конечному}. Максимальное самоограничение Троицы~--- её отношение к конечному. Троица~--- это не личность, и Верховное Существо не является исключительной персонализацией Троицы, но Верховный~--- это наибольшее приближение к сосредоточению мощи\hyp{}личности Троицы, которое может быть понято конечными созданиями. Поэтому о Троице по отношению к конечному иногда говорят как о Троице Верховности.
\vs p010 5:5 \li{2.}\bibemph{Отношение к абсонитному}. Райская Троица учитывает те уровни существования, которые более чем конечны, но менее чем абсолютны, и это отношение иногда именуется Троицей Предельности. Ни Абсолютное, ни Верховное не представляют Райскую Троицу полностью, но в ограниченном смысле и каждый для своего уровня, по\hyp{}видимому, представляет Троицу на протяжении доличностных эпох эмпирически\hyp{}мощностного развития.
\vs p010 5:6 \li{3.}\bibemph{Абсолютное отношение} Райской Троицы~--- отношение к абсолютным существованиям и достигает кульминации в действии тотального Божества.
\vs p010 5:7 \pc Троица Бесконечная включает согласованное действие всех отношений триединства Первого Источника и Центра~--- как необожествлённых, так и обожествлённых,~--- и поэтому личностям очень трудно её понять. Созерцая Троицу как нечто бесконечное, не забывайте о семи триединствах\fnst{См. \bibref[104:4]{p104 4:1}.}; тем самым можно избежать определённых трудностей понимания и частично разрешить некоторые парадоксы.
\vs p010 5:8 \pc Но я не владею языком, который позволил бы мне донести до ограниченного человеческого разума полную истину и вечное значение Райской Троицы и природу бесконечного взаимодействия трёх существ бесконечного совершенства.
\usection{СТАЦИОНАРНЫЕ СЫНЫ ТРОИЦЫ}
\vs p010 6:1 Всякий закон берёт начало в Первом Источнике и Центре; \bibemph{он есть закон}. Управление духовным законом~--- прерогатива Второго Источника и Центра. Откровение закона, провозглашение и толкование божественных уставов~--- функция Третьего Источника и Центра. Применение закона, правосудие, входит в компетенцию Райской Троицы и осуществляется определёнными Сынами Троицы.
\vs p010 6:2 \pc \bibemph{Правосудие} присуще всеобщему полновластию Райской Троицы, но доброта, милосердие и истина представляют собой вселенское служение божественных личностей, союз Божеств которых составляет Троицу. Правосудие не является отношением Отца, Сына или Духа. Правосудие есть отношение этих личностей любви, милосердия и служения как Троицы. Ни одно из Райских Божеств не способствует отправлению правосудия. Правосудие никогда не является личным отношением; оно всегда~--- множественная функция.
\vs p010 6:3 \pc \bibemph{Свидетельства,} основа справедливости (правосудие в гармонии с милосердием), предоставляются личностями Третьего Источника и Центра, совместного представителя Отца и Сына всем мирам и умам разумных существ всего творения.
\vs p010 6:4 \pc \bibemph{Вынесение приговора,} окончательное исполнение правосудия в соответствии со свидетельством, представленным личностями Бесконечного Духа,~--- есть работа Стационарных Сынов Троицы, существ, разделяющих природу Троицы~--- природу объединённого Отца, Сына и Духа.
\vs p010 6:5 \pc В эту группу Сынов Троицы входят следующие личности:
\vs p010 6:6 \li{1.}Тринитизованные Секреты Верховности.
\vs p010 6:7 \li{2.}От Века Вечные.
\vs p010 6:8 \li{3.}От Века Древние.
\vs p010 6:9 \li{4.}От Века Совершенные.
\vs p010 6:10 \li{5.}От Века Недавние.
\vs p010 6:11 \li{6.}От Века Единые.
\vs p010 6:12 \li{7.}От Века Верные.
\vs p010 6:13 \li{8.}Совершенствователи Мудрости.
\vs p010 6:14 \li{9.}Божественные Советники.
\vs p010 6:15 \li{10.}Всеобщие Цензоры.
\vs p010 6:16 \pc Мы~--- дети трёх Райских Божеств, функционирующих как Троица, ибо я как раз принадлежу к десятой категории этой группы~--- Всеобщим Цензорам. Эти категории не отражают позицию Троицы в универсальном смысле; они выражают эту коллективную позицию Божества только в области исполнительного судопроизводства~--- правосудия. Они были специально задуманы Троицей для конкретной работы, которая им поручена, и они представляют Троицу только в тех функциях, для которых они были персонализированы.
\vs p010 6:17 От Века Древние и их партнёры по происхождению от Троицы выносят приговоры верховной справедливости в семи сверхвселенных. В центральной вселенной такие функции существуют только теоретически; там справедливость самоочевидна в совершенстве, а совершенство Хавоны исключает любую возможность дисгармонии.
\vs p010 6:18 Правосудие~--- это коллективная мысль о праведности; милосердие~--- его личное выражение. Милосердие~--- это отношение любви; точность характеризует действие закона; Божественный суд~--- это душа справедливости, всегда подчиняющаяся правосудию Троицы, всегда осуществляющая божественную любовь Бога. При полном осознании и полном понимании праведное правосудие Троицы и милосердная любовь Всеобщего Отца совпадают. Но у человека нет такого полного понимания божественной справедливости. Таким образом, в Троице, с точки зрения человека, личности Отца, Сына и Духа приспособлены к согласованному служению любви и закона в эмпирических вселенных времени.
\usection{СВЕРХКОНТРОЛЬ ВЕРХОВНОСТИ}
\vs p010 7:1 Первое, Второе и Третье Лица Божества равны друг другу и едины. <<Господь, Бог наш,~--- Бог единый>>. В божественной Троице вечных Божеств есть совершенство цели и единство исполнения. Отец, Сын и Совместный Вершитель воистину и божественно едины. Истинно написано: <<Я первый и я последний, и нет Бога, кроме меня>>.
\vs p010 7:2 \pc Так, как вещи представляются смертным на конечном уровне, Райская Троица, подобно Верховному Существу, занимается только целым~--- всей планетой, всей вселенной, всей сверхвселенной, всей большой вселенной. Это отношение к тотальности существует потому, что Троица является тотальностью Божества, а также по многим другим причинам.
\vs p010 7:3 Верховное Существо~--- это нечто меньшее и нечто иное, чем Троица, функционирующая в конечных вселенных; но в определённых пределах и в нынешнюю эпоху неполного синтеза мощи\hyp{}личности это эволюционное Божество, по\hyp{}видимому, действительно отражает позицию Троицы Верховности. Отец, Сын и Дух не действуют лично с Верховным Существом, но в нынешнюю вселенскую эпоху они сотрудничают с ним как Троица. Мы понимаем, что они поддерживают такие же отношения с Предельным. Мы часто строим догадки относительно того, какими будут личные отношения между Райскими Божествами и Богом Верховным, когда он, наконец, разовьётся, но, по правде говоря, мы этого не знаем.
\vs p010 7:4 \pc Мы не считаем, что сверхуправление Верховности полностью предсказуемо. Более того, оказывается, что эта непредсказуемость характеризуется некоторой неполнотой развития, что, несомненно, выдаёт неполноту Всевышнего и конечной реакции на Райскую Троицу.
\vs p010 7:5 Смертный разум может сразу подумать о тысяче и одной вещи~--- катастрофических физических событиях, кошмарных несчастных случаях, ужасных бедствиях, мучительных болезнях и мировых катастрофах~--- и спросить, не связаны ли эти <<кары>> с неведомым маневрированием вероятного функционирования Верховного Существа. Откровенно говоря, мы не знаем; мы в самом деле не уверены. Но мы замечаем, что с течением времени все эти сложные и более или менее загадочные ситуации \bibemph{всегда} оборачиваются во благо и к прогрессу вселенных. Возможно, обстоятельства существования и необъяснимые превратности жизни переплетаются в полный смысла драгоценный рисунок благодаря функции Верховного и сверхконтролю Троицы.
\vs p010 7:6 Как сын Бога, ты можешь различать личное отношение любви во всех действиях Бога Отца. Но ты не всегда будешь в состоянии понять, сколько вселенских действий Райской Троицы способствует благу отдельного смертного на эволюционных мирах пространства. В течение вечности действия Троицы будут раскрыты как вполне значимые и продуманные, но они не всегда представляются таковыми созданиям времени.
\usection{ТРОИЦА ЗА ПРЕДЕЛАМИ КОНЕЧНОГО}
\vs p010 8:1 Даже частично понять многие истины и факты, относящиеся к Райской Троице, можно только осознавая функцию, выходящую за пределы конечного.
\vs p010 8:2 Было бы нецелесообразно обсуждать функции Троицы Предельности, но можно отметить, что Бог Предельный~--- это проявление Троицы в понимании Трансценденталов. Мы склонны верить, что объединение главной вселенной является конечным актом Предельного и, вероятно, отражает некоторые, но не все, фазы абсонитного сверхуправления Райской Троицы. Предельный является ограниченным проявлением Троицы по отношению к абсонитному только в том смысле, в каком Верховный частично представляет Троицу по отношению к конечному.
\vs p010 8:3 \pc Всеобщий Отец, Вечный Сын и Бесконечный Дух в определённом смысле являются составными личностями тотального Божества. Их союз в Райской Троице и абсолютная функция Троицы приравниваются к функции тотального Божества. И такое завершение Божества превосходит как конечное, так и абсонитное.
\vs p010 8:4 Хотя ни одно из Райских Божеств не обладает всем потенциалом Божества, втроём они достигают этого коллективно. Три бесконечные личности, по\hyp{}видимому, являются минимальным числом существ, необходимым для активации доличностного и экзистенциального потенциала тотального Божества~--- Божественного Абсолюта.
\vs p010 8:5 Мы знаем Всеобщего Отца, Вечного Сына и Бесконечного Духа как \bibemph{личности,} но я не знаю лично Божественный Абсолют. Я люблю Бога Отца и поклоняюсь ему; я уважаю и почитаю Божественный Абсолют.
\vs p010 8:6 \pc Однажды я обитал во вселенной, где определённая группа существ учила, что в вечности завершители станут детьми Божественного Абсолюта. Но я не желаю принимать такое объяснение тайны, которая окутывает будущее завершителей.
\vs p010 8:7 Корпус Завершения включает среди прочих тех смертных времени и пространства, которые достигли совершенства во всём, что относится к воле Бога. Как создания и в пределах способностей создания, они полностью и истинно знают Бога. Найдя, таким образом, Бога как Отца всех созданий, эти завершители должны когда\hyp{}нибудь начать поиски сверхконечного Отца. Но этот поиск включает понимание абсонитной природы предельных атрибутов и характера Райского Отца. Вечность покажет, возможно ли такое достижение, но мы убеждены, что даже если завершители действительно постигнут эту предельность божественности, они, вероятно, не смогут достичь сверхпредельных уровней абсолютного Божества.
\vs p010 8:8 Возможно, что завершители частично достигнут Божественного Абсолюта, но даже если это им удастся, в вечности вечностей проблема Всеобщего Абсолюта будет продолжать интриговать, мистифицировать, сбивать с толку и бросать вызов восходящим и прогрессирующим завершителям, ибо мы ощущаем, что непостижимость космических отношений Всеобщего Абсолюта будет иметь тенденцию к росту по мере того, как материальные вселенные и их духовное управление будут продолжать расширяться.
\vs p010 8:9 \pc Только бесконечность может раскрыть Отца\hyp{}Бесконечного.
\vsetoff
\vs p010 8:10 [При поддержке Всеобщего Цензора, действующего с разрешения От Века Древних, проживающих на Уверсе.]
\quizlink
