\upaper{6}{ВЕЧНЫЙ СЫН}
\uminitoc{ИНДИВИДУАЛЬНОСТЬ ВЕЧНОГО СЫНА}
\uminitoc{ПРИРОДА ВЕЧНОГО СЫНА}
\uminitoc{СЛУЖЕНИЕ ЛЮБВИ ОТЦА}
\uminitoc{АТРИБУТЫ ВЕЧНОГО СЫНА}
\uminitoc{ОГРАНИЧЕНИЯ ВЕЧНОГО СЫНА}
\uminitoc{РАЗУМ ДУХА}
\uminitoc{ЛИЧНОСТЬ ВЕЧНОГО СЫНА}
\uminitoc{ОСОЗНАНИЕ ВЕЧНОГО СЫНА}
\author{Божественный Советник}
\vs p006 0:1 Вечный Сын --- это совершенное и окончательное выражение <<первой>> личной и абсолютной концепции Всеобщего Отца. Соответственно, всякий раз, когда и как Отец лично и абсолютно ни выражал бы себя, он делает это через своего Вечного Сына, который всегда был, есть и будет живым и божественным Словом. И этот Вечный Сын пребывает в центре всех вещей в ассоциации с Вечным и Всеобщим Отцом, непосредственно окружая собой его личное присутствие.
\vs p006 0:2 Мы говорим о <<первой>> мысли Бога и ссылаемся на невозможное происхождение Вечного Сына во времени для того, чтобы получить доступ к мыслительным каналам человеческого интеллекта. Такие языковые искажения представляют собой наши максимальные усилия по достижению контакта\hyp{}компромисса со связанным временем разумом смертных созданий. В смысле последовательности Всеобщий Отец никогда не мог иметь первую мысль, а у Вечного Сына никогда не могло быть начала. Но мне было поручено изобразить реальности вечности для ограниченных временем разумов смертных с помощью таких символов мысли и обозначить отношения вечности с помощью временн\'ых концепций последовательности.
\vs p006 0:3 Вечный Сын --- это духовная персонализация всеобщей и бесконечной концепции божественной реальности Райского Отца --- безусловного духа и абсолютной личности. И тем самым Сын представляет собой божественное откровение личности Всеобщего Отца как создателя. Совершенная личность Сына показывает, что Отец действительно является вечным и всеобщим источником всех смыслов и ценностей духовного, волевого, целенаправленного и личностного.
\vs p006 0:4 Стремясь дать возможность конечному разуму времени сформировать некую последовательную концепцию отношений вечных и бесконечных существ Райской Троицы, мы прибегаем к такой концептуальной вольности, как ссылка на <<первую личную, всеобщую и бесконечную концепцию Отца>>. Я не могу передать человеческому разуму хоть сколько\hyp{}нибудь адекватную идею о вечных отношениях Божеств; поэтому я использую такие термины, которые хотя бы до некоторой степени донесут до конечного разума идею взаимоотношений этих вечных существ в последующие эпохи времени. Мы верим, что Сын произошёл от Отца; нас учат, что оба безусловно вечны. Поэтому очевидно, что ни одно создание времени никогда не сможет полностью постичь эту тайну Сына, происходящего от Отца, но при этом равно вечного самому Отцу.
\usection{ИНДИВИДУАЛЬНОСТЬ ВЕЧНОГО СЫНА}
\vs p006 1:1 Вечный Сын --- изначальный и единородный Сын Бога. Он --- Бог Сын, Второе Лицо Божества и сотворец всех вещей. Как Отец --- это Первый Великий Источник и Центр, так Вечный Сын --- Второй Великий Источник и Центр.
\vs p006 1:2 Вечный Сын --- это духовный центр и божественный администратор духовного правительства вселенной вселенных. Всеобщий Отец --- сначала создатель, затем регулятор; Вечный Сын --- сначала совместный создатель, а затем \bibemph{духовный администратор}. <<Бог есть дух>>, а Сын --- личностное откровение этого духа. Первый Источник и Центр --- Волевой Абсолют; Второй Источник и Центр --- Личностный Абсолют.
\vs p006 1:3 Всеобщий Отец никогда не действует как создатель лично, но только в союзе с Сыном или согласованно с действием Сына. Если бы автор Нового Завета говорил о Вечном Сыне, он изрёк бы истину, когда написал: <<В начале было Слово, и Слово было с Богом, и Слово было Бог. Всё было создано им, и без него ничего не было создано, что было создано>>.
\vs p006 1:4 Когда Сын Вечного Сына появился на Урантии, те, кто подружился с этим божественным существом в образе человека, упоминали о нём, как: <<О том, кто был от начала, кого мы слышали, кого мы видели своими глазами, за кем наблюдали, и кого осязали, даже как Слово жизни>>. И этот посвящённый Сын произошёл от Отца точно также, как Изначальный Сын, что представлено в одной из его земных молитв: <<И теперь, о мой Отец, прославь меня вместе с собой той славой, какую я имел с тобой до появления этого мира>>.
\vs p006 1:5 \pc Вечный Сын известен в различных вселенных под разными именами. В центральной вселенной его знают как Равный Источник, Сотворец и Партнёр Абсолют. На Уверсе, столице сверхвселенной, мы называем Сына Равным Духом\hyp{}Центром и Вечным Духом\hyp{}Администратором. На Салвингтоне, столице вашей локальной вселенной, этот Сын известен как Второй Вечный Источник и Центр. Мелхиседеки говорят о нём как о Сыне Сынов. В вашем мире, но не в вашей системе обитаемых миров, этого Изначального Сына путали с равным Сыном Создателем, Михаилом из Небадона, кто даровал себя смертным расам Урантии.
\vs p006 1:6 Хотя любого из Райских Сынов уместно называть Сынами Бога, мы обычно сохраняем название <<Вечный Сын>> за этим Изначальным Сыном, Вторым Источником и Центром, сотворцом Всеобщего Отца центральной вселенной могущества и совершенства и сотворцом всех других божественных Сынов, происходящих от бесконечных Божеств.
\usection{ПРИРОДА ВЕЧНОГО СЫНА}
\vs p006 2:1 Вечный Сын так же неизменен и бесконечно надёжен, как Всеобщий Отец. Он так же духовен, как Отец, такой же поистине безграничный дух. Тебе, с твоим скромным происхождением, Сын показался бы более личностным, поскольку он на один шаг доступнее тебе, чем Всеобщий Отец.
\vs p006 2:2 Вечный Сын --- это вечное Слово Бога. Он всецело подобен Отцу; по сути, Вечный Сын и \bibemph{есть} Бог Отец, личностно явленный вселенной вселенных. И так было, есть и будет всегда верно в отношении Вечного Сына и всех равных Сынов Создателей: <<Видевший Сына видел Отца>>.
\vs p006 2:3 По природе Сын полностью подобен духу\hyp{}Отцу. Когда мы поклоняемся Всеобщему Отцу, фактически мы одновременно поклоняемся Богу Сыну и Богу Духу. Бог Сын так же божественно реален и вечен по своей природе, как и Бог Отец.
\vs p006 2:4 Сын не только обладает всей бесконечной и трансцендентной праведностью Отца, но также отражает всю святость его характера. Сын разделяет совершенство Отца и совместно разделяет ответственность за помощь всем несовершенным созданиям в их духовных усилиях по достижению божественного совершенства.
\vs p006 2:5 Вечный Сын обладает всем характером божественности и атрибутами духовности Отца. Сын --- \bibemph{это} полнота абсолютности Бога в личности и духе, и эти качества Сын раскрывает в своём личном руководстве духовным правительством вселенной вселенных.
\vs p006 2:6 Бог --- действительно всеобщий дух; Бог есть дух; и эта духовная природа Отца сосредоточена и персонализована в Божестве Вечного Сына. В Сыне все духовные характеристики, по\hyp{}видимому, значительно усилены дифференциацией от всеобщности Первого Источника и Центра. И как Отец разделяет природу своего духа с Сыном, так они вместе столь же полностью и безоговорочно разделяют божественный дух с Совместным Вершителем --- Бесконечным Духом.
\vs p006 2:7 В любви к истине и в создании красоты Отец и Сын равны, за исключением того, что Сын, \bibemph{кажется}, больше посвящает себя реализации исключительно духовной красоты вселенских ценностей.
\vs p006 2:8 В божественной доброте я не вижу разницы между Отцом и Сыном. Отец любит своих вселенских детей как отец; Вечный Сын смотрит на всех созданий и как отец, и как брат.
\usection{СЛУЖЕНИЕ ЛЮБВИ ОТЦА}
\vs p006 3:1 Сын разделяет справедливость и праведность Троицы, но затмевает эти черты божественности бесконечным воплощением любви и милосердия Отца; Сын --- это откровение божественной любви вселенным. Как Бог есть любовь, так Сын --- милосердие. Сын не может любить больше, чем Отец, но он может проявить милосердие к созданиям ещё одним способом, ибо он не только первичный создатель, как Отец, но и Вечный Сын того же самого Отца, поэтому разделяет опыт сыновства со всеми другими сынами Всеобщего Отца.
\vs p006 3:2 Вечный Сын --- великий служитель милосердия всему творению. Милосердие --- это суть духовного характера Сына. Поручения Вечного Сына, передаваемые по духовным контурам Второго Источника и Центра, настроены в тональности милосердия.
\vs p006 3:3 Чтобы постичь любовь Вечного Сына, ты должен сначала осознать её божественный источник --- Отца, который \bibemph{есть} любовь, --- а затем созерцать развитие этого бесконечного чувства в обширном служении Бесконечного Духа и его почти безграничного множества личностей\hyp{}служителей.
\vs p006 3:4 Служение Вечного Сына посвящено откровению Бога любви вселенной вселенных. Этот божественный Сын не занят недостойной задачей, пытаясь убедить своего милостивого Отца полюбить свои смиренные создания и проявить милосердие к грешникам времени. Как неверно представлять Вечного Сына взывающим к Всеобщему Отцу проявить милосердие к своим скромным созданиям материальных миров пространства! Такие представления о Боге оскорбительны\fnst{Для Бога.} и абсурдны. Скорее тебе следует осознать, что всё милосердное служение Сынов Бога --- это прямое откровение сердца Отца, исполненного всеобщей любви и бесконечного сострадания. Любовь Отца --- настоящий и вечный источник милосердия Сына.
\vs p006 3:5 Бог есть любовь, Сын --- милосердие. Милосердие --- это проявленная любовь, любовь Отца в действии в лице его Вечного Сына. Любовь этого вселенского Сына также является вселенской. В том смысле, как любовь понимается на планете разнополых существ, любовь Бога более сравнима с любовью отца, тогда как любовь Вечного Сына больше похожа на чувство матери. Конечно, такие иллюстрации примитивны, но я использую их в надежде передать человеческому разуму мысль, что существует разница --- не в божественном содержании, а в качестве и способе выражения --- между любовью Отца и любовью Сына.
\usection{АТРИБУТЫ ВЕЧНОГО СЫНА}
\vs p006 4:1 Вечный Сын поддерживает уровень духа космической реальности; духовное могущество Сына абсолютно по отношению ко всем вселенским актуальностям. Он осуществляет совершенный контроль над взаимосвязью всей недифференцированной энергии духа и над всей актуализированной реальностью духа через свой абсолютный контроль гравитации духа. Весь чистый нефрагментированный дух и все духовные существа и ценности реагируют на бесконечную притягивающую силу первоначального Сына Рая. И если вечное будущее станет свидетелем появления неограниченной вселенной, гравитация духа и могущество духа Изначального Сына окажутся полностью адекватными для духовного контроля и эффективного управления таким безграничным творением.
\vs p006 4:2 \pc Сын всемогущ только в духовной сфере. В вечной структуре управления вселенной никогда не встречается расточительное и ненужное повторение функций; Божества не занимаются бесполезным дублированием вселенского служения.
\vs p006 4:3 \pc Вездесущность Изначального Сына составляет духовное единство вселенной вселенных. Духовная сплочённость всего творения основывается на повсеместном активном присутствии божественного духа Вечного Сына. Когда мы представляем себе духовное присутствие Отца, нам трудно отделить его в своём мышлении от духовного присутствия Вечного Сына. Дух Отца вечно пребывает в духе Сына.
\vs p006 4:4 Отец должен быть духовно вездесущим, но такая вездесущность представляется неотделимой от повсеместной активности духа Вечного Сына. Однако мы верим,что во всех ситуациях присутствия Отца\hyp{}Сына, имеющего двойную духовную природу, дух Сына согласован с духом Отца.
\vs p006 4:5 В своём контакте с личностью Отец действует в личностном контуре. В своём личном и обнаруживаемом контакте с духовным творением он предстаёт во фрагментах тотальности своего Божества, и эти фрагменты Отца выполняют единственную, уникальную и исключительную функцию, где и когда ни появлялись бы они во вселенных. Во всех подобных ситуациях дух Сына согласован с духовной функцией фрагментированного присутствия Всеобщего Отца.
\vs p006 4:6 Духовно Вечный Сын вездесущ. Дух Вечного Сына, безусловно, находится с тобой и вокруг тебя, но не внутри тебя и не составляет часть тебя подобно Таинственному Наставнику. Пребывающий в тебе фрагмент Отца настраивает человеческий разум на всё более божественные отношения, после чего такой восходящий разум становится всё более восприимчивым к духовной притягивающей силе всемогущего контура гравитации духа Второго Источника и Центра.
\vs p006 4:7 \pc Изначальный Сын обладает всесторонним и духовным самосознанием. В мудрости Сын полностью равен Отцу. В сфере знания и всеведения мы не можем обнаружить различие между Первым и Вторым Источниками; как и Отец, Сын знает всё; он никогда не удивляется никакому вселенскому событию; он знает конец от начала.
\vs p006 4:8 \pc Отец и Сын действительно знают число и местонахождение всех духов и одухотворённых существ во вселенной вселенных. Сын не только знает всё благодаря своему вездесущему духу, но Сын наравне с Отцом и Совместным Вершителем полностью осведомлён относительно обширной информации через систему отражения Верховного Существа, чей разум в любой момент времени знает обо всём, что происходит во всех мирах семи сверхвселенных. Есть и другие способы, благодаря которым Райский Сын всеведущ.
\vs p006 4:9 \pc Вечный Сын, как любящая, милосердная и заботливая духовная личность, полностью и бесконечно равен Всеобщему Отцу; и в то же время во всех своих милосердных и исполненных любви личных контактах с восходящими существами низших миров Вечный Сын такой же добрый и внимательный, такой же терпеливый и сострадающий, как и его Райские Сыны в локальных вселенных, которые часто дарят себя эволюционным мирам времени.
\vs p006 4:10 Нет необходимости и дальше распространяться об атрибутах Вечного Сына. Учитывая упомянутые исключения, необходимо только изучить духовные атрибуты Бога Отца, чтобы понять и правильно оценить атрибуты Бога Сына.
\usection{ОГРАНИЧЕНИЯ ВЕЧНОГО СЫНА}
\vs p006 5:1 Вечный Сын не функционирует лично в физических областях и не действует, кроме как через Совместного Вершителя, на уровнях служения разума сотворённым существам. Во всём остальном эти условия никоим образом не ограничивают Вечного Сына в полном и свободном проявлении всех божественных атрибутов \bibemph{духовного} всеведения, вездесущности и всемогущества.
\vs p006 5:2 Вечный Сын лично не пронизывает потенциалы духа, свойственные бесконечности Божества Абсолюта, но, по мере актуализации этих потенциалов, они попадают под всемогущий контроль контура гравитации духа Сына.
\vs p006 5:3 Личность --- это исключительный дар Всеобщего Отца. Вечный Сын получает личность от Отца, но без Отца --- личностью не наделяет. Сын порождает огромное воинство духов, но такие производные не являются личностями. Когда Сын создаёт личность, он делает это в союзе с Отцом или с Совместным Вершителем, который в таких отношениях может действовать от имени Отца. Таким образом, Вечный Сын --- совместный создатель личностей, но сам он не наделяет личностью никакое существо, и никогда сам не создаёт личностных существ. Однако это ограничение действия не лишает Сына способности создавать любые или все типы неличностной реальности.
\vs p006 5:4 Вечный Сын ограничен в передаче прерогатив создателя. Отец, увековечивая Изначального Сына, даровал ему власть и привилегию впоследствии соединяться с Отцом в божественном акте создания дополнительных Сынов, обладающих созидательными атрибутами, и это происходило в прошлом и происходит сейчас. Но когда эти равные Сыны произведены, прерогативы созидания, по\hyp{}видимому, далее не передаются. Вечный Сын передаёт полномочия созидания только первой или непосредственной персонализации. Поэтому, когда Отец и Сын соединяются, чтобы персонализовать Сына Создателя, они достигают своей цели; но Сын Создатель, появившийся таким образом, уже не в состоянии передавать или делегировать прерогативы созидания различным категориям Сынов, которых он может впоследствии создать, несмотря на то, что в высших Сынах локальной вселенной всё же проявляется очень ограниченное отражение созидательных атрибутов Сына Создателя.
\vs p006 5:5 Вечный Сын, как бесконечное и исключительно личностное существо, не может фрагментировать свою природу, не может распределять и передавать индивидуализированные частицы своей личности другим сущностям или личностям, как это делают Всеобщий Отец и Бесконечный Дух. Но Сын может и действительно дарит себя в качестве неограниченного духа, чтобы омывать всё творение и непрерывно притягивать к себе всех духов\hyp{}личностей и духовные реальности.
\vs p006 5:6 Всегда помни, что Вечный Сын --- это личностное изображение духа\hyp{}Отца всему творению. В смысле Божества Сын личностен и только личностен; такая божественная и абсолютная личность не может быть дезинтегрирована\fnst{То есть <<разложена на части>>.} или фрагментирована. Бог Отец и Бог Дух истинно личностны, но они вдобавок к тому, что представляют собой личности Божества, также являются и всем остальным.
\vs p006 5:7 Хотя Вечный Сын не может лично участвовать в дарении Настройщиков Мыслей, в вечном прошлом он заседал в совете со Всеобщим Отцом, одобряя план и обещая бесконечное сотрудничество, когда Отец, планируя дар Настройщиков Мыслей, предложил Сыну: <<Создадим смертного человека по нашему образу>>. И как фрагмент духа Отца пребывает в тебе, так и присутствие духа Сына окружает тебя, в то время как оба они трудятся беспрестанно, как один, ради твоего духовного развития.
\usection{РАЗУМ ДУХА}
\vs p006 6:1 Вечный Сын есть дух и имеет разум, но не такой дух или разум, которые может понять смертный разум. Смертный человек воспринимает разум на конечном, космическом, материальном и личностном уровнях. Человек также наблюдает феномены разума в живых организмах, функционирующих на субличностном (животном) уровне, но ему трудно понять природу разума, когда он связан со сверхматериальными существами и является частью личностей исключительно духовных. Однако разум следует определять иначе, когда он относится к уровню существования духа и используется для обозначения функций интеллекта в духе. Такой тип разума, непосредственно связанный с духом, несравним ни с тем разумом, который координирует дух и материю, ни с тем разумом, который связан только с материей.
\vs p006 6:2 Дух всегда сознателен, рассудителен и обладает различными фазами индивидуальности. Без разума в какой\hyp{}либо фазе не было бы духовного сознания в братстве духов\hyp{}существ. Эквивалент разума --- способность знать и быть познаваемым --- врождённый для Божества. Божество может быть личностным, доличностным, сверхличностным или безличностным, но Божество никогда не бывает неразумным, то есть никогда не лишено способности хотя бы общаться с подобными себе сущностями, существами или личностями.
\vs p006 6:3 Разум Вечного Сына подобен разуму Отца, но отличен от любого другого разума во вселенной, а вместе с разумом Отца является прародителем разнообразных и далеко распространившихся разумов Совместного Вершителя. Разум Отца и Сына, тот интеллект, который является прародителем абсолютного разума Третьего Источника и Центра, возможно, лучше всего проиллюстрирован в предразуме Настройщиков Мыслей, ибо, хотя эти фрагменты Отца находятся полностью за пределами контуров разума Совместного Вершителя, у них есть некоторая форма предразума; они познают, как и познаваемы; они обладают эквивалентом человеческого мышления.
\vs p006 6:4 Вечный Сын полностью духовен; человек --- почти полностью материален; поэтому многое из того, что относится к духу\hyp{}личности Вечного Сына, к его семи духовным сферам, окружающим Рай, и к природе неличностных творений Райского Сына, должно будет дожидаться достижения тобой статуса духа, следующего за завершением тобой моронтийного восхождения в локальной вселенной Небадон. И затем, по мере твоего продвижения через сверхвселенную и далее к Хавоне, многие из этих скрытых в духе тайн прояснятся, когда ты начнёшь наделяться <<разумом духа>> --- духовной проницательностью.
\usection{ЛИЧНОСТЬ ВЕЧНОГО СЫНА}
\vs p006 7:1 Вечный Сын --- это та бесконечная личность, от чьих безусловных личностных оков Всеобщий Отец освободился с помощью метода тринитизации и, благодаря которой, он с тех пор продолжает с бесконечной щедростью дарить себя своей постоянно расширяющейся вселенной Создателей и созданий. Сын --- \bibemph{абсолютная личность;} Бог --- \bibemph{отец\hyp{}личность} --- источник личности, даритель личности, причина личности. Каждое личностное существо получает личность от Всеобщего Отца, точно так же, как Изначальный Сын вечно получает свою личность от Райского Отца.
\vs p006 7:2 Личность Райского Сына абсолютна и чисто духовна, и эта абсолютная личность является также божественным и вечным образцом, сначала, дара Отцом личности Совместному Вершителю и, впоследствии, его дара личности мириадам своих созданий по всей обширной вселенной.
\vs p006 7:3 Вечный Сын --- истинно милосердный служитель, божественный дух, духовная сила и реальная личность. Сын --- это духовная и личностная природа Бога, раскрытая вселенным --- сумма и содержание Первого Источника и Центра, освобожденная от всего неличностного, внебожественного, недуховного и чисто потенциального. Но невозможно передать человеческому разуму в словах картину красоты и величия небесной личности Вечного Сына. Всё, что имеет тенденцию затемнять Всеобщего Отца, действует почти с равным влиянием, препятствуя концептуальному осознанию Вечного Сына. Тебе придётся дождаться достижения тобой Рая, и тогда ты поймёшь, почему я не смог описать характер этой абсолютной личности для понимания конечного разума.
\usection{ОСОЗНАНИЕ ВЕЧНОГО СЫНА}
\vs p006 8:1 Что касается индивидуальности, природы и других атрибутов личности, Вечный Сын является полностью равным, совершенным дополнением и вечным партнёром Всеобщего Отца. В том же смысле, в каком Бог --- Всеобщий Отец, Сын --- это Всеобщая Мать. И все мы, и высшие и низшие, образуем их вселенскую семью.
\vs p006 8:2 Чтобы оценить характер Сына, тебе следует изучить откровение божественного характера Отца; они навсегда и нераздельно едины. Как божественные личности, они практически неразличимы для низших видов интеллекта. Их не так уж сложно различать тем, чьё происхождение лежит в созидательных актах самих Божеств. Существа, рождённые в центральной вселенной и в Раю, воспринимают Отца и Сына не только как личностное единство всеобщего контроля, но и как две отдельные личности, действующие в определенных областях управления вселенной.
\vs p006 8:3 В качестве личностей ты можешь воспринимать Всеобщего Отца и Вечного Сына как отдельных индивидуумов, ибо таковыми они и являются; но в управлении вселенными они настолько переплетены и взаимосвязаны, что различить их возможно не всегда. Когда в делах вселенных Отец и Сын встречаются в запутанных взаимодействиях, не всегда полезно пытаться разделить их действия; просто вспомни, что Бог --- это инициирующая мысль, а Сын --- выразительное слово. В каждой локальной вселенной эта неразделимость персонализуется в божественности Сына Создателя, который представляет как Отца, так и Сына для созданий 10\,000\,000 обитаемых миров.
\vs p006 8:4 Вечный Сын бесконечен, но он достижим через личности его Райских Сынов и благодаря терпеливому служению Бесконечного Духа. Без дара служения Райских Сынов и любящей помощи созданий Бесконечного Духа, существа материального происхождения вряд ли могли бы надеяться достичь Вечного Сына. И в равной степени верно следующее: с помощью и под руководством этих небесных сил Богосознающий смертный определённо достигнет Рая и однажды встанет в личном присутствии этого величественного Сына Сынов.
\vs p006 8:5 \pc Несмотря на то, что образцом достижения смертной личности является Вечный Сын, тебе легче усвоить реальность Отца и Духа, потому что Отец --- непосредственный даритель твоей человеческой личности, а Бесконечный Дух --- абсолютный источник твоего смертного разума. Но по мере твоего восхождения по Райскому пути духовного развития, личность Вечного Сына будет становиться для тебя всё более реальной, и реальность его бесконечно духовного разума станет более различимой для твоего постепенно одухотворяющегося разума.
\vs p006 8:6 Никогда не может концепция Вечного Сына засиять ярко в твоём материальном или последующем моронтийном разуме; до тех пор, пока ты не одухотворишься и не начнёшь своё духовное восхождение, понимание личности Вечного Сына не сравняется с яркостью твоего представления о личности Сына Создателя Райского происхождения, который, лично и как личность, однажды воплотился и жил на Урантии как человек среди людей.
\vs p006 8:7 На протяжении всего твоего опыта в локальной вселенной Сын Создатель, чья личность доступна человеческому пониманию, должен компенсировать твою неспособность постичь в полной мере значение более исключительно духовного, но, тем не менее, личностного, Вечного Сына Рая. По мере твоего продвижения через Орвонтон и Хавону, когда позади останутся яркий образ и сокровенные воспоминания о Сыне Создателе твоей локальной вселенной, прохождение этого материального и моронтийного опыта будет компенсироваться всё более расширяющимися концепциями и всё более углубляющимся пониманием Вечного Сына Рая, чья реальность и близость будут постоянно возрастать по мере твоего приближения к Раю.
\vs p006 8:8 \pc Вечный Сын --- это великая и славная личность. Хотя смертный и материальный разум не в силах осознать актуальность личности такого бесконечного существа, не сомневайся, --- он личность. Я знаю, о чём говорю. Почти бессчётное число раз я находился в божественном присутствии этого Вечного Сына и затем отправлялся во вселенную, чтобы исполнить его милосердное распоряжение.
\vsetoff
\vs p006 8:9 [Вдохновлено Божественным Советником, назначенным сформулировать это изложение, описывающее Вечного Сына Рая.
\quizlink
