\upaper{20}{РАЙСКИЕ СЫНЫ БОГА}
\uminitoc{НИСХОДЯЩИЕ СЫНЫ БОГА}
\uminitoc{СЫНЫ ПОВЕЛИТЕЛИ}
\uminitoc{СУДЕБНЫЕ ДЕЙСТВИЯ}
\uminitoc{МИССИИ ПОВЕЛИТЕЛЯ}
\uminitoc{ПОСВЯЩЕНИЕ РАЙСКИХ СЫНОВ БОГА}
\uminitoc{ПОСВЯЩЕНИЯ В ОБЛИКЕ СМЕРТНЫХ}
\uminitoc{ТРОИЧНЫЕ СЫНЫ УЧИТЕЛЯ}
\uminitoc{СЛУЖЕНИЕ ДАЙНАЛОВ В ЛОКАЛЬНЫХ ВСЕЛЕННЫХ}
\uminitoc{ПЛАНЕТАРНОЕ СЛУЖЕНИЕ ДАЙНАЛОВ}
\uminitoc{ОБЪЕДИНЁННОЕ СЛУЖЕНИЕ РАЙСКИХ СЫНОВ}
\author{Совершенствователь Мудрости}
\vs p020 0:1 По своим функциям в сверхвселенной Орвонтон, Сыны Бога подразделяются на три основные группы:
\vs p020 0:2 \li{1.}Нисходящие Сыны Бога.
\vs p020 0:3 \li{2.}Восходящие Сыны Бога.
\vs p020 0:4 \li{3.}Тринитизованные Сыны Бога.
\vs p020 0:5 \pc К нисходящим категориям сыновства относятся личности непосредственно и божественно созданные. Восходящие сыны, такие как смертные создания, обретают этот статус через эмпирическое участие в творческом процессе, известном как эволюция. Тринитизованные Сыны~--- это группа смешанного происхождения, которая включает все существа, объятые Райской Троицей, даже не имеющие непосредственно Троичного происхождения.
\usection{НИСХОДЯЩИЕ СЫНЫ БОГА}
\vs p020 1:1 Все нисходящие Сыны Бога имеют высокое и божественное происхождение. Они посвящены нисходящему служению на мирах и системах времени и пространства, чтобы способствовать прогрессу в Райском восхождении низших созданий эволюционного происхождения~--- восходящих сынов Бога. В этих повествованиях из многочисленных категорий нисходящих Сынов будут описаны семь. Те Сыны, которые происходят от Божеств на центральном Острове Света и Жизни, называются \bibemph{Райскими Сынами Бога} и охватывают следующие три категории\fnst{Следующие семь категорий распадаются на три группы: $7=3+3+1$. Название первого члена первой и второй групп выбрано из широко известных имён из древнееврейского текста Ветхого Завета: \bibemph{Михаил} от \textheb{מִיכָאֵל} \bibemph{ми ка эль} <<кто подобен Богу?>>, и \bibemph{Мелхиседек} от \textheb{מַלְכִּי־צֶדֶק} \bibemph{малки цедек} <<царь праведности>> (см. ниже). Остальные два имени каждой группы сконструированы из двух компонент: окончаний -el/-al и -dek, указывающих на родство с первым элементом группы, и корня, несущего основное лексическое значение. Так, \bibemph{Авонал} происходит, несомненно, от древнееврейского \textheb{עָוֺן} \bibemph{авон} <<беззаконие>>, <<наказание за беззаконие>> или <<бедствия, проистекающие от беззаконий>>, а \bibemph{Дайнал} от древнееврейского \textheb{דֵעָה} \bibemph{деа} <<знание>>, особенно <<знание \bibemph{о} Боге>>. Происхождение \bibemph{Ворондадек} от комбинаций саксонских корней vor- $+$ on- $+$ da- $=$ <<в точности такой, как Отец>> и \bibemph{Ланонандек} от lan- $+$ on- $+$ an- $=$ <<не совсем такой, как [Отец]>> весьма сомнительно.}:
\vs p020 1:2 \li{1.}Сыны Создатели~--- Михаилы.
\vs p020 1:3 \li{2.}Сыны Повелители\fnst{По сути, Авоналы~--- это божественные \bibemph{диктаторы}, несущие мирам справедливые наказания за многочисленные беззакония.}~--- Авоналы.
\vs p020 1:4 \li{3.}Троичные Сыны Учителя~--- Дайналы.
\vs p020 1:5 \pc Остальные четыре категории нисходящего сыновства известны как \bibemph{Сыны Бога Локальных Вселенных:}
\vs p020 1:6 \li{4.}Сыны Мелхис\'едеки\fnst{В древнееврейском тексте Ветхого Завета слово \textheb{מַלְכִּי־צֶדֶק} встречается лишь дважды (см.\,\bibref[93:9.9]{p093 9:9}): Бытие~14:18 и Псалом~110:4 (109:4 в русской Библии, использующей нумерацию Псалмов согласно Септуагинте). В древнееврейском языке добиблейских времён, также как в классическом арабском, существовали три окончания падежей: -u (именительный), -i (родительный) и -a (винительный). В имени собственном \textheb{מַלְכִּי־צֶדֶק}, означающем <<царь праведности>> (\bibemph{малки\hyp{}цедек} $\Rightarrow$ \bibemph{Мелхиседек}), сохранилось это древнее окончание родительного падежа -i, чем объясняется необычная конструктивная форма слова <<царь>> \textheb{מַלְכִּי} \bibemph{малки} вместо логически следующей из правил, но неверной \textheb{מֶלֶךְ} \bibemph{мелех}.}.
\vs p020 1:7 \li{5.}Сыны Воронд\'адеки.
\vs p020 1:8 \li{6.}Сыны Ланон\'андеки.
\vs p020 1:9 \li{7.}Носители Жизни.
\vs p020 1:10 \pc Мелхиседеки представляют собой совместное потомство Сына Создателя локальной вселенной, Созидательного Духа и Отца Мелхиседека. И Ворондадеки, и Ланонандеки были созданы Сыном Создателем и его партнёром~--- Созидательным Духом. Ворондадеки наиболее известны как Всевышние~--- Отцы Созвездий; Ланонандеки~--- как Властелины Систем и Планетарные Принцы. Тройственная категория Носителей Жизни создаётся Сыном Создателем и Созидательным Духом в контакте с одним из трёх От Века Древних соответствующей сверхвселенной. Но природа и деятельность этих Сынов Бога Локальных Вселенных точнее отражены в тех документах, которые относятся к локальным творениям.
\vs p020 1:11 \pc Райские Сыны Бога имеют тройственное происхождение: первичные, или Сыны Создатели, создаются Всеобщим Отцом и Вечным Сыном; вторичные, или Сыны Повелители,~--- дети Вечного Сына и Бесконечного Духа; Троичные Сыны Учителя~--- это потомство Отца, Сына и Духа. С точки зрения служения, поклонения и молитвы Райские Сыны едины; их дух един, и их работа идентична по качеству и завершённости.
\vs p020 1:12 Как Райские категории <<От Века>> проявили себя божественными администраторами, так категории Райских Сынов раскрыли себя как божественные служители~--- создатели, помощники, дарители, судьи, учителя и открыватели истины. Они охватывают вселенную вселенных от берегов вечного Острова до обитаемых миров времени и пространства, исполняя различные виды служения в центральной и сверхвселенных, не раскрытые в этих повествованиях. Они по\hyp{}разному организованы в зависимости от характера и места их служения, но в локальной вселенной и Сыны Повелители, и Сыны Учителя служат под руководством Сына Создателя, возглавляющего это владение.
\vs p020 1:13 По\hyp{}видимому, Сыны Создатели обладают сосредоточенным в их личности духовным даром, который они контролируют и могут посвящать, что и сделал ваш собственный Сын Создатель, когда излил свой дух на всю смертную плоть на Урантии. Каждый Сын Создатель наделён этой духовной силой притяжения в своём собственном владении; он лично осознаёт любое действие и эмоцию каждого нисходящего Сына Бога, служащего в его вселенной. В этом выражается божественное отражение~--- повторение в локальной вселенной той абсолютной духовной силы притяжения Вечного Сына, которая позволяет ему дотянуться до всех своих Райских Сынов, чтобы установить и поддерживать контакт с ними, где бы они не находились во всей вселенной вселенных.
\vs p020 1:14 Райские Сыны Создатели действуют не только как Сыны в нисходящих видах служения и посвящения, но по завершении пути своего посвящения каждый становится вселенским Отцом своего собственного творения, в то время как другие Сыны Бога продолжают служение посвящения и духовного возвышения, предназначенного привести планеты, одну за другой, к добровольному признанию исполненного любви правления Всеобщего Отца, достигая кульминации в посвящении созданий воле Райского Отца и в планетарной верности вселенскому владычеству его Сына Создателя.
\vs p020 1:15 В семикратном\fnst{То есть <<прошедшем семь посвящений>>.} Сыне Создателе Создатель и создание навечно объединены союзом понимания, сочувствия и милосердной связи. Вся категория Михаилов~--- Сынов Создателей~--- настолько уникальна, что природа и род их деятельности будут рассмотрены в следующем документе этой серии, в то время как данное повествование будет в основном касаться двух остальных категорий Райского сыновства: Сынов Повелителей и Троичных Сынов Учителей.
\usection{СЫНЫ ПОВЕЛИТЕЛИ}
\vs p020 2:1 Каждый раз, когда оригинальная и абсолютная концепция существа, сформулированная Вечным Сыном, объединяется с новым и божественным идеалом полного любви служения, задуманного Бесконечным Духом, производится новый и оригинальный Сын Бога~--- Райский Сын Повелитель. Эти Сыны составляют категорию Авоналов, в отличие от категории Михаилов, Сынов Создателей. Хотя они и не создатели в личностном смысле, во всей своей деятельности они тесно связаны с Михаилами. Авоналы являются планетарными служителями и судьями, местными судьями время\hyp{}пространственных сфер~--- всех рас, для всех миров и во всех вселенных.
\vs p020 2:2 У нас есть основания полагать, что общее число Сынов Повелителей в большой вселенной составляет около одного миллиарда. Это самоуправляющаяся категория, руководимая своим верховным советом на Рае, составленным из опытных Авоналов из служб всех вселенных. Но по назначении и прибытии в локальную вселенную они служат под руководством Сына Создателя этой области.
\vs p020 2:3 Авоналы~--- Райские Сыны служения и посвящения отдельным планетам локальных вселенных. Поскольку каждый Сын Авонал~--- исключительная личность, и среди них нет двух одинаковых, их работа индивидуально уникальна в сферах их пребывания, где они зачастую воплощаются в облике смертных, а иногда рождаются от земных матерей на эволюционных мирах.
\vs p020 2:4 \pc В дополнение к своему служению на высших административных уровнях Авоналы выполняют тройную функцию на обитаемых мирах:
\vs p020 2:5 \li{1.}\bibemph{Судебные действия}. Они действуют в конце планетарных диспенсаций\fnst{Планетарных <<Судных периодов>>.}. Со временем десятки или даже сотни подобных миссий могут осуществляться на каждом отдельном мире, и они могут посещать те же самые или другие миры бессчётное количество раз в качестве завершителей диспенсаций, освободителей спящих выживших созданий.
\vs p020 2:6 \li{2.}\bibemph{Миссии повелителя}. Планетарное посещение такого типа обычно происходит до прибытия Сына посвящения. В такой миссии Авонал появляется в облике зрелого создания данного мира с помощью техники воплощения, не связанной с рождением смертных. После этого первого и обычного визита в качестве повелителя Авоналы могут неоднократно служить повелителем на одной и той же планете как до, так и после появления Сына посвящения. В течение этих дополнительных миссий повелителя Авонал может появиться или не появиться в материальной и видимой форме, но ни в одной из них он не рождается в мир беспомощным младенцем.
\vs p020 2:7 \li{3.}\bibemph{Миссии посвящения}. Все Сыны Авоналы хотя бы однажды посвящают себя одной из смертных рас какого\hyp{}нибудь эволюционного мира. Судебные визиты многочисленны, миссии повелителя могут повторяться, но на каждой планете появляется только один Сын посвящения. Авоналы посвящения рождаются от женщины, как на Урантии воплотился Михаил Небадона.
\vs p020 2:8 \pc Нет предела для числа миссий повелителя и миссий посвящения, в которых могут служить Сыны Авоналы, но обычно, после семикратного повторения опыта, они уступают своё место тем, у кого меньше опыта в таком виде служения. Эти Сыны, получившие опыт многократных пришествий, назначаются в высший личный совет Сына Создателя, таким образом становясь участниками управления делами вселенной.
\vs p020 2:9 Во всей своей работе на обитаемых мирах и для них Сыны Повелители получают помощь от двух категорий существ локальной вселенной: Мелхиседеков и архангелов, а в миссиях посвящения их также сопровождают Блистательные Вечерние Звёзды, тоже берущие начало в локальных творениях. В любых планетарных начинаниях вторичные Райские Сыны, Авоналы, поддерживаются всем могуществом и властью первичного Райского Сына, Сына Создателя локальной вселенной их служения. По сути, их работа на обитаемых сферах столь же эффективна и приемлема, как и служение самог\'о Сына Создателя на таких мирах обитания смертных.
\usection{СУДЕБНЫЕ ДЕЙСТВИЯ}
\vs p020 3:1 Авоналы известны как Сыны Повелители потому, что они~--- высшие судьи\fnst{Игра слов в английском: magisterial \ldots\ magistrates.} миров, выносящие решения относительно последовательных диспенсаций миров времени. Они руководят пробуждением спящих выживших, заседают в суде над данным миром, завершают диспенсацию приостановленного правосудия, исполняют мандаты, связанные с эпохой испытательного милосердия, дают пространственным созданиям планетарного служения задания, связанные с задачами новой диспенсации, и после выполнения миссии возвращаются в столицу своей локальной вселенной.
\vs p020 3:2 Вынося судебное решение о судьбе эпохи, Авоналы определяют участь эволюционных рас, но, несмотря на то что они могут выносить приговоры об индивидуальном уничтожении личностных созданий, они не приводят в исполнение такие приговоры. Вердикты такого рода исполняются только властями сверхвселенной.
\vs p020 3:3 Прибытие Райского Авонала на эволюционный мир с целью завершения диспенсации и открытия новой эры планетарного развития не обязательно связано с миссией повелителя или миссией посвящения. Миссии повелителя иногда, а миссии посвящения всегда, являются воплощением; то есть в таких назначениях Авоналы служат на планете в материальной форме~--- буквально. Другие их посещения представляют <<технический характер>>, и в этом качестве Авонал не воплощается для планетарного служения. Если Сын Повелитель прибывает только для вынесения решения о завершении диспенсации, он посещает планету как духовное существо, невидимое материальным созданиям сферы. Такие технические посещения происходят неоднократно в течение долгой истории обитаемого мира.
\vs p020 3:4 Сыны Авоналы могут действовать в качестве планетарных судей до приобретения как опыта повелителя, так и опыта посвящения. Тем не менее в обеих этих миссиях воплотившийся Сын будет судить проходящую планетарную эпоху; так же действует и Сын Создатель, когда он воплощается в миссии посвящения в облике смертной плоти. Когда Райский Сын посещает эволюционный мир и становится похожим на одного из его обитателей, его присутствие завершает диспенсацию и вводит в силу судебное решение о данном мире.
\usection{МИССИИ ПОВЕЛИТЕЛЯ}
\vs p020 4:1 До появления на планете Сына посвящения, обитаемый мир обычно посещает Райский Авонал с миссией повелителя. Если это первоначальный визит повелителя, Авонал всегда воплощается как материальное существо. Он появляется на планете своего назначения как полноценный мужчина из смертных рас, существо, полностью видимое смертным созданиям своего времени и поколения, и поддерживает с ними физический контакт. Связь Сына Авонала с локальными и всеобщими духовными силами сохраняется полной и непрерывной на протяжении всего периода воплощения.
\vs p020 4:2 Планета может переживать много посещений повелителя как до, так и после появления Сына посвящения. Её может многократно посещать один и тот же или разные Авоналы, действующие в качестве диспенсационных судей, но такие технические миссии вынесения решения не являются ни посвящением, ни миссиями повелителя, и Авоналы никогда не воплощаются в таких случаях. Даже когда планета благословляется повторными миссиями повелителя, Авоналы не всегда прибегают к смертному воплощению; и когда они действительно служат в подобии смертной плоти, они всегда появляются как взрослые существа данного мира; они не рождаются от женщины.
\vs p020 4:3 Воплощаясь для миссии посвящения либо повелителя, Райские Сыны имеют опытных Настройщиков, и эти Настройщики различны для каждого воплощения. Настройщики, присутствующие в разумах воплощённых Сынов Бога, не имеют никакой надежды на обретение личности через слияние с человеко\hyp{}божественными существами их пребывания, но они часто персонализируются прямым актом Всеобщего Отца. Такие Настройщики образуют верховный совет Божеграда для управления, идентификации и отправки Таинственных Мониторов на обитаемые миры. Они также принимают и утверждают Настройщиков по их возвращении в <<лоно Отца>> после посмертного распада их земных сосудов. Таким образом верные Настройщики судей мира становятся возвышенными руководителями себе подобных.
\vs p020 4:4 \pc Урантия никогда не принимала Сына Авонала с миссией повелителя. Если бы Урантия следовала общему плану обитаемых миров, она была бы благословлена миссией повелителя где\hyp{}то между днями Адама и посвящением Христа Михаила. Но обычная последовательность посещений Райских Сынов на вашей планете была полностью расстроена появлением вашего Сына Создателя во время его заключительного посвящения 1900 лет назад\fnst{На момент работы над этим переводом~--- 2027 лет назад.}.
\vs p020 4:5 Урантию ещё может посетить Авонал, направленный для воплощения с миссией повелителя, но что касается будущего появления Райских Сынов, даже <<ангелы на небесах не знают времени или способа таких посещений>>\fnst{Матфея 24:36.}, ибо мир посвящения Михаила становится предметом индивидуальной и личной опеки Сына Властелина и, как таковой, целиком подчиняется его собственным планам и постановлениям. А с вашим миром всё ещё больше осложняется обещанием Михаила вернуться. Несмотря на недоразумения касательно пребывания Михаила Небадона на Урантии, одно, безусловно, достоверно~--- его обещание вернуться на ваш мир. Ввиду этой перспективы только время может раскрыть будущий порядок посещений Урантии Райскими Сынами Бога.
\usection{ПОСВЯЩЕНИЕ РАЙСКИХ СЫНОВ БОГА}
\vs p020 5:1 Вечный Сын~--- это вечное Слово Бога. Вечный Сын является совершенным выражением <<первой>> абсолютной и бесконечной мысли своего вечного Отца. Когда личностное воспроизведение или божественное продолжение этого Изначального Сына приступает к миссии посвящения в смертном воплощении, становится буквально истинным, что божественное <<Слово стало плотью>> и что Слово, таким образом, обитает среди низших существ животного происхождения.
\vs p020 5:2 На Урантии широко распространено мнение, что цель посвящения Сына~--- каким\hyp{}то образом повлиять на отношение Всеобщего Отца. Но ваша просвещённость должна подсказать вам, что это не так. Посвящения Сынов Авоналов и Михаилов являются необходимой частью эмпирического процесса, призванного сделать этих Сынов надёжными и сострадательными судьями и правителями народов и планет времени и пространства. Путь семикратного посвящения~--- верховная цель всех Райских Сынов Создателей. И все Сыны Повелители движимы тем же самым духом служения, который столь исчерпывающе характеризует первичных Сынов Создателей и Вечного Сына Рая.
\vs p020 5:3 Представитель какой\hyp{}нибудь категории Райских Сынов должен принести дар посвящения каждому миру, населённому смертными, чтобы Настройщики Мыслей могли поселиться в разумах всех нормальных человеческих существ данной сферы, ибо Настройщики не приходят ко \bibemph{всем} подлинно человеческим существам, пока Дух Истины не будет излит на всю плоть; а ниспослание Духа Истины зависит от возвращения в столицу вселенной Райского Сына, успешно завершившего свою миссию посвящения в эволюционирующем мире в облике смертного.
\vs p020 5:4 В течение долгой истории обитаемой планеты происходит множество диспенсационных судов и, возможно, более чем одна миссия повелителя, но Сын посвящения обычно служит на этой сфере только один раз. Для каждого обитаемого мира необходимо, чтобы только один Сын посвящения прожил полную жизнь в качестве смертного от рождения до смерти. Рано или поздно, независимо от духовного статуса, каждому населённому смертными миру предназначено принять Сына Повелителя с миссией посвящения, за исключением одной планеты в каждой локальной вселенной, которую Сын Создатель выбирает для своего посвящения в облике смертного.
\vs p020 5:5 \pc Понимая больше о Сынах посвящения, вы осознаёте, почему в истории Небадона проявляется такой интерес к Урантии. Ваша маленькая и незначительная планета представляет интерес для локальной вселенной просто потому, что это смертный родной мир Иисуса из Назарета. Она была сценой последнего и триумфального посвящения вашего Сына Создателя, ареной, на которой Михаил завоевал верховный личный суверенитет во вселенной Небадон.
\vs p020 5:6 В столице своей локальной вселенной Сын Создатель, особенно после завершения своего собственного посвящения в облике смертного, проводит значительную часть времени, консультируя и наставляя коллегию подобных себе Сынов, Сынов Повелителей и других. С любовью и преданностью, с чутким состраданием и нежным вниманием эти Сыны Повелители посвящают себя мирам пространства. И такие планетарные посвящения ни в чём не уступают посвящениям Михаилов в облике смертных. Это правда, что сферой своего последнего приключения на пути обретения опыта созданий ваш Сын Создатель выбрал ту, с которой случились необыкновенные несчастья. Но ни одна планета никогда не могла оказаться в таком состоянии, чтобы для её духовного восстановления потребовалось посвящение Сына Создателя. В равной степени было бы достаточно любого Сына из группы посвящения, ибо во всей своей работе на мирах локальной вселенной Сыны Повелители столь же божественно эффективны и мудры, как и их Райский брат, Сын Создатель.
\vs p020 5:7 \pc Хотя воплощения во время посвящений этих Райских Сынов всегда сопряжены с опасностью, я ещё не видел записи о неудаче или невыполнении обязательств Сыном Повелителем или Сыном Создателем в миссии посвящения. Оба они по происхождению слишком близки к абсолютному совершенству, чтобы потерпеть неудачу. Они действительно подвергают себя риску, становясь подобными смертным созданиям из плоти и крови, и тем самым приобретают уникальный опыт создания, но, по моим наблюдениям, они всегда добиваются успеха. Они неизменно достигают цели миссии посвящения. Рассказ об их посвящениях и планетарном служении по всему Небадону составляет наиболее благородную и захватывающую главу в истории вашей локальной вселенной.
\usection{ПОСВЯЩЕНИЯ В ОБЛИКЕ СМЕРТНЫХ}
\vs p020 6:1 Метод, посредством которого Райский Сын становится готовым к воплощению в облике смертного в качестве Сына посвящения и оказывается в утробе матери на планете посвящения, является всеобщей тайной; и любая попытка обнаружить принцип действия этого метода Сынограда обречена на неудачу. Пусть возвышенное знание смертной жизни Иисуса из Назарета глубоко проникнет в ваши души, но не тратьте понапрасну силы на бесполезные размышления о том, как произошло это таинственное воплощение Михаила Небадона. Давайте же радоваться знанию и уверенности в том, что такие достижения возможны для божественной природы, и не будем тратить время на тщетные догадки о методе, используемом божественной мудростью для осуществления таких явлений.
\vs p020 6:2 \pc Для миссии посвящения в облике смертного Райский Сын всегда рождается от женщины и растёт как ребёнок мужского пола данного мира, как это произошло с Иисусом на Урантии. Все Сыны высшего служения проходят такой же путь, как и человек, от младенчества через юность к зрелости. Во всех отношениях они уподобляются смертным той расы, в которой родились. Они обращаются с прошениями к Отцу, как и дети тех сфер, где они служат. С материальной точки зрения эти человеко\hyp{}божественные Сыны живут обычной жизнью с одним только исключением: они не производят потомства на мирах их пребывания, это общее ограничение для всех категорий Райских Сынов посвящения.
\vs p020 6:3 Как Иисус трудился на вашем мире в качестве сына плотника, так и другие Райские Сыны трудятся в различных качествах на планетах своего посвящения. Вряд ли найдётся профессия, которой бы не занимался кто\hyp{}либо из райских Сынов в своём посвящении на какой\hyp{}нибудь из эволюционных планет времени.
\vs p020 6:4 Когда Сын посвящения овладевает опытом смертной жизни, когда он достигает совершенной гармонии со своим внутренним Настройщиком, он приступает к той части своей планетарной миссии, которая призвана просвещать разум и вдохновлять души его собратьев по плоти. Как учители, эти Сыны посвящены исключительно духовному просвещению смертных рас на мирах своего пребывания.
\vs p020 6:5 \pc Во многом похожие, пути посвящения Михаилов и Авоналов в облике смертных не во всём идентичны. Никогда Сын Повелитель не провозглашает: <<Всякий, видевший Сына, видел Отца>>,\fnst{Иоанна 14:9.} как это сделал ваш Сын Создатель во время своего пребывания на Урантии во плоти. Но совершивший посвящение Авонал заявляет: <<Всякий, видевший меня, видел Вечного Сына Бога>>. Сыны Повелители не происходят непосредственно от Всеобщего Отца и не воплощаются по воле Отца; но всегда посвящают себя как Райские \bibemph{Сыны,} исполняющие волю Вечного Сына Рая.
\vs p020 6:6 \pc Проходя через врата смерти, Сыны посвящения, Создатели или Повелители, снова появляются на третий день. Но не ст\'оит думать, что их всегда ожидает столь трагичный конец, с каким столкнулся Сын Создатель на вашем мире 1900\fnst{См. сноску к \bibref[20:4.4]{p020 4:4}.} лет назад. Чрезвычайный и необычайно жестокий опыт, через который прошёл Иисус из Назарета, послужил причиной тому, что Урантия стала локально известна как <<мир креста>>. Вовсе не обязательно, чтобы Сын Бога подвергался столь бесчеловечному обращению, и подавляющее большинство планет оказывает им более деликатный приём, позволяя завершить свой смертный путь, положить конец эпохе, вынести решение в отношении выживших спящих и открыть новую диспенсацию, не предавая насильственной смерти. Сын посвящения должен встретиться со смертью, должен пройти через весь реальный опыт смертных данного мира, но божественный план не требует, чтобы эта смерть была насильственной или необычной.
\vs p020 6:7 Сыны посвящения, не подвергшиеся смерти насильно, добровольно оставляют свои жизни и проходят через врата смерти не для удовлетворения требований <<сурового правосудия>> или <<божественного гнева>>, а чтобы завершить посвящение, <<испить чашу>> пути воплощения и личного опыта во всём, что составляет жизнь создания, как она проживается на планетах смертного существования. Посвящение~--- это планетарная и вселенская необходимость, а физическая смерть~--- не более чем необходимая часть миссии посвящения.
\vs p020 6:8 Когда воплощение в облике смертного завершено, Авонал, исполнявший служение, отправляется в Рай, принимается Всеобщим Отцом, возвращается в локальную вселенную назначения и получает признание Сына Создателя. Затем Авонал посвящения и Сын Создатель посылают своего совместного Духа Истины действовать в сердцах смертных рас, живущих на мире посвящения. В эпоху до обретения полновластия в локальной вселенной это~--- общий дух обоих Сынов, реализованный Созидательным Духом. Он несколько отличается от Духа Истины, который характеризует эпохи локальной вселенной после седьмого посвящения Михаила.
\vs p020 6:9 После завершения последнего посвящения Сына Создателя природа Духа Истины, ранее посланного во все миры посвящения Авоналов этой локальной вселенной, изменяется, становясь в более буквальном смысле духом полновластного Михаила. Этот феномен происходит одновременно с освобождением Духа Истины для служения на планете посвящения Михаила в облике смертного. После этого каждый мир, удостоенный посвящения Повелителя, получает такого же духа Утешителя от семикратного Сына Создателя, в союзе с этим Сыном Повелителем, какого он получил бы, если бы Суверен локальной вселенной лично посвятил себя этому миру.
\usection{ТРОИЧНЫЕ СЫНЫ УЧИТЕЛЯ}
\vs p020 7:1 Эти высоколичностные и высокодуховные Райские Сыны порождаются Райской Троицей. Они известны в Хавоне как категория Дайналов\fnst{О происхождении термина \bibemph{Дайнал} см. сноску к \bibref[20:1.1]{p020 1:1}.}. В Орвонтоне они зарегистрированы как Троичные Сыны Учителя, названные так из-за своего происхождения. На Спасограде их иногда называют Райскими Духовными Сынами.
\vs p020 7:2 Число Сынов Учителей постоянно увеличивается. По данным последней трансляции всеобщей переписи, количество этих Троичных Сынов, функционирующих в центральной вселенной и в сверхвселенных, немногим превышает 21\,000\,000\,000, и это без учёта Райских резервов, которые включают более одной трети всех существующих Троичных Сынов Учителей.
\vs p020 7:3 Категория сынов Дайналов не является органической частью администраций локальных или сверхвселенных. Её члены не являются ни создателями, ни спасателями, ни судьями, ни правителями. Они заботятся не столько об управлении вселенной, сколько о нравственном просвещении и духовном развитии. Они всеобщие наставники, посвящённые духовному пробуждению и нравственному руководству всех сфер. Их служение тесно взаимосвязано со служением личностей Бесконечного Духа и близко связано с Райским восхождением созданных существ.
\vs p020 7:4 Эти Сыны Троицы разделяют объединённую природу трёх Райских Божеств, но в Хавоне, по всей видимости, они больше отражают природу Всеобщего Отца. В сверхвселенных они, по\hyp{}видимому, выражают природу Вечного Сына, тогда как в локальных творениях они, очевидно, проявляют характер Бесконечного Духа. Во всех вселенных они являют собой воплощение служения и благоразумие мудрости.
\vs p020 7:5 В отличие от своих Райских братьев, Михаилов и Авоналов, Троичные Сыны Учителя не получают предварительной подготовки в центральной вселенной. Они направляются прямо в столицы сверхвселенных и оттуда назначаются на службу в какой\hyp{}нибудь локальной вселенной. В своём служении этим эволюционным сферам они используют объединённое духовное влияние Сына Создателя и действующих совместно с ним Сынов Повелителей, ибо сами по себе Дайналы не обладают духовной силой притяжения.
\usection{СЛУЖЕНИЕ ДАЙНАЛОВ В ЛОКАЛЬНЫХ ВСЕЛЕННЫХ}
\vs p020 8:1 Райские Духовные Сыны~--- уникальные существа Троичного происхождения и единственные Троичные создания, столь полно связанные с руководством вселенными двойственного происхождения. Они с любовью преданы служению просвещения смертных созданий и низших категорий духовных существ. Начиная трудиться в локальных системах, они, в соответствии с опытом и достижениями, продвигаются внутрь через служение в созвездиях к наивысшей работе локального творения. После аттестации они могут стать духовными послами, представляющими локальные вселенные своего служения.
\vs p020 8:2 Я не знаю точного числа Сынов Учителей в Небадоне; их многие тысячи. К этой категории относятся многие руководители отделений школ Мелхиседеков, в то время как объединённый штат постоянно действующего Университета Спасограда охватывает более 100\,000, включая этих Сынов. Большое их число размещается на различных мирах моронтийного обучения, где, наряду с духовным и интеллектуальным развитием смертных созданий, они в равной степени заняты обучением серафических существ и других уроженцев локальных творений. Многие из их помощников привлекаются из рядов существ, тринитизованных созданиями.
\vs p020 8:3 Сыны Учителя составляют ту инстанцию, которая проводит все экзамены и тесты для квалификации и аттестации всех подчинённых фаз вселенского служения, от обязанностей стражей аванпостов до исследователей звёзд. Они проводят многовековой курс обучения, начиная от планетарных курсов до высшего Колледжа Мудрости, расположенного на Спасограде. Признание, свидетельствующее об усилиях и достижениях, даровано всем, кто заканчивает эти путешествия [adventures] в мудрость и истину, будь то восходящий смертный или целеустремлённый херувим.
\vs p020 8:4 Во всех вселенных все Сыны Бога признательны этим неизменно преданным и всесторонне эффективным Троичным Сынам Учителям. Они~--- возвышенные учителя всех духовных личностей и даже~--- надёжные и истинные учителя самих Сынов Бога. Но я вряд ли могу рассказать вам о бесчисленных деталях обязанностей и функций Сынов Учителей. Обширная сфера деятельности сынов Дайналов будет лучше понята на Урантии тогда, когда больше повысится уровень вашего интеллектуального развития и когда завершится духовная изоляция вашей планеты.
\usection{ПЛАНЕТАРНОЕ СЛУЖЕНИЕ ДАЙНАЛОВ}
\vs p020 9:1 Когда развитие событий на эволюционном мире указывает на то, что пришло время положить начало духовной эпохе, Троичные Сыны Учителя всегда выступают в качестве добровольцев для этого служения. Вы не знакомы с этой категорией сыновства, потому что Урантия никогда не переживала духовной эпохи~--- тысячелетия космического просвещения. Но Сыны Учителя даже сейчас посещают ваш мир для составления планов своего предполагаемого пребывания на вашей сфере. Они должны будут появиться на Урантии после того, как её жители достигнут сравнительного освобождения от оков животного наследия и пут материализма.
\vs p020 9:2 Троичные Сыны Учителя не имеют никакого отношения к завершению планетарных диспенсаций. Они не судят мёртвых и не преобразуют живых, но в каждой планетарной миссии их сопровождает Сын Повелитель, совершающий это служение. Сыны Учителя целиком посвящены делу зарождения духовной эпохи, рассвету эры духовных реальностей на эволюционной планете. Они превращают в реальность духовные аналоги материального знания и темпоральной мудрости.
\vs p020 9:3 Обычно Сыны Учителя остаются на планетах своего посещения в течение 1000 лет планетарного времени. Возглавляет тысячелетнее планетарное правление один Сын Учитель, и ему помогают 70 партнёров той же категории. Дайналы не воплощаются и не материализуются каким\hyp{}либо иным методом, чтобы сделаться видимыми для смертных существ; поэтому контакт с миром посещения поддерживается благодаря деятельности Блистательных Вечерних Звёзд~--- личностей локальной вселенной, связанных с Троичными Сынами Учителями.
\vs p020 9:4 Дайналы могут возвращаться на обитаемый мир много раз, и после их последней миссии планета утверждается в статусе сферы света и жизни, эволюционной цели всех населённых смертными миров нынешней вселенской эпохи. Смертный Корпус Завершения много занимается сферами, утверждёнными в свете и жизни, и их планетарная деятельность соприкасается с деятельностью Сынов Учителей. В действительности вся категория сыновства Дайналов тесно связана со всеми фазами деятельности завершителей в эволюционных творениях времени и пространства.
\vs p020 9:5 \pc Троичные Сыны Учителя кажутся настолько полно отождествлёнными с режимом продвижения смертных на ранних стадиях эволюционного восхождения, что мы часто строим догадки об их возможном сотрудничестве с завершителями в нераскрытом пути будущих вселенных. Мы замечаем, что часть администраторов сверхвселенных представлена личностями Троичного происхождения, а часть~--- восходящими эволюционными созданиями, объятыми Троицей. Мы твёрдо убеждены, что Сыны Учителя и завершители сейчас накапливают опыт взаимодействия во времени, что может послужить предварительной подготовкой для их тесного общения в каком\hyp{}то нераскрытом будущем предназначении. На Уверсе мы верим, что после окончательного утверждения сверхвселенных в свете и жизни эти Райские Сыны Учителя, столь глубоко изучившие проблемы эволюционных миров и так долго связанные с путём восхождения эволюционных смертных, вероятно, перейдут к вечному союзу с Райским Корпусом Завершения.
\usection{ОБЪЕДИНЁННОЕ СЛУЖЕНИЕ РАЙСКИХ СЫНОВ}
\vs p020 10:1 Все Райские Сыны Бога имеют божественное происхождение и природу. Труд каждого Райского Сына во благо каждого мира совершается так, как если бы Сын служения был первым и единственным Сыном Бога.
\vs p020 10:2 Райские Сыны представляют собой божественное выражение действующей природы трёх лиц Божества в сферах времени и пространства. Сыны Создатели, Повелители и Учителя~--- это дары вечных Божеств детям человеческим и всем другим вселенским созданиям, обладающим потенциалом восхождения. Эти Сыны Бога~--- божественные служители, неустанно преданные делу помощи созданиям времени в достижении высокой духовной цели вечности.
\vs p020 10:3 В Сынах Создателях любовь Всеобщего Отца смешивается с милосердием Вечного Сына и раскрывается локальным вселенным в созидательном могуществе, полном любви служении и сострадательном полновластии Михаилов. В Сынах Повелителях милосердие Вечного Сына, объединённое со служением Бесконечного Духа, раскрывается эволюционным сферам в жизненном пути этих Авоналов суда, служения и посвящения. В Троичных Сынах Учителях любовь, милосердие и служение трёх Райских Божеств координируются на высочайших время\hyp{}пространственных уровнях ценностей и представляются вселенным как живая истина, божественная доброта и подлинная духовная красота.
\vs p020 10:4 В локальных вселенных эти категории сыновства сотрудничают, раскрывая откровение Божеств Рая созданиям пространства: как Отец локальной вселенной, Сын Создатель демонстрирует бесконечный характер Всеобщего Отца; как Сыны милосердия во время посвящения, Авоналы раскрывают непревзойдённую природу Вечного Сына бесконечного сострадания; как истинные учителя восходящих личностей, Троичные Сыны Дайналы раскрывают учительскую личность Бесконечного Духа. В своём божественно совершенном сотрудничестве Михаилы, Авоналы и Дайналы способствуют актуализации и раскрытию личности и полновластия Бога Верховного во время\hyp{}пространственных вселенных и для них. В гармонии своей триединой деятельности эти Райские Сыны Бога всегда действуют в авангарде личностей Божества, участвующих в бесконечном распространении божественности Первого Великого Источника и Центра от вечного Острова Рай в неизведанные глубины пространства.
\vsetoff
\vs p020 10:5 [Представлено Совершенствователем Мудрости из Уверсы.]
\quizlink
