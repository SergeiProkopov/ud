\upaper{21}{РАЙСКИЕ СЫНЫ СОЗДАТЕЛИ}
\uminitoc{ПРОИСХОЖДЕНИЕ И ПРИРОДА СЫНОВ СОЗДАТЕЛЕЙ}
\uminitoc{СОЗДАТЕЛИ ЛОКАЛЬНЫХ ВСЕЛЕННЫХ}
\uminitoc{ПОЛНОВЛАСТИЕ В ЛОКАЛЬНОЙ ВСЕЛЕННОЙ}
\uminitoc{ПОСВЯЩЕНИЯ МИХАИЛОВ}
\uminitoc{ОТНОШЕНИЕ СЫНОВ ВЛАСТЕЛИНОВ КО ВСЕЛЕННОЙ}
\uminitoc{ПРЕДНАЗНАЧЕНИЕ ВЛАСТЕЛИНОВ МИХАИЛОВ}
\author{Совершенствователь Мудрости}
\vs p021 0:1 Сыны Создатели являются создателями и правителями локальных вселенных времени и пространства. Эти вселенские создатели и властелины имеют двойное происхождение и воплощают в себе характерные черты Бога Отца и Бога Сына. Но каждый Сын Создатель отличается от любого другого; каждый уникален как по своей природе, так и личностью; каждый является <<единородным Сыном>>~--- совершенным божественным идеалом своего происхождения.
\vs p021 0:2 В огромной работе по организации, развитию и совершенствованию локальной вселенной эти высокие Сыны всегда получают неизменную поддержку и одобрение Всеобщего Отца. Отношения Сынов Создателей со своим Райским Отцом трогательны и превосходны. Несомненно, глубокая привязанность Божеств\hyp{}родителей к своему божественному потомству есть источник той прекрасной и почти божественной любви, которую даже смертные родители испытывают к своим детям.
\vs p021 0:3 Эти первичные Райские Сыны персонализируются как Михаилы. Когда они отправляются из Рая, чтобы основать свои вселенные, они известны как Создатели Михаилы. Когда они утверждаются в верховной власти, их называют Властелинами Михаилами. Иногда мы говорим о властелине вашей вселенной Небадон как о Христе Михаиле. Неизменно и вечно они правят <<по чину Михаила>>\fnst{Псалом~109:4: <<Клялся Господь и не раскается: Ты священник вовек по чину Мелхиседека>>.}, ибо таково название первого Сына их категории и природы.
\vs p021 0:4 \pc Изначальный, или первородный, Михаил никогда не имел опыта воплощения в качестве материального существа, но семь раз он проходил через опыт восхождения духовных созданий по семи контурам Хавоны, продвигаясь от внешних сфер к самому внутреннему контуру центрального творения. Категория Михаилов познала большую вселенную от одного конца до другого; нет ни одного существенного опыта любого из детей времени и пространства, в котором Михаилы не принимали бы личного участия; они действительно причастны не только к божественной природе, но и к вашей природе, то есть к любой~--- от высшей до низшей.
\vs p021 0:5 Изначальный Михаил является председательствующим главой первичных Райских Сынов, когда они собираются для совещания в центре всего. Не так давно на Уверсе мы записали вселенскую трансляцию внеочередного конклава на вечном Острове 150\,000 Сынов Создателей, собравшихся в родительском присутствии и занятых в обсуждениях, касающихся прогресса объединения и стабилизации вселенной вселенных. Это была избранная группа Суверенных Михаилов, Сынов семичастного посвящения.
\usection{ПРОИСХОЖДЕНИЕ И ПРИРОДА СЫНОВ СОЗДАТЕЛЕЙ}
\vs p021 1:1 Когда полнота абсолютного духовного представления в Вечном Сыне встречается с полнотой абсолютной личностной концепции во Всеобщем Отце, когда такой творческий союз достигается окончательно и полностью, когда происходит абсолютное тождество духа и такое бесконечное единство личностной концепции, тогда,~--- прямо тогда и там,~--- без утраты прерогатив или чего\hyp{}либо личностного любым из бесконечных Божеств, молниеносно появляется полноценное существо~--- новый и оригинальный Сын Создатель, единородный Сын совершенного идеала и могущественной идеи, чей союз порождает личность этого нового создателя~--- могущественную и совершенную.
\vs p021 1:2 Каждый Сын Создатель является единородным и единственно могущим быть рождённым потомком совершенного союза оригинальных концепций двух бесконечных, вечных и совершенных разумов вечно существующих Создателей вселенной вселенных. Никогда не может быть другого такого Сына, потому что каждый Сын Создатель~--- это безусловное, завершённое и окончательное выражение и воплощение всего, что представляет собой любая фаза любой черты, любой возможности, любой божественной реальности, которые во всей вечности могли быть когда\hyp{}либо обнаружены в тех божественных потенциалах, выражены через них или развиться из них, которые были объединены для создания этого Сына Михаила. Каждый Сын Создатель есть абсолют объединённых концепций божества, составляющих его божественное происхождение.
\vs p021 1:3 В принципе, божественная природа этих Сынов Создателей в равной степени происходит из атрибутов обоих Райских родителей. Все причастны к полноте божественной природы Всеобщего Отца и созидательных прерогатив Вечного Сына, но, наблюдая за практической реализацией функций Михаилов во вселенных, мы замечаем очевидные различия. Некоторые Сыны Создатели кажутся более похожими на Бога Отца; другие~--- на Бога Сына. Например: тенденция управления во вселенной Небадон предполагает, что её Создатель и правящий Сын есть тот, чья природа и характер больше напоминают Вечного Сына\hyp{}Мать. Следует также отметить, что некоторые вселенные возглавляются Райскими Михаилами, которые в равной степени напоминают Бога Отца и Бога Сына. И эти наблюдения ни в каком смысле не содержат критики; они просто являются констатацией фактов.
\vs p021 1:4 Я не знаю точное число существующих Сынов Создателей, но у меня есть веские основания полагать, что их более 700\,000. Теперь мы знаем, что существует ровно 700\,000 От Века Единых и что больше их не создаётся. Мы также отмечаем, что предопределённые планы нынешней вселенской эпохи, по\hyp{}видимому, указывают на то, что один От Века Единый располагается в каждой локальной вселенной в качестве советника\hyp{}посла Троицы. Далее мы видим, что постоянно растущее число Сынов Создателей уже превышает неизменное число От Века Единых. Но относительно предназначения Михаилов за пределами 700\,000 нас никогда не информировали.
\usection{СОЗДАТЕЛИ ЛОКАЛЬНЫХ ВСЕЛЕННЫХ}
\vs p021 2:1 Райские Сыны первичной категории являются проектировщиками, создателями, строителями и администраторами своих соответствующих владений~--- локальных вселенных времени и пространства, основных созидательных единиц семи эволюционных сверхвселенных. Сыну Создателю разрешено выбирать в пространстве место для своей будущей космической деятельности, но прежде чем он сможет начать даже физическую организацию своей вселенной, он должен провести длительный период наблюдения, посвящённый изучению усилий своих старших братьев в различных творениях, расположенных в сверхвселенной его запланированной деятельности. А до всего этого Сын Михаил завершит свой долгий и уникальный опыт Райского наблюдения и Хавонского обучения.
\vs p021 2:2 \pc Когда Сын Создатель покидает Рай и отправляется в приключение по созданию вселенной, чтобы стать главой~--- фактически Богом~--- организованной им самим локальной вселенной, тогда он впервые оказывается в близком контакте с Третьим Источником и Центром и во многих отношениях зависимым от него. Бесконечный Дух хотя и пребывает с Отцом и Сыном в центре всего, предназначен функционировать как настоящий и эффективный помощник каждого Сына Создателя. Поэтому каждого Сына Создателя сопровождает Созидательная Дочь Бесконечного Духа, то существо, которому предназначено стать Божественным Служителем~--- Материнским Духом новой локальной вселенной.
\vs p021 2:3 Отбытие Сына Михаила в связи с этим событием навсегда освобождает его прерогативы создателя от Райских Источников и Центров при соблюдении только некоторых ограничений, присущих предсуществованию этих Источников и Центров, а также определённых других предшествующих влияний и присутствий. Среди этих ограничений в остальном неограниченных создательских прерогатив Отца локальной вселенной можно выделить следующие:
\vs p021 2:4 \li{1.}\bibemph{Энергия\hyp{}материя} находится под доминирующим влиянием Бесконечного Духа. Прежде чем могут быть созданы какие\hyp{}либо новые формы вещей, большие или малые, прежде чем могут быть предприняты какие\hyp{}либо новые преобразования энергии\hyp{}материи, Сын Создатель должен обеспечить согласие и рабочее сотрудничество Бесконечного Духа.
\vs p021 2:5 \li{2.}\bibemph{Образцы и типы созданий} контролируются Вечным Сыном. Прежде чем Сын Создатель сможет заняться созданием любого нового типа существа, любого нового образца создания, он должен получить согласие Вечного и Изначального Сына\hyp{}Матери.
\vs p021 2:6 \li{3.}\bibemph{Личность} планируется и посвящается Всеобщим Отцом.
\vs p021 2:7 \pc Типы и образцы \bibemph{разума} определяются предшествующими созданию факторами бытия. После их объединения для образования создания (личностного или иного), разум становится даром Третьего Источника и Центра~--- всеобщего источника служения разума всем существам ниже уровня Райских Создателей.
\vs p021 2:8 \pc Контроль образцов и типов \bibemph{духа} зависит от уровня их проявления. В конечном счёте духовный образец контролируется Троицей или пред\hyp{}Троичными духовными дарами личностей Троицы~--- Отца, Сыны и Духа.
\vs p021 2:9 \pc Когда такой совершенный и божественный Сын вступил во владение избранной областью пространства для своей вселенной; когда первоначальные проблемы материализации вселенной и общего равновесия решены; когда он заключил эффективный и совместно действующий союз с дополняющей его Дочерью Бесконечного Духа~--- тогда этот Вселенский Сын и этот Вселенский Дух инициируют то взаимодействие, которое предназначено дать начало бесчисленным сонмам детей их локальной вселенной. В связи с этим событием Созидательный Дух~--- фокализация Райского Бесконечного Духа~--- изменяет свою природу, приобретая личные качества Материнского Духа локальной вселенной.
\vs p021 2:10 Несмотря на то что все Сыны Создатели божественно подобны своим Райским родителям, ни один из них не похож в точности на другого; каждый уникален, отличен, исключителен и оригинален как своей \bibemph{природой,} так и личностью. И так как они являются архитекторами и создателями жизненных планов своих соответствующих сфер, то само это разнообразие гарантирует, что их владения также будут разнообразны во всех производных от Михаила формах и фазах живого существования, которое может быть там создано и впоследствии развито. Поэтому категории существ, рождённых в локальных вселенных, весьма разнообразны. Нет двух вселенных, которые управлялись бы и были населены местными существами двойственного происхождения и идентичными во всех отношениях. В любой сверхвселенной одна половина присущих ей атрибутов очень схожа, будучи унаследованной от единообразных Созидательных Духов; другая половина различна, будучи унаследованной от разнообразных Сынов Создателей. Но такое разнообразие не характерно ни для созданий, происходящих только от Созидательного Духа, ни для прибывших существ~--- уроженцев центральной или сверхвселенных.
\vs p021 2:11 \pc Когда Сын Михаил отсутствует в своей вселенной, её правительством руководит перворождённое местное существо~--- Яркая Утренняя Звезда\fnst{В английском тексте \bibemph{Bright and Morning Star,} буквально переводимое как \bibemph{Яркая и Утренняя Звезда}, тогда как в названии \bibemph{Блистательная Вечерняя Звезда} (\bibemph{Brilliant Evening Star} в \bibref[20:2.9]{p020 2:9}) этого союза \bibemph{и} нет.}, глава исполнительной власти локальной вселенной. В такие времена неоценимы рекомендации и советы От Века Единого. На время своего отсутствия Сын Создатель может наделять ассоциированного с ним Материнского Духа способностью сверхконтроля своего духовного присутствия на обитаемых мирах и в сердцах своих смертных детей. А Материнский Дух локальной вселенной всегда остаётся в её столице, распространяя свою заботу и духовное служение до самых отдалённых уголков такой эволюционной области.
\vs p021 2:12 Личное присутствие Сына Создателя в своей локальной вселенной не является необходимым для гладкого функционирования существующего материального творения. Такие Сыны могут путешествовать в Рай, а их вселенные будут продолжать вращаться в пространстве. Они могут сложить свои полномочия, чтобы воплотиться как дети времени; всё равно их сферы будут продолжать обращаться вокруг соответствующих центров. Ни одна материальная организация не является независимой от охвата абсолютной гравитации Рая или космического сверхконтроля, присущего пространственному присутствию Безусловного Абсолюта.
\usection{ПОЛНОВЛАСТИЕ В ЛОКАЛЬНОЙ ВСЕЛЕННОЙ}
\vs p021 3:1 Сыну Создателю даётся область вселенной с согласия Райской Троицы и с подтверждением руководящего Главного Духа соответствующей сверхвселенной. Такое действие утверждает право физического владения, космической аренды. Но возвышение Сына Михаила от этой начальной и самоограниченной стадии правления к эмпирической верховности заработанного им полновластия происходит в результате его личного опыта в работе по созданию вселенной и в воплощениях посвящения. До достижения полновластия, заработанного посвящениями, он правит как наместник Всеобщего Отца.
\vs p021 3:2 \pc Сын Создатель мог бы заявить о полной суверенной власти над своим личным творением в любое время, но он мудро решает не делать этого. Если перед тем, как пройти через посвящения в качестве созданий, он принял бы незаслуженное верховное полновластие, то Райские личности, постоянно проживающие в его локальной вселенной, покинули бы её. Но этого никогда не случалось во всех творениях времени и пространства.
\vs p021 3:3 Сам факт статуса создателя подразумевает полноту суверенной власти, но Михаилы предпочитают эмпирически \bibemph{заработать} её, тем самым сохраняя полное сотрудничество со всеми Райскими личностями, прикреплёнными к администрации локальной вселенной. Мы не знаем ни одного Михаила, который когда\hyp{}либо поступил бы иначе, хотя все они могли бы, ибо действительно являются Сынами, обладающими свободной волей.
\vs p021 3:4 \pc Полновластие Сына Создателя в локальной вселенной проходит через шесть, возможно~--- семь, стадий эмпирического проявления. Они появляются в следующем порядке:
\vs p021 3:5 \li{1.}Первоначальное наместническое полновластие~--- временная единоличная власть, осуществляемая Сыном Создателем до приобретения личностных качеств у связанного с ним Созидательного Духа.
\vs p021 3:6 \li{2.}Совместное наместническое полновластие~--- совместное правление Райской пары после достижения личности Вселенским Материнским Духом.
\vs p021 3:7 \li{3.}Возрастающее наместническое полновластие~--- возрастающая власть Сына Создателя в течение периода его семи посвящений в качестве созданий.
\vs p021 3:8 \li{4.}Верховное полновластие~--- утверждённая власть после завершения седьмого посвящения. В Небадоне верховное полновластие начинается с момента завершения посвящения Михаила на Урантии. Оно существует немногим более 1900\fnst{На момент работы над этим переводом~--- 2027 лет.} лет вашего планетарного времени.
\vs p021 3:9 \li{5.}Возрастающее верховное полновластие~--- углублённые отношения, вырастающие из установления в свете и жизни большинства областей, населённых созданиями. Этот этап относится к недостигнутому будущему вашей локальной вселенной.
\vs p021 3:10 \li{6.}Тринитарное полновластие~--- осуществляется после установления всей локальной вселенной в свете и жизни.
\vs p021 3:11 \li{7.}Нераскрытое полновластие~--- неизвестные отношения будущей вселенской эпохи.
\vs p021 3:12 \pc Принимая первоначальное наместническое полновластие в планируемой локальной вселенной, Создатель Михаил даёт клятву Троице не принимать на себя верховное полновластие до тех пор, пока семь посвящений в качестве созданий не будут завершены и подтверждены правителями сверхвселенной. Но если Сын Михаил не мог бы по своему желанию заявить о таком незаслуженном полновластии, не было бы смысла давать клятву не делать этого.
\vs p021 3:13 Даже в эпохи до посвящений Сын Создатель правит своим владением почти верховно, когда нет разногласий ни в одной из его частей. Ограниченность правления едва ли могла бы проявиться, если бы полновластие никогда не оспаривалось. Полновластие, осуществляемое Сыном Создателем до посвящения во вселенной без восстания, не больше, чем во вселенной с восстанием; но в первом случае ограничения полновластия не проявляются, а во втором они очевидны.
\vs p021 3:14 Если когда\hyp{}либо власти или правлению Сына Создателя бросают вызов, нападают на них или им угрожает опасность, он навечно связан клятвой поддерживать, защищать, оборонять и, при необходимости, восстанавливать своё личное творение. Таких Сынов могут беспокоить или тревожить только существа, созданные ими самими, или высшие существа, которых они сами выбрали. Можно предположить, что <<высшие существа>>, происходящие с уровней выше локальной вселенной, вряд ли станут беспокоить Сына Создателя, и это правда. Но они могли бы, если бы захотели. В случае личности добродетель зависит от воли; праведность не является автоматической у созданий, обладающих свободной волей.
\vs p021 3:15 До завершения пути посвящения Сын Создатель правит с некоторыми наложенными им самим ограничениями полновластия, но после завершения своего служения посвящения он правит на основании своего действительного опыта в форме и подобии своих разнообразных созданий. Когда Создатель семь раз временно прожил среди своих созданий, когда его путь посвящения завершён, тогда он верховным образом утверждается во вселенской власти; он становится Сыном Властелином~--- полновластным и верховным правителем.
\vs p021 3:16 \pc Процесс обретения верховного полновластия над локальной вселенной включает следующие семь эмпирических шагов:
\vs p021 3:17 \li{1.}Эмпирически проникнуть в семь уровней существования методом воплощённого посвящения в точном подобии созданий соответствующего уровня.
\vs p021 3:18 \li{2.}Осуществить эмпирическое посвящение каждой фазе семичастной воли Райского Божества, олицетворённого в Семи Главных Духах.
\vs p021 3:19 \li{3.}Пройти каждый из семи опытов на уровнях созданий одновременно с исполнением одного из семи посвящений воле Райского Божества.
\vs p021 3:20 \li{4.}На каждом уровне созданий эмпирически изобразить кульминацию жизни создания Райскому Божеству и всем разумным существам вселенной.
\vs p021 3:21 \li{5.}На каждом уровне созданий эмпирически раскрыть одну фазу семичастной воли Божества уровню посвящения\fnst{То есть <<уровню существ данного посвящения>>. Например, при воплощении в образе смертного эволюционного мира это откровение обращено к смертным данного эволюционного мира и всех подобных миров.} и всей вселенной.
\vs p021 3:22 \li{6.}Эмпирически объединить семикратный опыт в качестве создания с семичастным опытом посвящения раскрытию природы и воли Божества.
\vs p021 3:23 \li{7.}Достичь новых и более высоких отношений с Верховным Существом. Отражение совокупности этого опыта Создателя\hyp{}создания увеличивает сверхвселенскую реальность Бога Верховного и время\hyp{}пространственное полновластие Всемогущего Верховного, а также актуализирует верховное полновластие Райского Михаила в локальной вселенной.
\vs p021 3:24 \pc Разрешая проблему полновластия в локальной вселенной, Сын Создатель не только демонстрирует свою собственную пригодность к правлению, но также раскрывает природу и изображает семичастную позицию Райских Божеств. Конечное понимание и осознание созданиями первичности Отца потенциально раскрываются в приключениях Сына Создателя, когда он нисходит до того, чтобы принять форму и опыт своих созданий. Эти первичные Райские Сыны по\hyp{}настоящему раскрывают любящую природу и благодетельную власть Отца, того же самого Отца, который совместно с Сыном и Духом является всеобщим главой всей мощи, всех личностей и всех правительств во всех вселенских сферах.
\usection{ПОСВЯЩЕНИЯ МИХАИЛОВ}
\vs p021 4:1 Существует семь групп проходящих посвящения Сынов Создателей, и такая классификация соответствует числу их посвящений созданиям своих миров: от первоначального опыта и далее через пять дополнительных сфер постепенного посвящения, пока они не достигнут седьмого и последнего эпизода опыта создания\hyp{}Создателя.
\vs p021 4:2 Посвящения Авоналов всегда происходят в подобии смертной плоти, но семь посвящений Сына Создателя включают его появление на семи уровнях существования созданий и относятся к раскрытию семи первичных выражений воли и природы Божества. Все без исключения Сыны Создатели проходят через это семь раз, даруя себя своим сотворённым детям, прежде чем они примут на себя утверждённую и верховную юрисдикцию над вселенной своего собственного творения.
\vs p021 4:3 Хотя эти семь посвящений различаются в разных секторах и вселенных, они всегда включают в себя приключение смертного посвящения. В последнем посвящении Сын Создатель появляется как член одной из высших рас смертных на каком\hyp{}нибудь обитаемом мире, обычно как член той расовой группы, которая содержит наибольшее генетическое наследие Адамического рода, привнесённого ранее для улучшения физического статуса народов животного происхождения. Только однажды на своём семичастном пути посвящения Сына Райский Михаил рождается от женщины, как у вас написано о младенце из Вифлеема. Только однажды он живёт и умирает как представитель низшей категории эволюционных волевых созданий.
\vs p021 4:4 После каждого из своих посвящений Сын Создатель восходит к <<правой руке Отца>>\fnst{Псалом 109:1, Марка 16:19.}, чтобы достичь там признания Отцом своего посвящения и получить наставление для подготовки к следующему эпизоду вселенского служения. После седьмого и заключительного посвящения Сын Создатель получает от Всеобщего Отца верховную власть и юрисдикцию над своей вселенной.
\vs p021 4:5 \pc Как известно, последний из божественных Сынов, появлявшихся на вашей планете, был Райским Сыном Создателем, завершившим шесть фаз своего пути посвящения; поэтому, когда он сознательно перестал цепляться за жизнь во плоти на Урантии, он мог по праву сказать: <<Свершилось>>\fnst{Иоанна 19:30.}~--- буквально свершилось. Смерть на Урантии завершила его путь посвящения; это был последний шаг в исполнении священной клятвы Райского Сына Создателя. И когда этот опыт приобретён, такие Сыны становятся верховными властелинами вселенной; они больше не правят как наместники Отца, но по собственному праву и от собственного имени~--- как <<Царь Царей и Господь Господствующих>>\fnst{Откровение 19:16}. За некоторыми указанными исключениями, эти Сыны семичастного посвящения являются безусловно верховными во вселенной своего пребывания. Что касается его локальной вселенной, то <<вся власть на небе и на земле>>\fnst{Матфея 28:18.} передаётся этому торжествующему и возведённому на трон Сыну Властелину.
\vs p021 4:6 \pc После завершения своего пути посвящения Сыны Создатели считаются отдельной категорией~--- семичастными Сынами Властелинами. Как личности, Сыны Властелины идентичны Сынам Создателям, но они пережили столь уникальный опыт посвящения, что их обычно рассматривают как представителей другой категории. Когда Создатель снисходит совершить посвящение, то суждено произойти реальной и необратимой перемене. Правда, Сын посвящения тем не менее всё ещё является Создателем, но он добавил к своей природе опыт создания, который навсегда удаляет его с божественного уровня Сына Создателя и возвышает его на эмпирический уровень Сына Властелина,~--- того, кто заработал полное право править вселенной и управлять её мирами. Такие существа воплощают в себе всё, что может быть унаследовано от божественного происхождения, и охватывают всё, что может быть получено из опыта ставшего совершенным создания. Стоит ли человеку оплакивать своё низкое происхождение и вынужденный эволюционный путь, когда сами Боги должны пройти эквивалентный опыт, прежде чем они будут признаны эмпирически достойными и компетентными, чтобы окончательно и полностью править своими вселенскими владениями!
\usection{ОТНОШЕНИЕ СЫНОВ ВЛАСТЕЛИНОВ КО ВСЕЛЕННОЙ}
\vs p021 5:1 Могущество Михаила Властелина неограниченно, потому что проистекает из испытанного им взаимодействия с Райской Троицей, оно неоспоримо, потому что извлечено из реального опыта в качестве тех созданий, которые подчиняются такой власти. Природа полновластия семичастного Сына Создателя является верховной, потому что она:
\vs p021 5:2 \li{1.}Охватывает семичастную точку зрения Райского Божества.
\vs p021 5:3 \li{2.}Воплощает семичастную позицию время\hyp{}пространственных созданий.
\vs p021 5:4 \li{3.}В совершенстве объединяет позицию Рая и точку зрения созданий.
\vs p021 5:5 \pc Таким образом, это эмпирическое полновластие включает в себя всю божественность Бога Семичастного, достигающую кульминации в Верховном Существе. И личное полновластие семикратного Сына подобно будущему полновластию когда\hyp{}нибудь завершённого Верховного Существа, ибо оно уже сейчас охватывает максимально полное содержание могущества и власти Райской Троицы, какое может быть проявлено в рамках данных время\hyp{}пространственных пределов.
\vs p021 5:6 \pc С достижением верховного полновластия в локальной вселенной Сын Михаил утрачивает власть и возможность создавать совершенно новые типы созданных существ в течение нынешней вселенской эпохи. Но утрата Сыном Властелином власти порождать совершенно новые категории никоим образом не препятствует работе по развитию жизни уже установленной, а также находящейся в процессе раскрытия; эта обширная программа эволюции вселенной идёт непрерывно и без сокращений. Обретение верховного полновластия Сыном Властелином подразумевает ответственность личной преданности за развитие и управление тем, что уже было задумано и создано, и того, что впоследствии будет создано теми, кто был таким образом задуман и создан. Со временем может развиться почти бесконечная эволюция разнообразных существ, но ни один совершенно новый образец или тип разумного создания отныне не будет иметь прямого происхождения от Сына Властелина. Это является первым шагом, началом, устойчивого управления в любой локальной вселенной.
\vs p021 5:7 Возвышение Сына семичастного посвящения до неоспоримого полновластия в своей вселенной означает начало конца многовековой неопределённости и относительного беспорядка. После этого события то, что не может быть когда\hyp{}нибудь одухотворено, в конечном итоге будет развалено; то, что когда\hyp{}нибудь не может быть согласовано с космической реальностью, в конечном итоге будет уничтожено. Когда запасы бесконечного милосердия и невыразимого терпения будут исчерпаны в усилиях завоевать лояльность и преданность волевых созданий миров, тогда справедливость и праведность восторжествуют. То, что милосердие не может восстановить, правосудие в конечном итоге уничтожит.
\vs p021 5:8 \pc Михаилы Властелины становятся верховными в своих локальных вселенных после того, как они были утверждены полновластными правителями. Некоторые ограничения их правления являются ограничениями, присущими космическому предсуществованию определённых сил и личностей. В остальном эти Сыны Властелины являются верховными во власти, ответственности и административном могуществе в своих вселенных; как Создатели и Боги, они действительно являются верховными почти во всём. Невозможно проникнуть за пределы их мудрости в вопросах функционирования данной вселенной.
\vs p021 5:9 После своего возвышения до установившегося полновластия в локальной вселенной Райский Михаил полностью контролирует всех других Сынов Бога, функционирующих в его владении, и он может свободно править в соответствии со своим представлением о потребностях своих сфер. Сын Властелин может по своей воле изменять порядок духовного судопроизводства и эволюционной адаптации обитаемых планет. И такие Сыны действительно составляют и осуществляют планы по своему собственному выбору во всех вопросах особых планетарных нужд, особенно в отношении миров их временного пребывания в качестве созданий, и ещё более относительно мира заключительного посвящения, планеты воплощения в подобии смертной плоти.
\vs p021 5:10 Сыны Властелины, по\hyp{}видимому, находятся в совершенной связи с мирами своих посвящений, и не только с мирами своего личного пребывания, но и со всеми мирами, которым посвятили себя Сыны Повелители. Этот контакт поддерживается их собственным духовным присутствием, Духом Истины, который они способны <<излить на всякую плоть>>. Эти Сыны Властелины также поддерживают непрерывную связь с Вечным Сыном\hyp{}Матерью, находящимся в центре всего. Они обладают сочувствием, которое простирается от Всеобщего Отца на небесах до низших рас планетарной жизни в сферах времени.
\usection{ПРЕДНАЗНАЧЕНИЕ ВЛАСТЕЛИНОВ МИХАИЛОВ}
\vs p021 6:1 Никто не осмелится безапелляционно обсуждать природу или предназначение семичастных Властелинов Суверенов локальных вселенных; тем не менее мы все много размышляем по этому поводу. Нас учат, и мы верим, что каждый Райский Михаил является \bibemph{абсолютом} концепций двойственного божества своего происхождения; таким образом, он воплощает в себе актуальные фазы бесконечности Всеобщего Отца и Вечного Сына. Михаилы, должно быть, частичны по отношению к тотальной бесконечности, но они, вероятно, абсолютны по отношению к той части бесконечности, которая связана с их происхождением. Но, наблюдая их деятельность в нынешнюю вселенскую эпоху, мы не обнаруживаем действий, которые более чем конечны; любые предположительно более чем конечные способности, должно быть, являются изолированными и пока что нераскрытыми.
\vs p021 6:2 Завершение пути посвящения в качестве созданий и возвышение до верховного вселенского полновластия должно означать окончательное высвобождение способностей Михаила к конечным действиям, сопровождаемое появлением способности к более чем конечному служению. Ибо в этой связи мы отмечаем, что такие Сыны Властелины впоследствии ограничены в создании новых типов созданных существ, и что это ограничение становится неизбежным из\hyp{}за освобождения их сверхконечных потенциалов.
\vs p021 6:3 Весьма вероятно, что эти нераскрытые способности создателя останутся заключёнными в них самих в течение всей нынешней вселенской эпохи. Но когда\hyp{}нибудь в далёком будущем, в ныне подготавливаемых вселенных внешнего пространства, мы верим, что связь между семичастным Сыном Властелином и Созидательным Духом седьмой ступени может достичь абсонитных уровней служения, сопровождаемого появлением новых вещей, смыслов и ценностей на трансцендентных уровнях предельного вселенского значения.
\vs p021 6:4 Так же как Божество Верховного актуализируется благодаря эмпирическому служению, так и Сыны Создатели достигают личной реализации потенциалов Райской божественности, скрытых в их непостижимой природе. Когда\hyp{}то на Урантии Христос Михаил сказал: <<Я есть путь, истина и жизнь>>\fnst{Иоанна 14:6}. И мы верим, что в вечности Михаилы буквально предназначены быть <<путём, истиной и жизнью>>, всегда освещая путь для всех вселенских личностей, ведя их от верховной божественности через предельную абсонитность к вечной завершённости божества.
\vsetoff
\vs p021 6:5 [Представлено Совершенствователем Мудрости из Уверсы.]
\quizlink
