\upaper{21}{РАЙСКИЕ СЫНЫ СОЗДАТЕЛИ}
\uminitoc{ПРОИСХОЖДЕНИЕ И ПРИРОДА СЫНОВ СОЗДАТЕЛЕЙ}
\uminitoc{СОЗДАТЕЛИ ЛОКАЛЬНЫХ ВСЕЛЕННЫХ}
\uminitoc{ПОЛНОВЛАСТИЕ В ЛОКАЛЬНОЙ ВСЕЛЕННОЙ}
\uminitoc{ПОСВЯЩЕНИЯ МИХАИЛОВ}
\uminitoc{ОТНОШЕНИЕ СЫНОВ ВЛАСТЕЛИНОВ К ВСЕЛЕННОЙ}
\uminitoc{ПРЕДНАЗНАЧЕНИЕ ВЛАСТЕЛИНОВ МИХАИЛОВ}
\author{Совершенствователь Мудрости}
\vs p021 0:1 Сыны Создатели являются создателями и правителями локальных вселенных времени и пространства. Эти вселенские создатели и властелины имеют двойное происхождение и воплощают в себе характерные черты Бога Отца и Бога Сына. Но каждый Сын Создатель отличается от любого другого; каждый уникален как по своей природе, так и личностью; каждый является <<единородным Сыном>>~--- совершенным божественным идеалом своего происхождения.
\vs p021 0:2 В огромной работе по организации, развитию и совершенствованию локальной вселенной эти высокие Сыны всегда получают неизменную поддержку и одобрение Всеобщего Отца. Отношения Сынов Создателей со своим Райским Отцом трогательны и превосходны. Несомненно, глубокая привязанность Божеств\hyp{}родителей к своему божественному потомству есть источник той прекрасной и почти божественной любви, которую даже смертные родители испытывают к своим детям.
\vs p021 0:3 Эти первичные Райские Сыны персонализируются как Михаилы. Когда они отправляются из Рая, чтобы основать свои вселенные, они известны как Создатели Михаилы. Когда они утверждаются в верховной власти, их называют Властелинами Михаилами. Иногда мы говорим о властелине вашей вселенной Небадон как о Христе Михаиле. Неизменно и вечно они правят <<по чину Михаила>>\fnst{Псалом~109:4: <<Клялся Господь и не раскается: Ты священник вовек по чину Мелхиседека>>.}, ибо таково название первого Сына их категории и природы.
\vs p021 0:4 \pc Изначальный, или первородный, Михаил никогда не имел опыта воплощения в качестве материального существа, но семь раз он проходил через опыт восхождения духовных созданий по семи контурам Хавоны, продвигаясь от внешних сфер к самому внутреннему контуру центрального творения. Категория Михаилов познала большую вселенную от одного конца до другого; нет ни одного существенного опыта любого из детей времени и пространства, в котором Михаилы не принимали бы личного участия; они действительно причастны не только к божественной природе, но и к вашей природе, то есть к любой~--- от высшей до низшей.
\vs p021 0:5 Изначальный Михаил является председательствующим главой первичных Райских Сынов, когда они собираются для совещания в центре всего. Не так давно на Уверсе мы записали вселенскую трансляцию внеочередного конклава на вечном Острове 150\,000 Сынов Создателей, собравшихся в родительском присутствии и занятых в обсуждениях, касающихся прогресса объединения и стабилизации вселенной вселенных. Это была избранная группа Суверенных Михаилов, Сынов семичастного посвящения.
\usection{ПРОИСХОЖДЕНИЕ И ПРИРОДА СЫНОВ СОЗДАТЕЛЕЙ}
\vs p021 1:1 Когда полнота абсолютного духовного представления в Вечном Сыне встречается с полнотой абсолютной личностной концепции во Всеобщем Отце, когда такой творческий союз достигается окончательно и полностью, когда происходит абсолютное тождество духа и такое бесконечное единство личностной концепции, тогда,~--- прямо тогда и там,~--- без утраты прерогатив или чего\hyp{}либо личностного любым из бесконечных Божеств, молниеносно появляется полноценное существо~--- новый и оригинальный Сын Создатель, единородный Сын совершенного идеала и могущественной идеи, чей союз порождает личность этого нового создателя~--- могущественную и совершенную.
\vs p021 1:2 Каждый Сын Создатель является единородным и единственно могущим быть рождённым потомком совершенного союза оригинальных концепций двух бесконечных, вечных и совершенных разумов вечно существующих Создателей вселенной вселенных. Никогда не может быть другого такого Сына, потому что каждый Сын Создатель~--- это безусловное, завершённое и окончательное выражение и воплощение всего, что представляет собой любая фаза любой черты, любой возможности, любой божественной реальности, которые во всей вечности могли быть когда\hyp{}либо обнаружены в тех божественных потенциалах, выражены через них или развиться из них, которые были объединены для создания этого Сына Михаила. Каждый Сын Создатель есть абсолют объединённых концепций божества, составляющих его божественное происхождение.
\vs p021 1:3 В принципе, божественная природа этих Сынов Создателей в равной степени происходит из атрибутов обоих Райских родителей. Все причастны к полноте божественной природы Всеобщего Отца и созидательных прерогатив Вечного Сына, но, наблюдая за практической реализацией функций Михаилов во вселенных, мы замечаем очевидные различия. Некоторые Сыны Создатели кажутся более похожими на Бога Отца; другие~--- на Бога Сына. Например: тенденция управления во вселенной Небадон предполагает, что её Создатель и правящий Сын есть тот, чья природа и характер больше напоминают Вечного Сына\hyp{}Мать. Следует также отметить, что некоторые вселенные возглавляются Райскими Михаилами, которые в равной степени напоминают Бога Отца и Бога Сына. И эти наблюдения ни в каком смысле не содержат критики; они просто являются констатацией фактов.
\vs p021 1:4 Я не знаю точное число существующих Сынов Создателей, но у меня есть веские основания полагать, что их более 700\,000. Теперь мы знаем, что существует ровно 700\,000 От Века Единых и что больше их не создаётся. Мы также отмечаем, что предопределённые планы нынешней вселенской эпохи, по\hyp{}видимому, указывают на то, что один От Века Единый располагается в каждой локальной вселенной в качестве советника\hyp{}посла Троицы. Далее мы видим, что постоянно растущее число Сынов Создателей уже превышает неизменное число От Века Единых. Но относительно предназначения Михаилов за пределами 700\,000 нас никогда не информировали.
\usection{СОЗДАТЕЛИ ЛОКАЛЬНЫХ ВСЕЛЕННЫХ}
\vs p021 2:1 Райские Сыны первичной категории являются проектировщиками, создателями, строителями и администраторами своих соответствующих владений~--- локальных вселенных времени и пространства, основных созидательных единиц семи эволюционных сверхвселенных. Сыну Создателю разрешено выбирать в пространстве место для своей будущей космической деятельности, но прежде чем он сможет начать даже физическую организацию своей вселенной, он должен провести длительный период наблюдения, посвящённый изучению усилий своих старших братьев в различных творениях, расположенных в сверхвселенной его запланированной деятельности. А до всего этого Сын Михаил завершит свой долгий и уникальный опыт наблюдения в Раю и обучения в Хавоне.
\vs p021 2:2 \pc Когда Сын Создатель покидает Рай и отправляется в приключение по созданию вселенной, чтобы стать главой~--- фактически Богом~--- организованной им самим локальной вселенной, тогда он впервые оказывается в близком контакте с Третьим Источником и Центром и во многих отношениях зависимым от него. Бесконечный Дух, хотя и пребывает с Отцом и Сыном в центре всего, предназначен функционировать как настоящий и эффективный помощник каждого Сына Создателя. Поэтому каждого Сына Создателя сопровождает Созидательная Дочь Бесконечного Духа, то существо, которому предназначено стать Божественным Служителем~--- Материнским Духом новой локальной вселенной.
\vs p021 2:3 Отбытие Сына Михаила в связи с этим событием навсегда освобождает его прерогативы создателя от Райских Источников и Центров при соблюдении только некоторых ограничений, присущих предсуществованию этих Источников и Центров, а также определённых других предшествующих влияний и присутствий. Среди этих ограничений в остальном неограниченных создательских прерогатив Отца локальной вселенной, можно выделить следующие:
\vs p021 2:4 \li{1.}\bibemph{Энергия\hyp{}материя} находится под доминирующим влиянием Бесконечного Духа. Прежде чем могут быть созданы какие\hyp{}либо новые формы вещей, большие или малые, прежде чем могут быть предприняты какие\hyp{}либо новые преобразования энергии\hyp{}материи, Сын Создатель должен обеспечить согласие и рабочее сотрудничество Бесконечного Духа.
\vs p021 2:5 \li{2.}\bibemph{Образцы и типы созданий} контролируются Вечным Сыном. Прежде чем Сын Создатель сможет заняться созданием любого нового типа существа, любого нового образца создания, он должен получить согласие Вечного и Изначального Сына\hyp{}Матери.
\vs p021 2:6 \li{3.}\bibemph{Личность} планируется и посвящается Всеобщим Отцом.
\vs p021 2:7 \pc Типы и образцы \bibemph{разума} определяются предшествующими созданию факторами бытия. После их объединения для образования создания (личностного или иного), разум становится даром Третьего Источника и Центра~--- всеобщего источника служения разума всем существам ниже уровня Райских Создателей.
\vs p021 2:8 \pc Контроль образцов и типов \bibemph{духа} зависит от уровня их проявления. В конечном счёте, духовный образец контролируется Троицей или пред\hyp{}Троичными духовными дарами личностей Троицы~--- Отца, Сыны и Духа.
\vs p021 2:9 \pc Когда такой совершенный и божественный Сын вступил во владение избранной областью пространства для своей вселенной; когда первоначальные проблемы материализации вселенной и общего равновесия решены; когда он заключил эффективный и совместно действующий союз с дополняющей его Дочерью Бесконечного Духа~--- тогда этот Вселенский Сын и этот Вселенский Дух инициируют то взаимодействие, которое предназначено дать начало бесчисленным сонмам детей их локальной вселенной. В связи с этим событием Созидательный Дух~--- фокализация Райского Бесконечного Духа~--- изменяет свою природу, приобретая личные качества Материнского Духа локальной вселенной.
\vs p021 2:10 Несмотря на то, что все Сыны Создатели божественно подобны своим Райским родителям, ни один из них не похож в точности на другого; каждый уникален, отличен, исключителен и оригинален как своей \bibemph{природой,} так и личностью. И так как они являются архитекторами и создателями жизненных планов своих соответствующих сфер, то само это разнообразие гарантирует, что их владения также будут разнообразны во всех производных от Михаила формах и фазах живого существования, которое может быть там создано и впоследствии развито. Поэтому категории существ, рождённых в локальных вселенных, весьма разнообразны. Нет двух вселенных, которые управлялись бы и были населены местными существами дуального происхождения и идентичными во всех отношениях. В любой сверхвселенной одна половина присущих ей атрибутов очень схожа, будучи унаследованной от единообразных Созидательных Духов; другая половина различна, будучи унаследованной от разнообразных Сынов Создателей. Но такое разнообразие не характерно ни для созданий, происходящих только от Созидательного Духа, ни для прибывших существ~--- уроженцев центральной или сверхвселенных.
\vs p021 2:11 \pc Когда Сын Михаил отсутствует в своей вселенной, её правительством руководит перворождённое местное существо~--- Яркая Утренняя Звезда\fnst{В английском тексте \bibemph{Bright and Morning Star,} буквально переводимое как \bibemph{Яркая и Утренняя Звезда}, тогда как в названии \bibemph{Блистательная Вечерняя Звезда} (\bibemph{Brilliant Evening Star} в \bibref[20:2.9]{p020 2:9}) этого предлога \bibemph{и} нет.}, глава исполнительной власти локальной вселенной. В такие времена неоценимы рекомендации и советы От Века Единого. На время своего отсутствия Сын Создатель может наделять ассоциированного с ним Материнского Духа способностью сверхконтроля своего духовного присутствия на обитаемых мирах и в сердцах своих смертных детей. А Материнский Дух локальной вселенной всегда остаётся в её столице, распространяя свою заботу и духовное служение до самых отдалённых уголков такой эволюционной области.
\vs p021 2:12 Личное присутствие Сына Создателя в своей локальной вселенной не является необходимым для гладкого функционирования существующего материального творения. Такие Сыны могут путешествовать в Рай, а вселенные их будут продолжать вращаться в пространстве. Они могут сложить свои полномочия, чтобы воплотиться как дети времени; всё равно их сферы будут продолжать обращаться вокруг соответствующих центров. Ни одна материальная организация не является независимой от охвата абсолютной гравитации Рая или космического сверхконтроля, присущего пространственному присутствию Безусловного Абсолюта.
\usection{ПОЛНОВЛАСТИЕ В ЛОКАЛЬНОЙ ВСЕЛЕННОЙ}
\vs p021 3:1 
\vs p021 3:2 \pc 
\vs p021 3:3 
\vs p021 3:4 \pc 
\vs p021 3:5 
\vs p021 3:6 
\vs p021 3:7 
\vs p021 3:8 
\vs p021 3:9 
\vs p021 3:10 
\vs p021 3:11 
\vs p021 3:12 \pc 
\vs p021 3:13 
\vs p021 3:14 
\vs p021 3:15 
\vs p021 3:16 \pc 
\vs p021 3:17 
\vs p021 3:18 
\vs p021 3:19 
\vs p021 3:20 
\vs p021 3:21 
\vs p021 3:22 
\vs p021 3:23 
\vs p021 3:24 \pc 
\usection{ПОСВЯЩЕНИЯ МИХАИЛОВ}
\vs p021 4:1 
\vs p021 4:2 
\vs p021 4:3 
\vs p021 4:4 
\vs p021 4:5 \pc 
\vs p021 4:6 \pc 
\usection{ОТНОШЕНИЕ СЫНОВ ВЛАСТЕЛИНОВ К ВСЕЛЕННОЙ}
\vs p021 5:1 
\vs p021 5:2 
\vs p021 5:3 
\vs p021 5:4 
\vs p021 5:5 \pc 
\vs p021 5:6 \pc 
\vs p021 5:7 
\vs p021 5:8 \pc 
\vs p021 5:9 
\vs p021 5:10 
\usection{ПРЕДНАЗНАЧЕНИЕ ВЛАСТЕЛИНОВ МИХАИЛОВ}
\vs p021 6:1 
\vs p021 6:2 
\vs p021 6:3 
\vs p021 6:4 
\vsetoff
\vs p021 6:5 
\quizlink
