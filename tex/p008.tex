\upaper{8}{БЕСКОНЕЧНЫЙ ДУХ}
\uminitoc{БОГ ДЕЙСТВИЯ}
\uminitoc{Nature of the Infinite Spirit}
\uminitoc{Relation of the Spirit to the Father and the Son}
\uminitoc{The Spirit of Divine Ministry}
\uminitoc{The Presence of God}
\uminitoc{Personality of the Infinite Spirit}
\author{Божественный Советник}
\vs p008 0:1 У истоков вечности, когда <<первая>> бесконечная и абсолютная мысль Всеобщего Отца находит в Вечном Сыне столь совершенное и адекватное слово для своего божественного выражения, следует верховное желание Бога\hyp{}Мысли и Бога\hyp{}Слова обрести всеобщего и бесконечного посредника для совместного выражения и объединённого действия.
\vs p008 0:2 На рассвете вечности и Отец, и Сын бесконечно осознают свою взаимозависимость, своё вечное и абсолютное единство; и поэтому они заключают бесконечный и вечный завет божественного партнёрства. Этот нескончаемый договор заключён для исполнения их единых идей по всему кругу вечности; и с момента этого вечностного события Отец и Сын продолжают пребывать в этом божественном союзе.
\vs p008 0:3 Здесь мы вплотную подходим к происхождению в вечности Бесконечного Духа, Третьего Лица Божества. В то мгновение, когда Бог Отец и Бог Сын совместно задумывают идентичное и бесконечное действие --- исполнение абсолютной мысли\hyp{}плана, --- в этот самый момент Бесконечный Дух начинает существовать во всей полноте.
\vs p008 0:4 \pc Рассказывая таким образом о порядке происхождения Божеств, я поступаю так только для того, чтобы вы могли обдумать их взаимосвязь. В действительности все трое существуют извечно; они экзистенциальны. У них нет ни начала, ни конца дней; они согласованны, верховны, предельны, абсолютны и бесконечны. Они есть, всегда были и всегда будут. И это три отчётливо индивидуализированных, но вечно связанных личности: Бог Отец, Бог Сын и Бог Дух.
\usection{БОГ ДЕЙСТВИЯ}
\vs p008 1:1 В вечности прошлого, при персонализации Бесконечного Духа, божественный личностный цикл становится совершенным и завершённым. Появляется Бог Действия, и необъятная сцена пространства готова к грандиозной драме творения --- вселенскому приключению --- божественной панораме вечных эпох.
\vs p008 1:2 Первый акт Бесконечного Духа --- это изучение и признание своих божественных родителей, Отца\hyp{}Отца и Мать\hyp{}Сына. Он, Дух, безоговорочно идентифицирует их обоих. Он полностью осознаёт их раздельные личности и бесконечные атрибуты, а также их совокупную природу и объединённую функцию. Затем добровольно, с трансцендентной готовностью и воодушевляющей спонтанностью Третье Лицо Божества, несмотря на своё равенство с Первым и Вторым Лицами, клянётся в вечной верности Богу Отцу и признаёт вечную зависимость от Бога Сына.
\vs p008 1:3 В соответствии с природой этого действия и взаимным признанием индивидуальной независимости каждого и исполнительного союза всех трёх, устанавливается круг вечности. Райская Троица существует. Сцена универсального пространства установлена для разнообразной и нескончаемой панорамы творческого раскрытия замысла Всеобщего Отца через личность Вечного Сына и посредством исполнения Бога Действия, исполнительного органа представлений реальности, поставленных созидательным партнёрством Отца и Сына.
\vs p008 1:4 \pc Бог Действия функционирует, и мёртвые своды пространства приходят в движение. В мгновение возникает миллиард идеальных сфер. До этого гипотетического момента в вечности присущие Раю пространственные энергии существовали и были потенциально действующими, но не имели действительности бытия; и физическую гравитацию нельзя измерить иначе, как реакцией материальных реальностей на её непрерывное притяжение. В этот (предполагаемо) вечно удалённый момент материальной вселенной нет, но в тот самый миг, когда материализуется один миллиард миров, появляется гравитация, достаточная и адекватная для того, чтобы удерживать их в вечном охвате Рая.
\vs p008 1:5 Теперь сквозь творение Богов излучается второй вид энергии, и этот истекающий дух мгновенно охватывается духовной гравитацией Вечного Сына. Так, объятая двойной гравитацией вселенная соприкасается с энергией бесконечности и погружается в дух божественности. Таким образом почва жизни подготавливается для сознания разума, проявляющегося в объединённых интеллектуальных контурах Бесконечного Духа.
\vs p008 1:6 На эти семена потенциального существования, рассеянных по всему центральному творению Богов, воздействует Отец, и появляется личность создания. Тогда присутствие Райских Божеств заполняет всё организованное пространство и начинает эффективно притягивать все вещи и существа к Раю.
\vs p008 1:7 \pc Бесконечный Дух увековечивается одновременно с рождением миров Хавоны, эта центральная вселенная создана им, с ним и в нём в соответствии с объединёнными концепциями и единой волей Отца и Сына. Третье Лицо обожествляется благодаря этому самому акту совместного творения и, таким образом, навсегда становится Совместным Создателем.
\vs p008 1:8 \pc Это грандиозные и величественные времена созидательного распространения Отца и Сына благодаря их совместному партнёру и исключительному исполнителю, Третьему Источнику и Центру. Не существует никаких письменных свидетельств об этих волнующих временах. У нас есть лишь скудные откровения Бесконечного Духа, обосновывающие эти могучие действия, и он только подтверждает тот факт, что центральная вселенная и всё, что к ней относится, возникли одновременно с приобретением им личности и осознанного существования.
\vs p008 1:9 Вкратце, Бесконечный Дух свидетельствует о том, что, поскольку он вечен, вечна и центральная вселенная. Такова традиционная отправная точка истории вселенной вселенных. Абсолютно ничего не известно, и не существует никаких записей, касающихся какого\hyp{}либо события или процесса, предшествовавших этому колоссальному извержению творческой энергии и управляющей мудрости, которые кристаллизовали огромную вселенную, существующей и столь совершенно функционирующей в центре всего. За этим событием лежат непостижимые процессы вечности и глубины бесконечности --- абсолютная тайна.
\vs p008 1:10 \pc Таким образом мы описываем последовательное происхождение Третьего Источника и Центра как интерпретационное снисхождение к привязанному ко времени и обусловленному пространством разуму смертных созданий. Разум человека должен иметь отправную точку для визуализации истории вселенной, и мне было поручено предоставить этот метод подхода к исторической концепции вечности. В материальном разуме последовательность требует Первопричину; поэтому мы постулируем Всеобщего Отца как Первый Источник и Абсолютный Центр всего творения, в то же время инструктируя всех разумных созданий, что Сын и Дух совечны с Отцом на всех стадиях вселенской истории и во всех сферах созидания. И мы делаем это, ни в коем случае не пренебрегая реальностью и вечностью Острова Рай, Безусловного Абсолюта, Всеобщего Абсолюта и Божества Абсолюта.
\vs p008 1:11 Для материального разума детей времени достаточно того, что он в состоянии постичь Отца в вечности. Мы знаем, что любой ребёнок может лучше всего соотнести себя с реальностью, вначале овладев отношениями в ситуации ребёнок\hyp{}родитель, а затем расширив эту концепцию, чтобы охватить семью в целом. Впоследствии развивающийся ум ребёнка сможет адаптироваться к концепции семейных отношений, отношений в обществе, расе и в мире, а затем и во вселенной, сверхвселенной и даже вселенной вселенных.
\usection{Nature of the Infinite Spirit}
\vs p008 2:1 
\vs p008 2:2 \pc 
\vs p008 2:3 
\vs p008 2:4 \pc 
\vs p008 2:5 \pc 
\vs p008 2:6 
\vs p008 2:7 
\vs p008 2:8 
\usection{Relation of the Spirit to the Father and the Son}
\vs p008 3:1 
\vs p008 3:2 
\vs p008 3:3 
\vs p008 3:4 
\vs p008 3:5 \pc 
\vs p008 3:6 
\vs p008 3:7 
\vs p008 3:8 
\vs p008 3:9 
\usection{The Spirit of Divine Ministry}
\vs p008 4:1 
\vs p008 4:2 
\vs p008 4:3 
\vs p008 4:4 \pc 
\vs p008 4:5 
\vs p008 4:6 
\vs p008 4:7 \pc 
\vs p008 4:8 
\usection{The Presence of God}
\vs p008 5:1 
\vs p008 5:2 
\vs p008 5:3 \pc 
\vs p008 5:4 \pc 
\vs p008 5:5 
\vs p008 5:6 
\usection{Personality of the Infinite Spirit}
\vs p008 6:1 
\vs p008 6:2 
\vs p008 6:3 
\vs p008 6:4 
\vs p008 6:5 
\vs p008 6:6 \pc 
\vs p008 6:7 
\vsetoff
\vs p008 6:8 
\quizlink
