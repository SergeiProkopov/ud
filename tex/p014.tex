\upaper{14}{ЦЕНТРАЛЬНАЯ И БОЖЕСТВЕННАЯ ВСЕЛЕННАЯ}
\uminitoc{СИСТЕМА РАЙ\hyp{}ХАВОНА}
\uminitoc{УСТРОЙСТВО ХАВОНЫ}
\uminitoc{МИРЫ ХАВОНЫ}
\uminitoc{СОЗДАНИЯ ЦЕНТРАЛЬНОЙ ВСЕЛЕННОЙ}
\uminitoc{ЖИЗНЬ В ХАВОНЕ}
\uminitoc{НАЗНАЧЕНИЕ ЦЕНТРАЛЬНОЙ ВСЕЛЕННОЙ}
\author{Совершенствователь Мудрости}
\vs p014 0:1 Совершенная и божественная вселенная занимает центр всего творения; она --- вечная сердцевина, вокруг которой вращаются обширные творения времени и пространства. Рай --- это гигантский Остров, ядро абсолютной стабильности, который неподвижно покоится в самом сердце великолепной вечной вселенной. Это центральное планетарное семейство, называемое Хавоной, значительно удалено от локальной вселенной Небадон. Оно имеет громадные размеры, почти невероятную массу и состоит из миллиарда сфер невообразимой красоты и непревзойдённого величия, но истинные размеры этого огромного творения находятся за пределами понимания человеческого разума.
\vs p014 0:2 Это --- одно единственное стабильное, совершенное и утверждённое скопление миров. Это --- целиком созданная и совершенная вселенная; она не является результатом эволюционного развития. Это --- вечное ядро совершенства, вокруг которого кружится бесконечная процессия вселенных, составляющих грандиозный эволюционный эксперимент, смелое приключение Божьих Сынов Создателей, которые стремятся повторить во времени и воспроизвести в пространстве вселенную\hyp{}образец, идеал божественной завершённости, верховной законченности, предельной реальности и вечного совершенства.
\usection{СИСТЕМА РАЙ\hyp{}ХАВОНА}
\vs p014 1:1 Между периферией Рая и внутренними границами семи сверхвселенных существуют следующие семь пространственных состояний и движений:
\vs p014 1:2 \li{1.}Спокойные зоны промежуточного пространства, соприкасающиеся с Раем.
\vs p014 1:3 \li{2.}Движущиеся по часовой стрелке три контура Рая и семь контуров Хавоны.
\vs p014 1:4 \li{3.}Зона пространства относительного покоя, отделяющая контуры Хавоны от тёмных гравитационных тел центральной вселенной.
\vs p014 1:5 \li{4.}Внутренний, движущийся против часовой стрелки, пояс тёмных гравитационных тел.
\vs p014 1:6 \li{5.}Вторая уникальная зона пространства, разделяющая две пространственные траектории тёмных гравитационных тел.
\vs p014 1:7 \li{6.}Внешний пояс тёмных гравитационных тел, вращающихся по часовой стрелке вокруг Рая.
\vs p014 1:8 \li{7.}Третья зона пространства --- зона относительного покоя --- отделяющая внешний пояс тёмных гравитационных тел от внутренних контуров семи сверхвселенных.
\vs p014 1:9 \pc Миллиард миров Хавоны организован в семь концентрических контуров, непосредственно окружающих три контура спутников Рая. Более 35\,000\,000 миров находятся во внутреннем контуре Хавоны и более 245\,000\,000 --- на внешнем, с пропорциональными числами между ними. Каждый контур отличается от других, но все они идеально сбалансированы и изысканно организованы, и каждый пронизан особым представительством Бесконечного Духа --- одним из Семи Духов Контуров. В дополнение к другим функциям этот неличностный Дух координирует ведение небесных дел в каждом контуре.
\vs p014 1:10 Планетарные контуры Хавоны не накладываются друг на друга; их миры следуют друг за другом упорядоченными линейными рядами. Центральная вселенная вращается вокруг неподвижного Острова Рай в одной гигантской плоскости, состоящей из десяти концентрических стабилизированных единиц --- трёх контуров сфер Рая и семи контуров миров Хавоны. С физической точки зрения контуры Хавоны и Рая --- это одна и та же система; их разделение проявляется только в признании их функциональной и административной сегрегации.
\vs p014 1:11 \pc Счёт времени не ведётся на Рае; последовательность чередующихся событий заложена в концепции коренных жителей центрального Острова. Но время уместно для контуров Хавоны и для многочисленных существ как небесного, так и земного происхождения, пребывающих на них. Каждый мир Хавоны имеет своё местное время, определяемое его контуром. Все миры в данном контуре имеют одинаковую продолжительность года, поскольку они равномерно обращаются вокруг Рая, и продолжительность этих планетных лет уменьшается от внешнего контура к самому внутреннему.
\vs p014 1:12 Кроме контурного времени Хавоны, существует стандартный день системы Рай\hyp{}Хавона, но есть и другие показатели времени, которые определяются на семи Райских спутниках Бесконечного Духа и передаются оттуда. Стандартный день системы Рай\hyp{}Хавона основан на продолжительности времени, необходимом для планетарных обителей первого или внутреннего контура Хавоны, чтобы совершить один оборот вокруг Острова Рай; и хотя их скорость огромна из\hyp{}за их положения между тёмными гравитационными телами и гигантским Раем, этим сферам требуется почти 1000 лет, чтобы завершить свой круг. Твоему взору невольно открывается истина, когда ты читаешь утверждение: <<День у Бога как тысяча лет, как стража в ночи>>\fnst{<<Ибо пред очами Твоими тысяча лет, как день вчерашний, когда он прошёл, и \bibemph{как} стража в ночи>>. Псалом 89:5 (Синодальный перевод).}. Один день системы Рай\hyp{}Хавона всего на 7 минут, 3\bibfrac{1}{8} секунды короче 1000 лет по современному високосному календарю Урантии.
\vs p014 1:13 Этот день Рая\hyp{}Хавоны является стандартной единицей измерения времени для семи сверхвселенных, хотя каждая из них поддерживает свои собственные внутренние стандарты времени.
\vs p014 1:14 \pc На окраине этой обширной центральной вселенной, далеко за пределами седьмого пояса миров Хавоны, кружится невероятное число огромных тёмных гравитационных тел. Эти многочисленные тёмные массы во многом совершенно не похожи на другие космические тела, отличаясь даже по форме. Тёмные гравитационные тела не отражают и не поглощают свет; они не реагируют на свет физической энергии и так плотно окружают и окутывают Хавону, что скрывают её от взора даже близлежащих обитаемых вселенных времени и пространства.
\vs p014 1:15 Огромный пояс тёмных гравитационных тел разделён на два равных эллиптических контура внедрением уникального пространства. Внутренний пояс вращается против, внешний --- по часовой стрелке. Противоположные направления движения в сочетании с колоссальной массой тёмных тел настолько эффективно уравновешивают линии гравитации Хавоны, что превращают центральную вселенную в физически сбалансированное и идеально стабилизированное творение.
\vs p014 1:16 Внутренний ряд тёмных гравитационных тел имеет трубчатую форму и состоит из трёх круговых групп. Поперечное сечение этого контура представляет собой три концентрических круга примерно равной плотности. Внешний контур тёмных гравитационных тел расположен перпендикулярно и в 10\,000 раз выше внутреннего контура. Вертикальный диаметр внешнего контура в 50\,000 раз превышает его поперечный диаметр.
\vs p014 1:17 Промежуточное пространство, существующее между этими двумя контурами гравитационных тел, \bibemph{уникально} тем, что нигде во всей обширной вселенной больше нет ничего подобного. Эта зона характеризуется огромными волновыми движениями вверх и вниз и пронизана огромной энергетической активностью неизвестной природы.
\vs p014 1:18 По нашему мнению, ничто, подобное тёмным гравитационным телам центральной вселенной, не будет характерно для будущей эволюции внешних пространственных уровней; мы считаем эти противоположно движущиеся ряды громадных уравновешивающих гравитацию тел уникальными в главной вселенной.
\usection{УСТРОЙСТВО ХАВОНЫ}
\vs p014 2:1 Духовные существа не обитают в туманном пространстве; они не населяют эфирные миры; они поселяются на настоящих сферах материальной природы, таких же реальных, как те, на которых живут смертные. Миры Хавоны являются актуальными и буквальными, хотя их буквальная субстанция отличается от материальной организации планет семи сверхвселенных.
\vs p014 2:2 Физические реальности Хавоны представляют собой порядок организации энергии, радикально отличающийся от любого существующего в эволюционных вселенных пространства. Энергии Хавоны тройственны; сверхвселенские единицы энергии\hyp{}материи содержат двойной энергетический заряд, хотя одна форма энергии существует в отрицательной и положительной фазах. Сотворённое в центральной вселенной тройственно (Троица); то, что сотворено в локальной вселенной (напрямую) Сыном Создателем и Созидательным Духом --- двойственно.
\vs p014 2:3 Материя Хавоны состоит ровно из 1000 базовых химических элементов и сбалансированного функционирования семи форм энергии Хавоны. Каждый из этих основных видов энергии проявляет семь фаз возбуждения, так что уроженцы Хавоны реагируют на 49 различных возбудителей ощущений. Другими словами, с чисто физической точки зрения, уроженцы центральной вселенной обладают 49 специфическими формами ощущений. Моронтийных чувств всего 70, а более высокие духовные типы реакций варьируются у различных существ от 70 до 210.
\vs p014 2:4 Ни одно из физических существ центральной вселенной не было бы видимым для жителей Урантии. И ни один из физических раздражителей тех далёких миров не вызвал бы реакции в ваших грубых органах чувств. Если смертного Урантии можно было бы переместить в Хавону, он был бы там глухим, слепым и совершенно лишённым всех других ощущений; он мог бы функционировать только как ограниченное самосознающее существо, лишённое всех раздражителей окружающей среды и каких\hyp{}либо реакций на неё.
\vs p014 2:5 \pc В центральном творении происходят многочисленные физические явления и духовные реакции, неизвестные на мирах, подобных Урантии. Базовая организация тройственного творения совершенно не похожа на двойственную структуру созданных вселенных времени и пространства.
\vs p014 2:6 Все естественные законы согласованы на совершенно иной основе, чем в двойственных системах энергии развивающихся творений. Вся центральная вселенная организована в соответствии с тройственной системой совершенного и симметричного управления. По всей системе Рай\hyp{}Хавона поддерживается идеальный баланс между всеми космическими реальностями и всеми духовными силами. Рай, с его абсолютным охватом материального творения, идеально регулирует и поддерживает физические энергии этой центральной вселенной; Вечный Сын, в качестве части своего всеобъемлющего охвата духа, в полном совершенстве поддерживает духовный статус всех обитателей Хавоны. На Рае нет ничего экспериментального, а система Рай\hyp{}Хавона --- единое целое созидательного совершенства.
\vs p014 2:7 Всеобщая духовная гравитация Вечного Сына невероятно активна по всей центральной вселенной. Все ценности духа и духовные личности непрестанно притягиваются внутрь, к обители Богов. Это влечение к Богу интенсивно и неизбежно. Стремление достичь Бога в центральной вселенной сильнее не потому, что гравитация духа там сильнее, чем в отдалённых вселенных, а потому, что те существа, которые достигли Хавоны, являются более полностью одухотворёнными и, следовательно, более восприимчивыми к вездесущему действию всеобщего притяжения гравитации духа Вечного Сына.
\vs p014 2:8 Точно так же Бесконечный Дух притягивает все интеллектуальные ценности к Раю. По всей центральной вселенной гравитация разума Бесконечного Духа действует во взаимосвязи с гравитацией духа Вечного Сына, и вместе они составляют объединённое стремление восходящих душ найти Бога, достичь Божества, достичь Рая и познать Отца.
\vs p014 2:9 \pc Хавона --- это духовно совершенная и физически устойчивая вселенная. Управление и сбалансированная стабильность центральной вселенной представляются совершенными. Всё физическое или духовное совершенно предсказуемо, но это не так в отношении феноменов разума и волеизъявления личности. Мы приходим к выводу, что грех здесь не может произойти, но этот вывод основан на том, что обладающие свободой воли исконные создания Хавоны никогда не были виновны в нарушении воли Божества. На протяжении всей вечности эти небесные существа были неизменно верны От Века Вечным. Не было греха и ни в одном создании, когда\hyp{}либо вошедшем в Хавону пилигримом. Никогда не было случая проступка со стороны какого\hyp{}либо создания любой группы личностей, когда\hyp{}либо созданного в центральной вселенной Хавоне или допущенного в неё. Столь совершенны и божественны методы и средства отбора во вселенных времени, что ни разу в записях Хавоны не отмечалось погрешности; не было допущено ни одной ошибки; ни одна восходящая душа не была допущена в центральную вселенную преждевременно.
\usection{МИРЫ ХАВОНЫ}
\vs p014 3:1 Что касается правительства центральной вселенной, то его нет. Хавона столь изысканно совершенна, что не требуется никакой интеллектуальной системы управления. Нет ни регулярно учреждаемых судов, ни законодательных собраний; Хавоне требуется только административное руководство. Здесь наблюдаются высочайшие идеалы истинного \bibemph{само}\hyp{}управления.
\vs p014 3:2 Для столь совершенных и близких к совершенству интеллектов нет необходимости в правительстве. Они не нуждаются в предписаниях, ибо они --- существа врождённого совершенства, среди которых есть и эволюционные создания, давно прошедшие испытательную проверку верховных судов сверхвселенных.
\vs p014 3:3 Управление Хавоной не происходит автоматически, но оно удивительно совершенно и божественно эффективно. Административными полномочиями, главным образом планетарными, наделён постоянно проживающий От Века Вечный, причём каждая сфера Хавоны направляется одной из таких личностей, происходящих от Троицы. От Века Вечные не являются создателями, но они --- совершенные администраторы. Они учат с высочайшим мастерством и направляют своих планетных детей с совершенством мудрости, граничащим с абсолютностью.
\vs p014 3:4 Миллиард сфер центральной вселенной --- это учебные миры высших личностей, уроженцев Рая и Хавоны; кроме того, здесь проходят окончательное испытание восходящие создания эволюционных миров времени. Во исполнение великого плана Всеобщего Отца по восхождению созданий, пилигримы времени высаживаются на принимающих мирах внешнего, или седьмого контура, и после усиленного обучения и приобретения расширенного опыта они постепенно продвигаются внутрь, планета за планетой, контур за контуром, пока, наконец, не достигнут Божеств и постоянного местожительства на Рае.
\vs p014 3:5 Хотя в настоящее время сферы семи контуров поддерживаются во всей своей божественной славе, только около 1\% всех планетарных возможностей используется для продвижения вселенского плана Отца по восхождению смертных. Около 0,1\% площади этих огромных миров посвящено жизни и деятельности Корпуса Завершения, существ, вечно обитающих в свете и жизни, которые часто пребывают и служат на мирах Хавоны. Эти возвышенные существа имеют свои личные резиденции на Рае.
\vs p014 3:6 Планетарное устройство сфер Хавоны совершенно непохоже на устройство эволюционных миров и систем пространства. Более нигде во всей большой вселенной не целесообразно использовать такие огромные сферы в качестве обитаемых миров. Физическое устройство триаты в сочетании с балансирующим эффектом огромных тёмных гравитационных тел, позволяет идеально выравнивать физические силы и изысканно уравновешивать различные притяжения этого огромного творения. В организации материальных функций и духовной деятельности этих огромных миров используется также антигравитация.
\vs p014 3:7 Архитектура, освещение и обогрев, а также биологические и художественные украшения сфер Хавоны превосходят наивысший предел человеческого воображения. Невозможно многое рассказать тебе о Хавоне; чтобы понять её красоту и величие, тебе нужно её увидеть. Но на этих совершенных мирах есть настоящие реки и озера.
\vs p014 3:8 Духовное назначение этих миров идеально; они прекрасно служат надёжной гаванью многочисленным типам различных существ, функционирующих в центральной вселенной. На этих прекрасных мирах протекает самая разнообразная деятельность, намного превышающая человеческое понимание.
\usection{СОЗДАНИЯ ЦЕНТРАЛЬНОЙ ВСЕЛЕННОЙ}
\vs p014 4:1 
\vs p014 4:2 
\vs p014 4:3 
\vs p014 4:4 
\vs p014 4:5 
\vs p014 4:6 
\vs p014 4:7 
\vs p014 4:8 
\vs p014 4:9 \pc 
\vs p014 4:10 \pc 
\vs p014 4:11 
\vs p014 4:12 
\vs p014 4:13 
\vs p014 4:14 
\vs p014 4:15 
\vs p014 4:16 
\vs p014 4:17 
\vs p014 4:18 \pc 
\vs p014 4:19 
\vs p014 4:20 
\vs p014 4:21 
\vs p014 4:22 
\usection{ЖИЗНЬ В ХАВОНЕ}
\vs p014 5:1 
\vs p014 5:2 
\vs p014 5:3 
\vs p014 5:4 \pc 
\vs p014 5:5 
\vs p014 5:6 \pc 
\vs p014 5:7 
\vs p014 5:8 
\vs p014 5:9 
\vs p014 5:10 
\vs p014 5:11 
\usection{НАЗНАЧЕНИЕ ЦЕНТРАЛЬНОЙ ВСЕЛЕННОЙ}
\vs p014 6:1 
\vs p014 6:2 
\vs p014 6:3 
\vs p014 6:4 
\vs p014 6:5 \pc 
\vs p014 6:6 
\vs p014 6:7 
\vs p014 6:8 
\vs p014 6:9 
\vs p014 6:10 
\vs p014 6:11 
\vs p014 6:12 
\vs p014 6:13 
\vs p014 6:14 
\vs p014 6:15 
\vs p014 6:16 
\vs p014 6:17 
\vs p014 6:18 
\vs p014 6:19 
\vs p014 6:20 
\vs p014 6:21 
\vs p014 6:22 
\vs p014 6:23 
\vs p014 6:24 
\vs p014 6:25 
\vs p014 6:26 
\vs p014 6:27 
\vs p014 6:28 
\vs p014 6:29 
\vs p014 6:30 
\vs p014 6:31 
\vs p014 6:32 
\vs p014 6:33 
\vs p014 6:34 
\vs p014 6:35 
\vs p014 6:36 
\vs p014 6:37 
\vs p014 6:38 
\vs p014 6:39 
\vs p014 6:40 
\vs p014 6:41 \pc 
\vsetoff
\vs p014 6:42 
\quizlink
