\upaper{14}{ЦЕНТРАЛЬНАЯ И БОЖЕСТВЕННАЯ ВСЕЛЕННАЯ}
\uminitoc{СИСТЕМА РАЙ\hyp{}ХАВОНА}
\uminitoc{УСТРОЙСТВО ХАВОНЫ}
\uminitoc{МИРЫ ХАВОНЫ}
\uminitoc{СОЗДАНИЯ ЦЕНТРАЛЬНОЙ ВСЕЛЕННОЙ}
\uminitoc{ЖИЗНЬ В ХАВОНЕ}
\uminitoc{НАЗНАЧЕНИЕ ЦЕНТРАЛЬНОЙ ВСЕЛЕННОЙ}
\author{Совершенствователь Мудрости}
\vs p014 0:1 Совершенная и божественная вселенная занимает центр всего творения; она --- вечная сердцевина, вокруг которой вращаются обширные творения времени и пространства. Рай --- это гигантский Остров, ядро абсолютной стабильности, который неподвижно покоится в самом сердце великолепной вечной вселенной. Это центральное планетарное семейство, называемое Хавоной, значительно удалено от локальной вселенной Небадон. Оно имеет громадные размеры, почти невероятную массу и состоит из миллиарда сфер невообразимой красоты и непревзойдённого величия, но истинные размеры этого огромного творения находятся за пределами понимания человеческого разума.
\vs p014 0:2 Это --- одно единственное стабильное, совершенное и утверждённое скопление миров. Это --- целиком созданная и совершенная вселенная; она не является результатом эволюционного развития. Это --- вечное ядро совершенства, вокруг которого кружится бесконечная процессия вселенных, составляющих грандиозный эволюционный эксперимент, смелое приключение Божьих Сынов Создателей, которые стремятся повторить во времени и воспроизвести в пространстве вселенную\hyp{}образец, идеал божественной завершённости, верховной законченности, предельной реальности и вечного совершенства.
\usection{СИСТЕМА РАЙ\hyp{}ХАВОНА}
\vs p014 1:1 Между периферией Рая и внутренними границами семи сверхвселенных существуют следующие семь пространственных состояний и движений:
\vs p014 1:2 \li{1.}Спокойные зоны промежуточного пространства, соприкасающиеся с Раем.
\vs p014 1:3 \li{2.}Движущиеся по часовой стрелке три контура Рая и семь контуров Хавоны.
\vs p014 1:4 \li{3.}Зона пространства относительного покоя, отделяющая контуры Хавоны от тёмных гравитационных тел центральной вселенной.
\vs p014 1:5 \li{4.}Внутренний, движущийся против часовой стрелки, пояс тёмных гравитационных тел.
\vs p014 1:6 \li{5.}Вторая уникальная зона пространства, разделяющая две пространственные траектории тёмных гравитационных тел.
\vs p014 1:7 \li{6.}Внешний пояс тёмных гравитационных тел, вращающихся по часовой стрелке вокруг Рая.
\vs p014 1:8 \li{7.}Третья зона пространства --- зона относительного покоя --- отделяющая внешний пояс тёмных гравитационных тел от внутренних контуров семи сверхвселенных.
\vs p014 1:9 \pc Миллиард миров Хавоны организован в семь концентрических контуров, непосредственно окружающих три контура спутников Рая. Более 35\,000\,000 миров находятся во внутреннем контуре Хавоны и более 245\,000\,000 --- на внешнем, с пропорциональными числами между ними. Каждый контур отличается от других, но все они идеально сбалансированы и изысканно организованы, и каждый пронизан особым представительством Бесконечного Духа --- одним из Семи Духов Контуров. В дополнение к другим функциям этот неличностный Дух координирует ведение небесных дел в каждом контуре.
\vs p014 1:10 Планетарные контуры Хавоны не накладываются друг на друга; их миры следуют друг за другом упорядоченными линейными рядами. Центральная вселенная вращается вокруг неподвижного Острова Рай в одной гигантской плоскости, состоящей из десяти концентрических стабилизированных единиц --- трёх контуров сфер Рая и семи контуров миров Хавоны. С физической точки зрения контуры Хавоны и Рая --- это одна и та же система; их разделение проявляется только в признании их функциональной и административной сегрегации.
\vs p014 1:11 \pc Счёт времени не ведётся на Рае; последовательность чередующихся событий заложена в концепции коренных жителей центрального Острова. Но время уместно для контуров Хавоны и для многочисленных существ как небесного, так и земного происхождения, пребывающих на них. Каждый мир Хавоны имеет своё местное время, определяемое его контуром. Все миры в данном контуре имеют одинаковую продолжительность года, поскольку они равномерно обращаются вокруг Рая, и продолжительность этих планетных лет уменьшается от внешнего контура к самому внутреннему.
\vs p014 1:12 Кроме контурного времени Хавоны, существует стандартный день системы Рай\hyp{}Хавона, но есть и другие показатели времени, которые определяются на семи Райских спутниках Бесконечного Духа и передаются оттуда. Стандартный день системы Рай\hyp{}Хавона основан на продолжительности времени, необходимом для планетарных обителей первого или внутреннего контура Хавоны, чтобы совершить один оборот вокруг Острова Рай; и хотя их скорость огромна из\hyp{}за их положения между тёмными гравитационными телами и гигантским Раем, этим сферам требуется почти 1000 лет, чтобы завершить свой круг. Твоему взору невольно открывается истина, когда ты читаешь утверждение: <<День у Бога как тысяча лет, как стража в ночи>>\fnst{<<Ибо пред очами Твоими тысяча лет, как день вчерашний, когда он прошёл, и \bibemph{как} стража в ночи>>. Псалом 89:5 (Синодальный перевод).}. Один день системы Рай\hyp{}Хавона всего на 7 минут, 3\bibfrac{1}{8} секунды короче 1000 лет по современному високосному календарю Урантии.
\vs p014 1:13 Этот день Рая\hyp{}Хавоны является стандартной единицей измерения времени для семи сверхвселенных, хотя каждая из них поддерживает свои собственные внутренние стандарты времени.
\vs p014 1:14 \pc На окраине этой обширной центральной вселенной, далеко за пределами седьмого пояса миров Хавоны, кружится невероятное число огромных тёмных гравитационных тел. Эти многочисленные тёмные массы во многом совершенно не похожи на другие космические тела, отличаясь даже по форме. Тёмные гравитационные тела не отражают и не поглощают свет; они не реагируют на свет физической энергии и так плотно окружают и окутывают Хавону, что скрывают её от взора даже близлежащих обитаемых вселенных времени и пространства.
\vs p014 1:15 Огромный пояс тёмных гравитационных тел разделён на два равных эллиптических контура внедрением уникального пространства. Внутренний пояс вращается против, внешний --- по часовой стрелке. Противоположные направления движения в сочетании с колоссальной массой тёмных тел настолько эффективно уравновешивают линии гравитации Хавоны, что превращают центральную вселенную в физически сбалансированное и идеально стабилизированное творение.
\vs p014 1:16 Внутренний ряд тёмных гравитационных тел имеет трубчатую форму и состоит из трёх круговых групп. Поперечное сечение этого контура представляет собой три концентрических круга примерно равной плотности. Внешний контур тёмных гравитационных тел расположен перпендикулярно и в 10\,000 раз выше внутреннего контура. Вертикальный диаметр внешнего контура в 50\,000 раз превышает его поперечный диаметр.
\vs p014 1:17 Промежуточное пространство, существующее между этими двумя контурами гравитационных тел, \bibemph{уникально} тем, что нигде во всей обширной вселенной больше нет ничего подобного. Эта зона характеризуется огромными волновыми движениями вверх и вниз и пронизана огромной энергетической активностью неизвестной природы.
\vs p014 1:18 По нашему мнению, ничто, подобное тёмным гравитационным телам центральной вселенной, не будет характерно для будущей эволюции внешних пространственных уровней; мы считаем эти противоположно движущиеся ряды громадных уравновешивающих гравитацию тел уникальными в главной вселенной.
\usection{УСТРОЙСТВО ХАВОНЫ}
\vs p014 2:1 
\vs p014 2:2 
\vs p014 2:3 
\vs p014 2:4 
\vs p014 2:5 \pc 
\vs p014 2:6 
\vs p014 2:7 
\vs p014 2:8 
\vs p014 2:9 \pc 
\usection{МИРЫ ХАВОНЫ}
\vs p014 3:1 
\vs p014 3:2 
\vs p014 3:3 
\vs p014 3:4 
\vs p014 3:5 
\vs p014 3:6 
\vs p014 3:7 
\vs p014 3:8 
\usection{СОЗДАНИЯ ЦЕНТРАЛЬНОЙ ВСЕЛЕННОЙ}
\vs p014 4:1 
\vs p014 4:2 
\vs p014 4:3 
\vs p014 4:4 
\vs p014 4:5 
\vs p014 4:6 
\vs p014 4:7 
\vs p014 4:8 
\vs p014 4:9 \pc 
\vs p014 4:10 \pc 
\vs p014 4:11 
\vs p014 4:12 
\vs p014 4:13 
\vs p014 4:14 
\vs p014 4:15 
\vs p014 4:16 
\vs p014 4:17 
\vs p014 4:18 \pc 
\vs p014 4:19 
\vs p014 4:20 
\vs p014 4:21 
\vs p014 4:22 
\usection{ЖИЗНЬ В ХАВОНЕ}
\vs p014 5:1 
\vs p014 5:2 
\vs p014 5:3 
\vs p014 5:4 \pc 
\vs p014 5:5 
\vs p014 5:6 \pc 
\vs p014 5:7 
\vs p014 5:8 
\vs p014 5:9 
\vs p014 5:10 
\vs p014 5:11 
\usection{НАЗНАЧЕНИЕ ЦЕНТРАЛЬНОЙ ВСЕЛЕННО}
\vs p014 6:1 
\vs p014 6:2 
\vs p014 6:3 
\vs p014 6:4 
\vs p014 6:5 \pc 
\vs p014 6:6 
\vs p014 6:7 
\vs p014 6:8 
\vs p014 6:9 
\vs p014 6:10 
\vs p014 6:11 
\vs p014 6:12 
\vs p014 6:13 
\vs p014 6:14 
\vs p014 6:15 
\vs p014 6:16 
\vs p014 6:17 
\vs p014 6:18 
\vs p014 6:19 
\vs p014 6:20 
\vs p014 6:21 
\vs p014 6:22 
\vs p014 6:23 
\vs p014 6:24 
\vs p014 6:25 
\vs p014 6:26 
\vs p014 6:27 
\vs p014 6:28 
\vs p014 6:29 
\vs p014 6:30 
\vs p014 6:31 
\vs p014 6:32 
\vs p014 6:33 
\vs p014 6:34 
\vs p014 6:35 
\vs p014 6:36 
\vs p014 6:37 
\vs p014 6:38 
\vs p014 6:39 
\vs p014 6:40 
\vs p014 6:41 \pc 
\vsetoff
\vs p014 6:42 
\quizlink
