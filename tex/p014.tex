\upaper{14}{ЦЕНТРАЛЬНАЯ И БОЖЕСТВЕННАЯ ВСЕЛЕННАЯ}
\uminitoc{СИСТЕМА РАЙ\hyp{}ХАВОНА}
\uminitoc{УСТРОЙСТВО ХАВОНЫ}
\uminitoc{МИРЫ ХАВОНЫ}
\uminitoc{СОЗДАНИЯ ЦЕНТРАЛЬНОЙ ВСЕЛЕННОЙ}
\uminitoc{ЖИЗНЬ В ХАВОНЕ}
\uminitoc{НАЗНАЧЕНИЕ ЦЕНТРАЛЬНОЙ ВСЕЛЕННОЙ}
\author{Совершенствователь Мудрости}
\vs p014 0:1 Совершенная и божественная вселенная занимает центр всего творения; она~--- вечная сердцевина, вокруг которой вращаются обширные творения времени и пространства. Рай~--- это гигантский Остров, ядро абсолютной стабильности, который неподвижно покоится в самом сердце великолепной вечной вселенной. Это центральное планетарное семейство, называемое Хавоной, значительно удалено от локальной вселенной Небадон. Оно имеет громадные размеры, почти невероятную массу и состоит из миллиарда сфер невообразимой красоты и непревзойдённого величия, но истинные размеры этого огромного творения находятся за пределами понимания человеческого разума.
\vs p014 0:2 Это~--- одно единственное стабильное, совершенное и утверждённое скопление миров. Это~--- целиком созданная и совершенная вселенная; она не является результатом эволюционного развития. Это~--- вечное ядро совершенства, вокруг которого кружится бесконечная процессия вселенных, составляющих грандиозный эволюционный эксперимент, смелое приключение Божьих Сынов Создателей, которые стремятся повторить во времени и воспроизвести в пространстве вселенную\hyp{}образец, идеал божественной завершённости, верховной законченности, предельной реальности и вечного совершенства.
\usection{СИСТЕМА РАЙ\hyp{}ХАВОНА}
\vs p014 1:1 Между периферией Рая и внутренними границами семи сверхвселенных существуют следующие семь пространственных состояний и движений:
\vs p014 1:2 \li{1.}Спокойные зоны промежуточного пространства, соприкасающиеся с Раем.
\vs p014 1:3 \li{2.}Движущиеся по часовой стрелке три контура Рая и семь контуров Хавоны.
\vs p014 1:4 \li{3.}Зона пространства относительного покоя, отделяющая контуры Хавоны от тёмных гравитационных тел центральной вселенной.
\vs p014 1:5 \li{4.}Внутренний, движущийся против часовой стрелки, пояс тёмных гравитационных тел.
\vs p014 1:6 \li{5.}Вторая уникальная зона пространства, разделяющая две пространственные траектории тёмных гравитационных тел.
\vs p014 1:7 \li{6.}Внешний пояс тёмных гравитационных тел, вращающихся по часовой стрелке вокруг Рая.
\vs p014 1:8 \li{7.}Третья зона пространства~--- зона относительного покоя~--- отделяющая внешний пояс тёмных гравитационных тел от внутренних контуров семи сверхвселенных.
\vs p014 1:9 \pc Миллиард миров Хавоны организован в семь концентрических контуров, непосредственно окружающих три контура Райских спутников. Более 35\,000\,000 миров находятся во внутреннем контуре Хавоны и более 245\,000\,000~--- на внешнем, с пропорциональными числами между ними. Каждый контур отличается от других, но все они идеально сбалансированы и изысканно организованы, и каждый пронизан особым представительством Бесконечного Духа~--- одним из Семи Духов Контуров. В дополнение к другим функциям этот неличностный Дух координирует ведение небесных дел в каждом контуре.
\vs p014 1:10 Планетарные контуры Хавоны не накладываются друг на друга; их миры следуют друг за другом упорядоченными линейными рядами. Центральная вселенная вращается вокруг неподвижного Острова Рай в одной гигантской плоскости, состоящей из десяти концентрических стабилизированных единиц~--- трёх контуров сфер Рая и семи контуров миров Хавоны. С физической точки зрения контуры Хавоны и Рая~--- это одна и та же система; их разделение проявляется только в признании их функциональной и административной сегрегации.
\vs p014 1:11 \pc Счёт времени не ведётся на Рае; последовательность чередующихся событий заложена в концепции коренных жителей центрального Острова. Но время уместно для контуров Хавоны и для многочисленных существ как небесного, так и земного происхождения, пребывающих на них. Каждый мир Хавоны имеет своё местное время, определяемое его контуром. Все миры в данном контуре имеют одинаковую продолжительность года, поскольку они равномерно обращаются вокруг Рая, и продолжительность этих планетных лет уменьшается от внешнего контура к самому внутреннему.
\vs p014 1:12 Кроме контурного времени Хавоны, существует стандартный день системы Рай\hyp{}Хавона, но есть и другие показатели времени, которые определяются на семи Райских спутниках Бесконечного Духа и передаются оттуда. Стандартный день системы Рай\hyp{}Хавона основан на продолжительности времени, необходимом для планетарных обителей первого или внутреннего контура Хавоны, чтобы совершить один оборот вокруг Острова Рай; и хотя их скорость огромна из\hyp{}за их положения между тёмными гравитационными телами и гигантским Раем, этим сферам требуется почти 1000 лет, чтобы завершить свой круг. Твоему взору невольно открывается истина, когда ты читаешь утверждение: <<День у Бога как тысяча лет, как стража в ночи>>\fnst{<<Ибо пред очами Твоими тысяча лет, как день вчерашний, когда он прошёл, и \bibemph{как} стража в ночи>>. Псалом 89:5 (Синодальный перевод).}. Один день системы Рай\hyp{}Хавона всего на 7 минут, 3\bibfrac{1}{8} секунды короче 1000 лет по современному високосному календарю Урантии.
\vs p014 1:13 Этот день Рая\hyp{}Хавоны является стандартной единицей измерения времени для семи сверхвселенных, хотя каждая из них поддерживает свои собственные внутренние стандарты времени.
\vs p014 1:14 \pc На окраине этой обширной центральной вселенной, далеко за пределами седьмого пояса миров Хавоны, кружится невероятное число огромных тёмных гравитационных тел. Эти многочисленные тёмные массы во многом совершенно не похожи на другие космические тела, отличаясь даже по форме. Тёмные гравитационные тела не отражают и не поглощают свет; они не реагируют на свет физической энергии и так плотно окружают и окутывают Хавону, что скрывают её от взора даже близлежащих обитаемых вселенных времени и пространства.
\vs p014 1:15 Огромный пояс тёмных гравитационных тел разделён на два равных эллиптических контура внедрением уникального пространства. Внутренний пояс вращается против, внешний~--- по часовой стрелке. Противоположные направления движения в сочетании с колоссальной массой тёмных тел настолько эффективно уравновешивают линии гравитации Хавоны, что превращают центральную вселенную в физически сбалансированное и идеально стабилизированное творение.
\vs p014 1:16 Внутренний ряд тёмных гравитационных тел имеет трубчатую форму и состоит из трёх круговых групп. Поперечное сечение этого контура представляет собой три концентрических круга примерно равной плотности. Внешний контур тёмных гравитационных тел расположен перпендикулярно и в 10\,000 раз выше внутреннего контура. Вертикальный диаметр внешнего контура в 50\,000 раз превышает его поперечный диаметр.
\vs p014 1:17 Промежуточное пространство, существующее между этими двумя контурами гравитационных тел, \bibemph{уникально} тем, что нигде во всей обширной вселенной больше нет ничего подобного. Эта зона характеризуется огромными волновыми движениями вверх и вниз и пронизана огромной энергетической активностью неизвестной природы.
\vs p014 1:18 По нашему мнению, ничто, подобное тёмным гравитационным телам центральной вселенной, не будет характерно для будущей эволюции внешних пространственных уровней; мы считаем эти противоположно движущиеся ряды громадных уравновешивающих гравитацию тел уникальными в главной вселенной.
\usection{УСТРОЙСТВО ХАВОНЫ}
\vs p014 2:1 Духовные существа не обитают в туманном пространстве; они не населяют эфирные миры; они поселяются на настоящих сферах материальной природы, таких же реальных, как те, на которых живут смертные. Миры Хавоны являются актуальными и буквальными, хотя их буквальная субстанция отличается от материальной организации планет семи сверхвселенных.
\vs p014 2:2 Физические реальности Хавоны представляют собой порядок организации энергии, радикально отличающийся от любого существующего в эволюционных вселенных пространства. Энергии Хавоны тройственны; сверхвселенские единицы энергии\hyp{}материи содержат двойной энергетический заряд, хотя одна форма энергии существует в отрицательной и положительной фазах. Сотворённое в центральной вселенной тройственно (Троица); то, что сотворено в локальной вселенной (напрямую) Сыном Создателем и Созидательным Духом,~--- двойственно.
\vs p014 2:3 Материя Хавоны состоит ровно из 1000 базовых химических элементов и сбалансированного функционирования семи форм энергии Хавоны. Каждый из этих основных видов энергии проявляет семь фаз возбуждения, так что уроженцы Хавоны реагируют на 49 различных возбудителей ощущений. Другими словами, с чисто физической точки зрения уроженцы центральной вселенной обладают 49 специфическими формами ощущений. Моронтийных чувств всего 70, а более высокие духовные типы реакций варьируются у различных существ от 70 до 210.
\vs p014 2:4 Ни одно из физических существ центральной вселенной не было бы видимым для жителей Урантии. И ни один из физических раздражителей тех далёких миров не вызвал бы реакции в ваших грубых органах чувств. Если смертного Урантии можно было бы переместить в Хавону, он был бы там глухим, слепым и совершенно лишённым всех других ощущений; он мог бы функционировать только как ограниченное самосознающее существо, лишённое всех раздражителей окружающей среды и каких\hyp{}либо реакций на неё.
\vs p014 2:5 \pc В центральном творении происходят многочисленные физические явления и духовные реакции, неизвестные на мирах, подобных Урантии. Базовая организация тройственного творения совершенно не похожа на двойственную структуру созданных вселенных времени и пространства.
\vs p014 2:6 Все естественные законы согласованы на совершенно иной основе, чем в двойственных системах энергии развивающихся творений. Вся центральная вселенная организована в соответствии с тройственной системой совершенного и симметричного управления. По всей системе Рай\hyp{}Хавона поддерживается идеальный баланс между всеми космическими реальностями и всеми духовными силами. Рай, с его абсолютным охватом материального творения, идеально регулирует и поддерживает физические энергии этой центральной вселенной; Вечный Сын, в качестве части своего всеобъемлющего охвата духа, в полном совершенстве поддерживает духовный статус всех обитателей Хавоны. На Рае нет ничего экспериментального, а система Рай\hyp{}Хавона~--- единое целое созидательного совершенства.
\vs p014 2:7 Всеобщая духовная гравитация Вечного Сына невероятно активна по всей центральной вселенной. Все ценности духа и духовные личности непрестанно притягиваются внутрь, к обители Богов. Это влечение к Богу интенсивно и неизбежно. Стремление достичь Бога в центральной вселенной сильнее не потому, что гравитация духа там сильнее, чем в отдалённых вселенных, а потому, что те существа, которые достигли Хавоны, являются более полно одухотворёнными и, следовательно, более восприимчивыми к вездесущему действию всеобщего притяжения гравитации духа Вечного Сына.
\vs p014 2:8 Точно так же Бесконечный Дух притягивает все интеллектуальные ценности к Раю. По всей центральной вселенной гравитация разума Бесконечного Духа действует во взаимосвязи с гравитацией духа Вечного Сына, и вместе они составляют объединённое стремление восходящих душ найти Бога, достичь Божества, достичь Рая и познать Отца.
\vs p014 2:9 \pc Хавона~--- это духовно совершенная и физически устойчивая вселенная. Управление и сбалансированная стабильность центральной вселенной представляются совершенными. Всё физическое или духовное совершенно предсказуемо, но это не так в отношении феноменов разума и волеизъявления личности. Мы приходим к выводу, что грех здесь не может произойти, но этот вывод основан на том, что обладающие свободой воли исконные создания Хавоны никогда не были виновны в нарушении воли Божества. На протяжении всей вечности эти небесные существа были неизменно верны От Века Вечным. Не было греха и ни в одном создании, когда\hyp{}либо вошедшем в Хавону пилигримом. Никогда не было случая проступка со стороны какого\hyp{}либо создания любой группы личностей, когда\hyp{}либо созданного в центральной вселенной Хавоне или допущенного в неё. Столь совершенны и божественны методы и средства отбора во вселенных времени, что ни разу в записях Хавоны не отмечалось погрешности; не было допущено ни одной ошибки; ни одна восходящая душа не была допущена в центральную вселенную преждевременно.
\usection{МИРЫ ХАВОНЫ}
\vs p014 3:1 Что касается правительства центральной вселенной, то его нет. Хавона столь изысканно совершенна, что не требуется никакой интеллектуальной системы управления. Нет ни регулярно учреждаемых судов, ни законодательных собраний; Хавоне требуется только административное руководство. Здесь наблюдаются высочайшие идеалы истинного \bibemph{само}\hyp{}управления.
\vs p014 3:2 Для столь совершенных и близких к совершенству интеллектов нет необходимости в правительстве. Они не нуждаются в предписаниях, ибо они~--- существа врождённого совершенства, среди которых есть и эволюционные создания, давно прошедшие испытательную проверку верховных судов сверхвселенных.
\vs p014 3:3 Управление Хавоной не происходит автоматически, но оно удивительно совершенно и божественно эффективно. Административными полномочиями, главным образом планетарными, наделён постоянно проживающий От Века Вечный, причём каждая сфера Хавоны направляется одной из таких личностей, происходящих от Троицы. От Века Вечные не являются создателями, но они~--- совершенные администраторы. Они учат с высочайшим мастерством и направляют своих планетных детей с совершенством мудрости, граничащим с абсолютностью.
\vs p014 3:4 Миллиард сфер центральной вселенной~--- это учебные миры высших личностей, уроженцев Рая и Хавоны; кроме того, здесь проходят окончательное испытание восходящие создания эволюционных миров времени. Во исполнение великого плана Всеобщего Отца по восхождению созданий пилигримы времени высаживаются на принимающих мирах внешнего, или седьмого контура, и после усиленного обучения и приобретения расширенного опыта они постепенно продвигаются внутрь, планета за планетой, контур за контуром, пока, наконец, не достигнут Божеств и постоянного местожительства на Рае.
\vs p014 3:5 Хотя в настоящее время сферы семи контуров поддерживаются во всей своей божественной славе, только около 1\% всех планетарных возможностей используется для продвижения вселенского плана Отца по восхождению смертных. Около 0,1\% площади этих огромных миров посвящено жизни и деятельности Корпуса Завершения, существ, вечно обитающих в свете и жизни, которые часто пребывают и служат на мирах Хавоны. Эти возвышенные существа имеют свои личные резиденции на Рае.
\vs p014 3:6 Планетарное устройство сфер Хавоны совершенно непохоже на устройство эволюционных миров и систем пространства. Более нигде во всей большой вселенной не целесообразно использовать такие огромные сферы в качестве обитаемых миров. Физическое устройство триаты в сочетании с балансирующим эффектом огромных тёмных гравитационных тел позволяет идеально выравнивать физические силы и изысканно уравновешивать различные притяжения этого огромного творения. В организации материальных функций и духовной деятельности этих огромных миров используется также антигравитация.
\vs p014 3:7 Архитектура, освещение и обогрев, а также биологические и художественные украшения сфер Хавоны превосходят наивысший предел человеческого воображения. Невозможно многое рассказать тебе о Хавоне; чтобы понять её красоту и величие, тебе нужно её увидеть. Но на этих совершенных мирах есть настоящие реки и озёра.
\vs p014 3:8 Духовное назначение этих миров идеально; они прекрасно служат надёжной гаванью многочисленным типам различных существ, функционирующих в центральной вселенной. На этих прекрасных мирах протекает самая разнообразная деятельность, намного превышающая человеческое понимание.
\usection{СОЗДАНИЯ ЦЕНТРАЛЬНОЙ ВСЕЛЕННОЙ}
\vs p014 4:1 На мирах Хавоны есть семь основных форм живых сущностей и существ, и каждая из этих основных форм существует в трёх различных фазах. Каждая из этих трёх фаз разделена на 70 больших подразделений, каждое большое подразделение состоит из 1000 малых подразделений с ещё другими подразделениями и так далее. Эти основные группы жизни могут быть классифицированы как:
\vs p014 4:2 \li{1.}Материальная.
\vs p014 4:3 \li{2.}Моронтийная.
\vs p014 4:4 \li{3.}Духовная.
\vs p014 4:5 \li{4.}Абсонитная.
\vs p014 4:6 \li{5.}Предельная.
\vs p014 4:7 \li{6.}Коабсолютная.
\vs p014 4:8 \li{7.}Абсолютная.
\vs p014 4:9 \pc Распад и смерть не являются частью цикла жизни на мирах Хавоны. В центральной вселенной низшие живые существа претерпевают преобразование материализации. Они просто меняют форму и проявление, но не исчезают в результате процесса распада и клеточной смерти.
\vs p014 4:10 \pc Все уроженцы Хавоны~--- потомки Райской Троицы. У них нет родителей\hyp{}созданий, и они не воспроизводятся. Мы не можем описать создание этих граждан центральной вселенной, существ, которые никогда не были созданы. Весь рассказ о создании Хавоны~--- это попытка отобразить в терминах время\hyp{}пространства [to time\hyp{}space] факт вечности, не имеющий отношения ни к времени, ни к пространству в том смысле, как их понимает смертный человек. Но мы должны сделать уступку человеческой философии в вопросе происхождения; даже личности, намного превышающие человеческий уровень, нуждаются в концепции <<начала>>. Тем не менее система Рай\hyp{}Хавона вечна.
\vs p014 4:11 Уроженцы Хавоны живут на миллиарде сфер центральной вселенной в том же смысле, в каком другие категории постоянного гражданства обитают на своих соответствующих сферах рождения. Как материальная категория сыновства поддерживает материальное, интеллектуальное и духовное хозяйство миллиарда локальных систем в сверхвселенной, так, в более широком смысле, уроженцы Хавоны живут и функционируют на миллиарде миров центральной вселенной. Ты мог бы считать обитателей Хавоны материальными существами при условии, что смысл слова <<материальный>> был бы расширен для описания физических реальностей божественной вселенной.
\vs p014 4:12 Есть жизнь, присущая только Хавоне и имеющая значение сама по себе. Обитатели Хавоны разными способами служат нисходящим из Рая и восходящим из сверхвселенных, но они живут также уникальной для центральной вселенной жизнью, обладающей относительным смыслом независимо от Рая или сверхвселенных.
\vs p014 4:13 Как поклонение сынов веры эволюционных миров служит удовлетворению любви Всеобщего Отца, так и возвышенное обожание уроженцев Хавоны насыщает совершенные идеалы божественной красоты и истины. Как смертный человек стремится исполнять волю Бога, так и эти существа центральной вселенной живут, чтобы удовлетворять идеалы Райской Троицы. По самой своей природе они \bibemph{есть} воля Бога. Человек радуется доброте Бога, хавонцы ликуют в божественной красоте, в то время как вы вместе наслаждаетесь служением свободы живой истины.
\vs p014 4:14 У жителей Хавоны есть как необязательное настоящее, так и будущее нераскрытое предназначение. И существует путь развития местных созданий, характерный для центральной вселенной, не включающий в себя ни восхождение к Раю, ни проникновение в сверхвселенные. Это развитие к более высокому статусу в Хавоне можно представить следующим образом:
\vs p014 4:15 \li{1.}Эмпирическое продвижение наружу~--- от первого контура к седьмому.
\vs p014 4:16 \li{2.}Продвижение вовнутрь~--- от седьмого контура к первому.
\vs p014 4:17 \li{3.}Внутриконтурное продвижение~--- развитие в пределах миров данного контура.
\vs p014 4:18 \pc Кроме уроженцев Хавоны, жители центральной вселенной включают в себя многочисленные классы образцовых существ для различных вселенских групп~--- советников, руководителей и учителей, подобных им и предназначенных для них по всему творению. Все существа во всех вселенных созданы по образцу созданий какого\hyp{}либо одного порядка, живущих на одном из миллиарда миров Хавоны. Даже цели и идеалы существования смертных представлены на внешних контурах этих образцовых высших сфер.
\vs p014 4:19 Также есть и существа, достигшие Всеобщего Отца и получившие право уходить и приходить сюда вновь, назначаемые здесь и там во вселенных выполнять миссии особого служения. И на каждом мире Хавоны найдутся кандидаты на достижение, кто физически достиг центральной вселенной, но ещё не обладает достаточным духовным развитием, позволяющим претендовать на местожительство в Раю.
\vs p014 4:20 Бесконечный Дух представлен на мирах Хавоны множеством личностей, существ милосердия и славы, управляющих деталями сложных интеллектуальных и духовных дел центральной вселенной. На этих мирах божественного совершенства они выполняют работу, необходимую для нормального функционирования этого огромного творения, и, кроме того, решают разнообразные задачи обучения, подготовки и служения огромного количества восходящих созданий, поднявшихся к славе из тёмных миров пространства.
\vs p014 4:21 Существуют многочисленные группы уроженцев системы Рай\hyp{}Хавона, напрямую не связанные с программой восхождения созданий и достижения ими совершенства; поэтому они исключены из классификаций личностей, представленных смертным расам. Здесь представлены только основные группы сверхчеловеческих существ и те категории, которые напрямую связаны с вашим опытом выживания.
\vs p014 4:22 Хавона изобилует всеми фазами жизни разумных существ, стремящихся продвинуться от низших контуров к высшим в своих усилиях достичь более высоких уровней осознания божественности и расширенного понимания верховных смыслов, предельных ценностей и абсолютной реальности.
\usection{ЖИЗНЬ В ХАВОНЕ}
\vs p014 5:1 На Урантии ты проходишь короткое и интенсивное испытание в течение своей первоначальной жизни материального существования. На обительских мирах и выше~--- на уровнях вашей системы, созвездия и локальной вселенной тебя ждут моронтийные фазы восхождения. На учебных мирах сверхвселенной ты пройдёшь истинные стадии развития духа и подготовишься для последующего транзита в Хавону. На семи контурах Хавоны твои достижения будут интеллектуальными, духовными и эмпирическими. И на каждом из миров каждого из этих контуров должна быть выполнена определённая задача.
\vs p014 5:2 Жизнь на божественных мирах центральной вселенной так богата и полна, так совершенна и насыщенна, что полностью превосходит человеческие представления обо всём, что может испытать созданное существо. Социальная и экономическая деятельность этого вечного творения совершенно не похожа на занятия материальных созданий, живущих на эволюционных мирах, подобных Урантии. Даже способ мышления Хавоны отличается от мыслительного процесса на Урантии.
\vs p014 5:3 Нормы жизни центральной вселенной уместны и естественны по своей природе; в правилах поведения нет произвола. В каждом требовании Хавоны раскрываются основания праведности и принципы правосудия. На Урантии объединение этих двух факторов было бы названо \bibemph{справедливостью}. По прибытии в Хавону ты будешь испытывать естественное удовольствие от того, что поступаешь именно так, как следует.
\vs p014 5:4 \pc Когда разумные существа впервые достигают центральной вселенной, их принимают и поселяют на направляющем мире седьмого контура Хавоны. По мере духовного развития, когда новоприбывшие достигают понимания личности Главного Духа своей сверхвселенной, они переводятся на шестой контур. (Именно благодаря аналогии с этим порядком в центральной вселенной, получили название круги прогресса человеческого разума). После того как восходящие создания достигли осознания Верховности и тем самым подготовились к приключению, связанному с Божествами, они попадают на пятый контур; и после достижения Бесконечного Духа~--- переходят на четвёртый. После обретения Вечного Сына они перемещаются на третий; а когда узна\'ют Всеобщего Отца,~--- отправляются на второй контур миров, где ближе знакомятся с Райскими хозяевами\fnst{Или <<воинствами Рая>> (англ. Paradise hosts).}. Прибытие на первый контур Хавоны означает принятие кандидатов времени на Райскую службу. На неопределённое время, в зависимости от продолжительности и характера восхождения созданий, они будут оставаться на внутреннем контуре прогрессивного духовного достижения. Из этого внутреннего контура восходящие пилигримы проходят внутрь к Райской резиденции и допускаются в Корпус Завершения.
\vs p014 5:5 В течение своего пребывания в Хавоне в качестве восходящего пилигрима ты сможешь свободно посещать миры контура твоего назначения. Тебе также будет позволено возвращаться на планеты уже пройденных тобой контуров. И всё это возможно для пребывающих на контурах Хавоны без необходимости супернафимирования\fnst{Транспортный перенос внутри супернафима, аналогичный переносу внутри транспортного серафима в пределах локальных творений.}. Пилигримы времени способны сами снарядить себя для пересечения <<достигнутого>> пространства, но должны полагаться на предписанную технику для преодоления <<недостигнутого>> пространства; пилигрим не может покинуть Хавону или продвинуться вперёд за пределы своего контура без помощи транспортного супернафима.
\vs p014 5:6 \pc В этом огромном центральном творении есть освежающая оригинальность. За исключением физической организации материи и фундаментального строения основных категорий разумных существ и других живых сущностей, между мирами Хавоны нет ничего общего. Каждая из этих планет~--- оригинальное, уникальное и эксклюзивное творение; каждая планета представляет собой несравненное, величественное и совершенное произведение. И это разнообразие индивидуальности распространяется на все черты физического, интеллектуального и духовного аспектов планетарного существования. Каждая из миллиарда этих сфер совершенства была разработана и украшена в согласии с планами постоянно пребывающего От Века Вечного. И именно поэтому среди них нет даже двух похожих.
\vs p014 5:7 Лишь когда ты пересечёшь последний из контуров Хавоны и посетишь последний из миров Хавоны, стремление к приключению и стимул любопытства оставят тебя на пути твоего продвижения. И тогда побуждение~--- импульс вечности, направленный вперёд,~--- заменит предшествующий ему соблазн приключений, свойственный времени.
\vs p014 5:8 Однообразие свидетельствует о незрелости творческого воображения и о бездействии интеллектуального согласования с духовным даром. Ко времени начала исследования этих небесных миров восходящий смертный уже достиг эмоциональной, интеллектуальной и социальной, если не духовной, зрелости.
\vs p014 5:9 Тебя ждут не только невообразимые изменения, с которыми ты столкнёшься по мере своего продвижения от контура к контуру в Хавоне, но и невыразимое изумление при перемещении от планеты к планете в пределах каждого контура. Каждый из миллиарда этих учебных миров~--- настоящий университет неожиданностей. Непрерывное удивление, нескончаемое изумление~--- вот опыт тех, кто пересекает эти контуры и путешествует по этим гигантским сферам. Однообразие отнюдь не свойственно пути через Хавону.
\vs p014 5:10 Любовь к приключениям, любопытство и боязнь однообразия~--- эти черты, свойственные эволюционирующей человеческой природе, были заложены не для того, чтобы досаждать и раздражать тебя в течение твоего короткого пребывания на земле, а, скорее, чтобы навести тебя на мысль, что смерть~--- это только начало бесконечного приключения, вечной жизни в предвкушении, вечного путешествия открытий.
\vs p014 5:11 Любознательность~--- дух исследования, стремление к открытиям, стимул к исследованиям~--- это часть врождённого и божественного дара эволюционных созданий пространства. Эти естественные импульсы были даны тебе не для того, чтобы ты ощущал разочарование или подавленность. Правда, в течение твоей короткой жизни на земле эти страстные желания нередко приходится сдерживать, зачастую испытывая разочарования, но они\fnst{Желания, а не разочарования, разумеется.} непременно полностью реализуются и будут великолепно удовлетворены в течение долгих грядущих веков.
\usection{НАЗНАЧЕНИЕ ЦЕНТРАЛЬНОЙ ВСЕЛЕННОЙ}
\vs p014 6:1 Диапазон видов деятельности семиконтурной Хавоны огромен. В целом их можно описать как:
\vs p014 6:2 \li{1.}Связанные с Хавоной.
\vs p014 6:3 \li{2.}Связанные с Раем.
\vs p014 6:4 \li{3.}Связанные с восходящим\hyp{}конечным~--- эволюционным Верховным\hyp{}Предельным.
\vs p014 6:5 \pc В Хавоне нынешней вселенской эпохи происходит множество сверхконечных видов деятельности, включающих неисчислимое разнообразие абсонитных и других фаз разума и функций духа. Возможно, центральная вселенная служит многим целям, которые мне не открыты, поскольку она действует множеством способов, превосходящих понимание созданного разума. Тем не менее я попытаюсь описать, как это совершенное творение служит нуждам и способствует удовлетворению семи категорий вселенского разума:
\vs p014 6:6 \li{1.}\bibemph{Всеобщий Отец}~--- Первый Источник и Центр. Бог Отец получает высшее родительское удовлетворение от совершенства центрального творения. Он испытывает наслаждение от насыщения любовью на почти равных ему уровнях. Совершенный Создатель божественно удовлетворён обожанием совершенного создания.
\vs p014 6:7 Хавона вознаграждает Отца удовлетворением от высшего достижения. Реализация совершенства в Хавоне компенсирует время\hyp{}пространственную задержку вечного стремления к бесконечному расширению.
\vs p014 6:8 Отец наслаждается от того, насколько Хавона отвечает божественной красоте. Божественный разум с удовлетворением предоставляет совершенный образец изысканной гармонии всем развивающимся вселенным.
\vs p014 6:9 Наш Отец смотрит на центральную вселенную с совершенным удовольствием, потому что она~--- достойное откровение духовной реальности для всех личностей вселенной вселенных.
\vs p014 6:10 Бог вселенных одобрительно относится к Хавоне и Раю как к вечному ядру мощи для всего последующего расширения вселенной во времени и пространстве.
\vs p014 6:11 Вечный Отец с нескончаемым удовлетворением рассматривает творение Хавоны как достойную и привлекательную цель для восходящих кандидатов времени, своих смертных внуков пространства, достигающих вечного дома их Создателя\hyp{}Отца. И Богу доставляет удовольствие вселенная Рай\hyp{}Хавона как вечный дом Божества и божественной семьи.
\vs p014 6:12 \li{2.}\bibemph{Вечный Сын}~--- Второй Источник и Центр. Для Вечного Сына величественное центральное творение предоставляет вечное доказательство эффективности сотрудничества божественной семьи~--- Отца, Сына и Духа. Оно служит духовным и материальным основанием для абсолютной уверенности во Всеобщем Отце.
\vs p014 6:13 Хавона предоставляет Вечному Сыну практически неограниченную основу для постоянно расширяющейся реализации могущества духа. Центральная вселенная предоставила Вечному Сыну арену, на которой он может безопасно и надёжно демонстрировать дух и метод служения посвящения для обучения связанных с ним Райских Сынов.
\vs p014 6:14 Хавона~--- это фундамент реальности для контроля гравитации духа Вечного Сына во вселенной вселенных. Эта вселенная доставляет Сыну удовлетворение родительского желания, духовного воспроизводства.
\vs p014 6:15 Миры Хавоны и их совершенные обитатели~--- первая и в вечности последняя демонстрация того, что Сын есть Слово Отца. Тем самым самосознание Сына как бесконечного дополнения Отца совершенно удовлетворяется.
\vs p014 6:16 И эта вселенная предоставляет возможность для реализации взаимного равноправного братства между Всеобщим Отцом и Вечным Сыном, что служит вечным доказательством бесконечной личности каждого из них.
\vs p014 6:17 \li{3.}\bibemph{Бесконечный Дух}~--- Третий Источник и Центр. Вселенная Хавона служит Бесконечному Духу доказательством того, что он есть Совместный Вершитель, бесконечный представитель единства Отца\hyp{}Сына. В Хавоне Бесконечный Дух получает совместное удовлетворение функционирования как от творческой деятельности, так и от абсолютного сосуществования с этим божественным достижением.
\vs p014 6:18 В Хавоне Бесконечный Дух нашёл арену, на которой он мог продемонстрировать способность и готовность быть полезным в качестве потенциального служителя милосердия. В этом совершенном творении Дух готовился к приключению служения в эволюционных вселенных.
\vs p014 6:19 Это совершенное творение предоставило Бесконечному Духу возможность участвовать во вселенском управлении вместе с обоими божественными родителями~--- управлять вселенной в качестве потомка, являющегося партнёром\hyp{}Создателем, тем самым готовясь к совместному управлению локальными вселенными в лице Созидательных Духов~--- партнёров Сынов Создателей.
\vs p014 6:20 Миры Хавоны~--- это лаборатория разума для создателей космического разума и служителей разуму каждого существующего создания. Разум различен на каждом мире Хавоны и служит образцом для интеллекта всех духовных и материальных созданий.
\vs p014 6:21 Эти совершенные миры представляют собой высшие школы разума для всех существ, чьё предназначение~--- Райское сообщество. Они\fnst{Миры.} предоставили Духу богатые возможности испытывать метод служения разума на надёжных и готовых дать совет личностях.
\vs p014 6:22 Хавона~--- это вознаграждение Бесконечному Духу за его обширную и бескорыстную работу во вселенных пространства. Хавона~--- совершенный дом и убежище для неутомимого Служителя Разума времени и пространства.
\vs p014 6:23 \li{4.}\bibemph{Верховное Существо}~--- эволюционное объединение эмпирического Божества. Творение Хавоны есть вечное и совершенное доказательство духовной реальности Верховного Существа. Это безупречное творение является откровением совершенной и симметричной духовной природы Бога Верховного до начала синтеза мощи и личности конечных отражений Райских Божеств в эмпирических вселенных времени и пространства.
\vs p014 6:24 В Хавоне потенциалы мощи Всемогущего объединены с духовной природой Верховного. Это центральное творение иллюстрирует пример будущего вечного единства Верховного.
\vs p014 6:25 Хавона~--- это совершенный образец потенциала универсальности Верховного. Эта вселенная~--- законченное описание будущего совершенства Верховного и наводит на размышления о потенциале Предельного.
\vs p014 6:26 Хавона демонстрирует завершённость ценностей духа, существующих в виде живых волевых созданий верховного и совершенного самообладания; разум, существующий в конечном счёте как эквивалент духа; реальность и единство интеллекта с неограниченным потенциалом.
\vs p014 6:27 \li{5.}\bibemph{Равные Сыны Создатели}. Хавона~--- это учебная тренировочная площадка, где Райские Михаилы подготавливаются для своих последующих приключений в создании вселенных. Это божественное и совершенное творение~--- образец для каждого Сына Создателя. Он стремится сделать так, чтобы его собственная вселенная в конечном итоге достигла этих уровней совершенства Рая\hyp{}Хавоны.
\vs p014 6:28 Сын Создатель использует создания Хавоны в качестве возможных личностей\hyp{}образцов для своих собственных смертных детей и духовных существ. Михаилы и другие Райские Сыны рассматривают Рай и Хавону как божественное предназначение детей времени.
\vs p014 6:29 Сыны Создатели знают, что центральное творение~--- это реальный источник того незаменимого вселенского сверхуправления, которое стабилизирует и объединяет их локальные вселенные. Они знают, что личное присутствие повсеместного влияния Верховного и Предельного находится в Хавоне.
\vs p014 6:30 Хавона и Рай~--- источник созидательного могущества любого Сына Михаила. Здесь обитают существа, которые сотрудничают с ним в создании вселенной. Из Рая приходят Вселенские Материнские Духи, совместные создатели локальных вселенных.
\vs p014 6:31 Райские Сыны считают центральное творение домом своих божественных родителей,~--- своим домом. Это место, куда они с удовольствием возвращаются время от времени.
\vs p014 6:32 \li{6.}\bibemph{Равные Дочери Служения}. Вселенские Материнские Духи, совместные создатели локальных вселенных, получают своё доличностное обучение на мирах Хавоны в тесном сотрудничестве с Духами Контуров. В центральной вселенной Дочери\hyp{}Духи локальных вселенных были должным образом обучены методам сотрудничества с Сынами Рая, при этом подчиняясь воле Отца.
\vs p014 6:33 На мирах Хавоны Дух и Дочери Духа находят образцы разума для всех своих групп духовных и материальных интеллектов, и эта центральная вселенная станет когда\hyp{}нибудь предназначением тех созданий, которых отправит\fnst{Речь идёт об окончании обучения восходящего смертного в рамках локальной вселенной, когда Сын Создатель <<спонсирует>> его дальнейшее обучение в направлении к центру сверхвселенной (англ. sponsors). См.\,\bibref[30:4.21]{p030 4:21}.} один из Материнских Вселенских Духов вместе с соответствующим Сыном Создателем.
\vs p014 6:34 Мать\hyp{}Создательница Вселенной помнит Рай и Хавону как место своего происхождения и дом Бесконечного Материнского Духа, обитель присутствия личности Бесконечного Разума.
\vs p014 6:35 Из этой центральной вселенной также были дарованы личные созидательные прерогативы, благодаря которым Божественная Служительница Вселенной дополняет Сына Создателя в труде созидания живых волевых созданий.
\vs p014 6:36 И наконец, поскольку эти Дочерние Духи Бесконечного Материнского Духа вряд ли когда\hyp{}либо вернутся в свой Райский дом, они получают огромное удовлетворение от всеобщего феномена отражения, связанного с Верховным Существом в Хавоне и персонализированного в Мажестоне на Рае.
\vs p014 6:37 \li{7.}\bibemph{Эволюционные Смертные на Пути Восхождения}. Хавона~--- это дом образцовых личностей любого типа смертных и дом всех сверхчеловеческих личностей, которые связаны со смертными, но не являются уроженцами творений времени.
\vs p014 6:38 Эти миры создают стимул для всех человеческих порывов к достижению истинных духовных ценностей на высочайшем из постижимых уровней реальности. Хавона~--- это предрайская учебная цель каждого восходящего смертного. Здесь смертные достигают пред\hyp{}Райского Божества~--- Верховного Существа. Хавона предстаёт перед каждым волевым созданием преддверием к Раю и достижению Бога.
\vs p014 6:39 Рай~--- это дом, а Хавона~--- мастерская и игровая площадка завершителей. И каждый Богопознавший смертный жаждет стать завершителем.
\vs p014 6:40 Центральная вселенная~--- это не только установленное предназначение человека, но и отправная точка вечного пути завершителей, поскольку когда\hyp{}нибудь им предстоит начать нераскрытое и вселенское приключение в опыте исследования бесконечности Всеобщего Отца.
\vs p014 6:41 \pc Хавона, несомненно, продолжит функционировать с абсонитным значением даже в будущие вселенские эпохи, которые, возможно, станут свидетелями попыток пилигримов пространства найти Бога на сверхконечных уровнях. Хавона обладает способностью служить образовательной вселенной для абсонитных существ. Вероятно, она будет школой завершения, когда семь сверхвселенных будут функционировать как средняя школа для окончивших начальные школы внешнего пространства. И мы склоняемся к мнению, что потенциалы вечной Хавоны поистине безграничны, что центральная вселенная обладает вечной способностью служить эмпирической учебной вселенной для всех прошлых, настоящих или будущих типов созданных существ.
\vsetoff
\vs p014 6:42 [Представлено Совершенствователем Мудрости, уполномоченным для этих действий От Века Древними Уверсы.]
\quizlink
