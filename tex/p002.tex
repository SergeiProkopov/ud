\upaper{2}{ПРИРОДА БОГА}
\uminitoc{БЕСКОНЕЧНОСТЬ БОГА}
\uminitoc{ВЕЧНОЕ СОВЕРШЕНСТВО ОТЦА}
\uminitoc{ПРАВОСУДИЕ И ПРАВЕДНОСТЬ}
\uminitoc{БОЖЕСТВЕННОЕ МИЛОСЕРДИЕ}
\uminitoc{ЛЮБОВЬ БОГА}
\uminitoc{ДОБРОТА БОГА}
\uminitoc{БОЖЕСТВЕННАЯ ИСТИНА И КРАСОТА}
\author{Божественный Советник}
\vs p002 0:1 Поскольку высшее возможное представление человека о Боге заключается в человеческой идее и в идеале первичной и бесконечной личности, позволительно и может оказаться полезным изучить определённые особенности божественной природы, которые составляют характер Божества. Природа Бога может лучше всего быть понята через откровение Отца, которое Михаил Небадона развернул в своих разносторонних учениях и в своей возвышенной смертной жизни во плоти. Божественная природа может также быть лучше понята человеком, если он ощущает себя как дитя Бога и смотрит на Райского Создателя как на истинного духовного Отца.
\vs p002 0:2 Природу Бога можно изучать в откровении высших идей, божественный характер можно представлять как изображение высочайших идеалов, однако самое просвещающее и духовно наставляющее из всех откровений божественной природы можно найти в постижении религиозной жизни Иисуса из Назарета~--- как до, так и после достижения им полного сознания божественности. Если взять жизнь Михаила во плоти в качестве основы откровения Бога человеку, то мы можем попытаться облечь в человеческие словесные символы определённые идеи и идеалы, касающиеся божественной природы, которые, возможно, внесут свой вклад в дальнейшее освещение и объединение человеческой концепции природы и характера личности Всеобщего Отца.
\vs p002 0:3 Во всех наших усилиях расширить и одухотворить человеческую концепцию Бога нам чрезвычайно мешают ограниченные способности смертного разума. Мы также испытываем серьёзные трудности при исполнении нашего задания из-за ограничений языка и скудости материала, который может быть использован в целях иллюстрации или сравнения в наших попытках изобразить божественные ценности и представить духовные смыслы конечному, смертному разуму человека. Все наши усилия по расширению человеческой концепции Бога были бы почти бесполезными, если бы не тот факт, что в смертном разуме пребывает дарованный Настройщик Всеобщего Отца, и что он\fnst{Смертный разум.} пронизан Духом Истины Сына Создателя. Поэтому, полагаясь на присутствие этих божественных духов в человеческом сердце для помощи расширения концепции Бога, я с радостью беру на себя выполнение своего поручения предпринять попытку дальнейшего изображения природы Бога разуму человека.
\usection{БЕСКОНЕЧНОСТЬ БОГА}
\vs p002 1:1 <<Касательно Бесконечного\fnst{Англ. Touching the Infinite~--- см. Иов~37:23 (KJV): <<\bibemph{Touching} the Almighty, we cannot find him out>>. Здесь, также как и во всех 30 случаях в KJV (Библия короля Якова, 1611), где использовано слово <<touching>>, смысл <<касательно>>,<<относительно>> чего-либо. Тот же смысл подразумевается и в \bibref[96:6.4]{p096 6:4}. Однако, возможно также значение <<прикасаясь>>, особенно подходящее к продолжению <<мы не можем найти его>>, но это совершенно иной и привнесённый извне смысл по отношению к цитируемому из Библии отрывку.}, мы не можем найти его. Божественные следы не познаны>>. <<Его понимание бесконечно, и величие его неисследимо>>. Ослепляющий свет присутствия Отца таков, что его низшим созданиям он кажется <<обитающим в кромешной тьме>>. Не только мысли и планы его неисследимы, но он <<творит дела великие и чудные без числа>>. <<Бог велик; мы не понимаем его, и непостижимо число лет его>>. <<Поистине, Богу ли жить на земле? Вот, небеса (вселенная) и небеса небес (вселенная вселенных) не могут вместить его>>. <<Как непостижимы суждения его и неисповедимы пути его!>>
\vs p002 1:2 <<Есть только один Бог, бесконечный Отец, который является также верным Создателем>>. <<Божественный Создатель есть также Всеобщий Распорядитель, источник и предназначение душ. Он~--- Верховная Душа, Первичный Разум и Безграничный Дух всего творения>>. <<Великий Регулятор не совершает ошибок. Он великолепен в величии и славе>>. <<Бог Создатель полностью лишён страха и враждебности. Он бессмертен, вечен, самосущ, божественен и щедр>>. <<Как чист и прекрасен, как глубок и непостижим небесный Прародитель всех вещей!>> <<Бесконечный особенно превосходен тем, что он наделяет собою людей. Он~--- начало и конец, Отец всякого доброго и совершенного замысла>>. <<С Богом всё возможно; вечный Создатель есть причина причин>>.
\vs p002 1:3 \pc Несмотря на бесконечность грандиозных проявлений вечной и универсальной личности Отца, он безусловно осознаёт как свою бесконечность, так и свою вечность; таким же образом он полностью сознает своё совершенство и могущество. Он~--- единственное существо во вселенной, помимо своих божественно равных, кто даёт совершенную, подобающую и полную самооценку.
\vs p002 1:4 Отец постоянно и неизменно удовлетворяет различные степени потребности в нём, время от времени изменяющиеся в разных частях его главной вселенной. Великий Бог знает и понимает себя; он бесконечно осознаёт все свои первичные атрибуты совершенства. Бог~--- это не космическая случайность; не является он и вселенским экспериментатором. Властелины Вселенной могут принимать участие в приключениях\fnst{Англ. adventure~--- смелом, рискованном предприятии.}; Отцы Созвездий могут экспериментировать; главы систем могут практиковаться; но Всеобщий Отец видит конец от начала, а его божественный план и вечный замысел фактически охватывают и содержат в себе все эксперименты и все приключения всех его подчинённых в каждом мире, системе и созвездии в каждой вселенной его необъятных владений.
\vs p002 1:5 Ничто не ново для Бога, и никакое космическое событие никогда не приходит как неожиданность; он обитает на круге вечности. Он не имеет ни начала, ни конца дней. Для Бога нет прошлого, настоящего или будущего; всё время присутствует в любом данном моменте. Он~--- великий и единственный Я ЕСТЬ.
\vs p002 1:6 \pc Всеобщий Отец абсолютно и безусловно бесконечен во всех своих атрибутах; и этот факт, сам по себе, автоматически отстраняет его от всякого прямого личного общения с конечными материальными созданиями и другими низшими типами созданных разумных существ.
\vs p002 1:7 И всё это требует таких мер по контакту и общению с его многочисленными созданиями, что были предписаны, во\hyp{}первых, в личностях Райских Сынов Бога, которые, хотя и совершенны в своей божественности, часто приобщаются к природе сам\'ой плоти и крови планетарных рас, становясь одним из вас и одним с вами; таким образом Бог как бы становится человеком, что произошло в посвящении Михаила, кого называли то Сыном Божьим, то Сыном Человеческим. И, во\hyp{}вторых, существуют личности Бесконечного Духа~--- различные категории серафических воинств и других небесных разумных существ, которые сближаются с материальными созданиями низшего происхождения и столь многими способами помогают и служат им. И, в\hyp{}третьих, существуют безличностные Таинственные Мониторы~--- Настройщики Мыслей, действительный дар самог\'о великого Бога, без объявления и без объяснения посылаемые для того, чтобы обитать внутри таких, как люди Урантии. В бесконечном изобилии они нисходят с высот славы, чтобы удостоить своим присутствием и обитать в скромных умах тех смертных, кто обладает способностью к Богосознанию или на то потенциалом.
\vs p002 1:8 Этими способами и многими другими, неизвестными тебе и совершенно превосходящими конечное понимание,~--- Райский Отец с любовью и готовностью уменьшает и иным образом изменяет, разбавляет и ослабляет свою бесконечность, чтобы он мог приблизиться к конечным разумам своих детей\hyp{}созданий. И так, через серию распределений личности\fnst{Личности Бога Отца.}, которые являются всё менее абсолютными, бесконечный Отец получает возможность наслаждаться тесным контактом с различными разумными созданиями многочисленных миров его обширной вселенной.
\vs p002 1:9 Всё это он делал, делает сейчас и будет продолжать делать вечно, ни в малейшей степени не умаляя факта и реальности своей бесконечности, вечности и первичности. И все эти вещи абсолютно истинны, несмотря на сложность их понимания, тайну, окутывающую их, или невозможность полного их понимания созданиями, какие живут на Урантии.
\vs p002 1:10 \pc Поскольку Первый Отец бесконечен в своих планах и вечен в своих замыслах, ни одно конечное существо по своей сути не может когда\hyp{}либо охватить или постичь эти божественные планы и замыслы во всей их полноте. Смертный человек может только иногда и мельком заглянуть в замыслы Отца, по мере их раскрытия в связи с реализацией плана восхождения созданий на сменяющих друг друга уровнях его продвижения во вселенной. Хотя человек не может охватить значимость бесконечности, бесконечный Отец, несомненно, полностью понимает и с любовью обнимает\fnst{Или <<заключает в себе>>.} всю конечность всех своих детей во всех вселенных.
\vs p002 1:11 Божественность и вечность Отец разделяет с большим числом высших Райских существ, но мы сомневаемся, разделяются ли полностью бесконечность и вытекающая из неё вселенская первичность кем\hyp{}либо, кроме равных ему партнёров по Райской Троице. Бесконечность личности должна по необходимости охватывать всю конечность личности; отсюда истинность~--- буквальная истинность~--- учения, которое провозглашает, что <<в Нём мы живём, и движемся, и существуем>>. Тот фрагмент чистого Божества Всеобщего Отца, который обитает в смертном человеке, \bibemph{является} частью бесконечности Первого Великого Источника и Центра, Отца Отцов.
\usection{ВЕЧНОЕ СОВЕРШЕНСТВО ОТЦА}
\vs p002 2:1 Даже ваши древние пророки понимали вечную, не имеющую начала и конца, круговую природу Всеобщего Отца. Бог буквально и вечно присутствует в своей вселенной вселенных. Он обитает в настоящем моменте со всем своим абсолютным величием и вечным великолепием. <<Отец имеет жизнь в себе, и эта жизнь есть жизнь вечная>>. Испокон веков именно Отец <<дарует всем жизнь>>. В божественной целостности есть бесконечное совершенство. <<Я~--- Господь; Я не изменяюсь>>. Наше знание вселенной вселенных показывает не только то, что он Отец светов, но и то, что в вед\'ении им межпланетных дел <<нет изменчивости и ни тени перемены>>. Он <<возвещает конец от начала>>. Он говорит: <<Мой совет исполнится, и всё, что мне угодно, я сделаю>> <<согласно вечному замыслу, который я задумал в моём Сыне>>. Таким образом, все планы и замыслы Первого Источника и Центра подобны ему самому: вечны, совершенны и навсегда неизменны.
\vs p002 2:2 В указах Отца~--- завершённость полноты и совершенство наполненности. <<Всё, что делает Бог, пребывает вовеки; к тому нечего прибавить и от того нечего убавить>>. Всеобщий Отец не раскаивается в своих изначальных мудрых и совершенных замыслах. Его планы непоколебимы, его совет неизменен, в то время как действия его божественны и непогрешимы. <<Пред очами его тысяча лет, как день вчерашний, когда он прошёл, и как стража в ночи>>. Совершенство божественности и величие\fnst{Или <<величина>> (англ. magnitude).} вечности навсегда останутся за пределами полного понимания ограниченного разума смертного человека.
\vs p002 2:3 \pc Реакции неизменного Бога в исполнении своего вечного замысла могут показаться изменяющимися в соответствии с меняющимся отношением и изменчивым умом сотворённых им разумных существ; то есть на видимом уровне и поверхностно они могут изменяться, но в глубине и за всеми внешними проявлениями по\hyp{}прежнему присутствует неизменный замысел, предвечный план вечного Бога.
\vs p002 2:4 Во вселенных совершенство обязательно должно быть относительным термином, в центральной же вселенной и особенно на Рае совершенство неразбавленное; в определённых фазах оно даже абсолютно. Проявления Троицы изменяют демонстрацию божественного совершенства, но не ослабляют его.
\vs p002 2:5 \pc Первичное совершенство Бога заключается не в присвоенной праведности, а в неотъемлемом совершенстве доброты его божественной природы. Он~--- окончателен, закончен и совершенен. Нет изъяна в красоте и совершенстве его праведного характера. И вся схема живых существований в мирах пространства сосредоточена на божественном замысле возвышения всех волевых созданий до высокого предназначения~--- опыта разделения Райского совершенства Отца. Бог не эгоцентричен и не замкнут в себе; он никогда не прекращает даровать себя всем самосознающим созданиям необъятной вселенной вселенных.
\vs p002 2:6 Бог вечно и бесконечно совершенен; он не может лично познать несовершенство как свой собственный опыт; но он действительно разделяет сознание всего опыта несовершенства всех преодолевающих трудности созданий эволюционных вселенных всех Райских Сынов Создателей. Личное и освобождающее прикосновение Бога совершенства окружает сердца и объединяет сущности всех тех смертных созданий, которые поднялись на вселенский уровень нравственного различения. Таким путём, а также через контакты божественного присутствия, Всеобщий Отец действительно участвует в опыте \bibemph{совместно} с незрелостью и несовершенством на эволюционном пути каждого нравственного существа во всей вселенной.
\vs p002 2:7 Человеческие ограничения, потенциальное зло, не являются частью божественной природы, но смертный опыт \bibemph{со} злом и все взаимодействия человека с ним несомненно являются частью постоянно расширяющейся самореализации Бога в детях времени~--- созданиях, наделённых нравственной ответственностью, которые были созданы или развиты любым Сыном Создателем, исходящим из Рая.
\usection{ПРАВОСУДИЕ И ПРАВЕДНОСТЬ}
\vs p002 3:1 Бог праведен; следовательно, он справедлив. <<Праведен Господь во всех путях своих>>. <<Я не напрасно сделал всё то, что я сделал,~--- говорит Господь>>. <<Суды Господни истинны и всецело праведны>>. На правосудие Всеобщего Отца не могут повлиять действия или поведение его созданий, <<ибо нет у Господа, Бога нашего, ни беззакония, ни лицеприятия, ни мздоимства>>.
\vs p002 3:2 \pc Как же бесполезно обращаться к такому Богу с ребяческими призывами изменить его неизменные указы так, чтобы мы могли избежать справедливых последствий действия его мудрых естественных законов и праведных духовных наказов! <<Не обманывайтесь; Бог осмеян не бывает\fnst{Цитата из Гал~6:7 \textgreek{θεὸς οὐ μυκτηρίζεται}.}, ибо что посеет человек, то и пожнёт>>. Правда, даже в справедливом пожинании урожая проступка это божественное правосудие всегда смягчается милосердием. Бесконечная мудрость~--- вот вечный арбитр, определяющий пропорции справедливости и милосердия, которые должны быть отмерены в каждом конкретном случае. Величайшее наказание (в действительности, неизбежное следствие) за преступление и умышленное восстание против правления Бога~--- это утрата существования в качестве отдельного субъекта этого правления. Окончательный результат откровенного греха~--- аннигиляция. В конечном счёте такие отождествлённые с грехом индивидуумы уничтожили себя сами, ибо, отдавшись в объятия беззакония, они стали полностью нереальными. Фактическое исчезновение такого создания, однако, всегда откладывается до тех пор, пока предписанный порядок правосудия, действующий в данной вселенной, не будет полностью соблюдён.
\vs p002 3:3 Распоряжение о прекращении существования обычно отдаётся при диспенсационном или эпохальном вынесении судебного решения миру или мирам. В таком мире, как Урантия, это происходит в конце планетарной диспенсации. В такое время решение о прекращении существования может быть вынесено координированным действием всех судебных трибуналов, начиная от планетарного совета через суды Сына Создателя вплоть до судебных трибуналов От Века Древних. Мандат о растворении\fnst{То есть о ликвидации личности (англ. dissolution).} исходит от высших судов сверхвселенной вслед за неопровержимым подтверждением обвинительного акта, берущего начало на сфере обитания преступника; и затем, когда приговор об уничтожении\fnst{То есть <<прекращении существования>> (англ. extinction). В Откровении иногда используются термины, связанные с биологическими эволюционными процессами (вымирание вида и т.\,д.), применительно к процессам духовной эволюции, как цивилизации в целом, так и индивидуума.} утверждён наверху, он приводится в исполнение прямым действием тех судей, которые находятся в столице сверхвселенной и работают оттуда.
\vs p002 3:4 Когда этот приговор окончательно подтверждается, отождествлённое с грехом существо мгновенно исчезает, как если бы его никогда и не было. От такой участи нет воскресения; она постоянна и вечна. С помощью трансформации времени и метаморфоз пространства факторы живой энергии индивидуума возвращаются в те космические потенциалы, откуда они когда\hyp{}то появились. Что касается личности беззаконника, то она лишается инструмента непрерывной жизни отказом создания сделать такой выбор и принять такие решения, которые обеспечили бы вечную жизнь. Когда продолжающееся принятие греха ассоциированным разумом\fnst{То есть разумом, связанным с данной личностью.} достигает кульминации в полном самоотождествлении с беззаконием, то~--- после прекращения жизни, после космического растворения~--- такая изолированная личность поглощается в сверхдушу творения, становясь частью эволюционного опыта Верховного Существа. Никогда вновь не появится она как личность; её индивидуальность становится такой, как если бы её никогда не было. В случае личности, в которой обитал Настройщик, эмпирические ценности духа сохраняются в реальности продолжающего существовать Настройщика.
\vs p002 3:5 \pc В любом вселенском противостоянии между актуальными уровнями реальности личность более высокого уровня в конечном счёте торжествует над личностью более низкого уровня. Этот неизбежный исход вселенского разногласия неотъемлемо присутствует в том факте, что божественность качества равняется степени реальности или актуальности любого волевого создания. Неразбавленное зло, полная ошибка, умышленный грех и отъявленное беззаконие по сути и автоматически самоубийственны. Такие отношения космической нереальности могут выживать во вселенной только благодаря вр\'еменному милосердию\hyp{}терпимости в ожидании действия механизмов определения справедливости и поиска справедливого решения вселенскими трибуналами праведных судов.
\vs p002 3:6 Правление Сынов Создателей в локальных вселенных~--- это правление созидания и одухотворения. Эти Сыны посвящают себя эффективному исполнению Райского плана постепенного восхождения смертных, реабилитации\fnst{Перевоспитанию.} мятежников и неверно мыслящих; но когда все эти исполненные любви усилия окончательно и навсегда отвергаются, окончательный приказ о растворении\fnst{Растворении личности (англ. dissolution).} исполняется силами, действующими под юрисдикцией От Века Древних.
\usection{БОЖЕСТВЕННОЕ МИЛОСЕРДИЕ}
\vs p002 4:1 Милосердие~--- это просто правосудие, смягчённое той мудростью, которая произрастает из совершенства знания и полного признания естественных слабостей и препятствий среды обитания конечных созданий. <<Наш Бог полон сострадания, милосерден, долготерпелив и многомилостив>>. Поэтому <<всякий, кто призовёт имя Господне, спасётся>>, <<ибо он щедро простит>>. <<Милость Господня из вечности в вечность>>; да, <<милость его пребывает вовек>>. <<Я~--- Господь, творящий милость, суд и праведность на земле, ибо в этом моя радость>>. <<Не по изволению сердца своего я наказываю и огорчаю сынов человеческих>>, ибо я~--- <<Отец милостей и Бог всякого утешения>>.
\vs p002 4:2 Бог является по своей сути добрым, по природе сострадательным и вечно милосердным. И никогда не требуется никакого воздействия на Отца, чтобы вызвать его любовь\hyp{}доброту. Потребности создания целиком достаточно, чтобы обеспечить полный поток нежного милосердия и спасительной благосклонности Отца. Так как Бог знает всё о своих детях, ему легко прощать. Чем лучше человек понимает своего ближнего, тем легче ему будет прощать его, даже любить его.
\vs p002 4:3 \pc Лишь проницательность бесконечной мудрости позволяет праведному Богу вершить правосудие и милосердие одновременно и в любой данной вселенской ситуации. Небесного Отца никогда не раздирают противоречивые отношения к своим вселенским детям; Бог никогда не бывает жертвой антагонизмов отношений. Всезнание Бога неизменно направляет его свободную волю в выборе того вселенского поведения, которое совершенно, одновременно и одинаково удовлетворяет требованиям всех его божественных атрибутов и бесконечных качеств его вечной природы.
\vs p002 4:4 Милосердие~--- это естественное и неизбежное порождение доброты и любви. Добрый характер любящего Отца просто не позволил бы отказать в мудрой помощи милосердия хотя бы одному члену какой\hyp{}либо группы своих вселенских детей. Вечное правосудие и божественное милосердие вместе образуют то, что в человеческом опыте называется \bibemph{справедливостью}.
\vs p002 4:5 Божественное милосердие представляет собой метод справедливого примирения между вселенскими уровнями совершенства и несовершенства. Милосердие~--- это правосудие Верховности, адаптированное к ситуациям эволюционирующего конечного, праведность вечности, изменённая так, чтобы удовлетворять высшим интересам и вселенскому благополучию детей времени. Милосердие~--- это не нарушение правосудия, но, скорее, чуткая интерпретация требований верховного правосудия в его справедливом применении к подчинённым духовным существам и материальным созданиям развивающихся вселенных. Милосердие есть правосудие Райской Троицы, с мудростью и любовью совершаемое над разнообразными разумами творений времени и пространства, формулируемое божественной мудростью и определяемое всезнающим разумом и суверенной свободной волей Всеобщего Отца и всех связанных с ним Создателей.
\usection{ЛЮБОВЬ БОГА}
\vs p002 5:1 <<Бог есть любовь>>, поэтому его единственным личным отношением к делам вселенной всегда является проявление божественной любви. Отец любит нас достаточно, чтобы подарить нам свою жизнь. <<Он повелевает солнцу своему восходить над злыми и добрыми и посылает дождь на праведных и неправедных>>.
\vs p002 5:2 \pc Неверно думать, будто жертвы его Сыновей или ходатайство подчинённых ему созданий уговаривают Бога полюбить своих детей, <<ибо Отец сам любит вас>>. Именно в соответствии с этой родительской любовью Бог и посылает чудесных Настройщиков обитать внутри разумов людей. Любовь Бога универсальна; <<всякий, кто желает, может прийти>>. Он хотел, <<чтобы все люди спаслись, придя к знанию истины>>. Он <<не желает, чтобы кто\hyp{}нибудь погиб>>.
\vs p002 5:3 Создатели являются самыми первыми, кто пытается спасти человека от гибельных последствий неразумного нарушения им божественных законов. Любовь Бога по природе своей~--- любовь отеческая; поэтому иногда он <<наказывает нас для нашей же пользы, чтобы мы могли быть причастными его святости>>. Даже в самых суровых испытаниях помни, что <<во всех скорбях наших он скорбит вместе с нами>>.
\vs p002 5:4 Бог божественно добр к грешникам. Когда мятежники обращаются к праведности, они принимаются милосердно, <<ибо наш Бог щедро простит>>. <<Я тот, кто изглаживает преступления твои ради себя самого, и грехов твоих я не вспомню>>. <<Вот какой любовью одарил нас Отец, чтобы мы назывались сынами Божьими>>.
\vs p002 5:5 В конце концов, величайшее свидетельство доброты Бога и верховная причина любить его есть обитающий внутри тебя дар Отца~--- Настройщик, который так терпеливо ожидает того часа, когда вы оба навечно соединитесь в одно целое. Хотя ты и не можешь найти Бога исследованием, если ты подчинишься руководству обитающего внутри тебя духа, он безошибочно поведёт тебя шаг за шагом, жизнь за жизнью, через вселенную за вселенной и эпоху за эпохой, пока, наконец, ты не окажешься в присутствии Райской личности Всеобщего Отца.
\vs p002 5:6 \pc Как неразумно, что ты лишаешь себя поклонения Богу только потому, что ограничения человеческой природы и недостатки твоего материального происхождения не позволяют тебе увидеть его. Между тобой и Богом существует колоссальное расстояние (физическое пространство), которое предстоит пройти. Существует также огромная пропасть духовного дифференциала, которая должна быть преодолена; но несмотря на всё то, что физически и духовно отделяет тебя от Райского личного присутствия Бога, остановись и задумайся о том серьёзном факте, что Бог живёт внутри тебя; он по\hyp{}своему уже перекинул мост через пропасть. Он послал часть себя, свой дух, чтобы жить в тебе и трудиться с тобой по мере того, как ты следуешь по своему вечному вселенскому пути.
\vs p002 5:7 Я нахожу лёгким и приятным поклоняться тому, кто так велик и одновременно с такой любовью посвящён служению, возвышающему его низших созданий. Я естественно люблю того, кто столь могуществен в творении и управлении им и, к тому же, кто так совершенен в доброте и так верен в любящем милосердии, постоянно окружающем нас. Думаю, что я любил бы Бога так же сильно, если бы он не был так велик и могуществен, но оставался бы таким же добрым и милосердным. Все мы больше любим Отца из-за его природы, чем из-за признания его поразительных атрибутов.
\vs p002 5:8 Когда я наблюдаю Сынов Создателей и их подчинённых администраторов, борющихся столь доблестно с различными трудностями времени, присущими эволюции вселенных пространства, я обнаруживаю в себе огромную и глубокую любовь к этим меньшим правителям вселенных. В конце концов, я думаю, что все мы, включая смертных миров, любим Всеобщего Отца и всех остальных существ~--- божественных или человеческих,~--- потому что понимаем, что эти личности истинно любят нас. Переживание любви очень во многом является прямой реакцией на то, что ты любим. Зная, что Бог любит меня, я продолжал бы очень сильно любить его, будь он даже лишён всех своих атрибутов верховности, предельности и абсолютности.
\vs p002 5:9 Любовь Отца сопровождает нас сейчас и на всём протяжении бесконечного круга вечных эпох. Когда ты думаешь о любящей природе Бога, возможна только одна разумная и естественная личностная реакция: ты будешь всё больше любить своего Создателя; твоё чувство к Богу будет сравнимо с тем, что испытывает ребёнок к земному родителю; ибо как отец, настоящий отец, истинный отец, любит своих детей, так и Всеобщий Отец любит и постоянно ищет благополучия своим созданным сынам и дочерям.
\vs p002 5:10 Но любовь Бога~--- это разумное и дальновидное родительское чувство. Божественная любовь действует в едином союзе с божественной мудростью и всеми остальными бесконечными качествами совершенной природы Всеобщего Отца. Бог есть любовь, но любовь не есть Бог. Величайшее проявление божественной любви к смертным существам наблюдается в посвящении Настройщиков Мыслей, но величайшее для тебя откровение любви Отца можно увидеть в жизни посвящения его Сына Михаила, когда он жил на земле идеальной духовной жизнью. Именно обитающий внутри Настройщик индивидуализирует любовь Бога к каждой человеческой душе.
\vs p002 5:11 \pc Временами я почти испытываю боль, будучи вынужден изображать божественное чувство небесного Отца к его вселенским детям используя человеческий словесный символ \bibemph{любовь}. Этот термин, хотя и выражает самую высокую человеческую концепцию смертных отношений уважения и преданности, так часто используется для обозначения столь многого в человеческих отношениях, являющегося постыдным, что делает его крайне непригодным для того, чтобы стать тем словом, которое выражало бы также несравненное чувство живого Бога к его вселенским созданиям! Как жаль, что я не могу воспользоваться каким\hyp{}нибудь неземным и исключительным термином, который передал бы человеческому разуму истинную природу и утончённо прекрасный смысл божественной любви Райского Отца.
\vs p002 5:12 \pc Когда человек теряет из виду любовь личностного Бога, царство Божье становится просто царством добра. Несмотря на бесконечное единство божественной природы, любовь является доминирующей характеристикой всех личностных отношений Бога со своими созданиями.
\usection{ДОБРОТА БОГА}
\vs p002 6:1 В физической вселенной мы можем видеть божественную красоту, в интеллектуальном мире можем заметить вечную истину, но доброту Бога можно найти только в духовном мире личного религиозного опыта. В своей истинной сущности религия есть доверие через веру в доброту Бога. В философии Бог может быть великим и абсолютным, так или иначе даже разумным и личностным, но в религии Бог должен быть также нравственным; он должен быть добрым. Человек может бояться великого Бога, но он доверяет только доброму Богу и только доброго Бога любит. Эта доброта Бога является частью личности Бога, и её полное раскрытие появляется только в личном религиозном опыте верующих сынов Бога.
\vs p002 6:2 Религия подразумевает, что сверхмир природы духа знает о фундаментальных нуждах человеческого мира и откликается на них. Эволюционная религия может стать этической, но только религия откровения становится истинно и духовно нравственной. Древнее представление, что Бог есть Божество, в котором преобладает царственная мораль, было возвышено Иисусом до нежного и трогательного уровня интимно\hyp{}семейной морали, присущей отношению родителя и ребёнка, нежнее и прекраснее которого нет во всём опыте смертных.
\vs p002 6:3 \pc <<Богатство доброты Бога ведёт ошибающегося человека к покаянию>>. <<Всякий благой дар и всякий совершенный дар нисходит от Отца светил>>. <<Бог добр; он есть вечное убежище душ человеческих>>. <<Господь Бог милосерден и милостив. Он долготерпелив и изобилует добротой и истиной>>. <<Вкусите, и увидите, что Господь добр! Благословен человек, который доверяет ему>>. <<Господь милостив и полон сострадания. Он есть Бог спасения>>. <<Он исцеляет сокрушённых сердцем и перевязывает раны души. Он~--- всемогущий благодетель человека>>.
\vs p002 6:4 \pc Концепция Бога как царя\hyp{}судьи,~--- хотя она и способствовала появлению высокого морального стандарта и создала законопослушный народ как группу,~--- оставила индивидуального верующего в прискорбном положении неуверенности относительно его статуса во времени и вечности. Более поздние еврейские пророки провозгласили Бога Отцом Израиля; Иисус же раскрыл Бога как Отца каждого человеческого существа. Всё представление смертных о Боге трансцендентно освещается жизнью Иисуса. Бескорыстие присуще родительской любви. Бог любит не \bibemph{подобно} отцу, а \bibemph{будучи} отцом. Он~--- Райский Отец каждой вселенской личности.
\vs p002 6:5 \pc Праведность подразумевает, что Бог является источником нравственного закона вселенной. Истина показывает Бога как открывателя, как учителя. Любовь же даёт чувство и жаждет чувства, ищет взаимопонимания в общении, подобного тому, какое существует между родителем и ребёнком. Праведность может быть божественной мыслью, но любовь~--- это отношение отца. Ошибочное предположение, что праведность Бога несовместима с бескорыстной любовью небесного Отца, допускало отсутствие единства в природе Божества и вело напрямую к развитию доктрины искупления, которая представляет собой философское оскорбление как единства, так и свободной воли Бога.
\vs p002 6:6 Любящий небесный Отец, чей дух обитает в его детях на земле, не является разделённой личностью~--- часть от правосудия и часть от милосердия; не требуется и посредник, чтобы обеспечить благосклонность или прощение Отца. Божественная праведность не подчинена строгому карающему правосудию; Бог как отец превосходит Бога как судью.
\vs p002 6:7 \pc Бог никогда не бывает гневным, мстительным или сердитым. Верно, что мудрость часто сдерживает его любовь, а правосудие обусловливает его отвергнутое милосердие. Его любовь к праведности не может не демонстрироваться как равносильная ненависть к греху. Отец не является противоречивой личностью; божественное единство совершенно. В Райской Троице существует абсолютное единство, несмотря на вечные индивидуальности равных Богу существ.
\vs p002 6:8 \pc Бог любит грешника и \bibemph{ненавидит} грех: такое утверждение истинно в философском смысле, но Бог является трансцендентной личностью, и личности могут любить и ненавидеть только другие личности. Грех не есть личность. Бог любит грешника, потому что он личностная (потенциально вечная) реальность, в то время как по отношению к греху Бог не проявляет никакого личного отношения, ибо грех не является духовной реальностью; он не является личностным; поэтому только правосудие Бога замечает его существование. Любовь Бога спасает грешника; закон Бога уничтожает грех. Эта позиция божественной природы, очевидно, изменилась бы, если бы грешник полностью и окончательно отождествил себя с грехом,~--- точно так, как тот же самый смертный разум может полностью отождествить себя с обитающим внутри него духовным Настройщиком. Такой отождествивший себя с грехом смертный после этого стал бы по своей природе целиком недуховным (и, следовательно, лично нереальным) и испытал бы окончательное угасание существования. Нереальность, даже неполнота природы создания, не может существовать вечно во всё более реальной и всё более духовной вселенной.
\vs p002 6:9 \pc Обращаясь к миру личности, Бог раскрывается как любящая личность; обращаясь к духовному миру, он~--- личная любовь; в религиозном опыте он~--- и то, и другое. Любовь определяет волевое желание Бога. Доброта Бога лежит в основе божественной свободной воли~--- универсального стремления любить, выказывать милосердие, проявлять терпение и прощать.
\usection{БОЖЕСТВЕННАЯ ИСТИНА И КРАСОТА}
\vs p002 7:1 Всякое конечное знание и понимание создания \bibemph{относительны}. Информация и сведения, добытые даже из высоких источников, лишь относительно полны, локально точны и лично истинны.
\vs p002 7:2 Физические факты достаточно однородны, но истина~--- это живой и гибкий фактор в философии вселенной. Эволюционирующие личности лишь отчасти мудры и относительно правдивы в своём общении. Они могут быть уверены лишь настолько, насколько распространяется их личный опыт. То, что в одном месте кажется полностью истинным, может быть только относительно истинным в другом сегменте творения.
\vs p002 7:3 Божественная истина, окончательная истина, однородна и универсальна, но повествование о вещах духовных, как об этом рассказывается многочисленными индивидуумами, происходящими из разных сфер, может иногда варьироваться в деталях вследствие этой относительности в полноте знания и в насыщенности личного опыта, а также продолжительности и глубины этого опыта. Хотя законы и указы, мысли и отношения Первого Великого Источника и Центра являются вечно, бесконечно и универсально истинными, их применение и адаптация к любой вселенной, системе, миру и созданному разумному существу происходят согласно планам и методу Сынов Создателей, действующих в соответствующих вселенных, а также в гармонии с локальными планами и процедурами Бесконечного Духа и всех других связанных с ним небесных личностей.
\vs p002 7:4 \pc Материализм, будучи лженаукой, приговорил бы смертного человека стать изгоем во вселенной. Такое частичное знание есть потенциальное зло: это~--- знание, состоящее как из добра, так и зла. Истина прекрасна, ибо она и полна, и симметрична. Когда человек ищет истину, он стремится к божественно реальному.
\vs p002 7:5 Философы совершают грубейшую ошибку, когда уходят в заблуждения абстракции~--- практику фокусирования внимания на одном аспекте реальности, затем провозглашая этот изолированный аспект всей истиной. Мудрый философ всегда будет искать творческий замысел, который скрывается за всеми вселенскими феноменами и предшествует им. Мысль создателя неизменно предшествует созидательному действию.
\vs p002 7:6 Интеллектуальное самосознание может раскрыть красоту истины, её духовное качество, не только по философской согласованности её концепций, но гораздо точнее и увереннее по безошибочному отклику вездесущего Духа Истины. Счастье проистекает от признания истины, потому что она может быть \bibemph{выражена в действии;} она может быть пережита. Разочарование и печаль сопровождают заблуждение, ибо, не будучи реальностью, оно не может быть реализовано на опыте. Божественная истина лучше всего узнаётся по её \bibemph{духовному аромату}.
\vs p002 7:7 \pc Вечный поиск существует ради объединения, ради божественной согласованности. Обширная физическая вселенная согласуется в Острове Рай; интеллектуальная вселенная согласуется в Боге разума, Совместном Вершителе; духовная вселенная согласована в личности Вечного Сына. Но отдельный смертный времени и пространства согласуется в Боге Отце через прямое отношение между обитающим внутри него Настройщиком Мыслей и Всеобщим Отцом. Настройщик человека есть частица Бога, и она вечно стремится к божественному объединению; она\fnst{Частица.} согласуется с Райским Божеством Первого Источника и Центра и в нём.
\vs p002 7:8 \pc Различение верховной красоты~--- это открытие и интеграция реальности: различение же божественной доброты в вечной истине~--- вот что является предельной красотой. Даже очарование человеческого искусства состоит в гармонии его единства.
\vs p002 7:9 Великой ошибкой еврейской религии была её неспособность связать доброту Бога с фактическими истинами науки и привлекательной красотой искусства. Цивилизация развивалась, а поскольку религия продолжала следовать тем же немудрым путём чрезмерного подчёркивания доброты Бога за счёт относительного исключения истины и пренебрежения красотой, то у определённых типов людей развилась всё более усиливающаяся тенденция отворачиваться от абстрактной и оторванной концепции изолированной доброты. Чрезмерно подчёркиваемая и обособленная мораль современной религии, которая не может удержать преданность и верность многих людей XX века, могла бы реабилитировать себя, если бы в дополнение к своим моральным наставлениям она уделяла должное внимание истинам науки, философии и духовного опыта, а также красотам физического творения, очарованию интеллектуального искусства и величию достижения настоящего характера.
\vs p002 7:10 Религиозный вызов этой эпохи обращён к тем дальновидным и прозорливым мужчинам и женщинам, обладающим духовной проницательностью, которые осмелятся построить новую и привлекательную философию жизни из расширенных и тонко интегрированных современных представлений о космической истине, вселенской красоте и божественной доброте. Такое новое и праведное в\'идение морали привлечёт всё хорошее, что есть в разуме человека и бросит вызов тому, что есть лучшее в человеческой душе. Истина, красота и доброта суть божественные реальности, и по мере восхождения человека по лестнице духовной жизни эти верховные качества Вечного становятся всё больше согласованными и объединёнными в Боге, который есть любовь.
\vs p002 7:11 \pc Всякая истина~--- материальная, философская или духовная~--- и прекрасна, и добра. Любая настоящая красота~--- материальное искусство или духовная симметрия~--- и истинна, и добра. Любая настоящая доброта~--- личная нравственность, социальная справедливость или божественное служение~--- одинаково истинна и прекрасна. Здоровье, здравый ум и счастье суть интеграции истины, красоты и доброты в том виде, в каком они смешаны в человеческом опыте. Такие уровни эффективного образа жизни появляются через объединение систем энергий, систем идей и систем духа.
\vs p002 7:12 Истина согласует, красота привлекает, доброта укрепляет. И когда эти ценности реального скоординированы в опыте личности, результатом является высокая степень любви, обусловленная мудростью и определяемая преданностью. Настоящая цель всего вселенского образования заключается в осуществлении лучшего согласования изолированного чада миров с б\'ольшими реальностями его расширяющегося опыта. Реальность конечна на человеческом уровне, бесконечна и вечна на более высоких и божественных уровнях.
\vsetoff
\vs p002 7:13 [Представлено Божественным Советником, уполномоченным От Века Древними Уверсы.]
\quizlink
\begin{thebibliography}{100}
\bibitem{Knudson1}
Albert C. Knudson.
{<<The Doctrine of God>>.}
{\em New York: Abingdon-Cokesbury Press}, 1930.
\bibitem{Hume1}
Robert Ernest Hume, M.A., Ph.D.,
{<<Treasure\hyp{}House of the Living Religions: Selections from Their Sacred Scriptures>>.}
{\em New York: Charles Scribner's Sons}, 1932.
\bibitem{Overstreet1}
H.A. Overstreet
{<<The Enduring Quest: A Search for a Philosophy of Life>>.}
{\em New York: W. Norton \&\ Company, Inc.}, 1931.
\end{thebibliography}
