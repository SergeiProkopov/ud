\upaper{16}{СЕМЬ ГЛАВНЫХ ДУХОВ}
\uminitoc{ОТНОШЕНИЕ К ТРИЕДИНОМУ БОЖЕСТВУ}
\uminitoc{ОТНОШЕНИЕ К БЕСКОНЕЧНОМУ ДУХУ}
\uminitoc{ИНДИВИДУАЛЬНОСТЬ И РАЗНООБРАЗИЕ ГЛАВНЫХ ДУХОВ}
\uminitoc{АТРИБУТЫ И ФУНКЦИИ ГЛАВНЫХ ДУХОВ}
\uminitoc{ОТНОШЕНИЕ К СОЗДАНИЯМ}
\uminitoc{КОСМИЧЕСКИЙ РАЗУМ}
\uminitoc{НРАВСТВЕННОСТЬ, ДОБРОДЕТЕЛЬ И ЛИЧНОСТЬ}
\uminitoc{УРАНТИЙСКАЯ ЛИЧНОСТЬ}
\uminitoc{РЕАЛЬНОСТЬ ЧЕЛОВЕЧЕСКОГО СОЗНАНИЯ}
\author{Всеобщий Цензор}
\vs p016 0:1 Семь Главных Духов Рая~--- первичные личности Бесконечного Духа. В этом семичастном созидательном акте самоповторения Бесконечный Дух исчерпал комбинаторные возможности, математически свойственные фактическому существованию трёх лиц Божества. Если бы возможно было создать большее число Главных Духов, они были бы созданы, но существует семь, и только семь, возможных комбинаций, присущих трём Божествам. Это объясняет, почему вселенная управляется в семи больших разделах, а число семь является фундаментальным в основе её организации и управления.
\vs p016 0:2 Cемь Главных Духов, таким образом, берут своё начало в следующих семи прообразах, наследуя от них индивидуальные характерные черты:
\vs p016 0:3 \li{1.}Всеобщий Отец.
\vs p016 0:4 \li{2.}Вечный Сын.
\vs p016 0:5 \li{3.}Бесконечный Дух.
\vs p016 0:6 \li{4.}Отец и Сын.
\vs p016 0:7 \li{5.}Отец и Дух.
\vs p016 0:8 \li{6.}Сын и Дух.
\vs p016 0:9 \li{7.}Отец, Сын и Дух.
\vs p016 0:10 \pc Мы очень мало знаем о деятельности Отца и Сына в создании Главных Духов. По всей видимости, они появились в результате личных действий Бесконечного Духа, однако нас со всей определённостью учили, что и Отец, и Сын также участвовали в этом созидательном акте.
\vs p016 0:11 По своей духовной природе и сущности Семь Духов Рая едины, но очень непохожи во всех других аспектах существа, и по оценке их действий в сверхвселенных индивидуальные отличия каждого безошибочно распознаваемы. Все последующие планы семи сегментов большой вселенной~--- и даже соответствующих сегментов внешнего пространства~--- обусловлены внедуховным разнообразием Семи Главных Духов в верховном и предельном сверхконтроле.
\vs p016 0:12 Главные Духи выполняют множество функций, но в настоящее время основная сфера их деятельности~--- центральное руководство семью сверхвселенными. Каждый Главный Дух поддерживает огромный фокально\hyp{}силовой центр, медленно обращающийся вокруг периферии Рая, всегда сохраняя позицию напротив вселенной непосредственного контроля и в Райском фокусе специализированного управления мощью и сегментированного распределения энергии. Радиальные граничные линии любой сверхвселенной действительно сходятся в Райском центре руководящего Главного Духа.
\usection{ОТНОШЕНИЕ К ТРИЕДИНОМУ БОЖЕСТВУ}
\vs p016 1:1 Совместный Создатель~--- Бесконечный Дух~--- необходим для завершения триединой персонализации неделимого Божества. Эта тройная персонализация Божества является, по сути, семичастной в возможности индивидуального и ассоциативного выражения; поэтому последующий план создания вселенных, населённых разумными и потенциально духовными существами, должным образом выражающими Отца, Сына и Духа, сделал персонализацию Семи Главных Духов неизбежной. Обычно мы говорим о тройной персонализации Божества как об \bibemph{абсолютной неизбежности,} в то же время рассматривая появление Семи Главных Духов как \bibemph{субабсолютную неизбежность}.
\vs p016 1:2 Хотя Семь Главных Духов едва ли выражают \bibemph{трёхчастное} Божество, они являются вечным изображением \bibemph{семичастного} Божества, активными и ассоциативными функциями трёх вечно существующих лиц Божества. С помощью этих Семи Духов, в них и через них, Всеобщий Отец, Вечный Сын или Бесконечный Дух, или любая их парная комбинация, способны действовать как таковые. Когда Отец, Сын и Дух действуют вместе, они могут и действительно функционируют через Главного Духа Номер Семь, но не как Троица. Порознь и совместно Главные Духи представляют любую и все возможные функции Божества, одну или несколько, но не совокупные, не Троицу. Главный Дух Номер Семь лично не функционирует по отношению к Райской Троице, и именно по этой причине он может функционировать \bibemph{лично} от имени Верховного Существа.
\vs p016 1:3 Но когда Семь Главных Духов покидают свои индивидуальные места личной власти и сверхвселенских полномочий и собираются вокруг Совместного Вершителя в триедином присутствии Райского Божества, они коллективно становятся представителями функциональной власти, мудрости и полномочий неделимого Божества~--- Троицы~--- по отношению к развивающимся вселенным и в них. Такой Райский союз первичного семичастного выражения Божества действительно охватывает, буквально заключает в себе, все атрибуты и отношения трёх вечных Божеств в Верховности и в Предельности. Фактически Семь Главных Духов действительно и сразу охватывают функциональную область Верховно\hyp{}Предельного в главной вселенной и по отношению к ней.
\vs p016 1:4 Насколько мы можем судить, эти Семь Духов связаны с божественной деятельностью трёх вечных лиц Божества; мы не обнаруживаем никакого свидетельства непосредственной связи с функционирующими присутствиями трёх вечных фаз Абсолюта. Во взаимосвязи Главные Духи представляют Райских Божеств в том, что можно приблизительно представить себе как конечную область действия. Эта область может включать многое из того, что является предельным, но \bibemph{не} абсолютным.
\usection{ОТНОШЕНИЕ К БЕСКОНЕЧНОМУ ДУХУ}
\vs p016 2:1 Как Вечный и Изначальный Сын раскрывается через лица постоянно увеличивающегося числа божественных Сынов, так и Бесконечный и Божественный Дух раскрывается через каналы Семи Главных Духов и связанные с ними группы духов. В центре центров Бесконечный Дух доступен, но не все, кто достигает Рая, способны сразу же распознать его личность и дифференцированное присутствие; но все, кто достигает центральной вселенной, могут непосредственно общаться, и действительно общаются, с одним из Семи Главных Духов, возглавляющим ту сверхвселенную, откуда поступил вновь прибывший пилигрим пространства.
\vs p016 2:2 Со вселенной вселенных Райский Отец говорит только через своего Сына, в то же время вместе, он и Сын, действуют только через Бесконечного Духа. Вне Рая и Хавоны Бесконечный Дух \bibemph{говорит} только голосами Семи Главных Духов.
\vs p016 2:3 \pc Бесконечный Дух оказывает влияние \bibemph{личного присутствия} внутри системы Рай\hyp{}Хавона; за её пределами присутствие его личного духа проявляется одним из Семи Главных Духов и через него. Поэтому сверхвселенское присутствие духа Третьего Источника и Центра на любом мире или в любом индивидууме обусловлено уникальной природой руководящего Главного Духа данного сегмента творения. И наоборот, объединённые линии силы духа и разума проходят внутрь, к Третьему Лицу Божества, через Семь Главных Духов.
\vs p016 2:4 \pc Семь Главных Духов коллективно наделяются верховно\hyp{}предельными атрибутами Третьего Источника и Центра. Хотя каждый индивидуально получает этот дар, только коллективно они раскрывают атрибуты всемогущества, всеведения и вездесущности. Поэтому ни один из них не может функционировать универсально; как индивидуумы, в проявлении таких полномочий верховности и предельности, каждый лично ограничен сверхвселенной непосредственного контроля.
\vs p016 2:5 Всё сказанное тебе о божественности и личности Совместного Вершителя в равной и полной мере относится к Семи Главным Духам, которые столь эффективно распределяют Бесконечный Дух по семи сегментам большой вселенной в соответствии со своим божественным даром и согласно своей различной и индивидуально\hyp{}уникальной природе. Поэтому вполне уместно применить к коллективной группе из семи любое или все имена Бесконечного Духа. Вместе они едины с Совместным Создателем на всех субабсолютных уровнях.
\usection{ИНДИВИДУАЛЬНОСТЬ И РАЗНООБРАЗИЕ ГЛАВНЫХ ДУХОВ}
\vs p016 3:1 Семь Главных Духов~--- неописуемые существа, но они отчётливо и определённо личностны. У них есть имена, но мы предпочитаем знакомить с ними по номерам. Будучи первичными персонализациями Бесконечного Духа, они родственны друг другу, но как первичные выражения семи возможных комбинаций триединого Божества они существенно отличаются друг от друга по природе, и это различие природы определяет различие в их управлении сверхвселенными. Эти Семь Главных Духов можно описать следующим образом.
\vs p016 3:2 \bibemph{Главный Дух Номер Один}. Особым образом этот Дух непосредственно представляет Райского Отца. Он~--- своеобразное и действенное проявление могущества, любви и мудрости Всеобщего Отца. Он является близким партнёром главы Таинственных Мониторов, того существа, которое возглавляет Коллегию Персонализированных Настройщиков на Божеграде. Во всех ассоциациях Семи Главных Духов от имени Всеобщего Отца всегда говорит Главный Дух Номер Один.
\vs p016 3:3 Этот Дух возглавляет первую сверхвселенную и, неизменно демонстрируя божественную природу первичной персонализации Бесконечного Духа, по всей видимости, по характеру больше похож на Всеобщего Отца. Он всегда поддерживает личную связь с семью Отражательными Духами в столице первой сверхвселенной.
\vs p016 3:4 \pc \bibemph{Главный Дух Номер Два}. Этот Дух~--- адекватное изображение несравненной природы и обаятельного характера Вечного Сына, первородного Сына всего творения. Он всегда находится в тесном партнёрстве со всеми категориями Сынов Бога, когда бы ни случилось им оказаться в родной вселенной лично или на радостном конклаве. На всех собраниях Семи Главных Духов он всегда говорит за Вечного Сына и от его имени.
\vs p016 3:5 Этот Дух руководит судьбой второй сверхвселенной и правит этим огромным владением так же, как это делал бы Вечный Сын. Он всегда поддерживает связь с семью Отражательными Духами, расположенными в столице второй сверхвселенной.
\vs p016 3:6 \pc \bibemph{Главный Дух Номер Три}. Этот Дух\hyp{}личность особенно напоминает Бесконечного Духа и руководит передвижениями и работой многих высоких личностей Бесконечного Духа. Он возглавляет их собрания и тесно связан со всеми личностями, которые берут своё исключительное начало от Третьего Источника и Центра. Когда Семь Главных Духов собираются на совет, именно Главный Дух Номер Три всегда говорит от имени Бесконечного Духа.
\vs p016 3:7 Этот Дух отвечает за сверхвселенную номер три и управляет делами этого сегмента во многом так же, как это делал бы Бесконечный Дух. Он всегда поддерживает связь с Отражательными Духами в столице третьей сверхвселенной.
\vs p016 3:8 \pc \bibemph{Главный Дух Номер Четыре}. Разделяя объединённую природу Отца и Сына, этот Главный Дух оказывает определяющее влияние касательно стратегии и тактики Отца\hyp{}Сына в советах Семи Главных Духов. Этот Дух~--- главный руководитель и советник тех восходящих существ, которые достигли Бесконечного Духа и, таким образом, стали кандидатами на встречу с Сыном и Отцом. Он поддерживает ту огромную группу личностей, которые происходят от Отца и Сына. Когда необходимо представлять Отца и Сына в ассоциациях Семи Главных Духов, всегда говорит именно Главный Дух Номер Четыре.
\vs p016 3:9 Этот Дух заботится о четвёртом сегменте большой вселенной в соответствии с его уникальной комбинацией атрибутов Всеобщего Отца и Вечного Сына. Он всегда поддерживает личную связь с Отражательными Духами в столице четвёртой сверхвселенной.
\vs p016 3:10 \pc \bibemph{Главный Дух Номер Пять}. Эта божественная личность, в совершенстве сочетающая в себе характер Всеобщего Отца и Бесконечного Духа, является советником огромной группы существ, известных как управляющие мощью, центры мощи и физические регуляторы. Этот Дух также заботится обо всех личностях, происходящих от Отца и Совместного Вершителя. В советах Семи Главных Духов, когда возникает вопрос о точке зрения Отца\hyp{}Духа, всегда говорит именно Главный Дух Номер Пять.
\vs p016 3:11 Этот Дух управляет благополучием пятой сверхвселенной так, как это предполагает объединённый характер действий Всеобщего Отца и Бесконечного Духа. Он всегда поддерживает связь с Отражательными Духами в столице пятой сверхвселенной.
\vs p016 3:12 \pc \bibemph{Главный Дух Номер Шесть}. Это божественное существо, по всей видимости, изображает объединённый характер Вечного Сына и Бесконечного Духа. Всякий раз, когда в центральной вселенной собираются создания, сотворённые совместно Сыном и Духом, их советником выступает именно этот Главный Дух; и всякий раз, когда в советах Семи Главных Духов необходимо говорить совместно за Вечного Сына и Бесконечного Духа, отвечает именно Главный Дух Номер Шесть.
\vs p016 3:13 Этот Дух управляет делами шестой сверхвселенной подобно тому, как это делали бы Вечный Сын и Бесконечный Дух. Он всегда поддерживает связь с Отражательными Духами в столице шестой сверхвселенной.
\vs p016 3:14 \pc \bibemph{Главный Дух Номер Семь}. Возглавляющий седьмую сверхвселенную Дух~--- это уникально равное изображение Всеобщего Отца, Вечного Сына и Бесконечного Духа. Седьмой Дух~--- наставник и советник всех существ триединого происхождения, является также советником и руководителем всех восходящих пилигримов Хавоны, тех скромных существ, которые достигли чертогов славы [courts of glory] благодаря объединённому служению Отца, Сына и Духа.
\vs p016 3:15 Седьмой Главный Дух не является органическим представителем Райской Троицы; но это известный факт, что его личностная и духовная природа \bibemph{есть} изображение Совместного Вершителя в равных пропорциях трёх бесконечных лиц, чей союз Божеств \bibemph{есть} Райская Троица, и чья функция как таковая \bibemph{есть} источник личностной и духовной природы Бога Верховного. Поэтому Седьмой Главный Дух раскрывает личностное и органическое отношение к духу\hyp{}лицу эволюционирующего Верховного. Поэтому в небесных советах Главных Духов, когда необходимо выбрать совместную личную позицию Отца, Сына и Духа или отобразить духовную позицию Верховного Существа, действует Главный Дух Номер Семь. Таким образом, по своей сущности он становится главой Райского совета Семи Главных Духов.
\vs p016 3:16 Ни один из Семи Духов не является органическим представителем Райской Троицы, но когда они объединяются как семичастное Божество, то этот союз в смысле божества~--- не в личностном смысле~--- равноценен функциональному уровню, связанному с функциями Троицы. В этом смысле <<Семичастный Дух>> способен функционально объединяться с Райской Троицей. Именно в этом же смысле Главный Дух Номер Семь иногда говорит в подтверждение точки зрения Троицы или, скорее, выражает позицию союза Семичастного Духа относительно позиции союза Трёхчастного Божества, точки зрения Райской Троицы.
\vs p016 3:17 Многочисленные функции Седьмого Главного Духа, таким образом, простираются от объединённого изображения \bibemph{личной природы} Отца, Сына и Духа через представление \bibemph{личной позиции} Бога Верховного до раскрытия \bibemph{позиции божества} Райской Троицы. И в некоторых отношениях этот главенствующий Дух одинаково выражает \bibemph{позиции} Предельного и Верховно\hyp{}Предельного.
\vs p016 3:18 Именно Главный Дух Номер Семь, благодаря своим многочисленным качествам, лично содействует прогрессу кандидатов на восхождение из миров времени в их попытках достичь понимания неделимого Божества Верховности. Такое понимание включает в себя постижение зкзистенциального полновластия Троицы Верховности, так скоординированного с концепцией растущего эмпирического полновластия Верховного Существа, чтобы сделать возможным постижение созданиями единства Верховности. Осознание созданиями этих трёх факторов соответствует хавонскому пониманию реальности Троицы и наделяет пилигримов времени способностью в конечном итоге проникнуть в Троицу, открыть три бесконечных лица Божества.
\vs p016 3:19 Неспособность пилигримов Хавоны полностью найти Бога Верховного компенсируется Седьмым Главным Духом, чья триединая природа таким своеобразным способом раскрывает духовное лицо Верховного. В течение настоящей вселенской эпохи, ввиду невозможности контакта с личностью Верховного, Главный Дух Номер Семь функционирует за Бога восходящих созданий в вопросах личных отношений. Он~--- единственное высшее духовное существо, которое все восходящие, несомненно, узн\'ают и отчасти поймут, когда достигнут центров славы.
\vs p016 3:20 Этот Главный Дух всегда поддерживает связь с Отражательными Духами Уверсы~--- столицы седьмой сверхвселенной, нашего собственного сегмента творения. Его управление Орвонтоном раскрывает изумительную симметрию согласованного сочетания божественной природы Отца, Сына и Духа.
\usection{АТРИБУТЫ И ФУНКЦИИ ГЛАВНЫХ ДУХОВ}
\vs p016 4:1 Семь Главных Духов есть полное представление Бесконечного Духа эволюционным вселенным. Они представляют Третий Источник и Центр в соотношениях энергии, разума и духа. Хотя они функционируют как координирующие главы всеобщего административного управления Совместного Вершителя, не забывай, что они берут своё начало в созидательных актах Райских Божеств. Буквально верно, что эти Семь Духов представляют собой персонализированную физическую мощь, космический разум и духовное присутствие триединого Божества, <<Семь Духов Бога, посланных по всей вселенной>>.
\vs p016 4:2 Главные Духи уникальны тем, что они действуют на всех вселенских уровнях реальности, кроме абсолютного. Поэтому они являются эффективными и совершенными руководителями всех фаз административных дел на всех уровнях сверхвселенской активности. Смертному разуму трудно понять очень многое о Главных Духах, потому что их работа является одновременно чрезвычайно специализированной, но всеобъемлющей, столь исключительно материальной и в то же время столь изысканно духовной. Эти разносторонние создатели космического разума~--- прародители Управляющих Вселенской Мощью и сами являются верховными управляющими огромного и обширного творения духовных созданий.
\vs p016 4:3 Семь Главных Духов являются создателями Управляющих Вселенской Мощью и их помощников~--- сущностей, незаменимых для организации, контроля и регулирования физических энергий большой вселенной. И эти же Главные Духи весьма существенно помогают Сынам Создателям в работе по формированию и организации локальных вселенных.
\vs p016 4:4 Мы не имеем возможности проследить какую бы то ни было личностную связь между деятельностью Главных Духов, связанной с космической энергией, и силовыми функциями Безусловного Абсолюта. Все энергетические проявления, находящиеся под юрисдикцией Главных Духов, управляются с периферии Рая; они не кажутся связанными напрямую с силовыми феноменами, отождествляемыми с нижней поверхностью Рая.
\vs p016 4:5 Несомненно, что, сталкиваясь с функциональной активностью различных Управляющих Моронтийной Мощью [Morontia Power Supervisors], мы оказываемся лицом к лицу с определённой нераскрытой деятельностью Главных Духов. Кто, кроме этих прародителей как физических регуляторов, так и духовных помощников, смог бы придумать, как соединить и связать материальную и духовную энергии таким образом, чтобы создать несуществующую ранее фазу вселенской реальности~--- моронтийную субстанцию и моронтийный разум?
\vs p016 4:6 Значительная часть реальности духовных миров относится к моронтийному типу~--- фазе вселенской реальности, совершенно неизвестной на Урантии. Цель существования личности~--- духовная, но моронтийные творения всегда занимают промежуточное положение, перекидывая мост через пропасть между материальными мирами происхождения смертных и сверхвселенскими сферами развивающегося духовного статуса. Именно в этой области Главные Духи вносят свой великий вклад в план Райского восхождения человека.
\vs p016 4:7 У Семи Главных Духов есть личные представители, которые действуют по всей большой вселенной; но ввиду того, что подавляющее большинство этих подчинённых существ не связано непосредственно с планом восхождения смертных по пути Райского совершенства, о них почти ничего не раскрыто. Многое, очень многое из деятельности Семи Главных Духов остаётся скрытым от человеческого понимания, ибо никоим образом не связано непосредственно с вашей задачей Райского восхождения.
\vs p016 4:8 \pc Весьма вероятно, хотя мы и не можем представить конкретных доказательств, что Главный Дух Орвонтона оказывает определённое влияние в следующих сферах деятельности:
\vs p016 4:9 \li{1.}Процедуры зарождения жизни Носителями Жизни локальной вселенной.
\vs p016 4:10 \li{2.}Активация жизни адъютантами разумо\hyp{}духами, дарованными мирам Созидательным Духом локальной вселенной.
\vs p016 4:11 \li{3.}Флуктуации энергетических проявлений, представленные единицами организованной материи, которые реагируют на линейную гравитацию.
\vs p016 4:12 \li{4.}Поведение появляющейся энергии, полностью освобождённой от охвата Безусловного Абсолюта и таким образом становящейся восприимчивой к прямому воздействию линейной гравитации и манипуляциям Управляющих Вселенской Мощью и их помощников.
\vs p016 4:13 \li{5.}Дар духа помощи Созидательного Духа локальной вселенной, известного на Урантии как Святой Дух.
\vs p016 4:14 \li{6.}Последующий дар духа посвящения Сынов, называемого на Урантии Утешителем, или Духом Истины.
\vs p016 4:15 \li{7.}Механизм отражательности локальных вселенных и сверхвселенной. Многие особенности, связанные с этим необычным феноменом, едва ли могут быть разумно объяснены или рационально поняты без постулирования деятельности Главных Духов в ассоциации с Совместным Вершителем и Верховным Существом.
\vs p016 4:16 \pc Несмотря на нашу неспособность в полной мере понять разнообразные действия Семи Главных Духов, мы уверены, что в огромной области вселенской деятельности существуют две сферы, к которым они не имеют абсолютно никакого отношения: дар и служение Настройщиков Мыслей и непостижимые функции Безусловного Абсолюта.
\usection{ОТНОШЕНИЕ К СОЗДАНИЯМ}
\vs p016 5:1 Каждый сегмент большой вселенной, каждая отдельная вселенная и мир пользуются преимуществами объединённого совета и мудрости всех Семи Главных Духов, но несут в себе индивидуальный штрих и оттенок только одного. И личная природа каждого Главного Духа всецело пронизывает и уникальным образом обусловливает его сверхвселенную.
\vs p016 5:2 Благодаря этому личному влиянию Семи Главных Духов каждое создание каждого вида разумных существ, вне Рая и Хавоны, должно носить характерный отпечаток индивидуальности, указывающий на наследственную природу одного из Семи Райских Духов. Что касается семи сверхвселенных, то каждое рождённое в них существо, человек или ангел, будет вечно носить этот знак натальной идентификации.
\vs p016 5:3 Семь Главных Духов не вторгаются непосредственно в материальный разум индивидуальных созданий на эволюционных мирах пространства. Смертные Урантии не ощущают личного присутствия разумо\hyp{}духовного влияния Главного Духа Орвонтона. Если этот Главный Дух и достигает какого\hyp{}либо рода контакта с отдельным смертным разумом на ранних этапах эволюции обитаемого мира, то это происходит через служение Созидательного Духа локальной вселенной, супруги и партнёра Божьего Сына Создателя, руководящего судьбами каждого локального творения. Но именно этот Созидательный Материнский Дух по своей природе и характеру совершенно подобен Главному Духу Орвонтона.
\vs p016 5:4 Физический отпечаток, накладываемый Главным Духом, является частью материального происхождения человека. Весь моронтийный путь проходит под постоянным влиянием того же Главного Духа. Неудивительно, что последующий духовный путь такого восходящего смертного никогда полностью не искореняет характерный отпечаток этого руководящего Духа. Отпечаток Главного Духа~--- основа самог\'о существования любой пред\hyp{}Хавонской стадии восхождения смертного.
\vs p016 5:5 Отличительные тенденции личности, проявляемые в жизненном опыте эволюционных смертных, которые характерны для каждой сверхвселенной и которые непосредственно выражают природу доминирующего Главного Духа, никогда не стираются полностью даже после того, как такие восходящие подвергаются долгой подготовке и объединяющей дисциплине, с которыми они сталкиваются на миллиарде образовательных сфер Хавоны. Даже последующей интенсивной Райской культуры недостаточно для того, чтобы искоренить отличительные признаки сверхвселенского происхождения. Через всю вечность восходящий смертный будет проявлять черты, свидетельствующие о руководящем Духе сверхвселенной его рождения. Даже в Корпусе Завершения, когда желают достичь \bibemph{полного} отношения Троицы к эволюционному творению или изобразить его, всегда собирается группа из семи завершителей~--- по одному от каждой сверхвселенной.
\usection{КОСМИЧЕСКИЙ РАЗУМ}
\vs p016 6:1 Главные Духи~--- это семичастный источник космического разума, интеллектуального потенциала большой вселенной. Этот космический разум является субабсолютным проявлением разума Третьего Источника и Центра и определённым образом функционально связан с разумом эволюционирующего Верховного Существа.
\vs p016 6:2 В мире, подобном Урантии, мы не сталкиваемся с прямым влиянием Семи Главных Духов на дела человеческих рас. Вы живёте под непосредственным влиянием Созидательного Духа Небадона. Тем не менее эти же Главные Духи доминируют в основных реакциях разума всех созданий, ибо они~--- действительные источники интеллектуального и духовного потенциалов, специализированных в локальных вселенных для функционирования в жизни тех индивидуумов, которые населяют эволюционные миры времени и пространства.
\vs p016 6:3 Факт существования космического разума объясняет родство различных типов человеческих и сверхчеловеческих разумов. Не только родственные духи тянутся друг к другу, но и родство разума также сближает и способствует сотрудничеству друг с другом. Иногда можно наблюдать, как человеческие разумы функционируют в каналах поразительного сходства и необъяснимого согласия.
\vs p016 6:4 \pc Во всех личностных ассоциациях космического разума существует качество, которое можно было бы назвать <<откликом реальности>>. Именно этот всеобщий космический дар волевых созданий не даёт им стать беспомощными жертвами априорных предположений науки, философии и религии. Эта чувствительность космического разума к реальности реагирует на определённые фазы реальности точно так же, как энергия\hyp{}материя реагирует на гравитацию. Было бы ещё правильнее сказать, что эти сверхматериальные реальности так реагируют на разум космоса.
\vs p016 6:5 Космический разум неизменно откликается (распознаёт ответ) на трёх уровнях вселенской реальности. Эти ответы самоочевидны рассудительным и глубоко мыслящим умам. Вот эти уровни реальности:
\vs p016 6:6 \li{1.}\bibemph{Причинность}~--- область реальности физических чувств, научные области логической однородности, дифференциация фактического и нефактического, рефлективные заключения, основанные на космическом отклике. Это~--- математическая форма космического различения.
\vs p016 6:7 \li{2.}\bibemph{Долг}~--- область реальности нравственности в философской сфере, арена рассудка, признание относительности правильного и неправильного. Это~--- судебная форма космического различения.
\vs p016 6:8 \li{3.}\bibemph{Поклонение}~--- духовная область реальности религиозного опыта, личное осознание божественного братства, признание духовных ценностей, уверенность в вечном выживании, восхождение от статуса слуг Бога к радости и свободе сынов Бога. Это~--- высшая проницательность космического разума, благоговейная и полная поклонения форма космического различения.
\vs p016 6:9 \pc Научная, нравственная и духовная проницательность~--- эти космические отклики являются врождёнными в космическом разуме, которым наделены все волевые создания. Опыт жизни неизменно развивает эти три космические интуиции; они составляют основу самосознания рефлективного мышления. Однако приходится с грустью отметить, что так мало людей на Урантии получают удовольствие в развитии этих качеств смелого и независимого космического мышления.
\vs p016 6:10 \pc С дарованием разума в локальных вселенных эти три проницательности космического разума образуют априорные допущения, позволяющие человеку функционировать как рациональная и самосознающая личность в сферах науки, философии и религии. Иначе говоря, осознание \bibemph{реальности} этих трёх проявлений Бесконечного происходит благодаря космическому методу самооткровения. Материя\hyp{}энергия осознаётся математической логикой чувств; разум\hyp{}рассудок интуитивно знает свой нравственный долг; дух\hyp{}вера (поклонение) является религией реальности духовного опыта. Эти три основных фактора рефлективного мышления могут быть объединены и скоординированы в развитии личности, либо они могут стать несоразмерными и практически не связанными в своих соответствующих функциях. Но объединённые, они создают сильный характер, состоящий во взаимосвязи фактической науки, нравственной философии и подлинного религиозного опыта. И именно эти три космические интуиции придают объективную обоснованность~--- реальность~--- человеческому опыту в вещах, значениях и ценностях.
\vs p016 6:11 Цель образования~--- развить и усилить эти врождённые способности человеческого разума; цивилизации~--- выразить их; жизненного опыта~--- реализовать их; религии~--- облагородить их; и личности~--- объединить их.
\usection{НРАВСТВЕННОСТЬ, ДОБРОДЕТЕЛЬ И ЛИЧНОСТЬ}
\vs p016 7:1 Наличием одного лишь интеллекта невозможно объяснить нравственную природу. Нравственность и добродетель присущи человеческой личности. Нравственная интуиция~--- осознание долга~--- есть одна из составляющих дарования человеческого разума и связана с другими неотъемлемыми свойствами человеческой природы: научной любознательностью и духовной проницательностью. Интеллект человека далеко превосходит интеллект его животных сородичей, но именно его нравственная и религиозная природа особенно отличают его от животного мира.
\vs p016 7:2 Избирательная реакция животного ограничена двигательным уровнем поведения. Предполагаемая проницательность высших животных существует на двигательном уровне и обычно появляется только после опыта двигательных проб и ошибок. Человек же способен проявлять научную, нравственную и духовную проницательность до любых исследований или экспериментов.
\vs p016 7:3 Только личность может знать, чт\'о она делает до того, как она это сделает; только личности обладают проницательностью, опережающей опыт. Личность может подумать прежде, чем прыгать, и поэтому может учиться как думая, так и прыгая. Неличностное животное обычно учится только прыгая.
\vs p016 7:4 В результате опыта животное становится способным рассмотреть различные способы достижения цели и выбрать подход, основанный на накопленном опыте. А личность может также рассмотреть саму цель и вынести суждение о её значимости и ценности. Одного интеллекта достаточно для выбора лучших средств достижения беспорядочных целей, но нравственное существо обладает проницательностью, которая позволяет ему провести различие как в целях, так и в средствах. И нравственное существо, выбирая добродетель, остаётся тем не менее разумным. Оно знает, что оно делает, почему оно это делает, куда оно идёт и как туда попадёт.
\vs p016 7:5 Когда человеку не удаётся различить цели своего земного устремления, он, по сути, функционирует на животном уровне существования. Он не смог воспользоваться высшими преимуществами той материальной сообразительности, нравственного различения и духовной проницательности, которые являются неотъемлемой частью дарованного ему как личностному существу космического разума.
\vs p016 7:6 \pc Добродетель есть праведность~--- соответствие космосу. Назвать добродетели~--- не значит дать им определение, но жить ими~--- значит знать их. Добродетель не есть просто знание, и ещё не мудрость, но, скорее, реальность приобретения постепенного опыта в достижении восходящих космических уровней. В повседневной жизни смертного человека добродетель реализуется последовательным выбором добра, а не зла, и такая способность выбора является свидетельством обладания нравственной природой.
\vs p016 7:7 На выбор человека между добром и злом оказывает влияние не только степень чуткости его нравственной природы, но и невежество, незрелость и заблуждение. Чувство меры также важно в проявлении добродетели, потому что зло может быть совершено, когда меньшее выбирается вместо большего в результате искажения или обмана. Искусство относительной оценки, или сравнительной меры, входит в практику добродетелей нравственной сферы.
\vs p016 7:8 \pc Нравственная природа человека была бы бессильна без искусства чувства меры и различения, воплощённого в его способности проникать в суть значений. Точно так же тщетным был бы нравственный выбор без той космической проницательности, которая ведёт к сознанию духовных ценностей. С точки зрения разума человек восходит к уровню нравственного существа потому, что он наделён личностью.
\vs p016 7:9 \pc Нравственность невозможно утверждать ни законом, ни силой. Это личное и добровольное дело, и распространяться она должна благодаря контактам людей, <<благоухающих нравственностью>>, с теми, кто менее нравственно отзывчив, но кто также в какой\hyp{}то мере желает исполнять волю Отца.
\vs p016 7:10 Нравственные действия~--- это те человеческие поступки, которые характеризуются высочайшей разумностью и руководствуются избирательным различением в выборе высших целей, а также в выборе нравственных средств для достижения этих целей. Такое поведение добродетельно. Итак, высшая добродетель~--- есть искреннее решение исполнять волю небесного Отца.
\usection{УРАНТИЙСКАЯ ЛИЧНОСТЬ}
\vs p016 8:1 Всеобщий Отец дарует личность многочисленным категориям существ, функционирующим на различных уровнях вселенской действительности. Человеческие существа Урантии наделяются личностью конечно\hyp{}смертного типа, действующей на уровне восходящих сынов Бога.
\vs p016 8:2 Хотя мы едва ли можем дать определение личности, мы можем попытаться изложить наше понимание известных факторов, входящих в состав ансамбля материальных, ментальных и духовных энергий, чья взаимосвязь составляет механизм, в котором, на основании которого и с помощью которого Всеобщий Отец вызывает функционирование дарованной им личности.
\vs p016 8:3 Личность~--- это уникальный дар оригинальной природы, чьё существование не зависит от дара Настройщиков Мыслей и предшествует ему. Тем не менее присутствие Настройщика усиливает качественное проявление личности. Настройщики Мыслей, исходящие от Отца, идентичны по своей природе, но личности~--- разнообразны, оригинальны и исключительны; и дальнейшее проявление личности обусловливается и определяется природой и качествами связанных энергий материальной, ментальной и духовной природы, которые составляют организм\hyp{}проводник для проявления личности.
\vs p016 8:4 Личности могут быть похожими, но никогда~--- одинаковыми. Лица данной серии, типа, вида или образца могут походить друг на друга, и действительно похожи, но они никогда не бывают идентичны. Личность~--- это та особенность индивидуума, которую мы \bibemph{знаем} и которая позволяет нам идентифицировать такое существо в будущем, независимо от характера и степени изменений в форме, разуме или статусе духа. Личность~--- это та часть любого индивидуума, которая позволяет нам распознать и определённо идентифицировать это лицо как то, которое мы знали прежде, независимо от того, насколько сильно оно могло измениться из-за модификации средства выражения и проявления его личности.
\vs p016 8:5 \pc Личность создания отличается двумя самопроявляющимися и характерными феноменами реагирующего поведения смертных: самосознанием и связанной с ним относительной свободной волей.
\vs p016 8:6 Самосознание состоит из интеллектуального осознания действительности личности; оно включает в себя умение осознавать реальность других личностей. Оно указывает на способность к индивидуализированному опыту с космическими реальностями и в них, что равносильно достижению статуса индивидуальности в личностных отношениях во вселенной. Самосознание означает признание актуальности служения разума и осознание относительной независимости созидательной и определяющей свободной воли.
\vs p016 8:7 \pc Относительная свободная воля, характеризующая самосознание человеческой личности, связана с:
\vs p016 8:8 \li{1.}Нравственным решением, высочайшей мудростью.
\vs p016 8:9 \li{2.}Духовным выбором, различением истины.
\vs p016 8:10 \li{3.}Бескорыстной любовью, братским служением.
\vs p016 8:11 \li{4.}Целеустремлённым сотрудничеством, верностью группе.
\vs p016 8:12 \li{5.}Космической проницательностью, постижением вселенских значений.
\vs p016 8:13 \li{6.}Посвящением личности, искренней преданностью исполнению воли Отца.
\vs p016 8:14 \li{7.}Поклонением, искренним стремлением к божественным ценностям и чистосердечной любовью к божественному Дарителю Ценностей.
\vs p016 8:15 \pc Урантийский тип человеческой личности можно рассматривать как функционирующий в физическом механизме, который представляет собой планетарную модификацию организма небадонского типа, принадлежащего к электрохимическому разряду активации жизни и наделённого небадонским типом орвонтонской серии космического разума родительского репродуктивного образца. Посвящение божественного дара личности наделённому разумом смертному механизму жалует достоинством космического гражданства и возможностью такому смертному существу незамедлительно реагировать на основополагающее признание трёх основных разумных реальностей космоса:
\vs p016 8:16 \li{1.}Математическое, или логическое, признание однородности физической причинности.
\vs p016 8:17 \li{2.}Обоснованное признание обязательности нравственного поведения.
\vs p016 8:18 \li{3.}Постижение верой братского поклонения Божеству, объединённое с исполненным любви служением человечеству.
\vs p016 8:19 \pc Полное функционирование такого личностного дара~--- начало осознания родства с Божеством. Такая индивидуальность, в которой обитает доличностный фрагмент Бога Отца, истинно и фактически является духовным сыном Бога. Такое создание не только раскрывает способность к принятию дара божественного присутствия, но и проявляет ответную реакцию на контур гравитации личности Райского Отца всех личностей.
\usection{РЕАЛЬНОСТЬ ЧЕЛОВЕЧЕСКОГО СОЗНАНИЯ}
\vs p016 9:1 Наделённое космическим разумом личностное существо, в котором обитает Настройщик, обладает врождённым признанием\hyp{}осознанием реальности энергии, реальности разума и реальности духа. Волевое создание оснащено таким образом, чтобы различать факт, закон и любовь Бога. Помимо этих трёх неотъемлемых свойств человеческого сознания, весь человеческий опыт действительно субъективен, за ислючением интуитивного осознания обоснованности, связанного с \bibemph{объединением} этих трёх реакций космического признания вселенской реальности.
\vs p016 9:2 Различающий Бога смертный способен чувствовать ценность объединения этих трёх космических качеств в эволюции выживающей души, высочайшем начинании человека в физической скинии, где нравственный разум сотрудничает с пребывающим в нём божественным духом для дуализации бессмертной души. С самого зарождения душа \bibemph{реальна;} она обладает космическими качествами выживания.
\vs p016 9:3 Если смертному человеку не удаётся выжить по наступлении естественной смерти, то реальные духовные ценности его человеческого опыта сохраняются как часть продолжающегося опыта Настройщика Мыслей. Личностные ценности такого невыжившего продолжают существовать как фактор внутри личности актуализирующегося Верховного Существа. Такие продолжающие своё существование качества личности лишены индивидуальности, но не эмпирических ценностей, накопленных в течение земной жизни во плоти. Выживание индивидуальности зависит от выживания бессмертной души моронтийного статуса и возрастающе божественной ценности. Индивидуальность личности сохраняется внутри души и благодаря её выживанию.
\vs p016 9:4 \pc Человеческое самосознание подразумевает признание реальности других <<я>>, отличных от сознательного <<я>>, а также взаимность такого осознания; что <<я>> познаваемо таким же образом, как познаёт само. Это проявляется чисто человеческим образом в общественной жизни человека. Но ты не можешь быть так же абсолютно уверен в реальности ближнего, как в реальности присутствия Бога, живущего внутри тебя. Общественное сознание не является таким же неотъемлемым, как Богосознание; оно является культурным наследием и зависит от знаний, символов и вклада основных дарований человека~--- науки, морали и религии. Из этих обобществлённых космических даров и состоит цивилизация.
\vs p016 9:5 Цивилизации нестабильны, потому что они не космичны; они не являются врождёнными для индивидуумов рас. Они должны взращиваться объединёнными достижениями основополагающих факторов человека~--- науки, морали и религии. Цивилизации приходят и уходят, но наука, мораль и религия всегда сохраняются после катаклизма.
\vs p016 9:6 Иисус не только раскрыл Бога человеку, но также по\hyp{}новому раскрыл человека самому себе и другим людям. В жизни Иисуса ты видишь человека в его лучших проявлениях. Человек становится, таким образом, столь прекрасно реальным, потому что в жизни Иисуса было так много от Бога, а осознание (признание) Бога является неотъемлемым и основополагающим во всех людях.
\vs p016 9:7 \pc Бескорыстие, за исключением родительского инстинкта, не является всецело естественным; не являются естественными любовь к другим людям и общественное служение. Для создания бескорыстного и альтруистического социального порядка требуются просвещение ума, мораль и побуждение религии, Богопознание. Осознание человеком собственной личности~--- самосознание~--- также непосредственно зависит от этого с\'амого факта врождённого осознания других, этой врождённой способности распознавать и понимать реальность другой личности~--- от человеческой до божественной.
\vs p016 9:8 Бескорыстное общественное сознание должно быть, по сути, религиозным сознанием; то есть если оно объективное; иначе оно является чисто субъективной философской абстракцией и поэтому лишённым любви. Только Богопознавший индивидуум может любить другого т\'ак, как он любит самого себя.
\vs p016 9:9 Самосознание по своей сути есть коммунальное сознание: Бог и человек, Отец и сын, Создатель и создание. В человеческом самосознании четыре осознания вселенской реальности являются скрытыми и неотъемлемыми:
\vs p016 9:10 \li{1.}Поиск знания, логика науки.
\vs p016 9:11 \li{2.}Поиск нравственных ценностей, чувство долга.
\vs p016 9:12 \li{3.}Поиск духовных ценностей, религиозный опыт.
\vs p016 9:13 \li{4.}Поиск личностных ценностей, способность признавать реальность Бога как личности и одновременно осознавать наши братские отношения с собратьями\hyp{}личностями.
\vs p016 9:14 \pc Ты начинаешь осознавать человека как своего собрата\hyp{}создание, потому что ты уже осознаёшь Бога как своего Отца\hyp{}Создателя. Отцовство~--- это отношение, которое убеждает нас в признании братства. И Отцовство становится, или может стать, вселенской реальностью для всех нравственных созданий, ибо Отец сам даровал личность всем таким существам и заключил их в пределы всеобщего личностного контура. Мы поклоняемся Богу, ибо, во\hyp{}первых, \bibemph{он существует,} затем, \bibemph{он существует в нас} и, наконец, \bibemph{мы существуем в нём}.
\vs p016 9:15 Разве странно, что космический разум, сознавая себя, осознаёт свой собственный источник~--- бесконечный разум Бесконечного Духа~--- и в то же время осознаёт физическую реальность необъятных вселенных, духовную реальность Вечного Сына и личностную реальность Всеобщего Отца?
\vsetoff
\vs p016 9:16 [При поддержке Всеобщего Цензора из Уверсы.]
\quizlink
\begin{thebibliography}{100}
\bibitem{Griffiths1}
Rees Griffiths.
{<<God in Idea and Experience, Or, The A Priori Elements of the Religious Consciousness: An Epistemological Study>>.}
{\em Edinburgh: T. \&\ T. Clark}, 1931.
\bibitem{Schoen1}
Max Schoen.
{<<A Scientific Basis for Moral Action>>.}
{\em The Scientific Monthly, Vol.\,48, No.\,3}, March 1939.
\end{thebibliography}
