\upaper{16}{СЕМЬ ГЛАВНЫХ ДУХОВ}
\uminitoc{ОТНОШЕНИЕ К ТРИЕДИНОМУ БОЖЕСТВУ}
\uminitoc{ОТНОШЕНИЕ К БЕСКОНЕЧНОМУ ДУХУ}
\uminitoc{ИНДИВИДУАЛЬНОСТЬ И РАЗНООБРАЗИЕ ГЛАВНЫХ ДУХОВ}
\uminitoc{АТРИБУТЫ И ФУНКЦИИ ГЛАВНЫХ ДУХОВ}
\uminitoc{ОТНОШЕНИЕ К СОЗДАНИЯМ}
\uminitoc{КОСМИЧЕСКИЙ РАЗУМ}
\uminitoc{НРАВСТВЕННОСТЬ, ДОБРОДЕТЕЛЬ И ЛИЧНОСТЬ}
\uminitoc{УРАНТИЙСКАЯ ЛИЧНОСТЬ}
\uminitoc{РЕАЛЬНОСТЬ ЧЕЛОВЕЧЕСКОГО СОЗНАНИЯ}
\author{Всеобщий Цензор}
\vs p016 0:1 Семь Главных Духов Рая --- первичные личности Бесконечного Духа. В этом семичастном созидательном акте самоповторения Бесконечный Дух исчерпал комбинаторные возможности, математически свойственные фактическому существованию трёх лиц Божества. Если бы возможно было создать большее число Главных Духов, они были бы созданы, но существует семь, и только семь, возможных комбинаций, присущих трём Божествам. Это объясняет, почему вселенная управляется в семи больших разделах, а число семь является фундаментальным в основе её организации и управления.
\vs p016 0:2 Cемь Главных Духов, таким образом, берут своё начало в следующих семи прообразах, наследуя от них индивидуальные характерные черты:
\vs p016 0:3 \li{1.}Всеобщий Отец.
\vs p016 0:4 \li{2.}Вечный Сын.
\vs p016 0:5 \li{3.}Бесконечный Дух.
\vs p016 0:6 \li{4.}Отец и Сын.
\vs p016 0:7 \li{5.}Отец и Дух.
\vs p016 0:8 \li{6.}Сын и Дух.
\vs p016 0:9 \li{7.}Отец, Сын и Дух.
\vs p016 0:10 \pc Мы очень мало знаем о деятельности Отца и Сына в создании Главных Духов. По всей видимости, они появились в результате личных действий Бесконечного Духа, однако нас со всей определённостью учили, что и Отец, и Сын также участвовали в этом созидательном акте.
\vs p016 0:11 По своей духовной природе и сущности Семь Духов Рая едины, но очень непохожи во всех других аспектах существа, и по оценке их действий в сверхвселенных индивидуальные отличия каждого безошибочно распознаваемы. Все последующие планы семи сегментов большой вселенной --- и даже соответствующих сегментов внешнего пространства --- обусловлены внедуховным разнообразием Семи Главных Духов в верховном и предельном сверхконтроле.
\vs p016 0:12 Главные Духи выполняют множество функций, но в настоящее время основная сфера их деятельности --- центральное руководство семью сверхвселенными. Каждый Главный Дух поддерживает огромный фокально\hyp{}силовой центр, медленно обращающийся вокруг периферии Рая, всегда сохраняя позицию напротив вселенной непосредственного контроля и в Райском фокусе специализированного управления мощью и сегментированного распределения энергии. Радиальные граничные линии любой сверхвселенной действительно сходятся в Райском центре руководящего Главного Духа.
\usection{ОТНОШЕНИЕ К ТРИЕДИНОМУ БОЖЕСТВУ}
\vs p016 1:1 Совместный Создатель --- Бесконечный Дух --- необходим для завершения триединой персонализации неделимого Божества. Эта тройная персонализация Божества является, по сути, семичастной в возможности индивидуального и ассоциативного выражения; поэтому последующий план создания вселенных, населённых разумными и потенциально духовными существами, должным образом выражающими Отца, Сына и Духа, сделал персонализацию Семи Главных Духов неизбежной. Обычно мы говорим о тройной персонализации Божества как об \bibemph{абсолютной неизбежности,} в то же время рассматривая появление Семи Главных Духов как \bibemph{субабсолютную неизбежность}.
\vs p016 1:2 Хотя Семь Главных Духов едва ли выражают \bibemph{трёхчастное} Божество, они являются вечным изображением \bibemph{семичастного} Божества, активными и ассоциативными функциями трёх вечно существующих лиц Божества. С помощью этих Семи Духов, в них и через них, Всеобщий Отец, Вечный Сын или Бесконечный Дух, или любая их парная комбинация, способны действовать как таковые. Когда Отец, Сын и Дух действуют вместе, они могут и действительно функционируют через Главного Духа Номер Семь, но не как Троица. Порознь и совместно Главные Духи представляют любую и все возможные функции Божества, одну или несколько, но не совокупные, не Троицу. Главный Дух Номер Семь лично не функционирует по отношению к Райской Троице, и именно по этой причине он может функционировать \bibemph{лично} от имени Верховного Существа.
\vs p016 1:3 Но когда Семь Главных Духов покидают свои индивидуальные места личной власти и сверхвселенских полномочий и собираются вокруг Совместного Вершителя в триедином присутствии Райского Божества, они коллективно становятся представителями функциональной власти, мудрости и полномочий неделимого Божества --- Троицы --- по отношению к развивающимся вселенным и в них. Такой Райский союз первичного семичастного выражения Божества действительно охватывает, буквально заключает в себе, все атрибуты и отношения трёх вечных Божеств в Верховности и в Предельности. Фактически Семь Главных Духов действительно и сразу охватывают функциональную область Верховного\hyp{}Предельного в главной вселенной и по отношению к ней.
\vs p016 1:4 Насколько мы можем судить, эти Семь Духов связаны с божественной деятельностью трёх вечных лиц Божества; мы не обнаруживаем никакого свидетельства непосредственной связи с функционирующими присутствиями трёх вечных фаз Абсолюта. Во взаимосвязи Главные Духи представляют Райских Божеств в том, что можно приблизительно представить себе как конечную область действия. Эта область может включать многое из того, что является предельным, но \bibemph{не} абсолютным.
\usection{ОТНОШЕНИЕ К БЕСКОНЕЧНОМУ ДУХУ}
\vs p016 2:1 Как Вечный и Изначальный Сын раскрывается через лица постоянно увеличивающегося числа божественных Сынов, так и Бесконечный и Божественный Дух раскрывается через каналы Семи Главных Духов и связанные с ними группы духов. В центре центров Бесконечный Дух доступен, но не все, кто достигает Рая, способны сразу же распознать его личность и дифференцированное присутствие; но все, кто достигает центральной вселенной, могут непосредственно общаться, и действительно общаются, с одним из Семи Главных Духов, возглавляющим ту сверхвселенную, откуда поступил вновь прибывший пилигрим пространства.
\vs p016 2:2 Со вселенной вселенных Райский Отец говорит только через своего Сына, в то же время вместе, он и Сын, действуют только через Бесконечного Духа. Вне Рая и Хавоны Бесконечный Дух \bibemph{говорит} только голосами Семи Главных Духов.
\vs p016 2:3 \pc Бесконечный Дух оказывает влияние \bibemph{личного присутствия} внутри системы Рай\hyp{}Хавона; за её пределами присутствие его личного духа проявляется одним из Семи Главных Духов и через него. Поэтому сверхвселенское присутствие духа Третьего Источника и Центра на любом мире или в любом индивидууме обусловлено уникальной природой руководящего Главного Духа данного сегмента творения. И наоборот, объединённые линии силы духа и разума проходят внутрь, к Третьему Лицу Божества, через Семь Главных Духов.
\vs p016 2:4 \pc Семь Главных Духов коллективно наделяются верховно\hyp{}предельными атрибутами Третьего Источника и Центра. Хотя каждый индивидуально получает этот дар, только коллективно они раскрывают атрибуты всемогущества, всеведения и вездесущности. Поэтому ни один из них не может функционировать универсально; как индивидуумы, в проявлении таких полномочий верховности и предельности, каждый лично ограничен сверхвселенной непосредственного контроля.
\vs p016 2:5 Всё сказанное тебе о божественности и личности Совместного Вершителя в равной и полной мере относится к Семи Главным Духам, которые столь эффективно распределяют Бесконечный Дух по семи сегментам большой вселенной в соответствии со своим божественным даром и согласно своей различной и индивидуально\hyp{}уникальной природе. Поэтому вполне уместно применить к коллективной группе из семи любое или все имена Бесконечного Духа. Вместе они едины с Совместным Создателем на всех субабсолютных уровнях.
\usection{ИНДИВИДУАЛЬНОСТЬ И РАЗНООБРАЗИЕ ГЛАВНЫХ ДУХОВ}
\vs p016 3:1 Семь Главных Духов --- неописуемые существа, но они отчётливо и определённо личностны. У них есть имена, но мы предпочитаем знакомить с ними по номерам. Будучи первичными персонализациями Бесконечного Духа, они родственны друг другу, но как первичные выражения семи возможных комбинаций триединого Божества они существенно отличаются друг от друга по природе, и это различие природы определяет дифференциал их сверхвселенского поведения\fnst{Или <<различие в управлении сверхвселенными>>.}. Эти Семь Главных Духов можно описать следующим образом.
\vs p016 3:2 \bibemph{Главный Дух Номер Один}. Особым образом этот Дух непосредственно представляет Райского Отца. Он --- своеобразное и действенное проявление могущества, любви и мудрости Всеобщего Отца. Он является близким партнёром главы Таинственных Мониторов, того существа, которое возглавляет Коллегию Персонализированных Настройщиков на Дивинингтоне. Во всех ассоциациях Семи Главных Духов от имени Всеобщего Отца всегда говорит Главный Дух Номер Один.
\vs p016 3:3 Этот Дух возглавляет первую сверхвселенную и, неизменно демонстрируя божественную природу первичной персонализации Бесконечного Духа, по всей видимости, по характеру больше похож на Всеобщего Отца. Он всегда поддерживает личную связь с семью Отражательными Духами в столице первой сверхвселенной.
\vs p016 3:4 \pc \bibemph{Главный Дух Номер Два}. Этот Дух --- адекватное изображение несравненной природы и обаятельного характера Вечного Сына, первородного Сына всего творения. Он всегда находится в тесном партнёрстве со всеми категориями Сынов Бога, когда бы ни случилось им оказаться в родной вселенной лично или на радостном конклаве. На всех собраниях Семи Главных Духов он всегда говорит за Вечного Сына и от его имени.
\vs p016 3:5 Этот Дух руководит судьбой второй сверхвселенной и правит этим огромным владением так же, как это делал бы Вечный Сын. Он всегда поддерживает связь с семью Отражательными Духами, расположенными в столице второй сверхвселенной.
\vs p016 3:6 \pc \bibemph{Главный Дух Номер Три}. Этот Дух\hyp{}личность особенно напоминает Бесконечного Духа и руководит передвижениями и работой многих высоких личностей Бесконечного Духа. Он возглавляет их собрания и тесно связан со всеми личностями, которые берут своё исключительное начало от Третьего Источника и Центра. Когда Семь Главных Духов собираются на совет, именно Главный Дух Номер Три всегда говорит от имени Бесконечного Духа.
\vs p016 3:7 Этот Дух отвечает за сверхвселенную номер три и управляет делами этого сегмента во многом так же, как это делал бы Бесконечный Дух. Он всегда поддерживает связь с Отражательными Духами в столице третьей сверхвселенной.
\vs p016 3:8 \pc \bibemph{Главный Дух Номер Четыре}. Разделяя объединённую природу Отца и Сына, этот Главный Дух оказывает определяющее влияние касательно стратегии и тактики Отца\hyp{}Сына в советах Семи Главных Духов. Этот Дух --- главный руководитель и советник тех восходящих существ, которые достигли Бесконечного Духа и, таким образом, стали кандидатами на встречу с Сыном и Отцом. Он поддерживает ту огромную группу личностей, которые происходят от Отца и Сына. Когда необходимо представлять Отца и Сына в ассоциациях Семи Главных Духов, всегда говорит именно Главный Дух Номер Четыре.
\vs p016 3:9 Этот Дух заботится о четвёртом сегменте большой вселенной в соответствии с его уникальной комбинацией атрибутов Всеобщего Отца и Вечного Сына. Он всегда поддерживает личную связь с Отражательными Духами в столице четвёртой сверхвселенной.
\vs p016 3:10 \pc \bibemph{Главный Дух Номер Пять}. Эта божественная личность, в совершенстве сочетающая в себе характер Всеобщего Отца и Бесконечного Духа, является советником огромной группы существ, известных как управляющие мощью, центры мощи и физические регуляторы. Этот Дух также заботится обо всех личностях, происходящих от Отца и Совместного Вершителя. В советах Семи Главных Духов, когда возникает вопрос о точке зрения Отца\hyp{}Духа, всегда говорит именно Главный Дух Номер Пять.
\vs p016 3:11 Этот Дух управляет благополучием пятой сверхвселенной так, как это предполагает объединённый характер действий Всеобщего Отца и Бесконечного Духа. Он всегда поддерживает связь с Отражательными Духами в столице пятой сверхвселенной.
\vs p016 3:12 \pc \bibemph{Главный Дух Номер Шесть}. Это божественное существо, по всей видимости, изображает объединённый характер Вечного Сына и Бесконечного Духа. Всякий раз, когда в центральной вселенной собираются создания, сотворённые совместно Сыном и Духом, их советником выступает именно этот Главный Дух; и всякий раз, когда в советах Семи Главных Духов необходимо говорить совместно за Вечного Сына и Бесконечного Духа, отвечает именно Главный Дух Номер Шесть.
\vs p016 3:13 Этот Дух управляет делами шестой сверхвселенной подобно тому, как это делали бы Вечный Сын и Бесконечный Дух. Он всегда поддерживает связь с Отражательными Духами в столице шестой сверхвселенной.
\vs p016 3:14 \pc \bibemph{Главный Дух Номер Семь}. Возглавляющий седьмую сверхвселенную Дух --- это уникально равное изображение Всеобщего Отца, Вечного Сына и Бесконечного Духа. Седьмой Дух --- наставник и советник всех существ триединого происхождения, является также советником и руководителем всех восходящих пилигримов Хавоны, тех скромных существ, которые достигли чертогов славы [courts of glory] благодаря объединённому служению Отца, Сына и Духа.
\vs p016 3:15 Седьмой Главный Дух не является органическим представителем Райской Троицы; но это известный факт, что его личностная и духовная природа \bibemph{есть} изображение Совместного Вершителя в равных пропорциях трёх бесконечных лиц, чей союз Божеств \bibemph{есть} Райская Троица, и чья функция как таковая \bibemph{есть} источник личностной и духовной природы Бога Верховного. Поэтому Седьмой Главный Дух раскрывает личностное и органическое отношение к духу\hyp{}лицу эволюционирующего Верховного. Поэтому в небесных советах Главных Духов, когда необходимо выбрать совместную личную позицию Отца, Сына и Духа или отобразить духовную позицию Верховного Существа, действует Главный Дух Номер Семь. Таким образом, по своей сущности он становится главой Райского совета Семи Главных Духов.
\vs p016 3:16 Ни один из Семи Духов не является органическим представителем Райской Троицы, но когда они объединяются как семичастное Божество, то этот союз в смысле божества --- не в личностном смысле --- равноценен функциональному уровню, связанному с функциями Троицы. В этом смысле <<Семичастный Дух>> способен функционально объединяться с Райской Троицей. Именно в этом же смысле Главный Дух Номер Семь иногда говорит в подтверждение точки зрения Троицы или, скорее, выражает позицию союза Семичастного Духа относительно позиции союза Трёхчастного Божества, точки зрения Райской Троицы.
\vs p016 3:17 Многочисленные функции Седьмого Главного Духа, таким образом, простираются от объединённого изображения \bibemph{личной природы} Отца, Сына и Духа через представление \bibemph{личной позиции} Бога Верховного до раскрытия \bibemph{позиции божества} Райской Троицы. И в некоторых отношениях этот главенствующий Дух одинаково выражает \bibemph{позиции} Предельного и Верховного\hyp{}Предельного.
\vs p016 3:18 Именно Главный Дух Номер Семь, благодаря своим многочисленным качествам, лично содействует прогрессу кандидатов на восхождение из миров времени в их попытках достичь понимания неделимого Божества Верховности. Такое понимание включает в себя постижение зкзистенциального полновластия Троицы Верховности, так скоординированного с концепцией растущего эмпирического полновластия Верховного Существа, чтобы сделать возможным постижение созданиями единства Верховности. Осознание созданиями этих трёх факторов соответствует хавонскому пониманию реальности Троицы и наделяет пилигримов времени способностью в конечном итоге проникнуть в Троицу, открыть три бесконечных лица Божества.
\vs p016 3:19 Неспособность пилигримов Хавоны полностью найти Бога Верховного компенсируется Седьмым Главным Духом, чья триединая природа таким своеобразным способом раскрывает духовное лицо Верховного. В течение настоящей вселенской эпохи, ввиду невозможности контакта с личностью Верховного, Главный Дух Номер Семь функционирует за Бога восходящих созданий в вопросах личных отношений. Он --- единственное высшее духовное существо, которое все восходящие, несомненно, узн\'ают и отчасти поймут, когда достигнут центров славы.
\vs p016 3:20 Этот Главный Дух всегда поддерживает связь с Отражательными Духами Уверсы --- столицы седьмой сверхвселенной, нашего собственного сегмента творения. Его управление Орвонтоном раскрывает изумительную симметрию согласованного сочетания божественной природы Отца, Сына и Духа.
\usection{АТРИБУТЫ И ФУНКЦИИ ГЛАВНЫХ ДУХОВ}
\vs p016 4:1 
\vs p016 4:2 
\vs p016 4:3 
\vs p016 4:4 
\vs p016 4:5 
\vs p016 4:6 
\vs p016 4:7 
\vs p016 4:8 \pc 
\vs p016 4:9 
\vs p016 4:10 
\vs p016 4:11 
\vs p016 4:12 
\vs p016 4:13 
\vs p016 4:14 
\vs p016 4:15 
\vs p016 4:16 \pc 
\usection{Relation to Creatures}
\vs p016 5:1 
\vs p016 5:2 
\vs p016 5:3 
\vs p016 5:4 
\vs p016 5:5 
\usection{The Cosmic Mind}
\vs p016 6:1 
\vs p016 6:2 
\vs p016 6:3 
\vs p016 6:4 \pc 
\vs p016 6:5 
\vs p016 6:6 
\vs p016 6:7 
\vs p016 6:8 
\vs p016 6:9 \pc 
\vs p016 6:10 \pc 
\vs p016 6:11 
\usection{Morals, Virtue, and Personality}
\vs p016 7:1 
\vs p016 7:2 
\vs p016 7:3 
\vs p016 7:4 
\vs p016 7:5 
\vs p016 7:6 \pc 
\vs p016 7:7 
\vs p016 7:8 \pc 
\vs p016 7:9 \pc 
\vs p016 7:10 
\usection{Urantia Personality}
\vs p016 8:1 
\vs p016 8:2 
\vs p016 8:3 
\vs p016 8:4 
\vs p016 8:5 \pc 
\vs p016 8:6 
\vs p016 8:7 \pc 
\vs p016 8:8 
\vs p016 8:9 
\vs p016 8:10 
\vs p016 8:11 
\vs p016 8:12 
\vs p016 8:13 
\vs p016 8:14 
\vs p016 8:15 \pc 
\vs p016 8:16 
\vs p016 8:17 
\vs p016 8:18 
\vs p016 8:19 \pc 
\usection{Reality of Human Consciousness}
\vs p016 9:1 
\vs p016 9:2 
\vs p016 9:3 
\vs p016 9:4 \pc 
\vs p016 9:5 
\vs p016 9:6 
\vs p016 9:7 \pc 
\vs p016 9:8 
\vs p016 9:9 
\vs p016 9:10 
\vs p016 9:11 
\vs p016 9:12 
\vs p016 9:13 
\vs p016 9:14 \pc 
\vs p016 9:15 
\vsetoff
\vs p016 9:16 
\quizlink
