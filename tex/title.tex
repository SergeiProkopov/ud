\makeatletter
\bib@raise@anchor{\bibpdfbookmark[0]{Титульный лист}{Ttl}}%
\makeatother

\vspace*{\stretch{0.1}}
\begin{center}
{%
\bibcovertitlefont
\titlefontsize
ПЯТОЕ ЭПОХАЛЬНОЕ\\
ОТКРОВЕНИЕ,\\
\bibemph{иначе именуемое}\\[2pt]
\titlefontsize
УРАНТИЙСКИЕ ДОКУМЕНТЫ\\
}%
\vspace*{\stretch{0.1}}
\itshape
Перевод с английского\\
под редакцией Тиграна Айвазяна\\
\vspace*{\stretch{0.4}}
\includegraphics[width=0.2\columnwidth]{images/Phoenix-Logo-Circles.jpg}\\
\vspace*{\stretch{0.3}}
\titlesepbig\\
\vspace*{\stretch{0.1}}
\end{center}

\titleframe

\newpage

\begin{center}
\vspace*{\stretch{0.3}}
\begin{center}\shadowbox{\strut\parbox{0.7\linewidth}{\normalsize\bfseries\itshape <<Одна из самых важных вещей в жизни человека --- понять, во что верил Иисус, открыть его идеалы и стремиться к достижению его возвышенной жизненной цели>>. \bibref[(196:1.3)]{p196 1:3}}}\end{center}
\vspace*{\stretch{0.6}}
\tunemarkup{pgnexus10}{\fontsize{12}{16}\itshape}
\tunemarkup{pgkoboaurahd}{\fontsize{10}{15}\itshape}
\parbox{0.9\linewidth}{\centering
Все замечания по адресу: {\makeatletter\upshape\bfseries aivazian.tigran@gmail.com\makeatother}\\[1ex]
\tux\ Вёрстка в \XeLaTeX\ (\TeX\ Live 2020) в системе Linux.\\
Текст набран шрифтом \textbf{\itshape\tunemarkup{garamond}{Adobe }\urantiamainfont} \urantiamainfontsize pt.\\[5pt]
\upshape\normalsize\bfseries Вариант: \tunemarkup{pgnexus10}{10" цветной}\tunemarkup{pgthinmob}{20:9 цветной}\tunemarkup{pgkoboaurahd}{7" ч/б}\\
\upshape\bfseries Дата: \mytoday{}\\
}
\vspace*{\stretch{0.1}}
\end{center}

\titleframe
