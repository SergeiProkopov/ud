\upaper{32}{ЭВОЛЮЦИЯ ЛОКАЛЬНЫХ ВСЕЛЕННЫХ}
\uminitoc{ФИЗИЧЕСКОЕ ВОЗНИКНОВЕНИЕ ВСЕЛЕННЫХ}
\uminitoc{ОРГАНИЗАЦИЯ ВСЕЛЕННОЙ}
\uminitoc{ЭВОЛЮЦИОННАЯ ИДЕЯ}
\uminitoc{ОТНОШЕНИЕ БОГА К ЛОКАЛЬНОЙ ВСЕЛЕННОЙ}
\uminitoc{ВЕЧНАЯ И БОЖЕСТВЕННАЯ ЦЕЛЬ}
\author{Могущественный Посланник}
\vs p032 0:1 Локальная вселенная есть дело рук Сына Создателя Райской категории Михаила. Она охватывает 100 созвездий, каждое из которых включает в себя 100 систем обитаемых миров. Каждая система в конечном итоге будет содержать примерно 1\,000 обитаемых сфер.
\vs p032 0:2 Все эти вселенные времени и пространства являются эволюционными. Созидательный план Райских Михаилов всегда идёт по пути постепенной эволюции и прогрессивного развития физической, интеллектуальной и духовной природы и способностей разнообразных созданий, которые населяют различные категории сфер, составляющих такую локальную вселенную.
\vs p032 0:3 Урантия принадлежит к локальной вселенной, властелином которой является Богочеловек Небадона~--- Иисус из Назарета и Михаил Спасограда. И все планы Михаила для этой вселенной были полностью одобрены Райской Троицей ещё до того, как он отправился в верховное приключение пространства.
\vs p032 0:4 Сыны Бога могут выбирать области для своей созидательной деятельности, но эти материальные творения были изначально спроектированы и запланированы Райскими Зодчими Главной Вселенной.
\usection{ФИЗИЧЕСКОЕ ВОЗНИКНОВЕНИЕ ВСЕЛЕННЫХ}
\vs p032 1:1 Довселенские манипуляции пространственной силой и изначальными энергиями являются работой Райских Главных Организаторов Силы; но в сверхвселенских областях, когда появляющаяся энергия начинает реагировать на локальную, или линейную, гравитацию, они уступают место управляющим мощью соответствующей сверхвселенной.
\vs p032 1:2 Эти управляющие мощью действуют обособленно на доматериальных и постсиловых фазах творения локальной вселенной. У Сына Создателя нет возможности начать организацию вселенной до тех пор, пока управляющие мощью не осуществят мобилизацию пространственных энергий [space\hyp{}energies] в достаточной степени, чтобы обеспечить материальную основу~--- буквальные солнца и материальные сферы~--- для возникающей вселенной.
\vs p032 1:3 \pc Все локальные вселенные обладают примерно одинаковым энергетическим потенциалом, хотя они сильно различаются по физическим размерам и время от времени могут различаться по содержанию видимой материи. Заряд мощи и наделение локальной вселенной потенциальной материей определяются манипуляциями управляющих мощью и их предшественников, а также действиями Сына Создателя и дарованием врождённого физического контроля, которым обладает его созидательный партнёр.
\vs p032 1:4 Энергетический заряд локальной вселенной составляет примерно \bibfrac{1}{100000} от силы, которой наделена её сверхвселенная. В случае Небадона, вашей локальной вселенной, материализация массы немного меньше. С физической точки зрения Небадон обладает таким же физическим даром энергии и материи, который можно обнаружить в любом из локальных творений Орвонтона. Единственное физическое ограничение на процесс эволюционного расширения вселенной Небадон заключается в количественном заряде пространственной энергии, удерживаемом гравитационным контролем связанных сил и личностей объединённого вселенского механизма.
\vs p032 1:5 \pc Когда энергия\hyp{}материя достигает определённой стадии материализации массы, на сцене появляется Райский Сын Создатель, сопровождаемый Созидательной Дочерью Бесконечного Духа. Одновременно с прибытием Сына Создателя начинается работа над архитектурной сферой, которой надлежит стать столичным миром проецируемой локальной вселенной. В течение долгих веков такое локальное творение эволюционирует, солнца стабилизируются, планеты формируются и устанавливаются на своих орбитах, в то время как работа по созданию архитектурных миров, которым предстоит служить столицами созвездий и столицами систем, продолжается.
\usection{ОРГАНИЗАЦИЯ ВСЕЛЕННОЙ}
\vs p032 2:1 В организации вселенных Сынам Создателям предшествуют управляющие мощью и другие существа, происходящие от Третьего Источника и Центра. Из предварительно организованных энергий пространства Михаил, ваш Сын Создатель, создал обитаемые миры вселенной Небадон и с тех пор неизменно посвящён кропотливому труду по их управлению. Из предсущей энергии эти божественные Сыны материализуют видимую материю, проецируют живые создания и при сотрудничестве вселенского присутствия Бесконечного Духа создают разнообразную свиту духов\hyp{}личностей.
\vs p032 2:2 Эти управляющие мощью и регуляторы энергии, которые задолго до Сына Создателя приступили к подготовительной физической работе по организации вселенной, впоследствии служат в величественной связи с этим Вселенским Сыном, навсегда сохраняя совместный контроль над теми энергиями, которые они первоначально организовали и заключили в контуры. На Спасограде и сейчас функционируют те же самые 100 центров мощи, которые сотрудничали с вашим Сыном Создателем при начальном формировании этой локальной вселенной.
\vs p032 2:3 \pc Первый завершённый акт физического творения в Небадоне состоял в организации столичного мира~--- архитектурной сферы Спасограда с его спутниками. С момента первоначальных действий центров мощи и физических регуляторов и до прибытия живого персонала на завершённые сферы Спасограда прошло немногим более одного миллиарда лет вашего нынешнего планетарного времени. За строительством Спасограда сразу же последовало создание 100 столичных миров планируемых созвездий и 10\,000 столичных сфер планируемых локальных систем планетарного контроля и управления вместе с их архитектурными спутниками. Такие архитектурные миры предназначены для размещения как физических, так и духовных личностей, а также промежуточных моронтийных, или переходных, стадий существования.
\vs p032 2:4 Спасоград~--- столица Небадона~--- расположен точно в центре энергии\hyp{}массы локальной вселенной. Но ваша локальная вселенная не является единой астрономической системой, хотя в её физическом центре и находится большая система.
\vs p032 2:5 Спасоград является личным центром Михаила Небадона, но найти его там можно не всегда. Хотя для гладкого функционирования вашей локальной вселенной больше не требуется постоянного присутствия Сына Создателя на столичной сфере, в более ранние эпохи физической организации это было не так. Сын Создатель не может покинуть свой столичный мир до тех пор, пока гравитационная стабилизация его владений не будет достигнута посредством материализации энергии, достаточной для того, чтобы позволить различным контурам и системам уравновешивать друг друга за счёт взаимного материального притяжения.
\vs p032 2:6 \pc Но вот физический план вселенной завершён, и Сын Создатель в союзе с Созидательным Духом разрабатывает свой план создания жизни; после чего данное выражение Бесконечного Духа начинает своё функционирование во вселенной как отдельная созидательная личность. Когда этот первый созидательный акт сформулирован и исполнен, рождается Яркая Утренняя Звезда, олицетворение этой первоначальной творческой концепции индивидуального существования [identity] и идеала божественности. Это глава исполнительной власти вселенной, личный партнёр Сына Создателя, подобный ему во всех аспектах характера, хотя заметным образом и ограниченный в атрибутах божественности.
\vs p032 2:7 И теперь, когда у Сына Создателя есть помощник~--- правая рука~--- и главный исполнитель, следует создание огромного и чудесного множества разнообразных существ. Появляются сыновья и дочери локальной вселенной, и вскоре после этого обеспечивается система правления таким творением, начиная от верховных советов вселенной до отцов созвездий и властелинов локальных систем~--- скоплений тех миров, которые предназначены впоследствии стать местами обитания разнообразных смертных рас волевых созданий; и каждый из этих миров будет возглавлять Планетарный Принц.
\vs p032 2:8 И затем, когда такая вселенная полностью организована и так обильно укомплектована, Сын Создатель приступает к исполнению предложения Отца создать смертного человека по их\fnst{Отца и Сына.} божественному образу.
\vs p032 2:9 \pc Организация планетарных обителей в Небадоне всё ещё продолжается, поскольку эта вселенная является довольно молодым скоплением среди звёздных и планетарных владений Орвонтона. Согласно последней регистрации в Небадоне находится 3\,840\,101 обитаемая планета, а Сатания~--- локальная система, к которой принадлежит ваш мир, является довольно типичной среди других систем.
\vs p032 2:10 Сатания не является однородной физической системой, единой астрономической единицей или организацией. Её 619 обитаемых миров расположены более чем в 500\fnst{Более детально: всего 562 системы, из которых 511 имеют одну планету, 46~--- две, 4~--- три и одна~--- четыре планеты.} различных физических системах. Только в пяти есть более двух обитаемых миров, и из них только одна имеет четыре населённые планеты, в то время как 46 имеют по два обитаемых мира.
\vs p032 2:11 Система обитаемых миров Сатании сильно удалена от Уверсы и того огромного скопления солнц, которое служит физическим, или астрономическим, центром седьмой сверхвселенной. От Иерусема, столицы Сатании, свыше 200\,000 световых лет до физического центра сверхвселенной Орвонтон, расположенного очень и очень далеко в плотном поперечнике Млечного Пути. Сатания находится на периферии локальной вселенной, а Небадон сейчас сильно сместился к краю Орвонтона. От самой удалённой системы обитаемых миров до центра сверхвселенной чуть меньше 250\,000 световых лет\fnst{Это означает, что современные оценки размера нашей Галактики (100\,000 световых лет) являются заниженными по крайней мере в пять раз.}.
\vs p032 2:12 Сейчас вселенная Небадон обращается далеко на юго\hyp{}востоке в сверхвселенском контуре Орвонтона. Ближайшими соседними вселенными являются: Авал\'он\fnst{Это слово встречается пять раз в Урантийских документах: \bibref[38:5.1]{p038 5:1}, \bibref[66:2.7]{p066 2:7}, \bibref[67:6.5]{p067 6:5}, \bibref[77:2.6]{p077 2:6}. Слово <<Авалон>> впервые встречается в <<Истории бриттов>> Гальфрида Монмутского (XII век), как название острова, где был выкован знаменитый меч короля Артура <<Экскалибур>> и куда впоследствии Артур был отправлен на лечение ран, полученных в Битве при Камлане~--- его последнем сражении. Слово <<Камлан>>, между прочим, означает <<искривлённый>>. В написанной несколько позднее <<Истории Мерлина>> этот остров называется <<Insula Pomorum>>~--- <<Остров яблок>>. По мнению исследователя сэра Джона Риса (<<Studies in the Arthurian Legend>>, 1891) слово <<Авалон>> происходит от <<Абаллах>>~--- имени тёмного кельтского божества и само место, по его мнению, находится на западном побережье в Корнуолле. Томас Мэлори (XV век) идентифицирует Авалон с Гластонбери, соответствующем кельтским легендам об <<Острове стекла>>, населённом умершими героями.}, Хенсел\'он, Сансел\'он, Портал\'он, В\'олверинг, Ф\'ановинг и \'Алворинг.
\vs p032 2:13 \pc Однако эволюция локальной вселенной~--- это долгая история. Документы, относящиеся к сверхвселенной, знакомят с этим предметом; документы данного раздела, посвящённые локальным творениям, продолжают это знакомство, а последующие, затрагивающие историю и предназначение Урантии, завершают этот рассказ. Ты сможешь адекватно понять предназначение смертных такого локального творения только тогда, когда внимательно прочитаешь повествования о жизни и учениях вашего Сына Создателя, однажды прожившего жизнь человека в облике смертной плоти на вашем собственном эволюционном мире.
\usection{ЭВОЛЮЦИОННАЯ ИДЕЯ}
\vs p032 3:1 Единственное совершенно установленное творение~--- это Хавона, центральная вселенная, созданная непосредственно мыслью Всеобщего Отца и словом Вечного Сына. Хавона есть экзистенциальная, совершенная и наполненная вселенная, окружающая дом вечных Божеств, центр всего. Творения семи сверхвселенных являются конечными, эволюционными и последовательно развивающимися.
\vs p032 3:2 Все физические системы времени и пространства имеют эволюционное происхождение. Они даже физически не стабилизированы до тех пор, пока не окажутся в установившихся контурах своих сверхвселенных. Так и локальная вселенная не утвердится в свете и жизни до тех пор, пока её физические возможности расширения и развития не будут исчерпаны, и пока духовный статус всех её обитаемых миров не будет навсегда утверждён и стабилизирован.
\vs p032 3:3 За исключением центральной вселенной, совершенство является постепенным достижением. В центральном творении мы имеем образец совершенства, но все другие сферы должны достичь такого совершенства методами, установленными для развития этих конкретных миров или вселенных. И почти бесконечное разнообразие характеризует планы Сынов Создателей по организации, развитию, упорядочению и стабилизации своих соответствующих локальных вселенных.
\vs p032 3:4 \pc За исключением божественного присутствия Отца, каждая локальная вселенная в определённом смысле является копией административной организации центрального, или образцового, творения. Хотя Всеобщий Отец лично присутствует во вселенной своего пребывания, он не пребывает в разуме существ, происходящих из этой вселенной, как он буквально обитает с душами смертных времени и пространства. По\hyp{}видимому, существует премудрая компенсация в корректировке и регулировании духовных дел обширного творения. В центральной вселенной Отец присутствует лично как таковой, но отсутствует в разумах детей этого совершенного творения; во вселенных пространства Отец отсутствует лично, будучи представлен своими Суверенными Сынами, в то время как он близко [intimately] присутствует в разумах своих смертных детей, будучи духовно представлен доличностным присутствием Таинственных Мониторов, пребывающих в умах этих волевых созданий.
\vs p032 3:5 На столице локальной вселенной пребывают все создатели и созидательные личности, представляющие самостоятельную власть и административную автономию, за исключением личного присутствия Всеобщего Отца. В локальной вселенной можно найти что-то от каждого и кого-то от почти любого класса разумных существ центральной вселенной, кроме Всеобщего Отца. Хотя Всеобщий Отец не присутствует лично в локальной вселенной, он лично представлен её Сыном Создателем, сначала~--- наместником Бога, а впоследствии~--- верховным и полновластным правителем по своему собственному праву.
\vs p032 3:6 Чем дальше вниз по шкале жизни, тем труднее становится найти невидимого Отца оком веры. Низшим созданиям~--- а иногда даже более высоким личностям~--- трудно разглядеть Всеобщего Отца в его Сынах Создателях. И вот, в ожидании времени своего духовного возвышения, когда совершенство развития позволит им увидеть Бога лично, они утомляются в движении вперёд, испытывают духовные сомнения, впадают в замешательство и таким образом изолируют себя от непрерывно меняющихся духовных целей своего времени и вселенной. Так они теряют способность увидеть Отца, глядя на Сына Создателя. Самая надёжная защита для создания в течение долгой борьбы по достижению Отца в то время, когда внутренние условия делают такое достижение невозможным,~--- это крепко держаться за истину\hyp{}факт присутствия Отца в его Сынах. Буквально и образно, духовно и личностно, Отец и Сыны~--- одно. Это факт: тот, кто видел Сына Создателя, видел Отца.
\vs p032 3:7 \pc Личности любой вселенной изначально устойчивы и надёжны только в соответствии со степенью их родства с Божеством. Когда происхождение создания отходит достаточно далеко от изначальных и божественных Источников, имеем ли мы дело с Сынами Бога или созданиями служения, принадлежащими Бесконечному Духу, возрастает вероятность дисгармонии, беспорядка, а иногда и восстания~--- греха.
\vs p032 3:8 \pc За исключением совершенных существ, происходящих от Божества, все волевые создания в сверхвселенных имеют эволюционную природу, начиная с низшего уровня и поднимаясь всё выше, а на самом деле~--- внутрь. Даже высокодуховные личности продолжают восходить по шкале жизни, постепенно переходя из жизни в жизнь и со сферы на сферу. А в случае тех, кто принимает Таинственных Мониторов, вообще нет предела возможным высотам их духовного восхождения и вселенских достижений.
\vs p032 3:9 Совершенство созданий времени, когда оно окончательно достигнуто, целиком является приобретением, настоящей личной собственностью. И, несмотря на щедрое присутствие элементов милосердия, тем не менее достижения созданий есть результат индивидуальных усилий и реальной жизни, личностной реакции на существующую среду.
\vs p032 3:10 Факт эволюционного животного происхождения не налагает клейма позора на какую\hyp{}либо личность с точки зрения вселенной, поскольку это исключительный метод создания одного из двух основных типов конечных разумных волевых созданий. Когда достигнуты высоты совершенства и вечности, тем больше чести тем, кто начал с самого низа и радостно поднимался по лестнице жизни, виток за витком, и кто, достигнув высот славы, обретает личный опыт, воплощающий в себе настоящее знание каждой фазы жизни,~--- снизу доверху.
\vs p032 3:11 Во всём этом проявляется мудрость Создателей. Всеобщему Отцу было бы так же легко сделать всех смертных совершенными существами, наделить совершенством своим божественным словом. Но это лишило бы их чудесного опыта приключений и обретения знаний, связанных с долгим и постепенным восхождением внутрь,~--- опыта, который может быть только у тех, кому посчастливилось начать с самого дна живого существования.
\vs p032 3:12 В окружающих Хавону вселенных ровно столько совершенных созданий, сколько необходимо для того, чтобы удовлетворить потребность в образцовых учителях\hyp{}проводниках для тех, кто восходит по эволюционной шкале жизни. Эмпирическая природа личности эволюционного типа является естественным космическим дополнением вечно совершенной природы созданий Рая\hyp{}Хавоны. В действительности и совершенные, и ставшие совершенными создания являются незавершёнными в том, что касается конечной тотальности. Но в дополняющем друг друга общении экзистенциально совершенных созданий системы Рай\hyp{}Хавона с эмпирически ставшими совершенными завершителями, восходящими из эволюционных вселенных, оба типа находят освобождение от присущих им ограничений и таким образом могут совместно пытаться достичь возвышенных высот предельного статуса созданий.
\vs p032 3:13 Эти взаимодействия созданий являются вселенскими последствиями действий и реакций внутри Семичастного Божества, где вечная божественность Райской Троицы соединяется с эволюционирующей божественностью Верховных Создателей время\hyp{}пространственных вселенных в актуализирующем мощь Божестве Верховного Существа, посредством его и через него.
\vs p032 3:14 Божественно совершенное создание и ставшее совершенным эволюционное создание равны в степени потенциала божественности, но различаются по природе. Одно должно полагаться на другого, чтобы достичь верховности служения. Эволюционные сверхвселенные зависят от совершенной Хавоны, обеспечивающей завершающую подготовку их восходящих граждан, но так же и совершенной центральной вселенной требуется существование совершенствующихся сверхвселенных, чтобы обеспечить полноту развития её нисходящих обитателей.
\vs p032 3:15 Два первичных проявления конечной реальности~--- врождённого совершенства и развитого совершенства, будь то личности или вселенные,~--- являются равноправными, зависимыми и интегрированными. Одно нуждается в другом для достижения завершения функции, служения и предназначения.
\usection{ОТНОШЕНИЕ БОГА К ЛОКАЛЬНОЙ ВСЕЛЕННОЙ}
\vs p032 4:1 
\vs p032 4:2 
\vs p032 4:3 \pc 
\vs p032 4:4 
\vs p032 4:5 \pc 
\vs p032 4:6 \pc 
\vs p032 4:7 
\vs p032 4:8 
\vs p032 4:9 \pc 
\vs p032 4:10 \pc 
\vs p032 4:11 
\vs p032 4:12 
\usection{ВЕЧНАЯ И БОЖЕСТВЕННАЯ ЦЕЛЬ}
\vs p032 5:1 
\vs p032 5:2 
\vs p032 5:3 \pc 
\vs p032 5:4 
\vs p032 5:5 
\vs p032 5:6 
\vs p032 5:7 \pc 
\vs p032 5:8 
\vsetoff
\vs p032 5:9 
\quizlink
