\upaper{25}{ВОИНСТВА ПОСЛАННИКОВ ПРОСТРАНСТВА}
\uminitoc{СЕРВИТАЛЫ ХАВОНЫ}
\uminitoc{ВСЕОБЩИЕ МИРОТВОРЦЫ}
\uminitoc{ОБШИРНОЕ СЛУЖЕНИЕ МИРОТВОРЦЕВ}
\uminitoc{ТЕХНИЧЕСКИЕ СОВЕТНИКИ}
\uminitoc{ХРАНИТЕЛИ ЗАПИСЕЙ НА РАЕ}
\uminitoc{НЕБЕСНЫЕ РЕГИСТРАТОРЫ}
\uminitoc{МОРОНТИЙНЫЕ СПУТНИКИ}
\uminitoc{РАЙСКИЕ СПУТНИКИ}
\author{Высокоуполномоченный}
\vs p025 0:1 Промежуточное положение в семье Бесконечного Духа занимают Воинства Посланников Пространства. Эти разносторонние существа функционируют как связующие звенья между высшими личностями и духами\hyp{}служителями. Воинства Посланников включают следующие категории небесных существ:
\vs p025 0:2 \li{1.}Сервиталы Хавоны.
\vs p025 0:3 \li{2.}Всеобщие Миротворцы.
\vs p025 0:4 \li{3.}Технические Советники.
\vs p025 0:5 \li{4.}Хранители Райских Записей.
\vs p025 0:6 \li{5.}Небесные Регистраторы.
\vs p025 0:7 \li{6.}Моронтийные Спутники.
\vs p025 0:8 \li{7.}Райские Спутники.
\vs p025 0:9 \pc Из семи перечисленных групп только три~--- сервиталы, миротворцы и Моронтийные Спутники~--- создаются как таковые; оставшиеся четыре представляют собой уровни достижения ангельских категорий. В соответствии с присущей им природой и достигнутым статусом, воинства посланников по\hyp{}разному служат во вселенной вселенных, но всегда под руководством тех, кто правит сферами их назначения.
\usection{СЕРВИТАЛЫ ХАВОНЫ}
\vs p025 1:1 Хотя и названные сервиталами, эти <<промежуточные создания>> центральной вселенной не являются слугами в каком\hyp{}либо унизительном смысле этого слова. В духовном мире не существует чёрной работы; любое служение~--- священно и радостно; и высшие категории существ не смотрят свысока на низшие формы существования.
\vs p025 1:2 \pc Сервиталы Хавоны являются результатом совместной творческой работы Семи Главных Духов и их помощников~--- Семи Верховных Управляющих Мощью. Это созидательное сотрудничество ближе всего к тому, чтобы стать образцом для длинного ряда воспроизведений дуального типа в эволюционных вселенных, начиная от создания Яркой Утренней Звезды посредством связи Сына Создателя и Созидательного Духа до полового размножения на мирах, подобных Урантии.
\vs p025 1:3 Число сервиталов непомерно, и постоянно создаются ещё. Они появляются группами по 1\,000 на третий момент\fnst{Мне не ясно, что это за <<третий момент>>, ибо на Рае, даже на его дальнем северном секторе, время не имеет смысла.}, следующий за собранием Главных Духов и Верховных Управляющих Мощью в их общей области~--- дальнем северном секторе Рая. Каждый четвёртый сервитал более физического типа, чем остальные; то есть из каждой 1\,000~--- 750, по\hyp{}видимому, относятся к духовному типу, а 250 являются полуфизическими по своей природе. Эти \bibemph{четвёртые создания} в некотором роде относятся к категории материальных существ (материальных в смысле Хавоны) и больше напоминают управляющих физической мощью, чем Главных Духов.
\vs p025 1:4 \pc В личностных взаимоотношениях духовное преобладает над материальным, хотя сейчас на Урантии кажется, что это не так; и в создании Сервиталов Хавоны преобладает закон доминирования духа; установленное соотношение даёт трёх духовных существ к одному полуфизическому.
\vs p025 1:5 \pc Вновь созданные сервиталы вместе с вновь появляющимися Проводниками Выпускников~--- все проходят через курсы обучения, которые постоянно ведут старшие проводники на каждом из семи контуров Хавоны. Затем сервиталы назначаются на те виды деятельности, для которых они лучше всего приспособлены, и, поскольку они бывают двух типов~--- духовные и полуфизические, не существует почти никаких ограничений для сферы работы, которую могут выполнять эти разносторонние существа. Высшие, или духовные, группы выборочно назначаются на службы Отца, Сына или Духа и на работу Семи Главных Духов. Время от времени в больших количествах они направляются служить в образовательные миры, окружающие столичные сферы семи сверхвселенных,~--- миры, посвящённые завершающей подготовке и духовной культуре восходящих душ времени, готовящихся к продвижению к контурам Хавоны. Как духи\hyp{}сервиталы, так и их более физические собратья назначаются помощниками и партнёрами Проводников Выпускников для помощи и наставления различных категорий восходящих созданий, которые достигли Хавоны и которые стремятся достичь Рая.
\vs p025 1:6 Сервиталы Хавоны и Проводники Выпускников проявляют трансцендентную преданность своей работе и трогательную привязанность друг к другу, привязанность, которую, хотя она и духовная, ты мог бы понять лишь в сравнении с феноменом человеческой любви. Есть божественный пафос в разлуке сервиталов с проводниками, как это часто случается, когда сервиталы направляются для исполнения миссий за пределами центральной вселенной; но они уходят с радостью, а не с печалью. Приносящая удовлетворение радость от исполнения высокого долга является эмоцией духовных существ, затмевающей все остальные. Печали нет места при осознании добросовестно исполненного божественного долга. И когда восходящая человеческая душа предстаёт перед Верховным Судьёй, обладающее вечной силой решение не будет определяться материальными успехами или количественными достижениями; приговор, разносящийся по высшим судам, гласит: <<Хорошая работа, добрый и \bibemph{верный} слуга; ты был верен в самом главном; ты будешь поставлен правителем над вселенскими реальностями>>.
\vs p025 1:7 На сверхвселенскую службу Сервиталы Хавоны всегда назначаются в область, возглавляемую Главным Духом, на которого они наиболее похожи в общих и особых духовных прерогативах. Они служат только на образовательных мирах, окружающих столицы семи сверхвселенных, и последний отчёт Уверсы показывает, что почти 138 миллиардов сервиталов служили на её 490 спутниках. Они заняты в бесконечном разнообразии деятельности, связанной с работой этих образовательных миров, составляющих сверхуниверситеты сверхвселенной Орвонтон. Здесь они~--- твои спутники; они спускаются с высот, которые станут твоим следующим этапом пути, чтобы изучать тебя и вдохновлять реальностью и уверенностью в твоём грядущем переходе из вселенных времени в сферы вечности. И в этих контактах сервиталы получают тот предварительный опыт служения восходящим созданиям времени, который так полезен в их последующей работе на контурах Хавоны в качестве помощников Проводников Выпускников или~--- как преображённые сервиталы~--- в качестве самих Проводников Выпускников.
\usection{ВСЕОБЩИЕ МИРОТВОРЦЫ}
\vs p025 2:1 На каждого созданного Сервитала Хавоны создаётся семь Всеобщих Миротворцев, по одному в каждой сверхвселенной. Этот творческий акт включает в себя определённый сверхвселенский метод отражательной реакции на процессы, происходящие на Рае.
\vs p025 2:2 На столичных мирах семи сверхвселенных функционируют семь отражений Семи Главных Духов. Трудно браться за описание материальному разуму природы этих Отражательных Духов. Они~--- истинные личности; тем не менее каждый член сверхвселенской группы точно отражает только одного из Семи Главных Духов. И каждый раз, когда Главные Духи объединяются с управляющими мощью с целью создания группы Сервиталов Хавоны, происходит одновременная фокализация на одном из Отражательных Духов в каждой из сверхвселенских групп, и незамедлительно появляется равное число полноценных Всеобщих Миротворцев на столичных мирах сверхтворений. Если при создании сервиталов Главный Дух Номер Семь проявит инициативу, никто, кроме Отражательных Духов седьмой категории, не станет беременным миротворцами; и одновременно с созданием 1\,000 сервиталов орвонтонского типа в каждой столице сверхвселенной появляется 1\,000 миротворцев седьмой категории. В результате этих актов, отражающих семичастную природу Главных Духов, происходят семь созданных категорий миротворцев, служащих в каждой сверхвселенной.
\vs p025 2:3 Миротворцы предрайского статуса не могут служить в любой сверхвселенной, будучи ограниченными своими родными сегментами творения. Поэтому любой сверхвселенский корпус, включающий одну седьмую часть каждой созданной категории, проводит очень долгое время под влиянием одного из Главных Духов, исключая остальных, ибо, хотя все семь \bibemph{отражены} на столицах сверхвселенных, только один из них является \bibemph{доминирующим} в каждом сверхтворении.
\vs p025 2:4 Каждое из семи сверхтворений действительно пронизано одним из Главных Духов, который руководит его судьбой. Таким образом, каждая сверхвселенная становится подобной гигантскому зеркалу, отражающему природу и характер возглавляющего её Главного Духа, всё это продолжается далее в каждой дочерней локальной вселенной через присутствие и функции Созидательных Материнских Духов. Влияние такой среды на эволюционный рост столь глубоко, что на своих постсверхвселенских путях миротворцы коллективно проявляют 49 эмпирических точек зрения, или проницательностей, каждая под одним углом, поэтому неполная, но все взаимно дополняют друг друга и вместе стремятся охватить круг Верховности.
\vs p025 2:5 \pc В каждой сверхвселенной Всеобщие Миротворцы оказываются удивительным и естественным образом разделёнными на группы по четыре~--- союзы, в которых они продолжают служить. В каждой группе трое являются духовными личностями, а один, как и четвёртые создания среди сервиталов, является полуматериальным существом. Этот квартет представляет собой миротворческую комиссию и укомплектован следующим образом:
\vs p025 2:6 \li{1.}\bibemph{Судья\hyp{}Арбитр}. Один из них, единогласно назначенный тремя другими как самый компетентный и квалифицированный, чтобы действовать в качестве главы судейской группы.
\vs p025 2:7 \li{2.}\bibemph{Дух\hyp{}Адвокат}. Назначается судьёй\hyp{}арбитром для представления доказательств и защиты прав всех личностей, вовлечённых в любое дело, переданное на рассмотрение миротворческой комиссии.
\vs p025 2:8 \li{3.}\bibemph{Божественный Палач}. Миротворец, обладающий присущей его природе способностью устанавливать контакт с материальными существами миров и исполнять решения комиссии. Божественные палачи, будучи четвёртыми созданиями~--- квазиматериальными существами,~--- почти, но не полностью, доступны ограниченному зрению смертных рас\fnst{Возможно, что именно о подобных \bibemph{палачах} объявил царю Навуходоносору пророк Даниил (см.\,Даниил~4:1--34), назвав их <<смотрителями>> (англ. \bibemph{watchers}, арам. \textheb{עִירִין}). Любопытно, что значение арамейского слова \textheb{עִיר} было незнакомо переводчикам Септуагинты и оставлено в греческой транслитерации \textgreek{ιρ}.}.
\vs p025 2:9 \li{4.}\bibemph{Регистратор}. Оставшийся член комиссии автоматически становится регистратором, секретарём трибунала. Он следит за тем, чтобы все записи были должным образом подготовлены для архивов сверхвселенной и для записей локальной вселенной. Если комиссия работает на эволюционном мире, то третий отчёт с помощью палача подготавливается для физических записей правительства системы, в юрисдикции которой находится данный мир.
\vs p025 2:10 \pc На заседании комиссия функционирует как группа из трёх, поскольку адвокат отстраняется на время рассмотрения дела и участвует в формулировании приговора только по завершении слушания. Поэтому эти комиссии иногда называют судейским трио.
\vs p025 2:11 \pc Миротворцы представляют огромную ценность для поддержания гладкого функционирования вселенной вселенных. Пересекая пространство с серафической, тройной, скоростью, они служат как передвижные суды миров~--- комиссии,~--- предназначенные для быстрого разрешения незначительных трудностей. Если бы не эти мобильные и в высшей степени справедливые комиссии, трибуналы сфер были бы безнадёжно перегружены делами о небольших разногласиях в мирах.
\vs p025 2:12 Эти судейские трио не рассматривают дела, имеющие отношение к вечности; душа, вечная перспектива создания времени, никогда не подвергается опасности их действиями. Миротворцы не занимаются вопросами, выходящими за пределы вр\'еменного существования и космического благополучия созданий времени. Но если комиссия однажды приняла проблему в судебное производство, то её решения окончательны и всегда единодушны; решение судьи\hyp{}арбитра не подлежит апелляции.
\usection{ОБШИРНОЕ СЛУЖЕНИЕ МИРОТВОРЦЕВ}
\vs p025 3:1 
\vs p025 3:2 
\vs p025 3:3 
\vs p025 3:4 
\vs p025 3:5 
\vs p025 3:6 
\vs p025 3:7 
\vs p025 3:8 
\vs p025 3:9 
\vs p025 3:10 
\vs p025 3:11 
\vs p025 3:12 
\vs p025 3:13 
\vs p025 3:14 \pc 
\vs p025 3:15 
\vs p025 3:16 
\vs p025 3:17 
\usection{ТЕХНИЧЕСКИЕ СОВЕТНИКИ}
\vs p025 4:1 
\vs p025 4:2 \pc 
\vs p025 4:3 
\vs p025 4:4 
\vs p025 4:5 
\vs p025 4:6 
\vs p025 4:7 
\vs p025 4:8 
\vs p025 4:9 
\vs p025 4:10 \pc 
\vs p025 4:11 
\vs p025 4:12 
\vs p025 4:13 
\vs p025 4:14 \pc 
\vs p025 4:15 
\vs p025 4:16 
\vs p025 4:17 \pc 
\vs p025 4:18 
\vs p025 4:19 
\vs p025 4:20 
\usection{ХРАНИТЕЛИ ЗАПИСЕЙ НА РАЕ}
\vs p025 5:1 
\vs p025 5:2 
\vs p025 5:3 
\vs p025 5:4 
\usection{НЕБЕСНЫЕ РЕГИСТРАТОРЫ}
\vs p025 6:1 
\vs p025 6:2 
\vs p025 6:3 
\vs p025 6:4 
\vs p025 6:5 
\vs p025 6:6 
\usection{МОРОНТИЙНЫЕ СПУТНИКИ}
\vs p025 7:1 
\vs p025 7:2 
\vs p025 7:3 
\vs p025 7:4 
\usection{РАЙСКИЕ СПУТНИКИ}
\vs p025 8:1 
\vs p025 8:2 
\vs p025 8:3 
\vs p025 8:4 
\vs p025 8:5 \pc 
\vs p025 8:6 
\vs p025 8:7 
\vs p025 8:8 \pc 
\vs p025 8:9 \pc 
\vs p025 8:10 
\vs p025 8:11 
\vsetoff
\vs p025 8:12 
\quizlink
