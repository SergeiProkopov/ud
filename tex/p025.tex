\upaper{25}{ВОИНСТВА ПОСЛАННИКОВ ПРОСТРАНСТВА}
\uminitoc{СЕРВИТАЛЫ ХАВОНЫ}
\uminitoc{ВСЕОБЩИЕ МИРОТВОРЦЫ}
\uminitoc{ОБШИРНОЕ СЛУЖЕНИЕ МИРОТВОРЦЕВ}
\uminitoc{ТЕХНИЧЕСКИЕ КОНСУЛЬТАНТЫ}
\uminitoc{ХРАНИТЕЛИ ЗАПИСЕЙ НА РАЕ}
\uminitoc{НЕБЕСНЫЕ ПИСЦЫ}
\uminitoc{МОРОНТИЙНЫЕ СПУТНИКИ}
\uminitoc{РАЙСКИЕ СПУТНИКИ}
\author{Высокоуполномоченный}
\vs p025 0:1 Промежуточное положение в семье Бесконечного Духа занимают Воинства Посланников Пространства. Эти разносторонние существа функционируют как связующие звенья между высшими личностями и духами\hyp{}служителями. Воинства Посланников включают следующие категории небесных существ:
\vs p025 0:2 \li{1.}Сервиталы Хавоны.
\vs p025 0:3 \li{2.}Всеобщие Миротворцы.
\vs p025 0:4 \li{3.}Технические Консультанты.
\vs p025 0:5 \li{4.}Хранители Райских Записей.
\vs p025 0:6 \li{5.}Небесные Писцы.
\vs p025 0:7 \li{6.}Моронтийные Спутники.
\vs p025 0:8 \li{7.}Райские Спутники.
\vs p025 0:9 \pc Из семи перечисленных групп только три~--- сервиталы, миротворцы и Моронтийные Спутники~--- создаются как таковые; оставшиеся четыре представляют собой уровни достижения ангельских категорий. В соответствии с присущей им природой и достигнутым статусом, воинства посланников по\hyp{}разному служат во вселенной вселенных, но всегда под руководством тех, кто правит сферами их назначения.
\usection{СЕРВИТАЛЫ ХАВОНЫ}
\vs p025 1:1 Хотя и названные сервиталами, эти <<промежуточные создания>> центральной вселенной не являются слугами в каком\hyp{}либо унизительном смысле этого слова. В духовном мире не существует чёрной работы; любое служение~--- священно и радостно; и высшие категории существ не смотрят свысока на низшие формы существования.
\vs p025 1:2 \pc Сервиталы Хавоны являются результатом совместной творческой работы Семи Главных Духов и их помощников~--- Семи Верховных Управляющих Мощью. Это созидательное сотрудничество ближе всего к тому, чтобы стать образцом для длинного ряда воспроизведений двойственного типа в эволюционных вселенных, начиная от создания Яркой Утренней Звезды посредством связи Сына Создателя и Созидательного Духа до полового размножения на мирах, подобных Урантии.
\vs p025 1:3 Число сервиталов непомерно, и постоянно создаются ещё. Они появляются группами по 1\,000 на третий момент\fnst{Мне не ясно, что это за <<третий момент>>, ибо на Рае, даже на его дальнем северном секторе, время не имеет смысла.}, следующий за собранием Главных Духов и Верховных Управляющих Мощью в их общей области~--- дальнем северном секторе Рая. Каждый четвёртый сервитал более физического типа, чем остальные; то есть из каждой 1\,000~--- 750, по\hyp{}видимому, относятся к духовному типу, а 250 являются полуфизическими по своей природе. Эти \bibemph{четвёртые создания} в некотором роде относятся к категории материальных существ (материальных в смысле Хавоны) и больше напоминают управляющих физической мощью, чем Главных Духов.
\vs p025 1:4 \pc В личностных взаимоотношениях духовное преобладает над материальным, хотя сейчас на Урантии кажется, что это не так; и в создании Сервиталов Хавоны преобладает закон доминирования духа; установленное соотношение даёт трёх духовных существ к одному полуфизическому.
\vs p025 1:5 \pc Вновь созданные сервиталы вместе с вновь появляющимися Проводниками Выпускников~--- все проходят через курсы обучения, которые постоянно ведут старшие проводники на каждом из семи контуров Хавоны. Затем сервиталы назначаются на те виды деятельности, для которых они лучше всего приспособлены, и, поскольку они бывают двух типов~--- духовные и полуфизические, не существует почти никаких ограничений для сферы работы, которую могут выполнять эти разносторонние существа. Высшие, или духовные, группы выборочно назначаются на службы Отца, Сына или Духа и на работу Семи Главных Духов. Время от времени в больших количествах они направляются служить в образовательные миры, окружающие столичные сферы семи сверхвселенных,~--- миры, посвящённые завершающей подготовке и духовной культуре восходящих душ времени, готовящихся к продвижению к контурам Хавоны. Как духи\hyp{}сервиталы, так и их более физические собратья назначаются помощниками и партнёрами Проводников Выпускников для помощи и наставления различных категорий восходящих созданий, которые достигли Хавоны и которые стремятся достичь Рая.
\vs p025 1:6 Сервиталы Хавоны и Проводники Выпускников проявляют трансцендентную преданность своей работе и трогательную привязанность друг к другу, привязанность, которую, хотя она и духовная, ты мог бы понять лишь в сравнении с феноменом человеческой любви. Есть божественный пафос в разлуке сервиталов с проводниками, как это часто случается, когда сервиталы направляются для исполнения миссий за пределами центральной вселенной; но они уходят с радостью, а не с печалью. Приносящая удовлетворение радость от исполнения высокого долга является эмоцией духовных существ, затмевающей все остальные. Печали нет места при осознании добросовестно исполненного божественного долга. И когда восходящая человеческая душа предстаёт перед Верховным Судьёй, обладающее вечной силой решение не будет определяться материальными успехами или количественными достижениями; приговор, разносящийся по высшим судам, гласит: <<Хорошая работа, добрый и \bibemph{верный} слуга; ты был верен в самом главном; ты будешь поставлен правителем над вселенскими реальностями>>.
\vs p025 1:7 На сверхвселенскую службу Сервиталы Хавоны всегда назначаются в область, возглавляемую Главным Духом, на которого они наиболее похожи в общих и особых духовных прерогативах. Они служат только на образовательных мирах, окружающих столицы семи сверхвселенных, и последний отчёт Уверсы показывает, что почти 138 миллиардов сервиталов служили на её 490 спутниках. Они заняты в бесконечном разнообразии деятельности, связанной с работой этих образовательных миров, составляющих сверхуниверситеты сверхвселенной Орвонтон. Здесь они~--- твои спутники; они спускаются с высот, которые станут твоим следующим этапом пути, чтобы изучать тебя и вдохновлять реальностью и уверенностью в твоём грядущем переходе из вселенных времени в сферы вечности. И в этих контактах сервиталы получают тот предварительный опыт служения восходящим созданиям времени, который так полезен в их последующей работе на контурах Хавоны в качестве помощников Проводников Выпускников или~--- как преображённые сервиталы~--- в качестве самих Проводников Выпускников.
\usection{ВСЕОБЩИЕ МИРОТВОРЦЫ}
\vs p025 2:1 На каждого созданного Сервитала Хавоны создаётся семь Всеобщих Миротворцев, по одному в каждой сверхвселенной. Этот творческий акт включает в себя определённый сверхвселенский метод отражательной реакции на процессы, происходящие на Рае.
\vs p025 2:2 На столичных мирах семи сверхвселенных функционируют семь отражений Семи Главных Духов. Трудно браться за описание материальному разуму природы этих Отражательных Духов. Они~--- истинные личности; тем не менее каждый член сверхвселенской группы точно отражает только одного из Семи Главных Духов. И каждый раз, когда Главные Духи объединяются с управляющими мощью с целью создания группы Сервиталов Хавоны, происходит одновременная фокализация на одном из Отражательных Духов в каждой из сверхвселенских групп, и незамедлительно появляется равное число полноценных Всеобщих Миротворцев на столичных мирах сверхтворений. Если при создании сервиталов Главный Дух Номер Семь проявит инициативу, никто, кроме Отражательных Духов седьмой категории, не станет беременным миротворцами; и одновременно с созданием 1\,000 сервиталов орвонтонского типа в каждой столице сверхвселенной появляется 1\,000 миротворцев седьмой категории. В результате этих актов, отражающих семичастную природу Главных Духов, происходят семь созданных категорий миротворцев, служащих в каждой сверхвселенной.
\vs p025 2:3 Миротворцы предрайского статуса не могут служить в любой сверхвселенной, будучи ограниченными своими родными сегментами творения. Поэтому любой сверхвселенский корпус, включающий одну седьмую часть каждой созданной категории, проводит очень долгое время под влиянием одного из Главных Духов, исключая остальных, ибо, хотя все семь \bibemph{отражены} на столицах сверхвселенных, только один из них является \bibemph{доминирующим} в каждом сверхтворении.
\vs p025 2:4 Каждое из семи сверхтворений действительно пронизано одним из Главных Духов, который руководит его судьбой. Таким образом, каждая сверхвселенная становится подобной гигантскому зеркалу, отражающему природу и характер возглавляющего её Главного Духа, всё это продолжается далее в каждой дочерней локальной вселенной через присутствие и функции Созидательных Материнских Духов. Влияние такой среды на эволюционный рост столь глубоко, что на своих постсверхвселенских путях миротворцы коллективно проявляют 49 эмпирических точек зрения, или проницательностей, каждая под одним углом, поэтому неполная, но все взаимно дополняют друг друга и вместе стремятся охватить круг Верховности.
\vs p025 2:5 \pc В каждой сверхвселенной Всеобщие Миротворцы оказываются удивительным и естественным образом разделёнными на группы по четыре~--- союзы, в которых они продолжают служить. В каждой группе трое являются духовными личностями, а один, как и четвёртые создания среди сервиталов, является полуматериальным существом. Этот квартет представляет собой миротворческую комиссию и укомплектован следующим образом:
\vs p025 2:6 \li{1.}\bibemph{Судья\hyp{}Арбитр}. Один из них, единогласно назначенный тремя другими как самый компетентный и квалифицированный, чтобы действовать в качестве главы судейской группы.
\vs p025 2:7 \li{2.}\bibemph{Дух\hyp{}Адвокат}. Назначается судьёй\hyp{}арбитром для представления доказательств и защиты прав всех личностей, вовлечённых в любое дело, переданное на рассмотрение миротворческой комиссии.
\vs p025 2:8 \li{3.}\bibemph{Божественный Палач}. Миротворец, обладающий присущей его природе способностью устанавливать контакт с материальными существами миров и исполнять решения комиссии. Божественные палачи, будучи четвёртыми созданиями~--- квазиматериальными существами,~--- почти, но не полностью, доступны ограниченному зрению смертных рас\fnst{Возможно, что именно о подобных \bibemph{палачах} объявил царю Навуходоносору пророк Даниил (см.\,Даниил~4:1--34), назвав их <<смотрителями>> (англ. \bibemph{watchers}, арам. \textheb{עִירִין}). Любопытно, что значение арамейского слова \textheb{עִיר} было незнакомо переводчикам Септуагинты и оставлено в греческой транслитерации \textgreek{ιρ}.}.
\vs p025 2:9 \li{4.}\bibemph{Писец}. Оставшийся член комиссии автоматически становится писцом, секретарём трибунала. Он следит за тем, чтобы все записи были должным образом подготовлены для архивов сверхвселенной и для записей локальной вселенной. Если комиссия работает на эволюционном мире, то третий отчёт с помощью палача подготавливается для физических записей правительства системы, в юрисдикции которой находится данный мир.
\vs p025 2:10 \pc На заседании комиссия функционирует как группа из трёх, поскольку адвокат отстраняется на время рассмотрения дела и участвует в формулировании приговора только по завершении слушания. Поэтому эти комиссии иногда называют судейским трио.
\vs p025 2:11 \pc Миротворцы представляют огромную ценность для поддержания гладкого функционирования вселенной вселенных. Пересекая пространство с серафической, тройной, скоростью, они служат как передвижные суды миров~--- комиссии,~--- предназначенные для быстрого разрешения незначительных трудностей. Если бы не эти мобильные и в высшей степени справедливые комиссии, трибуналы сфер были бы безнадёжно перегружены делами о небольших разногласиях в мирах.
\vs p025 2:12 Эти судейские трио не рассматривают дела, имеющие отношение к вечности; душа, вечная перспектива создания времени, никогда не подвергается опасности их действиями. Миротворцы не занимаются вопросами, выходящими за пределы вр\'еменного существования и космического благополучия созданий времени. Но если комиссия однажды приняла проблему в судебное производство, то её решения окончательны и всегда единодушны; решение судьи\hyp{}арбитра не подлежит апелляции.
\usection{ОБШИРНОЕ СЛУЖЕНИЕ МИРОТВОРЦЕВ}
\vs p025 3:1 Миротворцы поддерживают центр группы на столице своей сверхвселенной, где содержится их главный резервный корпус. Их вторичные резервы размещены на столицах локальных вселенных. Более молодые и менее опытные члены комиссий начинают своё служение на низших мирах,~--- мирах, подобных Урантии, и продвигаются к вынесению решений по более крупным проблемам после того, как приобретут более зрелый опыт.
\vs p025 3:2 Категория миротворцев является полностью надёжной; ни один никогда не сбивался с пути. Хотя они и не являются непогрешимыми в мудрости и суждениях, они неоспоримо надёжны и безупречны в верности. Они берут своё начало на столицах сверхвселенных и в конечном итоге возвращаются туда, продвигаясь через следующие уровни вселенского служения:
\vs p025 3:3 \li{1.}\bibemph{Миротворцы миров}. Каждый раз, когда руководящие личности отдельных миров сильно озадачены или действительно оказываются в тупике относительно надлежащей процедуры при данных обстоятельствах, и если дело не является достаточно важным для передачи в официально учреждённые трибуналы данной сферы, тогда, при получении петиций от двух личностей, по одной от каждой спорящей стороны, немедленно начинает действовать миротворческая комиссия.
\vs p025 3:4 Когда эти административные и правовые затруднения передаются в руки миротворцев для изучения и вынесения решения, то их полномочия становятся верховными. Но они не выносят решения до тех пор, пока не будут заслушаны все доказательства, и не существует абсолютно никакого ограничения в их полномочиях для того, чтобы вызывать свидетелей отовсюду. И хотя их решения не могут быть обжалованы, иногда дело принимает такой оборот, что комиссия прекращает вести свои протоколы на определённом месте, высказывает своё мнение и передаёт всю проблему высшим трибуналам сферы.
\vs p025 3:5 Решения членов комиссии заносятся в планетарные записи и, при необходимости, приводятся в исполнение божественным палачом. Власть его весьма велика, и сфера его деятельности очень широка. Божественные палачи искусно манипулируют тем, что есть, в интересах того, что должно быть. Иногда очевидно, что их работа выполняется для благополучия сферы, а иногда их действия на мирах времени и пространства трудно поддаются объяснению. Хотя исполняемые ими приговоры не противоречат ни естественному закону, ни установленным обычаям сферы, они часто совершают свои странные действия и обеспечивают исполнение мандатов миротворцев в соответствии с высшими законами управления системой.
\vs p025 3:6 \li{2.}\bibemph{Миротворцы столиц систем}. От службы на эволюционных мирах эти комиссии четырёх продвигаются к службе на столице системы. Здесь у них много работы, и они показывают себя чуткими друзьями людей, ангелов и других существ\hyp{}духов. Судейское трио занято не столько разногласиями между личностями, сколько спорами между группами и недоразумениями, возникающими между различными категориями созданий; ведь на столице системы живут как духовные, так и материальные существа, а также существа комбинированного типа, такие как Материальные Сыны.
\vs p025 3:7 В тот момент, когда Создатели дарят существование развивающимся индивидуумам, обладающим способностью выбора, происходит отклонение от гладкого функционирования божественного совершенства; неизбежно возникают недоразумения, и необходимо предусмотреть возможность справедливой корректировки этих различных, но искренних точек зрения. Мы все должны помнить, что премудрые и всемогущие Создатели могли бы сделать локальные вселенные такими же совершенными, как Хавона. В центральной вселенной нет необходимости в миротворческих комиссиях. Но Создатели в своей премудрости решили не делать этого. И хотя они произвели вселенные, которые изобилуют различиями и наполнены трудностями, они также обеспечили механизм и средства для согласования всех этих различий и гармонизации этого кажущегося беспорядка.
\vs p025 3:8 \li{3.}\bibemph{Миротворцы созвездий}. От службы в системах миротворцы продвигаются к решению проблем созвездия, принимая на рассмотрение небольшие трудности, возникающие между сотней его систем обитаемых миров. Не так много проблем, возникающих на столицах созвездий, попадают под их юрисдикцию, но они постоянно заняты, посещая одну систему за другой, собирая доказательства и подготавливая предварительные заключения. Если спор является честным, если недоразумения возникают из-за искренних расхождений во мнениях и честного различия точек зрения, то не имеет значения, сколь мало лиц может быть вовлечено, не важно, насколько очевидно тривиальное недоразумение, миротворческая комиссия всегда готова вынести решение по существу спора.
\vs p025 3:9 \li{4.}\bibemph{Миротворцы локальных вселенных}. В этой ещё более обширной работе во вселенной члены комиссии оказывают огромную помощь как Мелхиседекам и Сынам Повелителям, так и правителям созвездий и воинствам личностей, занимающихся координацией и управлением ста созвездий. Различные категории серафимов и другие обитатели столичных сфер локальных вселенных также пользуются помощью и решениями судейских трио.
\vs p025 3:10 Почти невозможно объяснить природу тех разногласий, которые могут возникать в деталях происходящего в системе, созвездии или вселенной. Трудности действительно возникают, но они совершенно не похожи на мелкие неприятности и невзгоды материального существования, переживаемого на эволюционных мирах.
\vs p025 3:11 \li{5.}\bibemph{Миротворцы малых секторов сверхвселенной}. От проблем локальных вселенных члены комиссии переходят к изучению вопросов, возникающих в малых секторах своей сверхвселенной. Чем дальше они восходят внутрь от отдельных планет, тем меньше остаётся материальных обязанностей у божественного палача; постепенно он берёт на себя новую роль интерпретатора милосердия\hyp{}правосудия, в то же время~--- будучи квазиматериальным~--- держа комиссию в целом в полном сочувствия контакте с материальными аспектами её расследований.
\vs p025 3:12 \li{6.}\bibemph{Миротворцы больших секторов сверхвселенной}. Характер работы членов комиссии продолжает меняться по мере их продвижения. Становится всё меньше и меньше недоразумений, требующих вынесения судебных решений, и всё больше и больше загадочных явлений, которые необходимо объяснить и истолковать. От стадии к стадии они эволюционируют от арбитров по разногласиям до \bibemph{объясняющих тайны}~--- судей, становящихся учителями\hyp{}интерпретаторами. Когда\hyp{}то они были арбитрами тех, кто из-за своего невежества допускал возникновение трудностей и недоразумений; но теперь они становятся наставниками тех, кто достаточно разумен и терпим, чтобы избегать конфликтов разума и войн мнений. Чем выше образование создания, тем больше у него уважения к знанию, опыту и мнениям других.
\vs p025 3:13 \li{7.}\bibemph{Миротворцы сверхвселенной}. Здесь миротворцы становятся равными~--- четыре взаимно понимающих и в совершенстве функционирующих арбитра\hyp{}учителя. Божественный палач лишается карательной власти и становится физическим голосом духовного трио. К этому времени эти советники и учителя хорошо знакомы с большинством действительных проблем и трудностей, с которыми приходится сталкиваться при ведении дел сверхвселенной. Таким образом они становятся замечательными советниками и мудрыми учителями восходящих пилигримов, проживающих на образовательных сферах, которые окружают столичные миры сверхвселенных.
\vs p025 3:14 \pc Все миротворцы служат под общим надзором От Века Древних и под непосредственным руководством Помощников Изображения вплоть до своего перевода в Рай. Во время своего Райского пребывания они подотчётны Главному Духу, возглавляющему сверхвселенную их происхождения.
\vs p025 3:15 В реестрах сверхвселенных не числятся те миротворцы, которые вышли за пределы их юрисдикции, а такие комиссии разбросаны по всей большой вселенной. Согласно последнему отчёту Уверсы о регистрации, число действующих комиссий в Орвонтоне составляет почти 18 триллионов~--- более 70 триллионов индивидуумов. Но это лишь очень небольшая часть от всего множества миротворцев, созданных в Орвонтоне; их численность в целом является более высокой величиной и эквивалентна общему числу Сервиталов Хавоны с учётом преображённых в Проводников Выпускников.
\vs p025 3:16 Время от времени, по мере увеличения числа миротворцев сверхвселенной, они преображаются [translated] в совет совершенства на Рае, из которого впоследствии выходят как координационный корпус, развитый Бесконечным Духом для вселенной вселенных; это~--- изумительная группа существ, число и эффективность которой постоянно возрастает. Благодаря эмпирическому восхождению и Райскому обучению они приобретают уникальное понимание возникающей реальности Верховного Существа и странствуют по вселенной вселенных, исполняя специальные задания.
\vs p025 3:17 Члены миротворческой комиссии никогда не разлучаются. Группа четырёх всегда служит вместе в том составе, в каком изначально была объединена. Даже в своём прославленном служении они продолжают функционировать как квартеты накопленного космического опыта и усовершенствованной эмпирической мудрости. Их вечный союз является воплощением верховного правосудия времени и пространства.
\usection{ТЕХНИЧЕСКИЕ КОНСУЛЬТАНТЫ}
\vs p025 4:1 Эти правовые и технические разумы духовного мира не были созданы как таковые. Из числа первых супернафимов и омниафимов один миллион самых дисциплинированных умов были выбраны Бесконечным Духом в качестве ядра этой обширной и разносторонней группы. И всегда с того далёкого времени реальный опыт применения законов совершенства к планам эволюционного творения необходим для всех, кто стремится стать Техническим Консультантом.
\vs p025 4:2 \pc Технические Консультанты набираются из рядов следующих категорий личностей:
\vs p025 4:3 \li{1.}Супернафимы.
\vs p025 4:4 \li{2.}Секонафимы.
\vs p025 4:5 \li{3.}Терциафимы.
\vs p025 4:6 \li{4.}Омниафимы.
\vs p025 4:7 \li{5.}Серафимы.
\vs p025 4:8 \li{6.}Определённые типы восходящих смертных.
\vs p025 4:9 \li{7.}Определённые типы восходящих промежуточных созданий.
\vs p025 4:10 \pc В настоящее время, не считая смертных и промежуточных созданий вр\'еменного назначения, число технических консультантов, зарегистрированных на Уверсе и действующих в Орвонтоне, немногим превышает 61 триллион.
\vs p025 4:11 Технические Консультанты часто действуют индивидуально, но организованы в группы по семь для несения службы; они поддерживают общие центры на сферах назначения. В каждой группе не менее пяти должны иметь постоянный статус, а двое могут быть временно связаны с этой группой. Восходящие смертные и восходящие промежуточные создания служат в этих консультативных комиссиях во время восхождения к Раю, но они не поступают на регулярные курсы подготовки Технических Консультантов и никогда не становятся постоянными членами этой категории.
\vs p025 4:12 Те смертные и промежуточные создания, которые временно служат вместе с консультантами, избираются для такой работы благодаря их глубокому знанию концепции всеобщего закона и верховного правосудия. По мере того как ты путешествуешь к своей Райской цели, постоянно приобретая дополнительные знания и возрастающее мастерство, тебе всё время предоставляется возможность делиться с другими мудростью и опытом, которые ты уже накопил; на всём пути к Хавоне ты будешь играть роль ученика\hyp{}учителя. Ты будешь продвигаться по восходящим уровням этого обширного эмпирического университета, передавая тем, кто находится чуть ниже тебя, новые обретённые знания на своём пути восхождения. Во всеобщем режиме ты не будешь считаться обладателем знания и истины до тех пор, пока не продемонстрируешь свою способность и готовность передавать это знание и истину другим.
\vs p025 4:13 После долгого обучения и приобретения реального опыта, любым из духов\hyp{}помощников выше статуса херувимов разрешается получить постоянное назначение в качестве Технических Консультантов. Все кандидаты добровольно вступают в эту категорию служения; но однажды приняв на себя такие обязанности, они не могут отказаться от них. Только от Века Древние вправе перевести этих консультантов на другую деятельность.
\vs p025 4:14 \pc Подготовка Технических Консультантов, начатая в колледжах Мелхиседеков локальных вселенных, продолжается в судах От Века Древних. После этого обучения в сверхвселенной они отправляются в <<школы семи кругов>>, расположенные на контрольных мирах контуров Хавоны. И с контрольных миров они принимаются в <<колледж этики закона и метода Верховности>>~--- Райскую образовательную школу для совершенствования Технических Консультантов.
\vs p025 4:15 Эти консультанты~--- больше чем эксперты в области права; они~--- студенты и учителя \bibemph{прикладного} права, законов вселенной, применяемых к жизням и судьбам всех, кто населяет огромные области обширного творения. Со временем они становятся живыми библиотеками законов времени и пространства, предотвращающими бесконечные проблемы и ненужные задержки, наставляя личности времени относительно форм и способов действий, наиболее приемлемых для правителей вечности. Они могут давать работникам пространства советы, позволяющие им функционировать в согласии с требованиями Рая; для всех существ они являются учителями метода Создателей.
\vs p025 4:16 Такую живую библиотеку прикладного права невозможно было создать; такие существа должны были развиться посредством реального опыта. Бесконечные Божества экзистенциальны, поэтому у них компенсирован недостаток опыта; они знают всё ещё до того, как всё испытают, но они не передают это неэмпирическое знание своим подчинённым созданиям.
\vs p025 4:17 \pc Технические Консультанты посвящены работе по предотвращению задержек, содействию прогрессу и консультированию относительно достижений. Всегда существует \bibemph{лучший} и \bibemph{правильный} способ сделать что\hyp{}либо; всегда существует метод совершенства~--- божественный метод,~--- и эти консультанты знают, как направить нас всех к поиску этого лучшего пути.
\vs p025 4:18 Эти чрезвычайно мудрые и практичные существа всегда тесно связаны со службой и работой Всеобщих Цензоров. Мелхиседеки обеспечены умелым корпусом. Правители систем, созвездий, вселенных и секторов сверхвселенных~--- все в изобилии снабжены этими техническими, или правовыми, справочными умами духовного мира. Особая группа действует в качестве правовых советников для Носителей Жизни, консультируя этих Сынов о степени допустимого отклонения от установленного порядка распространения жизни и иным образом наставляя относительно их прерогатив и широты функций. Они являются консультантами всех классов существ относительно надлежащих правил и способов всех процессов мира духа. Но они не имеют дел напрямую и лично с материальными созданиями миров.
\vs p025 4:19 Помимо консультирования касательно установленных обычаев, Технические Консультанты в равной мере посвящены эффективному толкованию всех законов, касающихся сотворённых существ~--- физических, ментальных [mindal] и духовных. Они доступны Всеобщим Миротворцам и всем другим, кто желает познать истину закона; иными словами, познать, на какую реакцию Верховности Божества можно рассчитывать в данной ситуации при имеющихся факторах установленного физического, ментального и духовного порядка. Они пытаются даже прояснить метод Предельного.
\vs p025 4:20 Технические Консультанты являются избранными и испытанными существами; я никогда не знал ни одного из них, кто бы сбился с пути. У нас на Уверсе нет записей о том, что они когда\hyp{}либо были осуждены за неуважение к божественным законам, которые они так эффективно интерпретируют и столь красноречиво разъясняют. Нет известного предела области их служения, равно как ничем не ограничено их развитие. Они остаются консультантами вплоть до врат Рая; вся вселенная закона и опыта открыта для них.
\usection{ХРАНИТЕЛИ ЗАПИСЕЙ НА РАЕ}
\vs p025 5:1 Из числа третичных супернафимов Хавоны некоторые из старших главных писцов избираются Хранителями Записей~--- хранителями официальных архивов Острова Света; эти архивы полностью отличаются от живых записей регистрации в разумах хранителей знания, иногда называемых <<живой библиотекой Рая>>.
\vs p025 5:2 Регистрирующие ангелы обитаемых планет являются источником всех индивидуальных записей. Во вселенных функционируют и другие писцы, делающие как официальные, так и живые записи. От Урантии до Рая встречаются оба вида записей; в локальной вселенной больше письменных записей и меньше живых; на Рае больше живых и меньше официальных; на Уверсе оба вида встречаются одинаково.
\vs p025 5:3 Каждый значительный эпизод в организованном и обитаемом творении подлежит регистрации. В то время как события исключительно местного значения регистрируются только в локальных записях, события более широкого значения рассматриваются соответствующим образом. Всё, что имеет вселенское значение, начиная с планет, систем и созвездий Небадона, отправляется на Спасоград; и из таких столиц вселенных эти эпизоды передаются в записи более высоких уровней, касающиеся правительств секторов и сверхвселенных. На Рае также имеется соответствующая сводка данных по сверхвселенным и Хавоне; и это историческое и совокупное повествование о вселенной вселенных находится в в\'едении этих возвышенных третичных супернафимов.
\vs p025 5:4 Хотя некоторые их этих существ были направлены в сверхвселенные служить в качестве Главных Писцов для руководства деятельностью Небесных Писцов, ни один никогда не был удалён из постоянного списка своей категории.
\usection{НЕБЕСНЫЕ ПИСЦЫ}
\vs p025 6:1 Это те писцы, которые делают все записи дважды, создавая первоначальную духовную запись и полуматериальный дубликат,~--- то, что можно было бы назвать копией <<под копирку>> [carbon copy]. Они могут делать это благодаря своей особой способности одновременно манипулировать как духовной, так и материальной энергией. Небесные Писцы не создаются как таковые; они являются восходящими серафимами из локальных вселенных. Их принимают, классифицируют и назначают в свои сферы работы советы Главных Писцов на столицах семи сверхвселенных. Там же располагаются школы для подготовки Небесных Писцов. Школой на Уверсе руководят Совершенствователи Мудрости и Божественные Советники.
\vs p025 6:2 По мере продвижения во вселенском служении писцы сохраняют систему двойной регистрации, тем самым делая свои записи всегда доступными всем классам существ~--- от представителей материальной категории до высоких духов света. В своём опыте перехода, восходя из этого материального мира, ты всегда сможешь обращаться к записям или иным образом знакомиться с историей и традициями сферы своего статуса.
\vs p025 6:3 Писцы~--- это испытанный и надёжный корпус. Я никогда не слышал о нарушении долга Небесным Писцом, и ни разу в их записях не было обнаружено фальсификаций. Они подвергаются двойному контролю: записи тщательно изучаются их возвышенными собратьями из Уверсы и Могущественными Посланниками, которые подтверждают соответствие квазифизических копий первоначальным записям духа.
\vs p025 6:4 В то время как продвигающиеся писцы, несущие службу на подчинённых регистрационных сферах, во вселенных Орвонтона исчисляются триллионами и триллионами, число тех, кто достиг статуса на Уверсе, меньше 8\,000\,000. Эти старшие, или выпускники\hyp{}писцы, являются сверхвселенскими хранителями и экспедиторами предоставленных записей времени и пространства. Их постоянные центры расположены в кольцевых обителях, окружающих область записей на Уверсе. Они никогда не передают эти записи на хранение другим; они могут отсутствовать лично, но число отсутствующих никогда не бывает значительным.
\vs p025 6:5 Как и для тех супернафимов, которые стали Хранителями Записей, назначение в корпус Небесных Писцов является постоянным. Серафимы и супернафимы, поступающие на эти службы, остаются соответственно Небесными Писцами и Хранителями Записей до того дня, когда появится новое и модифицированное управление полной персонализации Бога Верховного.
\vs p025 6:6 На Уверсе эти старшие Небесные Писцы могут показать записи всего, что имеет космическое значение во всём Орвонтоне, начиная с далёких времён прибытия От Века Древних, в то время как на вечном Острове Хранители Записей охраняют архивы этой сферы, свидетельствующие о процессах на Рае со времён персонификации Бесконечного Духа.
\usection{МОРОНТИЙНЫЕ СПУТНИКИ}
\vs p025 7:1 Эти дети Материнских Духов локальных вселенных~--- друзья и партнёры всех, кто живёт восходящей моронтийной жизнью. Они не обязательны для реальной работы в продвижении восходящих созданий; и также никоим образом не заменяют работы серафических хранителей, которые часто сопровождают своих смертных товарищей в путешествии к Раю. Моронтийные Спутники~--- это просто милосердные любезные хозяева\fnst{В смысле <<гость и хозяин [дома]>>.} для тех, кто только начинает долгое восхождение внутрь. Они также являются искусными организаторами игр, и им умело помогают в этой работе управляющие реверсией.
\vs p025 7:2 Хотя тебе предстоит выполнять серьёзные и всё более сложные задания на моронтийных учебных мирах Небадона, тебе всегда будут предоставляться регулярные периоды отдыха и реверсии. На протяжении всего путешествия к Раю всегда будет время для отдыха и игр духа; и на пути к свету и жизни всегда есть время для поклонения и новых достижений.
\vs p025 7:3 Эти Моронтийные Спутники столь дружелюбные товарищи, что, когда ты наконец покинешь последнюю фазу моронтийного опыта, готовясь приступить к сверхвселенскому приключению духа, ты искренне будешь жалеть, что эти общительные создания не могут сопровождать тебя, ибо они служат исключительно в локальных вселенных. На любом этапе восходящего пути все личности, с которыми можно вступить в контакт, будут дружелюбны и общительны, но пока ты не встретишь Райских Спутников, ты не найдёшь иной группы, столь посвящённой дружбе и общению.
\vs p025 7:4 Деятельность Моронтийных Спутников более полно описана в повествованиях, посвящённых вашей локальной вселенной.
\usection{РАЙСКИЕ СПУТНИКИ}
\vs p025 8:1 Райские Спутники являются смешанной, или сборной, группой, набранной из рядов серафимов, секонафимов, супернафимов и омниафимов. Хотя они служат по вашим меркам необычайно продолжительное время, они не имеют постоянного статуса. Когда это служение завершается, как правило (но не всегда), они возвращаются к тем обязанностям, которые выполняли, когда были призваны на Райскую службу.
\vs p025 8:2 Члены ангельских воинств назначаются на эту службу Материнскими Духами локальных вселенных, Отражательными Духами сверхвселенных и Мажестоном Рая. Они призываются на центральный Остров и назначаются Райскими Спутниками одним из Семи Главных Духов. За исключением постоянного статуса на Рае, эта вр\'еменная служба Райского общения является высшей честью, когда\hyp{}либо оказываемой духам\hyp{}помощникам.
\vs p025 8:3 Эти избранные ангелы посвящены служению дружеского общения и назначаются в качестве партнёров всем классам существ, которые могут остаться в одиночестве на Рае, главным образом восходящим смертным, но также и всем другим, кто оказывается один на центральном Острове. Райские Спутники не должны совершать что-то особенное для тех, с кем они дружески общаются; они~--- просто спутники. Почти все остальные существа, с которыми вы, смертные, встретитесь в течение своего пребывания на Рае, за исключением своих собратьев\hyp{}пилигримов, будут делать нечто определённое вместе с вами и для вас; но эти спутники назначены только для того, чтобы быть с вами и общаться с вами как личностные товарищи. В их служении им часто помогают милосердные и выдающиеся Граждане Рая.
\vs p025 8:4 Смертные приходят от очень общительных рас. Создатели хорошо знают, что <<не хорошо быть человеку одному>>, и поэтому создали условия для дружеского общения даже на Рае.
\vs p025 8:5 \pc Если ты, как восходящий смертный, достигнешь Рая в обществе спутника или близкого партнёра твоего земного пути, или случится, что твой серафический хранитель судьбы прибудет с тобой или будет ждать тебя, тогда тебе не будет назначен постоянный спутник. Но если ты прибудешь один, то спутник обязательно поприветствует тебя, когда ты проснёшься на Острове Света от заключительного сна времени. Даже если известно, что тебя будет сопровождать кто-то из восходящей группы, будут назначены вр\'еменные спутники, чтобы приветствовать тебя на вечных берегах и сопроводить в место, приготовленное для принятия тебя и твоих товарищей. Ты можешь быть уверен, что тебя радушно встретят, когда переживёшь опыт воскресения в вечность на вечных берегах Рая.
\vs p025 8:6 Принимающие спутники назначаются в течение завершающих дней вр\'еменного проживания восходящих на последнем контуре Хавоны, и они тщательно изучают записи о происхождении смертных и полное событий восхождение через миры пространства и контуры Хавоны. Когда они приветствуют смертных времени, они уже хорошо осведомлены о путях этих прибывающих пилигримов и тотчас же показывают себя чуткими и интересными [intriguing] товарищами.
\vs p025 8:7 Если в течение твоего вр\'еменного пребывания на Рае, до того, как станешь завершителем, по какой\hyp{}либо причине ты окажешься временно разлучён со своим товарищем по восходящему пути~--- смертным или серафическим,~--- Райский Спутник тотчас же будет назначен для совета и общения. Будучи однажды назначенным восходящему смертному, одиноко проживающему на Рае, спутник остаётся с ним до тех пор, пока к нему не присоединятся его восходящие товарищи или он не будет должным образом принят в Корпус Завершения.
\vs p025 8:8 \pc Райские Спутники получают назначение в порядке очерёдности, за исключением того, что восходящий никогда не передаётся на попечение спутника, чья природа отлична от его сверхвселенского типа. Если бы смертный с Урантии прибыл на Рай сегодня, ему был бы назначен первый ожидающий спутник либо орвонтонского происхождения, либо иным образом обладающий природой Седьмого Главного Духа. Поэтому омниафимы не служат\fnst{Омниафимы создаются Бесконечным Духом и Семью Верховными Исполнителями, и их природа не соответствует ни одному сверхвселенскому типу отдельно.} с восходящими созданиями из семи сверхвселенных.
\vs p025 8:9 \pc Райские Спутники оказывают множество дополнительных услуг. Если восходящий смертный достигает центральной вселенной один и, пересекая Хавону, потерпит неудачу в какой\hyp{}либо фазе приключения, связанного с Божеством, то в должное время он будет отправлен обратно во вселенные времени, и тотчас же будет сделан вызов в резервы Райских Спутников. Одному из этой категории будет поручено следовать за потерпевшим неудачу пилигримом, быть с ним, утешать и подбадривать его, оставаясь с ним, пока он не вернётся в центральную вселенную, чтобы возобновить Райское восхождение.
\vs p025 8:10 Если восходящий пилигрим потерпел поражение в приключении, связанном с Божеством, при пересечении Хавоны в компании восходящего серафима~--- ангела\hyp{}хранителя смертного пути,~--- то этот ангел\hyp{}хранитель предпочтёт следовать за своим смертным товарищем. Такие серафимы всегда вызываются добровольцами и получают разрешение сопровождать своих давних смертных товарищей назад на службу времени и пространства.
\vs p025 8:11 Но иначе обстоит дело с двумя тесно связанными восходящими смертными: если один достигает Бога, а другой временно терпит неудачу, то преуспевший индивидуум неизменно выбирает вернуться в эволюционные творения вместе с разочарованной личностью, но это не разрешается. Вместо этого посылается вызов в резервы Райских Спутников, и выбирается один из добровольцев для сопровождения разочарованного пилигрима. Тогда Райский Гражданин добровольно присоединяется к преуспевшему смертному, который задерживается на центральном Острове в ожидании потерпевшего поражение товарища и преподаёт тем временем в некоторых Райских школах, рассказывая полную приключений историю эволюционного восхождения.
\vsetoff
\vs p025 8:12 [При поддержке Высокоуполномоченного из Уверсы.]
\quizlink
