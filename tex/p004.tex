\upaper{4}{СВЯЗЬ БОГА СО ВСЕЛЕННОЙ}
\uminitoc{ВСЕЛЕНСКАЯ ПОЗИЦИЯ ОТЦА}
\uminitoc{БОГ И ПРИРОДА}
\uminitoc{НЕИЗМЕННЫЙ ХАРАКТЕР БОГА}
\uminitoc{ОСОЗНАНИЕ БОГА}
\uminitoc{ОШИБОЧНЫЕ ПРЕДСТАВЛЕНИЯ О БОГЕ}
\author{Божественный Советник}
\vs p004 0:1 Всеобщий Отец имеет вечный замысел, касающийся материальных, интеллектуальных и духовных явлений вселенной вселенных, который он выполняет на протяжении всего времени. Бог создал вселенные по своей собственной свободной и суверенной воле, и он создал их согласно своему премудрому и вечному замыслу. Сомнительно, что кто\hyp{}либо, кроме Райских Божеств и их высочайших партнёров, действительно очень много знает о вечном замысле Бога. Даже возвышенные граждане Рая придерживаются самых разных мнений о природе вечного замысла Божеств.
\vs p004 0:2 Легко сделать вывод, что целью создания совершенной центральной вселенной Хавоны было просто удовлетворение божественной природы. Хавона может служить образцовым творением для всех других вселенных и завершающей школой для пилигримов времени на их пути к Раю; тем не менее, такое небесное творение должно существовать прежде всего для удовольствия и удовлетворения совершенных и бесконечных Создателей.
\vs p004 0:3 Изумительный план совершенствования эволюционных смертных и, после их достижения Рая и Корпуса Завершения, дальнейшего обучения для какой\hyp{}то нераскрытой будущей работы, представляется в настоящее время одной из главных забот семи сверхвселенных и их многочисленных подразделений; но эта схема восхождения для одухотворения и обучения смертных времени и пространства ни в коем случае не является исключительным занятием вселенских разумных существ. Действительно, есть много других увлекательных занятий, которые занимают время и задействуют энергию небесных воинств.
\usection{ВСЕЛЕНСКАЯ ПОЗИЦИЯ ОТЦА}
\vs p004 1:1 Веками обитатели Урантии неправильно понимали провидение Бога. В вашем мире имеет место провидение божественного воздействия, но это не та детская, произвольная и материальная помощь, какой её представляли себе многие смертные. Провидение Бога состоит во взаимосвязанных действиях небесных существ и божественных духов, которые в соответствии с космическим законом непрестанно трудятся во славу Бога и для духовного продвижения его вселенских детей.
\vs p004 1:2 Разве ты не можешь продвинуться вперёд в твоём представлении об отношениях Бога с человеком до того уровня, где ты осознаешь, что девиз вселенной~--- \bibemph{прогресс?} В течение долгих веков человеческий род боролся, чтобы достичь своего нынешнего положения. На протяжении всех этих тысячелетий Провидение разрабатывало план прогрессивной эволюции. Эти две мысли не противоречат друг другу на практике, а только в ошибочных представлениях человека. Божественное провидение никогда не препятствует истинному человеческому прогрессу~--- ни мирскому, ни духовному. Провидение всегда соответствует неизменной и совершенной природе верховного Законодателя.
\vs p004 1:3 <<Бог верен>>, и <<все заповеди его справедливы>>. <<Его верность утверждена на самих небесах>>. <<На веки, Господи, твоё слово утверждено на небесах. Твоя верность~--- всем поколениям; ты утвердил землю, и она пребывает>>. <<Он~--- верный Создатель>>.
\vs p004 1:4 Нет ограничений на силы и личности, которые Отец может использовать для утверждения своего замысла и поддержки своих созданий. <<Вечный Бог~--- наше прибежище, и под нами~--- руки вечные>>. <<Тот, кто обитает в тайной обители Всевышнего, пребудет под тенью Всемогущего>>. <<Смотри, хранящий нас не воздремлет и не уснёт>>. <<Мы знаем, что всё работает вместе во благо любящих Бога>>, <<ибо очи Господа обращены к праведникам, а уши его открыты для их молитв>>.
\vs p004 1:5 Бог поддерживает <<всё словом силы своей>>. И когда рождаются новые миры, он <<посылает Сынов своих~--- и они создаются>>. Бог не только творит, но и <<сохраняет их всех>>. Бог постоянно поддерживает всё творение материальное и всех существ духовных. Вселенные вечно устойчивы. Среди кажущейся неустойчивости существует стабильность. Среди энергетических потрясений и физических катаклизмов звёздных миров~--- скрытый порядок и безопасность.
\vs p004 1:6 Всеобщий Отец не отстранился от управления вселенными; он не бездействующее Божество. Если Бог устранится, будучи вседержителем всего творения, то немедленно наступит всеобщий крах. Если бы не Бог, не было бы \bibemph{реальности.} В этот самый момент, так же как и в отдалённые эпохи прошлого и в вечном будущем, Бог продолжает поддерживать. Божественная досягаемость простирается по кругу вечности. Вселенная не заведена, как часы, чтобы работать какое\hyp{}то время, а затем перестать функционировать; всё постоянно обновляется. Отец непрестанно изливает энергию, свет и жизнь. Работа Бога не только духовна, но и буквальна. <<Он простирает север над пустым пространством и подвешивает землю ни на чём>>.
\vs p004 1:7 \pc Существо моего порядка способно обнаружить предельную гармонию и заметить далеко идущую и глубокую координацию в повседневных делах вселенского управления. Многое из того, что смертному разуму кажется разрозненным и случайным, кажется мне упорядоченным и конструктивным. Но во вселенных происходит очень многое, чего я не могу полностью понять. Я давно изучаю и более или менее знаком с общеизвестными силами, энергиями, разумами, моронтиями, духами и личностями локальных вселенных и сверхвселенных. У меня есть общее понимание того, как действуют эти факторы и личности, и я близко знаком с работой аккредитованных духовных разумов большой вселенной. Несмотря на мои знания вселенских феноменов, я постоянно сталкиваюсь с космическими реакциями, которые не могу полностью понять. Я непрерывно встречаюсь с кажущимся случайным слаженным взаимодействием сил, энергий, интеллектов и духов, которые не могу удовлетворительно объяснить.
\vs p004 1:8 Я полностью компетентен, чтобы проследить и проанализировать действие всех феноменов, непосредственно проистекающих из функционирования Всеобщего Отца, Вечного Сына, Бесконечного Духа и, в значительной степени, Острова Рай. Моё недоумение вызвано столкновением с тем, что кажется действием их таинственных со\hyp{}управителей, трёх Абсолютов потенциальности. Эти Абсолюты, похоже, вытесняют материю, выходят за пределы разума и превосходят дух. Я постоянно нахожусь в растерянности и часто озадачен своей неспособностью понять эти сложные операции, которые я приписываю присутствиям и действиям Безусловного Абсолюта, Божественного Абсолюта и Всеобщего Абсолюта.
\vs p004 1:9 Эти Абсолюты, должно быть, не полностью раскрытые по всей вселенной присутствия, которые в явлениях пространственной потенции и в функции других сверхпредельностей делают невозможным для физиков, философов или даже верующих предсказывать с уверенностью, к\'ак именно первоисточники силы, идеи или духа будут реагировать на требования, предъявляемые в условиях сложной реальности, включающей верховные адаптации и предельные ценности.
\vs p004 1:10 \pc Во вселенных времени и пространства существует также органическое единство, которое, кажется, лежит в основе всей ткани космических событий. Это живое присутствие эволюционирующего Верховного Существа, эта Имманентность Спроецированного Незавершённого, необъяснимо проявляется время от времени тем, что, кажется, является удивительно случайной координацией явно не связанных между собой вселенских событий. Это, должно быть, и есть функция Провидения~--- сфера Верховного Существа и Совместного Вершителя.
\vs p004 1:11 Я склонен верить, что именно этот обширный и, в целом, нераспознаваемый контроль координации и взаимосвязанности всех фаз и форм вселенской деятельности приводит к тому, что такая разнообразная и, казалось бы, безнадёжно запутанная смесь физических, ментальных, моральных и духовных феноменов так безошибочно работает во славу Бога и на благо людей и ангелов.
\vs p004 1:12 Но в более широком смысле кажущиеся <<случайности>> космоса несомненно являются частью конечной драмы время\hyp{}пространственного приключения Бесконечного в его вечном манипулировании Абсолютами.
\usection{БОГ И ПРИРОДА}
\vs p004 2:1 Природа~--- это, в ограниченном смысле, физическая привычка\fnst{Англ. habit.} Бога. Поведение или действие Бога определяется и условно модифицируется экспериментальными планами и эволюционными образцами локальной вселенной, созвездия, системы или планеты. Бог действует в соответствии с чётко определённым, неизменным, незыблемым законом по всей широко раскинувшейся главной вселенной; но он модифицирует шаблоны своего действия так, чтобы способствовать скоординированному и сбалансированному поведению каждой вселенной, созвездия, системы, планеты и личности в соответствии с локальными объектами, целями и планами конечных проектов эволюционного развития.
\vs p004 2:2 Следовательно, природа, как её понимает смертный человек, представляет лежащий в основании фундамент и основной фон для неизменного Божества и его незыблемых законов~--- модифицируемых, колеблющихся и переживающих потрясения в результате действия локальных планов, замыслов, образцов и условий, которые были введены в действие и выполняются локальной вселенной, созвездием, системой и планетарными силами и личностями. Например: законы Бога, установленные в Небадоне, были модифицированы планами, установленными Сыном Создателем и Созидательным Духом этой локальной вселенной; и вдобавок ко всему этому на действие этих законов оказали дальнейшее влияние ошибки, нарушения и восстания определённых существ, проживающих на вашей планете и принадлежащих к вашей непосредственной планетарной системе Сатания\fnst{Слово <<Сат\'ания>>, возможно, происходит от древнееврейского \textheb{שָׂטָן} (сатан), означающего <<противник>>, что приводит к смыслу <<враждебная система>>. Но возможно и обратное: имя <<врага рода человеческого>> может само происходить от названия локальной системы и означать нечто совершенно иное и нам пока неизвестное.}.
\vs p004 2:3 \pc Природа является время\hyp{}пространственным результатом двух космических факторов: во\hyp{}первых, незыблемости, совершенства и прямоты Райского Божества, а во\hyp{}вторых, экспериментальных планов, грубых ошибок руководства, заблуждений мятежников, незавершённости развития и несовершенства мудрости вне\hyp{}Райских созданий, от высших до низших. Поэтому природа несёт в себе единообразную, неизменную, величественную и удивительную нить совершенства из круга вечности; но в каждой вселенной, на каждой планете и в каждой индивидуальной жизни эта природа видоизменяется, уточняется и, возможно, искажается действиями, ошибками и неверностью созданий эволюционных систем и вселенных; и поэтому природе всегда присуще изменчивое настроение, к тому же причудливое и при этом стабильное внутри, и изменяется оно в соответствии с действующими процедурами локальной вселенной.
\vs p004 2:4 Природа~--- это совершенство Рая, поделённое на неполноту, зло и грех незавершённых вселенных. Таким образом, это частное выражает как совершенное, так и частичное, как вечное, так и временное. Продолжающаяся эволюция изменяет природу, увеличивая долю Райского совершенства и уменьшая долю зла, заблуждения и дисгармонии относительной реальности.
\vs p004 2:5 \pc Бог не присутствует лично в природе или в каких\hyp{}либо силах природы, поскольку феномен природы~--- это наложение несовершенств прогрессивной эволюции, а иногда и последствий мятежного восстания, на Райские основания универсального закона Бога. В том виде, в каком она проявляется в таком мире, как Урантия, природа никогда не может быть адекватным выражением, истинным представлением, верным изображением премудрого и бесконечного Бога.
\vs p004 2:6 Природа вашего мира~--- это ограничение законов совершенства эволюционными планами локальной вселенной. Какая пародия~--- поклоняться природе, потому что она в узком, ограниченном смысле пронизана Богом; потому что это фаза универсальной и, следовательно, божественной силы! Природа является также проявлением незавершённого, неполного, несовершенного результата развития, роста и прогресса вселенского эксперимента в космической эволюции.
\vs p004 2:7 Видимые недостатки мира природы не указывают на какие\hyp{}то соответствующие недостатки в характере Бога. Скорее, такие наблюдаемые несовершенства являются просто неизбежными остановками в демонстрации вечно движущейся киноленты\hyp{}экранизации бесконечности. Именно эти дефекты\hyp{}прерывания совершенства\hyp{}непрерывности позволяют конечному разуму материального человека уловить мимолётный проблеск божественной реальности во времени и пространстве. Материальные проявления божественности кажутся дефектными эволюционному разуму человека только потому, что смертный человек упорно продолжает рассматривать явления природы природными глазами, используя человеческое зрение без помощи моронтийной моты\fnst{Синтетическое слово <<мота>>, возможно, происходит от латинского \bibemph{motus} (движение), потому что сверхзнание, соответствующее истинной гармонии науки, философии и религии, является внутренне активным и связанным с движением.} или откровения,~--- её компенсационного заменителя в мирах времени.
\vs p004 2:8 И природа испорчена, её прекрасное лицо покрыто шрамами, черты её иссечены восстанием, неправильным поведением, ошибочным мышлением мириад существ, которые являются частью природы, но способствуют её обезображиванию во времени. Нет, природа~--- это не Бог. Природа~--- не объект поклонения.
\usection{НЕИЗМЕННЫЙ ХАРАКТЕР БОГА}
\vs p004 3:1 Слишком долго человек думал о Боге как о себе подобном. Бог не ревнует, не ревновал и никогда не будет ревновать человека или любое другое существо во вселенной вселенных. Зная, что Сын Создатель задумал, чтобы человек стал шедевром планетарного творения, чтобы он был правителем всей земли, вид того, как над ним господствуют его собственные низменные страсти, зрелище его преклонения перед идолами из дерева, камня, золота и эгоистичных амбиций~--- эти отвратительные сцены побуждают Бога и его Сынов ревновать \bibemph{о} человеке, а не его.
\vs p004 3:2 Вечный Бог неспособен на ярость и гнев в смысле этих человеческих эмоций, то есть как человек понимает подобные реакции. Эти эмоции низки и презренны; они едва ли заслуживают того, чтобы называться человеческими, ещё менее~--- божественными; и такие чувства абсолютно чужды совершенной природе и милосердному характеру Всеобщего Отца.
\vs p004 3:3 \pc В значительной степени трудности, с которыми смертные Урантии сталкиваются в понимании Бога, связаны с далеко идущими последствиями восстания Люцифера и предательства Калигастии\fnst{Синтетическое слово <<Калигастия>>, возможно, происходит от латинского \bibemph{caligo} (быть во тьме), придающее имени смысл <<тот, кто стоит во тьме кромешной>>. Так же, как и в ситуации с названием нашей локальной системы Сатании, данная интерпретация предполагает локальный, временный смысл, данный смертным уже после предательства бывшего Планетарного Принца 200\,000 лет назад.}. В мирах, не подвергнутых изоляции из\hyp{}за греха, эволюционные расы в состоянии формулировать гораздо лучшие идеи о Всеобщем Отце; они меньше страдают от путаницы, искажения и извращения понятий.
\vs p004 3:4 \pc Бог не раскаивается ни в чём, что он когда\hyp{}либо сделал, делает сейчас или когда\hyp{}либо сделает. Он~--- премудрый, равно как и всемогущий. Мудрость человека произрастает из проб и ошибок человеческого опыта; мудрость Бога заключается в безусловном совершенстве его бесконечной вселенской проницательности, и это божественное предвидение эффективно направляет созидательную свободную волю.
\vs p004 3:5 Всеобщий Отец никогда не делает ничего, что могло бы вызвать впоследствии печаль или сожаление, но волевые создания, спроектированные и сотворённые его личностями\hyp{}Создателями в отдалённых вселенных, своим злополучным выбором иногда вызывают эмоции божественной печали в личностях их родителей\hyp{}Создателей. Но хотя Отец не совершает ошибок, не питает сожалений и не переживает печали, он~--- существо с отцовской привязанностью, и сердце его, несомненно, огорчается, когда его детям не удаётся достичь тех духовных уровней, которых они способны достичь с помощью, столь щедро предоставленной планами духовных достижений и вселенским курсом восхождения смертных.
\vs p004 3:6 Бесконечная доброта Отца находится за пределами понимания конечного разума времени; следовательно, для эффективной демонстрации всех фаз относительной доброты необходимо постоянно предоставлять возможность контрастного сравнения с относительным злом (не грехом). Совершенство божественной доброты может быть различимо несовершенной проницательностью смертных только потому, что оно находится в контрастной ассоциации с относительным несовершенством в отношениях времени и материи в движениях пространства.
\vs p004 3:7 Характер Бога бесконечно сверхчеловеческий\fnst{Англ. superhuman, не superhumane!}; поэтому такая природа божественности должна быть персонализирована~--- например, в божественных Сынах, прежде чем она может быть постигнута верой конечным разумом человека.
\usection{ОСОЗНАНИЕ БОГА}
\vs p004 4:1 Бог~--- единственное неподвижное, автономное и неизменное существо во всей вселенной вселенных, не имеющее ничего ни вне, ни выше себя, ни прошлого, ни будущего. Бог~--- это целенаправленная энергия (созидательный дух) и абсолютная воля, а они универсальны и существуют сами по себе.
\vs p004 4:2 Поскольку Бог существует сам по себе, он абсолютно независим. Сама личность\fnst{Англ. identity.} Бога несовместима с изменением. <<Я, Господь, не меняюсь>>. Бог неизменен; но пока ты не достигнешь Райского статуса, ты не можешь даже начать понимать, как Бог может перейти от простоты к сложности, от идентичности к вариации, от покоя к движению, от бесконечности к конечности, от божественного к человеческому и от единства к двойственности и триединству. И таким образом Бог может изменять проявления своей абсолютности, потому что божественная неизменность не предполагает неподвижности; Бог обладает волей~--- он \bibemph{есть} воля.
\vs p004 4:3 Бог~--- это существо абсолютного самоопределения; у его вселенских реакций нет пределов, кроме тех, которые установлены им самим, а его добровольные действия обусловлены только теми божественными качествами и совершенными атрибутами, которые по своей сути характеризуют его вечную природу. Следовательно, Бог относится ко вселенной как существо окончательной доброты и свободной воли созидательной бесконечности.
\vs p004 4:4 Отец\hyp{}Абсолют~--- создатель центральной и совершенной вселенной и Отец всех остальных Создателей. Личность, доброту и множество других качеств Бог разделяет с человеком и другими существами, но бесконечность воли свойственна лишь ему одному. Бог ограничен в своих творческих актах только чувствами своей вечной природы и велениями своей бесконечной мудрости. Бог лично выбирает только то, что бесконечно совершенно, отсюда божественное совершенство центральной вселенной; и хотя Сыны Создатели полностью разделяют его божественность, даже части его абсолютности, они не всецело ограничены той окончательностью мудрости, которая направляет бесконечность воли Отца. Следовательно, в ранге сыновства Михаила созидательная свободная воля становится ещё более активной, полностью божественной и почти предельной, если не абсолютной. Отец бесконечен и вечен, но отрицание возможности его волевого самоограничения равносильно отрицанию самой концепции его волевой абсолютности.
\vs p004 4:5 \pc Абсолютность Бога пронизывает все семь уровней вселенской реальности. И вся эта абсолютная природа подчинена отношению Создателя к его вселенской семье созданий. Точность может характеризовать тринитарное правосудие во вселенной вселенных, но во всех своих отношениях в обширной семье созданий времени Бог вселенных руководствуется \bibemph{божественным чувством}. В общем и целом~--- вечно~--- бесконечный Бог является \bibemph{Отцом}. Из всех возможных имён, которыми он мог бы быть приемлемо назван, мне было поручено изобразить Бога всего творения как Всеобщего Отца.
\vs p004 4:6 В Боге Отце добровольные действия не управляются силой и не направляются одним только интеллектом; божественная личность определяется как заключающаяся в духе и проявляющая себя вселенным как любовь. Поэтому во всех своих личных отношениях с личностями\hyp{}созданиями вселенных Первый Источник и Центр постоянно и последовательно выступает как любящий Отец. Бог~--- это Отец в самом высоком смысле этого слова. Он вечно движим совершенным идеализмом божественной любви, и эта нежная натура находит своё сильнейшее выражение и величайшее удовлетворение в том, чтобы любить и быть любимым.
\vs p004 4:7 \pc В науке Бог~--- Первопричина; в религии~--- вселенский и любящий Отец; в философии~--- единственное существо, которое существует само по себе, не зависит ни от какого другого существа, чтобы существовать, но великодушно дарует реальность существования всему и всем другим существам. Однако требуется откровение, чтобы показать, что Первопричина науки и существующее само по себе Единство философии~--- это Бог религии, исполненный милосердия и доброты и обещавший устроить вечное выживание своих земных детей.
\vs p004 4:8 Мы страстно желаем создать концепцию Бесконечного, но поклоняемся опыту\hyp{}идее Бога, нашей способности в любом месте и в любое время постигать факторы личности и божественности нашей высшей концепции Божества.
\vs p004 4:9 Сознание победоносной человеческой жизни на земле рождается из такой веры создания, которая осмеливается бросить вызов каждому повторяющемуся эпизоду существования, сталкиваясь с ужасным зрелищем человеческих ограничений, неизменным заявлением: даже если я не могу этого сделать, во мне живёт тот, кто может и сделает это, часть Отца\hyp{}Абсолюта вселенной вселенных. А это и есть та <<победа, которая побеждает мир, твоя вера>>.
\usection{ОШИБОЧНЫЕ ПРЕДСТАВЛЕНИЯ О БОГЕ}
\vs p004 5:1 Религиозная традиция~--- это несовершенным образом сохранившиеся записи об опыте Богопознавших людей прошлых эпох, но такие записи ненадёжны в качестве руководства для религиозной жизни или как источник достоверной информации о Всеобщем Отце. Такие древние верования постоянно изменялись из\hyp{}за того, что первобытный человек был мифотворцем.
\vs p004 5:2 Один из величайших источников заблуждения на Урантии относительно природы Бога проистекает из того, что в ваших священных книгах нет чёткого различия между личностями Райской Троицы, а также между Райским Божеством с одной стороны и создателями и администраторами локальной вселенной~--- с другой. Во время прошлых диспенсаций частичного понимания вашим священникам и пророкам не удавалось чётко различать Планетарных Принцев, Властелинов Систем, Отцов Созвездий, Сынов Создателей, Правителей Сверхвселенных, Верховного Существа и Всеобщего Отца. Многие из посланий подчинённых личностей, таких как Носители Жизни и различные категории ангелов, в ваших записях представлены как исходящие от самог\'о Бога. Урантийская религиозная мысль всё ещё путает ассоциированные личности Божества с самим Всеобщим Отцом, так что все они включены под одним названием.
\vs p004 5:3 \pc Люди Урантии продолжают страдать от влияния примитивных концепций Бога. Боги, неистовствующие в бурю; сотрясающие землю в своей ярости и поражающие людей в своём гневе; осуждающие своим недовольством во времена голода и наводнений~--- это боги примитивной религии; это не те Боги, которые живут и правят во вселенных. Такие концепции являются пережитком тех времён, когда люди полагали, что вселенная находится под руководством и господством прихотей таких воображаемых богов. Но смертный человек начинает понимать, что он живёт в сфере относительной законности и порядка в том, что касается административных правил и поведения Верховных Создателей и Верховных Регуляторов.
\vs p004 5:4 \pc Варварская идея умиротворения разгневанного Бога, умилостивления оскорблённого Господа, завоевания благосклонности Божества через жертвы и покаяние и даже через пролитие крови представляет собой религию совершенно незрелую и примитивную, философию, недостойную просвещённого века науки и истины. Такие верования крайне отталкивают небесных существ и божественных правителей, которые служат и правят во вселенных. Это оскорбление для Бога~--- верить, полагать или учить, что невинная кровь должна быть пролита, чтобы завоевать его расположение или отвратить воображаемый божественный гнев.
\vs p004 5:5 Евреи верили, что <<без пролития крови не бывает прощения грехов>>\fnst{Евреям~9:22.}. Они не смогли избавиться от древней и языческой идеи, что Богов нельзя умиротворить, кроме как видом крови, хотя Моисей и сделал явный шаг вперёд, когда запретил человеческие жертвоприношения и заменил их в примитивных детских умах своих бедуинских последователей церемониальным жертвоприношением животных.
\vs p004 5:6 Дар Райского Сына вашему миру был неотъемлемой частью ситуации завершения планетарной эпохи; он был неизбежным, и он не был сделан из\hyp{}за необходимости завоевать благосклонность Бога. Этот дар оказался также последним личным актом Сына Создателя в долгом приключении по обретению эмпирического полновластия над своей вселенной. Какой пародией на бесконечный характер Бога является это учение о том, что его отцовское сердце во всей своей суровой холодности и жестокости настолько не трогали несчастья и горести его созданий, что его нежное милосердие не проявлялось до тех пор, пока он не увидел своего невинного Сына, истекающего кровью и умирающего на кресте Голгофы!
\vs p004 5:7 Но жители Урантии скоро найдут избавление от этих древних заблуждений и языческих суеверий, касающихся природы Всеобщего Отца. Появляется откровение истины о Боге, и человеческому роду суждено познать Всеобщего Отца во всей той красоте характера и привлекательности качеств, столь великолепно изображённых Сыном Создателем, который обитал на Урантии как Сын Человеческий и Сын Божий.
\vsetoff
\vs p004 5:8 [Представлено Божественным Советником Уверсы.]
\quizlink
\begin{thebibliography}{100}
\bibitem{Knudson1}
Albert C. Knudson.
{<<The Doctrine of God>>.}
{\em New York: Abingdon-Cokesbury Press}, 1930.
\bibitem{Hocking1}
William Ernest Hocking.
{<<The Meaning of God in Human Experience: A Philosophic Study of Religion>>}
{\em New Haven: Yale University Press}, 1912.
\end{thebibliography}
