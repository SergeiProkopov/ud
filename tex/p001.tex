\upaper{1}{ВСЕОБЩИЙ ОТЕЦ}
\author{Божественный Советник}
\vs p001 0:1 Всеобщий Отец --- Бог всего творения, Первый Источник и Центр всех вещей и существ. Прежде всего думайте о Боге как о создателе, затем как о регуляторе и, наконец, как о бесконечном вседержителе\fnst{О вечно поддерживающем бесконечную реальность.}. Истина о Всеобщем Отце начала пробиваться к человечеству, когда пророк сказал\fnst{Цитата сформирована из слов, приписываемых следующим лицам, перечисляемым здесь в порядке цитирования (с повторениями): Ездра (Неемия 9:6), Анна, мать Самуила (1~Царств 2:2), снова Ездра (Неемия 9:6), Давид (Псалом~32:6), снова Давид (Псалом~103:2).}: <<Един ты, Боже, и нет другого, кроме тебя. Ты сотворил небеса и небеса небес со всем их воинством; ты сохраняешь их и управляешь ими. Сынами Бога были сотворены вселенные. Создатель облачается в свет, словно в одеяние, простирает небеса, словно занавес>>. Только концепция Всеобщего Отца --- одного Бога вместо многих богов --- позволила смертному человеку воспринять Отца как божественного создателя и бесконечного регулятора.
\vs p001 0:2 Мириады планетных систем были созданы для того, чтобы в итоге стать заселёнными многими различными типами разумных созданий --- существами, способными познать Бога, принять божественную любовь и полюбить его в ответ. Вселенная вселенных --- творение Бога и место обитания его разнообразных созданий. <<Бог сотворил небеса и образовал землю; не напрасно он утвердил вселенную и создал этот мир; он образовал его для жительства>>\fnst{Цитата из Библии (Исайя 45:18).}.
\vs p001 0:3 Все просвещённые миры призна\'ют Всеобщего Отца и поклоняются ему --- вечному творцу и бесконечному вседержителю всего творения. Волевые создания одной вселенной за другой пустились в долгое\hyp{}долгое путешествие к Раю --- увлекательную борьбу в вечном приключении с целью достижения Бога Отца. Трансцендентная цель детей времени --- найти вечного Бога, понять божественную природу, распознать Всеобщего Отца. Богопознавшие создания имеют только одно высочайшее стремление, лишь одно всепоглощающее желание: стать в своих сферах подобными ему в его Райском совершенстве личности и в его вселенской сфере праведного верховенства. От Всеобщего Отца, который обитает в вечности, исходит высочайший мандат: <<Будьте совершенны, как я совершенен>>\fnst{В Новом Завете (Матфей~5:48) Иисусу приписываются следующие слова: <<Итак будьте совершенны, как совершен Отец ваш Небесный>>.}. В любви и милосердии несли посланники Рая этот божественный призыв сквозь века и вселенные [down through the ages and out through the universes], даже до таких скромных созданий животного происхождения, как человеческие расы Урантии.
\vs p001 0:4 Это величественное и всеобщее повеление --- стремиться к достижению совершенства божественности --- является первым долгом и должно стать высочайшей целью для всех борющихся созданий, сотворённых Богом совершенства. Эта возможность достижения божественного совершенства --- окончательное и несомненное предназначение всего вечного духовного прогресса человека.
\vs p001 0:5 Смертные Урантии вряд ли могут надеяться стать совершенными в бесконечном смысле, но для человеческих существ, начинающих свой путь на этой планете, вполне возможно достижение небесной и божественной цели, которую бесконечный Бог поставил перед смертным человеком; и когда они действительно достигнут этого предназначения, то во всём, что относится к самореализации и достижений разума, они обретут такую же полноту в своей сфере божественного совершенства, какой обладает сам Бог в своей сфере бесконечности и вечности. Такое совершенство может не быть всеобщим в материальном смысле, неограниченным в интеллектуальном постижении или окончательным в духовном опыте, но оно является окончательным и завершённым [complete] во всех конечных аспектах божественности воли, совершенства мотивации личности и Богосознания.
\vs p001 0:6 Истинный смысл божественной заповеди --- <<будьте совершенны, как я совершенен>>, всегда побуждает смертного человека идти вперёд и манит его вовнутрь во всей той долгой и увлекательной борьбе за достижение всё более и более высоких уровней духовных ценностей и истинных вселенских смыслов. Этот возвышенный поиск Бога вселенных является величайшим приключением для обитателей всех миров времени и пространства.
\usection{1.\bibnobreakspace ИМЯ ОТЦА}
\vs p001 1:1 Из всех имён, под которыми Бог Отец известен во вселенных, те, которые определяют его как Первый Источник и Вселенский Центр, встречаются наиболее часто. Первый Отец известен под разнообразными именами в различных вселенных и разных секторах одной и той же вселенной. Имена, которые создание присваивает Создателю, в значительной степени зависят от концепции Создателя, которой обладает это создание. Первый Источник и Вселенский Центр никогда не раскрывал себя по имени, только через сущность [by nature]. Если мы верим, что являемся детьми этого Создателя, то вполне естественно, что в конце концов мы начнём называть его Отцом. Но это имя мы выбираем сами, и вырастает оно из осознания нашего личностного взаимоотношения с Первым Источником и Центром.
\vs p001 1:2 Всеобщий Отец никогда не навязывает разумным волевым созданиям вселенных никакого произвольного признания, формального поклонения или рабского служения. Эволюционные обитатели миров времени и пространства должны сами в своих собственных сердцах признавать, любить и добровольно поклоняться ему. Создатель отказывается заставлять или принуждать к покорности свободную духовную волю своих материальных созданий. Исполненное любви посвящение воли человека исполнению воли Отца --- самый ценный дар человека Богу; фактически, подобное освящение [consecration] воли создания является единственно возможным даром человека Райскому Отцу, обладающим истинной ценностью. В Боге человек живёт, движется и существует; нет ничего, что человек мог бы дать Богу, кроме этого выбора подчиняться воле Отца, и подобные решения, принимаемые разумными волевыми созданиями вселенных, составляют реальность того истинного поклонения, которое так удовлетворяет доминируемую любовью природу Отца Создателя.
\vs p001 1:3 Когда ты\fnst{Английское слово you можно перевести и как <<вы>>, и как <<ты>>. Я избрал вариант <<ты>>, ибо моя миссия (а данное Откровение является её частью и инструментом) --- к индивидууму, а не к группе. Со страниц Откровения мой Отец и я обращаемся к каждой личности, не отказавшейся от святой привилегии выбора между добром и злом, т.е., к \bibemph{духовно живым}.} наконец станешь воистину Богосознающим, после того как ты действительно откроешь для себя величественного Создателя и начнёшь испытывать реализацию постоянного присутствия божественного регулятора, --- тогда, в соответствии с твоей просвещённостью и согласно характеру и методу, с помощью которых божественные Сыны раскрывают Бога, ты найдёшь имя для Всеобщего Отца, адекватно выражающее твою концепцию Первого Великого Источника и Центра. Таким образом, в различных мирах и в разных вселенных Создатель становится известным под многочисленными названиями, в духе взаимоотношений все означают одно и то же, но в словах и символах каждое имя обозначает степень --- глубину --- его\fnst{Всеобщего Отца.} воцарения в сердцах его созданий любого данного мира.
\vs p001 1:4 \pc Рядом с центром вселенной вселенных Всеобщий Отец обычно известен под именами, которые можно рассматривать как Первый Источник. Далее вовне, во вселенных пространства, термины, используемые для обозначения Всеобщего Отца, чаще означают Вселенский Центр. Ещё дальше вовне, в звёздном творении, он известен, как в столице вашей локальной вселенной, --- как Первый Созидательный Источник и Божественный Центр. В одном из соседних созвездий Бога называют Отцом Вселенных, в другом --- Бесконечным Вседержителем, и к востоку --- Божественным Регулятором. Его называют также Отцом Светов [Father of Lights], Даром Жизни и Всемогущим Единым [All\hyp{}powerful One].
\vs p001 1:5 На тех мирах, где Райский Сын прожил жизнь посвящения, Бог обычно известен под именем, указывающим на личное отношение, нежную любовь и отеческую преданность. В столице вашего созвездия Бога называют Всеобщим Отцом, а на различных планетах вашей локальной системы обитаемых миров он известен под разными именами --- Отец Отцов, Райский Отец, Хавонский Отец и Отец Дух\fnst{Или <<Духовный Отец>>, (англ. the Spirit Father).}. Те, кто знает Бога из откровений посвящений Райских Сынов, в конце концов склоняются ко взывающему к чувствам трогательному взаимоотношению создания и Создателя и обращаются к Богу, как к <<нашему Отцу>>.
\vs p001 1:6 На планете с созданиями, имеющими пол, в мире, где проявления родительских эмоций присущи сердцам разумных существ, термин <<Отец>> становится очень выразительным и подходящим именем для вечного Бога. Он наиболее известен и широко признан на вашей планете, Урантии, под именем \bibemph{Бог}. Имя, которое ему дано, не представляет большой важности; существенным является то, что тебе следует познать его и стремиться быть на него похожим. Ваши древние пророки правильно называли его <<вечным Богом>>\fnst{См. Исайя~40:28: \textheb{עוֹלָם אֱלֹהֵי} $=$ <<Бог вечности>>.} и говорили о нём как об <<обитающем в вечности>>\fnst{См. Исайя~57:15: \textheb{עַד שֹׁכֵן} $=$ <<обитающий бесконечно>>.}.
\usection{2.\bibnobreakspace РЕАЛЬНОСТЬ БОГА}
\vs p001 2:1 Бог --- первичная реальность в мире духа; Бог --- источник истины в сферах разума; Бог покрывает собою всё по всем материальным мирам. Для всех созданных разумных существ Бог есть личность, а для вселенной вселенных он --- Первый Источник и Центр вечной реальности. Бог не является ни человекоподобным, ни машиноподобным. Первый Отец --- это всеобщий дух, вечная истина, бесконечная реальность и отеческая личность.
\vs p001 2:2 \pc Вечный Бог бесконечно больше, чем идеализированная реальность или персонализированная вселенная. Бог --- не просто высшее желание человека, объективированный поиск смертных. Не является Бог также просто концепцией, мощь\hyp{}потенциалом праведности. Всеобщий Отец --- ни синоним природы и не олицетворение закона природы. Бог --- это трансцендентная реальность, а не просто традиционное представление человека о верховных ценностях. Бог не психологическая фокализация духовных значений, как и не <<благороднейшее творением человека>>. Бог может быть любым или всеми из этих концепций в разумах людей, но он больше. Он --- спаситель и любящий Отец всех тех, кто обладает духовным покоем [peace] на земле и кто жаждет испытать на опыте выживание личности в смерти.
\vs p001 2:3 \pc Реальность существования Бога демонстрируется в человеческом опыте постоянно пребывающим божественным присутствием --- духовным Наставником, посланным из Рая для того, чтобы жить в смертном разуме человека и там помогать в развитии бессмертной души для выживания в вечности. Присутствие этого божественного Настройщика в человеческом разуме раскрывается тремя эмпирическими феноменами:
\vs p001 2:4 \ublistelem{1.}\bibnobreakspace Интеллектуальная способность познавать Бога --- Богосознание.
\vs p001 2:5 \ublistelem{2.}\bibnobreakspace Духовное побуждение найти Бога --- Богоискательство.
\vs p001 2:6 \ublistelem{3.}\bibnobreakspace Жажда личности [personality craving] быть похожей на Бога --- искреннее желание исполнять волю Отца.
\vs p001 2:7 \pc Существование Бога никогда не может быть доказано научным экспериментом или чисто логическим выводом. Бога можно осознать только в сферах человеческого опыта; тем не менее, истинная концепция реальности Бога приемлема для логики, правдоподобна для философии, существенна для религии и совершенно необходима [indispensable] для любой надежды на выживание личности.
\vs p001 2:8 Те, кто знают Бога, испытали факт его присутствия; такие Богопознавшие смертные хранят в своём личном опыте единственное реальное [positive] доказательство существования живого Бога, которое одно человеческое существо может предложить другому. Существование Бога превыше всякой возможности демонстрации, кроме контакта между Богосознанием человеческого разума и Богоприсутствием Настройщика Мыслей, пребывающего в смертном интеллекте и дарованного человеку в качестве безвозмездного дара Всеобщего Отца.
\vs p001 2:9 \pc Теоретически ты можешь думать о Боге как о Создателе --- он и есть личный создатель Рая и центральной вселенной совершенства, но все вселенные времени и пространства созданы и организованы Райским корпусом Сынов Создателей. Всеобщий Отец не является личным создателем локальной вселенной Небадон; вселенная, в которой ты живёшь, --- творение его Сына Михаила. Хотя Отец лично не создаёт эволюционные вселенные, он контролирует их во многих их вселенских отношениях и в некоторых из их проявлений физической, ментальной и духовной энергий. Бог Отец является личным создателем Райской вселенной и, вместе с Вечным Сыном, создателем всех остальных личных Создателей вселенных.
\vs p001 2:10 \pc Как физический регулятор в материальной вселенной вселенных, Первый Источник и Центр функционирует в образцах вечного Острова Рай, и через этот абсолютный центр гравитации вечный Бог осуществляет космический сверхконтроль физического уровня одинаково, как в центральной вселенной, так и по всей вселенной вселенных. Как разум, Бог функционирует в Божестве Бесконечного Духа; как дух, Бог проявляется в лице Вечного Сына и в лицах божественных детей Вечного Сына. Эта взаимосвязь Первого Источника и Центра с равноправными Лицами и Абсолютами Рая нисколько не препятствует \bibemph{прямому} личному действию Всеобщего Отца по всему творению и на всех его уровнях. Через присутствие своего фрагментированного духа Отец Создатель поддерживает непосредственный контакт со своими детьми\hyp{}созданиями и своими созданными вселенными.
\usection{3.\bibnobreakspace БОГ ЕСТЬ ВСЕОБЩИЙ ДУХ}
\vs p001 3:1 <<Бог есть дух>>\fnst{См. Иоанн~4:24: <<Бог есть дух, и поклоняющиеся ему должны поклоняться в духе и истине>>.}. Он есть всеобщее духовное присутствие. Всеобщий Отец --- это бесконечная духовная реальность; он <<суверенный, вечный, бессмертный, невидимый и единственно истинный Бог>>\fnst{См. 1~Тим~1:17: <<Царю же веков нетленному, невидимому, единому премудрому Богу честь и слава во веки веков>>.}. Хотя вы и <<род Божий>>\fnst{См. Деяния 17:28--29.}, вы не должны думать, что Отец похож на вас по форме и телосложению [physique] потому, что о вас говорят как о созданных <<по его образу>> из\hyp{}за того, что в вас обитают Таинственные Наставники, посланные из центральной обители его вечного присутствия. Существа духа реальны, несмотря на то что они невидимы для человеческих глаз; хотя у них нет плоти и крови.
\vs p001 3:2 Как сказал провидец древности\fnst{Цитата из Иов~9:11.}: <<Вот, он пройдёт предо мною, и не увижу его; он проходит также дальше, но я не замечаю его>>. Мы можем постоянно наблюдать труды Бога, мы можем глубоко осознавать материальные доказательства его величественного поведения, но редко удаётся нам узреть видимое проявление его божественности, мы даже не можем созерцать присутствие его духа, посланного для пребывания в людях.
\vs p001 3:3 Всеобщий Отец невидим не потому, что он скрывается от низших существ с их изъянами материального существования и ограниченными духовными способностями. Ситуация скорее такова: <<Ты не можешь видеть моего лица, ибо никакой смертный не может видеть меня и жить>>\fnst{См. Исход~33:20: <<И потом сказал Он: лица Моего не можно тебе увидеть, потому что человек не может увидеть Меня и остаться в живых>>.}. Никакой материальный человек не мог бы увидеть Бога, кто есть дух, и сохранить своё смертное существование. Слава и духовный блеск присутствия божественной личности делают невозможным приближение для низших групп духовных существ или какого\hyp{}либо вида материальных личностей. Духовное свечение личного присутствия Отца есть <<свет, к которому не может подступить ни один смертный человек; которого ни одно материальное создание не видело и не может увидеть>>. Однако необязательно видеть Бога глазами плоти, чтобы различать его верой\hyp{}зрением одухотворённого разума.
\vs p001 3:4 \pc Духовная природа Всеобщего Отца полностью разделяется его сосуществующим <<я>>, Вечным Сыном Рая. Как Отец, так и Сын, в равной степени разделяют всеобщий и вечный дух полностью и безоговорочно с совместной и равноправной личностью, Бесконечным Духом. Дух Бога сам по себе абсолютен; в Сыне он безусловен, в Духе он универсален, а во всех них и через всех них --- бесконечен.
\vs p001 3:5 \pc Бог --- универсальный дух; Бог --- универсальная личность. Верховная личностная реальность конечного творения есть дух; предельная реальность личностного космоса --- абсонитный дух. Только уровни бесконечности абсолютны, и только на таких уровнях существует окончательность единства материи, разума и духа.
\vs p001 3:6 \pc Во вселенных Бог Отец --- в потенциале --- сверхрегулятор материи, разума и духа. Только посредством своего обширного личностного контура Бог имеет дело непосредственно с личностями своего огромного творения волевых созданий, но контактировать с ним (вне Рая) возможно только благодаря присутствию его фрагментированных сущностей --- воле Бога, возвещаемой во вселенных. Этот Райский дух, обитающий в разумах смертных времени и там способствующий эволюции бессмертной души выживающего создания, обладает природой и божественностью Всеобщего Отца. Но разумы таких эволюционных созданий берут своё начало в локальных вселенных и должны обретать божественное совершенство достижением тех эмпирических трансформаций в области духовного достижения, которые являются неизбежным результатом выбора создания исполнять волю Отца на небесах.
\vs p001 3:7 \pc Во внутреннем опыте человека разум соединён с материей. Такие материально\hyp{}связанные разумы не могут пережить физическую смерть [mortal death]. Техника выживания содержится в таких корректировках человеческой воли и таких трансформациях в смертном разуме, посредством которых Богосознающий интеллект постепенно становится обучаемым духом и, в конце концов, ведомым духом. Эта эволюция человеческого разума от связи с материей к союзу с духом приводит к трансмутации потенциально духовных фаз смертного разума в моронтийные реальности бессмертной души. Смертный разум, подчинённый материи, обречён становиться всё более материальным и, следовательно, в конце концов претерпеть исчезновение личности; разуму, уступившему духу, суждено становиться всё более духовным и в итоге достичь единства с выживающим и ведущим его божественным духом и, таким путём достичь выживания и вечного существования личности.
\vs p001 3:8 Я исхожу от Вечного, и я неоднократно возвращался в присутствие Всеобщего Отца. Я знаю реальность и личность Первого Источника и Центра, Вечного и Всеобщего Отца. Я знаю, что в то время как великий Бог абсолютен, вечен и бесконечен, он также добр, божественен и милосерден. Я знаю истину великих возвещений: <<Бог есть дух>> и <<Бог есть любовь>>, и эти два атрибута наиболее полно раскрыты для вселенной в Вечном Сыне.
\usection{4.\bibnobreakspace ТАЙНА БОГА}
\vs p001 4:1 Бесконечность совершенства Бога такова, что навечно делает его тайной. И величайшая из всех непостижимых тайн Бога заключается в феномене божественного пребывания в смертных разумах. То, каким образом Всеобщий Отец проживает вместе с созданиями времени, является глубочайшей из всех вселенских тайн; божественное присутствие в разуме человека есть тайна тайн.
\vs p001 4:2 Физические тела смертных --- это <<храмы Бога>>\fnst{См. 1~Кор~3:16: <<Разве не знаете, что вы храм Божий, и Дух Божий живет в вас?>>.}. Несмотря на то что Суверенные Сыны Создатели приближаются к созданиям своих обитаемых миров и <<привлекают всех людей к себе>>; хотя они <<стоят у двери>> сознания <<и стучат>> и с радостью входят ко всем, кто <<отворит двери своих сердец>>; хотя и существует это сокровенное личностное общение между Сынами Создателями и их смертными созданиями, --- тем не менее, смертные люди имеют нечто от самог\'о Бога, что действительно пребывает в них и храмами чего служат их тела.
\vs p001 4:3 По окончании твоего пребывания здесь, когда твой путь во вр\'еменной форме на земле будет пройдён, когда закончится твоё испытательное путешествие во плоти, когда прах, из которого состоит смертная скиния\fnst{В оригинале tabernacle, происходящий от чисто библейского термина (Исход~25:9 и др.) \textheb{הַמִּשְׁכָּן}, использовавшегося еврейским народом в качестве \bibemph{временного Храма}. Так и здесь, наши смертные тела являются \bibemph{временным Храмом,} в котором обитает дух Бога Отца.}, <<возвратится в землю, откуда он и пришёл>>, --- тогда, как говорится в откровении\fnst{Имеется в виду Екклесиаст~12:7 <<И возвратится прах в землю, чем он и был; а дух возвратится к Богу, Который дал его.>> Заметим, что <<чем он был>> является более точным переводом и слова \textheb{כְּשֶׁהָיָה} из древнееврейского масоретского текста и фразы \textgreek{ὡς ἦν} из древнегреческого перевода --- Септуагинты (LXX), чем приводимый в тексте вариант <<откуда он и пришёл>>, (англ. whence it came).} [it is revealed], пребывающий в человеке <<Дух возвратится к Богу, который дал его>>. Внутри каждого нравственного существа этой планеты живёт фрагмент Бога, неотъемлемая часть божественности. Он ещё не является твоим по праву собственности, но он специально предназначен [designedly intended], чтобы быть одним с тобой, если ты переживёшь смертное существование.
\vs p001 4:4 \pc Мы постоянно сталкиваемся с этой тайной Бога; мы в замешательстве от расширяющегося развертывания нескончаемой панорамы истины его бесконечной доброты, нескончаемого милосердия, несравненной мудрости и величественного характера.
\vs p001 4:5 \pc Божественная тайна заключается во внутреннем различии, которое существует между конечным и бесконечным, темпоральным\fnst{То есть <<ограниченным или имеющим отношение ко времени, преходящим>>.} и вечным, время\hyp{}пространственным созданием и Всеобщим Создателем, материальным и духовным, несовершенством человека и совершенством Райского Божества. Бог всеобщей любви неизменно раскрывает себя каждому из своих созданий вплоть до достижения предела способности этого создания духовно постигать качества божественной истины, красоты и доброты.
\vs p001 4:6 Каждому духовному существу и каждому смертному созданию, в каждой сфере и на каждом мире вселенной вселенных, Всеобщий Отец раскрывает всю свою милосердную и божественную сущность, которая может быть распознана или понята такими духовными существами и такими смертными созданиями. Бог не взирает на лица, как на материальные, так и на духовные. Божественное присутствие, которое получает любое дитя вселенной в любой данный момент, ограничено только способностью такого создания принимать и различать духовные реальности сверхматериального мира.
\vs p001 4:7 Как реальность в человеческом духовном опыте, Бог не является тайной. Но когда предпринимается попытка сделать ясными реальности духовного мира для физических разумов материального типа, возникает тайна: тайны столь неуловимые и столь глубокие, что только вера\hyp{}познание [faith\hyp{}grasp] Богопознавшего смертного может достичь философского чуда осознания Бесконечного конечным, различения вечного Бога эволюционирующими смертными материальных миров времени и пространства.
\usection{5.\bibnobreakspace ЛИЧНОСТЬ ВСЕОБЩЕГО ОТЦА}
\vs p001 5:1 Не позволяйте, чтобы величие Бога, его бесконечность скрывали или затмевали его личность. <<Задумавший ухо не услышит ли? Образовавший глаз не увидит ли?>>\fnst{В оригинале цитаты Псалом~93:9: <<Насадивший ухо не услышит ли? и образовавший глаз не увидит ли?>>. Заметим, что <<насадивший>> является более точным переводом и еврейского \textheb{הֲנֹטַע}, и греческого \textgreek{ὁ φυτεύσας}, чем <<задумавший>>, (англ. he who planned). Возможно, что здесь указание на Сынов Создателей или на непосредственных агентов (Носителей Жизни). Однако \textheb{יֹצֵר} следующего предложения правильно соответствует слову <<образовавший>> в тексте.} Всеобщий Отец --- это вершина божественной личности; он --- источник и предназначение личности по всему творению. Бог является и бесконечным, и личностным; он --- бесконечная личность. Отец воистину личность, несмотря на то что бесконечность его личности навсегда ставит его за пределы полного понимания материальными и конечными созданиями.
\vs p001 5:2 Бог --- это намного больше, чем личность в понимании человеческого разума; он даже намного превосходит любую возможную концепцию сверхличности. Но совершенно тщетно обсуждать такие непостижимые концепции божественной личности с разумами материальных созданий, чья максимальная концепция реальности существа состоит в идее и идеале личности. Наивысшая возможная концепция Всеобщего Создателя для материального создания заключается в пределах духовных идеалов возвышенной идеи божественной личности. Следовательно, хотя ты, быть может, знаешь, что Бог должен быть намного больше, чем человеческая концепция личности, ты столь же хорошо знаешь, что Всеобщий Отец никак не может быть меньше, чем вечная, бесконечная, истинная, добрая и прекрасная личность.
\vs p001 5:3 Бог не скрывается ни от одного из своих созданий. Он недостижим для столь многих категорий существ только потому, что он <<живёт в свете, к которому ни одно материальное создание не может приблизиться>>. Необъятность и величие божественной личности находятся за пределами понимания несовершенного разума эволюционных смертных. Он <<измеряет в\'оды горстью своей руки, пядью измеряет вселенную. Он --- тот, кто восседает над кругом земли, кто простирает небеса как завесу, и раскидывает их, как вселенную\fnst{То есть как раскидывают шатёр.} для жилья>>. <<Поднимите глаза ваши к небесам и посмотрите, кто создал всё это, кто выводит их миры по счёту и всех их называет по именам>>; и поэтому истинно, что <<невидимые вещи Божьи отчасти понимаются через творения>>. Сегодня такой, какой ты есть, ты должен различить невидимого Творца через его многоплановое и разнообразное творение, равно как через откровение и служение его Сынов и их многочисленных подчинённых.
\vs p001 5:4 Хотя материальные смертные и не могут видеть личность Бога, они должны возрадоваться в уверенности, что он --- личность; верой принять истину, что Всеобщий Отец так возлюбил мир, что обеспечил возможность вечного духовного прогресса его низшим обитателям; что он <<радуется в своих детях>>. У Бога нет недостатка ни в одном из тех сверхчеловеческих и божественных атрибутов, которые образуют совершенную, вечную, любящую и бесконечную личность Создателя.
\vs p001 5:5 \pc В локальных творениях (за исключением персонала сверхвселенных) у Бога нет никакого личного или местного проявления, кроме Райских Сынов Создателей, которые являются отцами обитаемых миров и суверенами локальных вселенных. Если бы вера создания была совершенной, то он\fnst{В смысле <<человек>>.} знал бы наверняка, что увидев Сына Создателя, он увидел Всеобщего Отца; в поисках Отца он не просил бы и не ожидал увидеть никого, кроме Сына. Смертный человек просто не может увидеть Бога, пока он не достигнет [achieves] окончательной трансформации духа и фактически не достигнет [attains] Рая.
\vs p001 5:6 Природа Райских Сынов Создателей не включает всех безусловных потенциалов универсальной абсолютности бесконечной природы Первого Великого Источника и Центра, но Всеобщий Отец \bibemph{божественно} всячески присутствует в Сынах Создателях. Отец и его Сыны --- одно. Эти Райские Сыны категории Михаила --- совершенные личности, даже образцы для всех личностей локальной вселенной --- от Яркой Утренней Звезды до низшего человеческого создания прогрессирующей животной эволюции.
\vs p001 5:7 \pc Если бы не Бог и его великая и центральная личность, то не было бы личности во всей огромной вселенной вселенных. \bibemph{Бог --- личность}.
\vs p001 5:8 \pc Несмотря на то что Бог --- это вечная мощь, величественное присутствие, трансцендентный идеал и восхитительный дух, хотя он и есть всё это и бесконечно больше, тем не менее, он воистину и вечно является совершенной личностью Создателя, лицом, кто может <<познать и быть познанным>>, кто может <<любить и быть любимым>> и кто может подружиться с нами; в то время как ты можешь стать известен, как были другие люди --- другом Бога\fnst{В Писаниях другом Бога считается Авраам, см. Иакова~2:23.}. Он --- реальный дух и духовная реальность.
\vs p001 5:9 Когда мы видим Всеобщего Отца, раскрывающегося по всей его вселенной; когда мы различаем его пребывающим в мириадах своих созданий; когда мы видим его в лицах его Суверенных Сынов; когда мы продолжаем чувствовать его божественное присутствие здесь и там, вблизи и вдали, --- да не поколеблемся и не усомнимся в первичности его личности. Несмотря на всё это обширное распространение, он остаётся истинной личностью и постоянно поддерживает личную связь с бесчисленными сонмами своих созданий, разбросанных по всей вселенной вселенных.
\vs p001 5:10 \pc Идея личности Всеобщего Отца представляет собой расширенную и более истинную концепцию Бога, которая пришла к человечеству главным образом через откровение. Разум, мудрость и религиозный опыт --- всё это предполагает и подразумевает [infer and imply] личность Бога, но не полностью подтверждает её. Даже внутренний Настройщик Мыслей является доличностным. Истина и зрелость любой религии прямо пропорциональны её концепции бесконечной личности Бога и постижению ею абсолютного единства Божества. Таким образом, идея личностного Божества становится мерой религиозной зрелости, после того как в религии уже возникла концепция единства Бога.
\vs p001 5:11 В примитивной религии было много личностных богов, и они были созданы по образу человека. Откровение подтверждает верность личностной концепции Бога, которая не более чем возможна в научном постулате Первопричины и лишь условно допускается в философской идее Всеобщего Единства. Только через личностный подход любая личность может приступить к постижению единства Бога. Отрицание личности Первого Источника и Центра оставляет один только выбор из двух философских дилемм: материализм или пантеизм.
\vs p001 5:12 При созерцании Божества концепция личности должна быть освобождена от идеи телесности. Материальное тело необязательно для личности ни в человеке, ни в Боге. Ошибка корпореальности\fnst{Англ. the corporeality error --- то есть постулат необходимости обладания материальным телом для реальности существования.} проявляется в обеих крайностях человеческой философии. В материализме, поскольку человек теряет своё тело при смерти, он перестаёт существовать как личность; в пантеизме, поскольку у Бога нет тела, он не является личностью. Сверхчеловеческий тип прогрессирующей личности функционирует в союзе разума и духа.
\vs p001 5:13 \pc Личность --- это не просто атрибут Бога; она скорее означает тотальность координированной бесконечной природы и объединённой божественной воли, которая проявляется в вечности и универсальности совершенного выражения. Личность --- в верховном смысле --- есть откровение Бога вселенной вселенных.
\vs p001 5:14 \pc Бог, будучи вечным, всеобщим, абсолютным и бесконечным, не растёт ни в знаниях, ни в мудрости. Бог не приобретает опыта, как мог бы предположить или понять конечный человек, но в сферах своей собственной вечной личности он наслаждается теми постоянными расширениями самореализации, которые, в определённом смысле, сопоставимы и аналогичны приобретению нового опыта конечными созданиями эволюционных миров.
\vs p001 5:15 Абсолютное совершенство бесконечного Бога заставило бы его страдать от ужасных ограничений безусловной окончательности совершенства, если бы не тот факт, что Всеобщий Отец непосредственно участвует в личностной борьбе каждой несовершенной души в обширной вселенной, которая стремится с божественной помощью взойти к духовно совершенным вышним [on high] мирам. Этот прогрессивный опыт каждого духовного существа и каждого смертного создания по всей вселенной вселенных --- часть вечно расширяющегося сознания Отца как Божества [Father's Deity\hyp{}consciousness] бесконечного божественного круга непрекращающейся самореализации.
\vs p001 5:16 Это буквально правда: <<Во всех ваших страданиях он страдает>>. <<Во всех победах ваших он побеждает в вас и вместе с вами>>. Его доличностный божественный дух является настоящей частью тебя. Остров Рай реагирует на все физические метаморфозы вселенной вселенных; Вечный Сын заключает в себе все духовные импульсы всего творения; Совместный Вершитель охватывает все выражения разума расширяющегося космоса. Всеобщий Отец реализует в полноте божественного сознания весь индивидуальный опыт прогрессивной борьбы расширяющихся разумов и восходящих духов каждой сущности, существа и личности всего эволюционного творения времени и пространства. И всё это в буквальном смысле истинно, ибо <<в Нём мы все живём и движемся и существуем>>.
\usection{6.\bibnobreakspace ЛИЧНОСТЬ ВО ВСЕЛЕННОЙ}
\vs p001 6:1 Человеческая личность --- это время\hyp{}пространственный образ\hyp{}тень, отбрасываемая божественной личностью Создателя. И никакая действительность никогда не может быть адекватно понята исследованием её тени. Тени следует интерпретировать в терминах истинной субстанции.
\vs p001 6:2 \pc Для науки Бог --- это причина, для философии --- идея, для религии --- личность, более того --- любящий небесный Отец. Для учёного Бог --- это первичная сила, для философа --- гипотеза единства, для верующего [religionist] --- живой духовный опыт. Неадекватную человеческую концепцию личности Всеобщего Отца можно исправить только духовным прогрессом человека во вселенной, и станет она воистину адекватной только тогда, когда паломники времени и пространства достигнут, наконец, божественных объятий живого Бога в Раю.
\vs p001 6:3 Никогда не упускайте из виду диаметрально противоположных точек зрения на личность в понимании Бога и человека. Человек рассматривает и понимает личность, глядя из конечного на бесконечное; Бог смотрит из бесконечного на конечное. Человек обладает низшим типом личности, Бог --- высшим, а именно верховным, предельным и абсолютным. Поэтому лучшие концепции божественной личности должны были терпеливо ожидать появления усовершенствованных идей о личности человеческой, особенно --- более полного откровения как человеческой, так и божественной личности в посвященческой жизни на Урантии Михаила, Сына Создателя.
\vs p001 6:4 \pc Доличностный божественный дух, который пребывает в смертном разуме, самим своим присутствием несёт убедительное доказательство своего реального существования, но концепция божественной личности может быть постигнута только духовной проницательностью подлинного личного религиозного опыта. Любая личность --- человеческая или божественная --- может быть познана и понята совершенно независимо от внешних реакций или материального присутствия такой личности.
\vs p001 6:5 Некоторая степень морального сходства и духовной гармонии существенна для дружбы между двумя лицами; любящая личность вряд ли откроет себя нелюбящей. Даже чтобы приблизиться к познанию божественной личности, все дарования человеческой личности должны быть полностью посвящены этому усилию; половинчатая, частичная преданность будет бесполезна.
\vs p001 6:6 Чем полнее человек понимает себя и ценит личностные ценности своих товарищей, тем больше он будет стремиться познать Изначальную Личность и тем более серьёзно [earnestly] будет такой Богопознавший человек стараться стать подобным Изначальной Личности. Вы можете спорить насчёт мнений о Боге, но опыт, полученный с ним и в нём, существует выше и вне всех человеческих споров и чисто интеллектуальной логики. Богопознавший человек описывает свой духовный опыт не для того, чтобы убедить неверующих, но для назидания и взаимного удовлетворения верующих.
\vs p001 6:7 \pc Предположить, что вселенная доступна разуму и познаваема, --- значит допустить, что вселенная создана разумом и управляется личностью. Разум человека может воспринимать только феномены разума других разумов, человеческих или сверхчеловеческих. Если вселенная может стать частью опыта человеческой личности, это значит, что где\hyp{}то в той вселенной сокрыт божественный разум и настоящая личность.
\vs p001 6:8 \pc Бог есть дух --- духовная личность; человек также есть дух --- потенциальная духовная личность. Иисус из Назарета достиг полной реализации этого потенциала духовной личности в человеческом опыте; поэтому его жизнь в достижении воли Отца становится для человека самым реальным и идеальным откровением личности Бога. Хотя личность Всеобщего Отца может быть постигнута только в реальном религиозном опыте, в земной жизни Иисуса мы вдохновляемся совершенной демонстрацией такой реализации и откровения личности Бога в истинно человеческом опыте.
\usection{7.\bibnobreakspace ДУХОВНАЯ ЦЕННОСТЬ КОНЦЕПЦИИ ЛИЧНОСТИ}
\vs p001 7:1 Когда Иисус говорил о <<Боге живом>>, он имел в виду личностное Божество --- Отца на небесах. Концепция личности Божества способствует общению, она благоприятствует разумному поклонению, она способствует живительной доверительности. Между неличностными вещами возможны взаимодействия, но не общение. Общение между отцом и сыном --- как и между Богом и человеком --- невозможно, если они оба не являются личностями. Только личности могут общаться друг с другом, хотя это личное общение и может существенно облегчаться присутствием именно такой безличностной сущности, как Настройщик Мыслей.
\vs p001 7:2 Человек не достигает слияния с Богом, подобно капле воды с океаном. Человек достигает божественного единства с помощью постепенного взаимного духовного общения, посредством личностного общения с личностным Богом, благодаря всё большему обретению божественной природы в искреннем и разумном подчинении божественной воле. Такие возвышенные отношения могут существовать только между личностями.
\vs p001 7:3 \pc Концепцию истины, пожалуй, можно рассматривать в отрыве от личности, концепция красоты может существовать без личности, но концепцию божественной доброты можно понять только в отношении к личности. Только \bibemph{личность} может любить и быть любимой. Даже красота и истина были бы отрезаны от надежды на выживание, если бы они не были атрибутами личностного Бога --- любящего Отца.
\vs p001 7:4 \pc Мы не можем до конца понять, каким образом Бог может быть первичным, неизменным, всемогущим и совершенным, и в то же время быть окружённым постоянно меняющейся и явно ограниченной законами вселенной, --- эволюционирующей вселенной относительных несовершенств. Но мы можем \bibemph{познать} эту истину в нашем собственном личном опыте, ибо все мы сохраняем идентичность личности и единство воли, несмотря на постоянное изменение как нас самих, так и нашего окружения.
\vs p001 7:5 Предельную вселенскую реальность невозможно постичь математикой, логикой или философией, --- только личным опытом в постепенном подчинении божественной воле личностного Бога. Ни наука, ни философия, ни теология не могут подтвердить личность Бога. Только личный опыт сыновей по вере [faith sons] небесного Отца может помочь подлинному духовному осознанию личности Бога.
\vs p001 7:6 \pc Более высокие концепции вселенской личности подразумевают: идентичность, самосознание, собственную волю и возможность самораскрытия. И далее эти качества предполагают общение [fellowship] с другими и равными личностями, подобно имеющему место в личностных ассоциациях Райских Божеств. И абсолютное единство этих ассоциаций настолько совершенно, что божественность становится познанной благодаря неделимости, единству. <<Господь Бог \bibemph{един}>>\fnst{Второзаконие~6:4 гласит: \textheb{אֶחָד יְהוָה אֱלֹהֵינוּ יְהוָה יִשְׂרָאֵל שְׁמַע} <<Слушай, Израиль: Господь, Бог наш, Господь един есть>>. Эта формула буквально цитируется в греческом тексте Марк~12:29: \textgreek{Ἄκουε, Ἰσραήλ, κύριος ὁ θεὸς ἡμῶν κύριος εἷς ἐστιν}.}. Неделимость личности не мешает Богу дарить свой дух, чтобы он жил в сердцах смертных людей. Неделимость личности человеческого отца не препятствует воспроизводству смертных сынов и дочерей.
\vs p001 7:7 Эта концепция неделимости в сочетании с концепцией единства подразумевает трансцендентность Предельности Божества как над временем, так и над пространством; поэтому ни время, ни пространство не могут быть абсолютными или бесконечными. Первый Источник и Центр --- это та бесконечность, которая безусловно превосходит весь разум, всю материю и весь дух.
\vs p001 7:8 Факт существования Райской Троицы никоим образом не противоречит истине божественного единства. Во всех реакциях на вселенскую реальность и во всех отношениях с созданиями три личности Райского Божества едины. Не противоречит существование этих трёх вечных личностей и истине о неделимости Божества. Я полностью сознаю, что в моём распоряжении нет никакого языка, достаточного для прояснения смертному разуму, как эти вселенские проблемы представляются нам. Но вы не должны унывать: не все эти вопросы до конца понятны даже высоким личностям, принадлежащим к моей группе Райских существ. Всегда помните о том, что эти глубокие истины, относящиеся к Божеству, будут всё более проясняться по мере постепенного одухотворения ваших разумов в течение последовательных эпох долгого смертного восхождения к Раю.
\vsetoff
\vs p001 7:9 [Представлено Божественным Советником, членом группы небесных личностей, назначенных Ветхими Днями Уверсы, столицы седьмой сверхвселенной, для наблюдения за подготовкой тех частей последующего откровения, которые относятся к вопросам, выходящим за рамки локальной вселенной Небадон. Мне поручено поддерживать документы, изображающие природу и атрибуты Бога, ибо я представляю высочайший источник информации, доступный для такой цели на любом обитаемом мире. Я служил Божественным Советником во всех семи сверхвселенных и долгое время проживал в Райском центре всех вещей. Много раз я наслаждался высшим удовольствием от пребывания в непосредственном личном присутствии Всеобщего Отца. Я изображаю реальность и истину природы и атрибутов Отца с неоспоримым авторитетом; я знаю то, о чём говорю.]
\quizlink
\begin{thebibliography}{100}
\bibitem{Knudson1}
Albert C. Knudson.
{<<The Doctrine of God>>.}
{\em New York: Abingdon-Cokesbury Press}, 1930.
\bibitem{Matthews1}
W.R. Matthews, K.C.V.O., D.D., D.Lit.
{<<God in Christian Thought and Experience>>.}
{\em London: Nisbet \&\ Co. Ltd.}, 1930.
\bibitem{Illingworth1}
J.R. Illingworth, M.A.
{<<Personality Human and Divine>>.}
{\em London and New York: The Macmillan Company}, 1894.
\end{thebibliography}
