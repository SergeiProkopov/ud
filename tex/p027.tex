\upaper{27}{СЛУЖЕНИЕ ПЕРВИЧНЫХ СУПЕРНАФИМОВ}
\uminitoc{ИНИЦИАТОРЫ ПОКОЯ}
\uminitoc{ГЛАВЫ НАЗНАЧЕНИЙ}
\uminitoc{ИНТЕРПРЕТАТОРЫ ЭТИКИ}
\uminitoc{РУКОВОДИТЕЛИ ПОВЕДЕНИЯ}
\uminitoc{ХРАНИТЕЛИ ЗНАНИЙ}
\uminitoc{МАСТЕРА ФИЛОСОФИИ}
\uminitoc{РУКОВОДИТЕЛИ ПОКЛОНЕНИЯ}
\author{Совершенствователь Мудрости}
\vs p027 0:1 Первичные супернафимы~--- это возвышенные слуги Божеств на вечном Острове Рай. Никогда они ещё не отклонялись от путей света и праведности. Их численность окончательна и неизменна; от вечности ни один из этого великолепного воинства не был потерян. Эти высокие супернафимы~--- совершенные существа, верховные в совершенстве, но они не абсонитны, как не являются они и абсолютными. Будучи самой сутью совершенства, эти дети Бесконечного Духа трудятся взаимозаменяемо и по собственной воле во всех фазах своих разнообразных обязанностей. Хотя они не действуют широко за пределами Рая, но участвуют в различных тысячелетних собраниях и групповых встречах центральной вселенной. Они также выступают в качестве специальных посланников Божеств, и многие из них поднимаются до статуса Технических Консультантов.
\vs p027 0:2 Первичные супернафимы также ставятся во главе серафических воинств, которые служат мирам, изолированным из\hyp{}за восстания. После того как Райский Сын посвящает себя такому миру, завершает свою миссию, восходит ко Всеобщему Отцу, принимается и возвращается как облечённый полномочиями спаситель этого изолированного мира, главы назначений всегда уполномочивают первичного супернафима принять руководство духами\hyp{}помощниками восстановленной сферы. Супернафимы во время этой специальной службы периодически сменяются. На Урантии нынешний <<глава серафимов>> является вторым представителем этой категории, исполняющим обязанности со времён посвящения Христа Михаила.
\vs p027 0:3 От вечности первичные супернафимы служили на Острове Света и отправлялись с миссиями руководства к мирам пространства, но текущая классификация их функций существует только с момента прибытия на Рай пилигримов времени из Хавоны. Сейчас эти высокие ангелы служат главным образом в следующих семи категориях:
\vs p027 0:4 \li{1.}Руководители поклонения.
\vs p027 0:5 \li{2.}Мастера философии.
\vs p027 0:6 \li{3.}Хранители знаний.
\vs p027 0:7 \li{4.}Руководители поведения.
\vs p027 0:8 \li{5.}Интерпретаторы этики.
\vs p027 0:9 \li{6.}Главы назначений.
\vs p027 0:10 \li{7.}Инициаторы покоя.
\vs p027 0:11 Только после того как восходящие пилигримы достигают статуса постоянных жителей Рая, они попадают под непосредственное влияние этих супернафимов, проходя затем обучение под руководством этих ангелов в последовательности, обратной вышеуказанной. То есть ты вступишь на свой Райский путь под опекой инициаторов покоя, и после ряда этапов, на каждом из которых тебя будет обучать представитель промежуточной категории, ты закончишь этот период подготовки с руководителями поклонения. И тогда ты будешь готов начать свой нескончаемый путь завершителя.
\usection{ИНИЦИАТОРЫ ПОКОЯ}
\vs p027 1:1 Инициаторы покоя~--- это инспекторы Рая, которые отправляются с центрального Острова к внутреннему контуру Хавоны, чтобы сотрудничать там со своими коллегами, дополнениями покоя из второй категории супернафимов. Одна из существенных составляющих для наслаждения Раем~--- это покой, божественный отдых; и эти инициаторы покоя~--- последние наставники, которые готовят пилигримов времени к вступлению в вечность. Они начинают свою работу на последнем круге достижения центральной вселенной и продолжают её, когда пилигрим пробуждается от последнего сна перехода, дремоты, которая выпускает создание пространства в царство вечности.
\vs p027 1:2 \pc Покой имеет семичастную природу: есть отдых сна и досуга у низших форм жизни, отдых открытия~--- у более высоких существ, и отдых поклонения~--- у высшего типа личности духа. Есть также обычный отдых для получения энергии, перезарядка существ физической или духовной энергией. И ещё существует сон перехода, бессознательный сон в объятиях серафима при перемещении с одной сферы на другую. Полностью отличается от всех этих форм глубокий сон метаморфозы, отдых перехода от одной стадии бытия к другой, от одной жизни к другой, от одного состояния существования к другому, сон, который всегда сопутствует переходу к новому вселенскому \bibemph{статусу,} в отличие от эволюции через различные \bibemph{стадии} какого\hyp{}либо одного статуса.
\vs p027 1:3 Однако последний сон метаморфозы~--- нечто большее, чем те предыдущие сны перехода, которые отмечали последовательные достижения следующего статуса на пути восхождения; с его помощью создания времени и пространства пересекают внутренние границы временн\'ого и пространственного, чтобы достичь статуса постоянного жительства во вневременн\'ых и внепространственных обителях Рая. Инициаторы покоя и дополнения покоя точно так же необходимы для этой трансцендентной метаморфозы, как серафимы и связанные с ними существа для выживания смертного существа после смерти.
\vs p027 1:4 \pc Ты погрузишься в покой на последнем контуре Хавоны и воскреснешь в вечности на Рае. И после духовной реперсонализации ты мгновенно узнаешь в приветствующем тебя на вечных берегах инициаторе покоя того самого первичного супернафима, который погрузил тебя в последний сон на внутреннем контуре Хавоны; и вспомнишь своё последнее великое усилие веры, когда ты вновь был готов вверить своё существование в руки Всеобщего Отца.
\vs p027 1:5 Ты насладился последним покоем времени; испытал последний сон перехода; теперь ты пробуждаешься к жизни вечной на берегах вечной обители. <<И не будет больше сна. Присутствие Бога и его Сына стоит перед вами, и вы слуги его навечно; ты увидел его лицо, и имя его~--- твой дух. Там не будет ночи; и они не нуждаются в свете солнца, ибо Великий Источник и Центр даёт им свет; они будут жить во веки веков. И Бог отрёт все слёзы с их глаз; не будет больше смерти, ни печали, ни плача, и не будет больше боли, ибо прежнее прошло>>.\fnst{Ср. Откровение\,21:23: <<И город не имеет нужды ни в солнце, ни в луне для освещения своего, ибо слава Божия осветила его, и светильник его~--- Агнец>>, Откровение\,21:4: <<И отрёт Бог всякую слезу с очей их, и смерти не будет уже; ни плача, ни вопля, ни болезни уже не будет, ибо прежнее прошло>>.}
\usection{ГЛАВЫ НАЗНАЧЕНИЙ}
\vs p027 2:1 Это группа, которую глава супернафимов, <<ангел оригинального образца>>, время от времени назначает возглавлять организацию всех трёх категорий ангелов данного типа: первичных, вторичных и третичных. Супернафимы~--- полностью самоуправляющаяся и саморегулирующаяся группа, за исключением функций их общего главы, первого ангела Рая, который всегда руководит всеми этими духами\hyp{}личностями.
\vs p027 2:2 Ангелы назначения тесно сотрудничают с прославленными смертными жителями Рая до принятия последних в Корпус Завершения. Занятия прибывающих на Рай не ограничиваются лишь изучением и обучением; служение также играет важную роль в предфинальном образовательном Райском опыте. И я наблюдал, как в периоды досуга восходящие смертные обнаруживают предрасположенность к дружескому общению с резервным корпусом супернафических глав назначений.
\vs p027 2:3 Когда вы, восходящие смертные, достигаете Рая, круг вашего общения включает гораздо больше, чем контакт с воинством возвышенных и божественных существ и со знакомой толпой прославленных смертных собратьев. Вам предстоит подружиться более чем с 3\,000 различных категорий Граждан Рая, с различными группами Трансценденталов и с многочисленными другими типами обитателей Рая, постоянных и вр\'еменных, не раскрытых на Урантии. После постоянного контакта с этими могучими интеллектами Рая, лучший отдых~--- общение с ангельскими типами разума; они напоминают смертным времени серафимов, с которыми у них был столь длительный контакт и освежающее общение.
\usection{ИНТЕРПРЕТАТОРЫ ЭТИКИ}
\vs p027 3:1 Чем выше ты поднимаешься по шкале жизни, тем больше внимания должно уделяться вселенской этике. Этическое сознание~--- это просто признание любым индивидуумом неотъемлемых прав любого и всех остальных индивидуумов. Но духовная этика намного превосходит смертные и даже моронтийные представления о личных и групповых отношениях.
\vs p027 3:2 Этика должным образом преподаётся и адекватно изучается пилигримами времени в их долгом восхождении к триумфам Рая. На всём протяжении своего пути восхождения внутрь, начиная от миров происхождения в пространстве, восходящие создания продолжают прибавлять к своему непрестанно расширяющемуся кругу вселенских товарищей всё новые группы существ. Каждая новая группа коллег, с которой они встречаются, добавляет следующий уровень этики, который надо осознать и соблюдать, и ко времени достижения восходящими смертными Рая, они действительно нуждаются в чьём\hyp{}то полезном дружеском совете в отношении этических интерпретаций. Их не нужно учить этике, но, столкнувшись лицом к лицу с неординарной задачей контакта с таким обилием нового, они нуждаются в правильной \bibemph{интерпретации} того, что с таким усердием освоили.
\vs p027 3:3 Интерпретаторы этики оказывают неоценимую помощь прибывающим на Рай, помогая им приспособиться к многочисленным группам величественных существ в течение этого насыщенного событиями периода, длящегося от достижения статуса постоянного жительства до официального вступления в Корпус Смертных Завершителей. Многих из большого числа типов Граждан Рая восходящие пилигримы уже встречали на семи контурах Хавоны. Прославленные смертные также имели близкий контакт с сынами, тринитизованными созданиями, из объединённого корпуса на внутреннем контуре Хавоны, где эти существа получают значительную часть своего образования. А на других контурах восходящие пилигримы встречали многочисленных нераскрытых постоянных жителей системы Рай\hyp{}Хавона, которые продолжают там групповое обучение в рамках подготовки к нераскрытым заданиям будущего.
\vs p027 3:4 Все эти небесные товарищеские отношения неизменно взаимны. Как восходящие смертные, вы не только получаете пользу от этих сменяющих друг друга вселенских спутников и столь многочисленных категорий всё более божественных товарищей, но также передаёте каждому из этих братских существ что\hyp{}то из своей личности и опыта, тем самым навсегда изменяя и делая лучше каждого, кто имел общение с восходящим смертным из эволюционных миров времени и пространства.
\usection{РУКОВОДИТЕЛИ ПОВЕДЕНИЯ}
\vs p027 4:1 Уже полностью обученные этике Райских отношений~--- не бессмысленным формальностям или диктату искусственных каст, а, скорее, приличиям, присущим им внутренне,~--- восходящие смертные считают полезным получить совет супернафических руководителей поведения, которые учат новых членов Райского общества нормам совершенного поведения высоких существ, пребывающих на центральном Острове Света и Жизни.
\vs p027 4:2 Гармония~--- это лейтмотив центральной вселенной, и на Рае царит очевидный порядок. Надлежащее поведение существенно для прогресса на пути знаний, проходящем через философию к духовным высотам спонтанного поклонения. Существуют божественные приёмы приближения к Божественности; и овладение ими возможно лишь по прибытии пилигримов на Рай. Дух этих приёмов был уже передан на контурах Хавоны, но последние штрихи в обучении пилигримов времени можно наложить только после того, как они действительно достигнут Острова Света.
\vs p027 4:3 Всё поведение на Рае полностью спонтанно, во всех смыслах естественно и непринуждённо. Но при этом на вечном Острове существует надлежащий и совершенный образ действий, и руководители поведения всегда рядом с <<чужестранцами, вошедшими в ворота>>, наставляя и направляя их шаги так, чтобы они чувствовали себя совершенно непринуждённо, и в то же время давая пилигримам возможность избежать той путаницы и неопределённости, которые в противном случае были бы неизбежны. Только таким образом можно было предотвратить бесконечный беспорядок; а его на Рае не бывает никогда.
\vs p027 4:4 Эти руководители поведения действительно служат как прославленные учителя и наставники. Их основная забота~--- познакомить новых смертных жителей с почти бесконечным рядом новых ситуаций и с незнакомыми обычаями. Несмотря на всю длительную подготовку и долгое путешествие к нему, Рай всё же остаётся невыразимо удивительным и неожиданно новым для тех, кто наконец обретает статус постоянного жителя.
\usection{ХРАНИТЕЛИ ЗНАНИЙ}
\vs p027 5:1 Супернафические хранители знаний~--- это высшие <<живые послания>>, известные и читаемые всеми, кто обитает на Рае. Это божественные записи истины, живые книги настоящего знания. Ты слышал о записях в <<книге жизни>>. Хранители знаний~--- именно такие живые книги, записи совершенства, запечатлённые на вечных скрижалях божественной жизни и верховного поручительства. Они реальные живые автоматические библиотеки. Факты вселенных неотъемлемо присутствуют в этих первичных супернафимах, буквально записаны в этих ангелах; и также сама суть этих совершенных и наполненных хранилищ истины вечности и информации времени не позволяет неправде поселиться в их разумах.
\vs p027 5:2 Эти хранители проводят неформальные курсы обучения жителей вечного Острова, но их главная функция~--- это справка и проверка. Любой проживающий на Рае может по желанию иметь рядом с собой живое хранилище конкретного факта или истины, которые он желает узнать. На северной оконечности Острова есть живые искатели знаний, которые укажут руководителя группы, хранящей искомую информацию; и тут же появятся блистательные существа, которые и \bibemph{есть} именно то, чт\'о ты желаешь знать. Тебе больше не нужно искать просвещения из написанных страниц; теперь ты общаешься с живой информацией лицом к лицу. Так ты обретаешь высшее знание от живых существ, которые являются его окончательными хранителями.
\vs p027 5:3 Когда ты найдёшь супернафима, который является именно тем, чт\'о ты хочешь выяснить, тебе станут доступными \bibemph{все} известные факты всех вселенных, ибо эти хранители знаний представляют собой окончательные и живые конспекты обширной сети ангелов\hyp{}писцов, начиная от серафимов и секонафимов локальных и сверхвселенных до главных писцов третичных супернафимов в Хавоне. И эта живая аккумуляция знаний отличается от официальных записей Рая, совокупного резюме вселенской истории.
\vs p027 5:4 Мудрость истины происходит из божественности центральной вселенной, но знание, эмпирическое знание, в значительной степени берёт своё начало в областях времени и пространства~--- отсюда необходимость поддержания обширных сверхвселенских организаций серафимов\hyp{}писцов и супернафимов, поддерживаемых Небесными Писцами.
\vs p027 5:5 Эти первичные супернафимы, которые от природы обладают вселенским знанием, также ответственны за его организацию и классификацию. Представляя собой живую справочную библиотеку вселенной вселенных, они классифицировали знания в семь больших разделов, каждый из которых объединяет около миллиона подразделов. Лёгкость, с которой жители Рая могут пользоваться этим обширным хранилищем знаний, единственно обязана добровольным и мудрым усилиям хранителей знаний. Хранители являются также возвышенными учителями центральной вселенной, щедро раздающими свои живые сокровища всем существам на любом из контуров Хавоны, и они широко, хотя и опосредованно, используются судами От Века Древних. Но эта живая библиотека, доступная для центральной и сверхвселенных, недоступна для локальных творений. Только косвенно и отражательно извлекается польза из Райских знаний в локальных вселенных.
\usection{МАСТЕРА ФИЛОСОФИИ}
\vs p027 6:1 Верховному удовлетворению от поклонения уступает лишь радость от занятий философией. На какие бы высоты ты не поднялся и как бы далеко не продвинулся, всегда остаются тысячи загадок, которые требуют применения философии в попытке найти их решение.
\vs p027 6:2 Мастера философии Рая с удовольствием направляют разумы его жителей, как исконных, так и восходящих, к воодушевляющему поиску решения вселенских проблем. Эти супернафические мастера философии~--- <<небесные мудрецы>>, существа мудрости, которые используют истину знания и факты опыта в своём стремлении постичь неизвестное. Они превращают знание в истину, а опыт~--- в мудрость. На Рае восходящие личности пространства достигают высот бытия: они обладают знанием; они знают истину; они могут философствовать~--- думать истину; они могут даже стремиться охватить концепции Предельного и попытаться постичь методы Абсолютов.
\vs p027 6:3 На южной оконечности обширных владений Рая мастера философии проводят углублённые курсы по семидесяти функциональным разделам мудрости. Здесь они рассуждают о планах и целях Бесконечности и стремятся согласовать опыт и систематизировать знания всех, кому доступна их мудрость. Они разработали в высшей степени специализированные подходы к различным вселенским проблемам, но их окончательные выводы всегда оказываются в полном согласии.
\vs p027 6:4 Эти Райские философы учат всеми возможными методами обучения, включая высший метод графов Хавоны и определённые Райские методы передачи информации. Все эти высшие приёмы сообщения знаний и передачи идей полностью выходят за рамки возможностей понимания даже самого высокоразвитого человеческого разума. Час обучения на Рае был бы эквивалентом 10\,000 лет использования методов запоминания слов, существующих на Урантии. Ты не можешь понять такие способы коммуникации, а в смертном опыте просто нет ничего, с чем их можно было бы сравнить и чему их можно было бы уподобить.
\vs p027 6:5 Мастера философии получают огромное удовольствие, делясь своей интерпретацией вселенной вселенных с существами, взошедшими из миров пространства. И хотя выводы философии никогда не могут быть столь же окончательны, как факты знания и истины опыта, тем не менее, услышав рассуждения этих первичных супернафимов о нерешённых проблемах вечности и действиях Абсолютов, ты испытаешь определённое и длительное удовлетворение в связи с этими неразрешёнными вопросами.
\vs p027 6:6 Эти интеллектуальные занятия Рая не транслируются; философия совершенства доступна только тем, кто присутствует лично. Окружающие творения знают об этих учениях только от тех, кто прошёл через этот опыт и кто впоследствии донёс эту мудрость до вселенных пространства.
\usection{РУКОВОДИТЕЛИ ПОКЛОНЕНИЯ}
\vs p027 7:1 Поклонение~--- высочайшая привилегия и первейшая обязанность всех созданных разумных существ. Поклонение~--- это сознательный и радостный акт осознания и признания истины и факта сокровенных и личных отношений Создателей со своими созданиями. Качество поклонения определяется глубиной восприятия создания; и по мере того как познание бесконечного характера Богов прогрессирует, акт поклонения становится всё более всеобъемлющим, пока в итоге не достигнет триумфа высочайшего эмпирического наслаждения и самого утончённого удовольствия, известного созданным существам.
\vs p027 7:2 \pc Хотя на Острове Рай есть определённые места для поклонения, он больше напоминает одно огромное святилище для богослужения. Поклонение~--- это первая и доминирующая страсть всех, кто добирается до его благословенных берегов,~--- спонтанный взрыв эмоций существ, которые узнали о Боге достаточно, чтобы достичь его присутствия. Круг за кругом во время путешествия внутрь через Хавону поклонение становится всё возрастающей страстью, пока на Рае не возникает необходимость направлять или иным образом контролировать его выражение.
\vs p027 7:3 Периодические, спонтанные, групповые и другие особые вспышки верховного поклонения и духовного прославления, которыми наслаждаются на Рае, проводятся под руководством особого корпуса первичных супернафимов. Под управлением этих руководителей поклонения такое почитание достигает цели высшего удовольствия создания и поднимается к высотам совершенства возвышенного самовыражения и личного наслаждения. Все первичные супернафимы стремятся быть руководителями поклонения; и все восходящие существа с радостью оставались бы вечно в состоянии поклонения, если бы главы назначений периодически не распускали эти собрания. Но ни от одного восходящего существа никогда не потребуется приступить к выполнению заданий вечного служения, пока оно не достигнет полного удовлетворения в поклонении.
\vs p027 7:4 \pc Задача руководителей поклонения~--- научить восходящие создания поклоняться так, чтобы они были способны получать удовлетворение от самовыражения и в то же время уделять внимание необходимой деятельности в соответствии с Райским режимом. Без усовершенствования метода поклонения среднему смертному, достигшему Рая, потребовались бы сотни лет, чтобы полностью и удовлетворительно выразить свои эмоции разумной признательности и благодарности восходящего. Руководители поклонения открывают новые и ранее неизвестные пути выражения так, чтобы эти чудесные дети лона пространства и родовых мук времени были в состоянии получать полное удовлетворение от поклонения за значительно более короткий срок.
\vs p027 7:5 Всё искусство всех существ целой вселенной, способное усилить и повысить возможности самовыражения и проявление признательности, используется максимальным образом в поклонении Райским Божествам. \bibemph{Поклонение~--- высочайшая радость Райского существования;} это освежающая Райская игра. Поклонение повлияет на ваши усовершенствованные души на Рае так же, как игра влияет на ваши изнурённые умы на земле. Способ поклонения на Рае находится далеко за пределами понимания смертных, но дух его ты можешь начать ценить даже здесь, на Урантии, ибо духи Богов даже сейчас пребывают в тебе, парят над тобой и вдохновляют тебя на истинное поклонение.
\vs p027 7:6 На Рае существуют назначенные сроки и места для поклонения, но их недостаточно, чтобы справиться с постоянно увеличивающимся потоком духовных эмоций растущего разума и расширяющегося признания божественности блистательными существами эмпирического восхождения к вечному Острову. Никогда со времён Грандфанды супернафимам не удавалось полностью удовлетворить дух поклонения на Рае. Всегда существует избыток поклонения в сравнении с его предварительной оценкой. И это происходит потому, что личности врождённого совершенства никогда не могут полностью оценить потрясающие реакции духовных эмоций существ, которые медленно и с трудом поднялись к славе Рая из глубин духовной темноты низших миров времени и пространства. Когда такие ангелы и смертные времени достигают присутствия Сил Рая, происходит выражение накопленных веками эмоций, зрелище, поражающее ангелов Рая и влекущее за собой верховную радость божественного удовлетворения в Райских Божествах.
\vs p027 7:7 Иногда весь Рай захлёстывает всепоглощающая волна духовного выражения, исполненного поклонения. Часто руководители поклонения не могут контролировать такие явления до тех пор, пока в обители Божеств не появится тройное колебание света, означающее, что божественное сердце Богов полностью и безраздельно удовлетворено искренним поклонением жителей Рая, совершенных граждан славы и восходящих созданий времени. Какое торжество метода! Какое исполнение вечного плана и замысла Богов, что разумная любовь ребёнка\hyp{}создания полностью удовлетворила безграничную любовь Отца\hyp{}Создателя!
\vs p027 7:8 \pc После достижения верховного удовлетворения от полноты поклонения, ты удостаиваешься права быть принятым в Корпус Завершения. Восходящий путь почти завершён, и готовится празднование седьмого юбилея. Первый юбилей ознаменовал собой соглашение смертного с Настройщиком Мыслей, когда была утверждена цель выжить; вторым было пробуждение в моронтийной жизни; третьим было слияние с Настройщиком Мыслей; четвёртым было пробуждение в Хавоне; пятый отпраздновал обретение Всеобщего Отца; шестой юбилей стал событием Райского пробуждения от последнего переходного сна времени. Седьмой юбилей знаменует вступление в корпус смертных завершителей и начало служения в вечности. Достижение седьмой стадии реализации духа завершителем, вероятно, будет означать празднование первого из юбилеев вечности.
\vs p027 7:9 \pc На этом заканчивается рассказ о супернафимах Рая, высочайшей категории духов\hyp{}помощников, о тех существах, которые, как вселенский класс, всегда сопровождают тебя от мира твоего происхождения до тех пор, пока руководители поклонения окончательно не попрощаются с тобой, когда ты дашь Троичную клятву вечности и будешь принят в Смертный Корпус Завершения.
\vs p027 7:10 Нескончаемое служение Райской Троице вот\hyp{}вот начнётся; и теперь завершитель стоит лицом к лицу с новой задачей~--- Богом Предельным.
\vsetoff
\vs p027 7:11 [Представлено Совершенствователем Мудрости из Уверсы.]
\quizlink
