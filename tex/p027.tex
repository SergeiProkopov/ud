\upaper{27}{СЛУЖЕНИЕ ПЕРВИЧНЫХ СУПЕРНАФИМОВ}
\uminitoc{ИНИЦИАТОРЫ ПОКОЯ}
\uminitoc{ГЛАВЫ НАЗНАЧЕНИЙ}
\uminitoc{ИНТЕРПРЕТАТОРЫ ЭТИКИ}
\uminitoc{РУКОВОДИТЕЛИ ПОВЕДЕНИЯ}
\uminitoc{ХРАНИТЕЛИ ЗНАНИЙ}
\uminitoc{МАСТЕРА ФИЛОСОФИИ}
\uminitoc{РУКОВОДИТЕЛИ ПОКЛОНЕНИЯ}
\author{Совершенствователь Мудрости}
\vs p027 0:1 Первичные супернафимы~--- это возвышенные слуги Божеств на вечном Острове Рай. Никогда они ещё не отклонялись от путей света и праведности. Их численность окончательна и неизменна; от вечности ни один из этого великолепного воинства не был потерян. Эти высокие супернафимы~--- совершенные существа, верховные в совершенстве, но они не абсонитны, как не являются они и абсолютными. Будучи самой сутью совершенства, эти дети Бесконечного Духа трудятся взаимозаменяемо и по собственной воле во всех фазах своих разнообразных обязанностей. Хотя они не действуют широко за пределами Рая, но участвуют в различных тысячелетних собраниях и групповых встречах центральной вселенной. Они также выступают в качестве специальных посланников Божеств, и многие из них поднимаются до статуса Технических Консультантов.
\vs p027 0:2 Первичные супернафимы также ставятся во главе серафических воинств, которые служат мирам, изолированным из\hyp{}за восстания. После того как Райский Сын посвящает себя такому миру, завершает свою миссию, восходит ко Всеобщему Отцу, принимается и возвращается как облечённый полномочиями спаситель этого изолированного мира, главы назначений всегда уполномочивают первичного супернафима принять руководство духами\hyp{}помощниками восстановленной сферы. Супернафимы во время этой специальной службы периодически сменяются. На Урантии нынешний <<глава серафимов>> является вторым представителем этой категории, исполняющим обязанности со времён посвящения Христа Михаила.
\vs p027 0:3 От вечности первичные супернафимы служили на Острове Света и отправлялись с миссиями руководства к мирам пространства, но текущая классификация их функций существует только с момента прибытия на Рай пилигримов времени из Хавоны. Сейчас эти высокие ангелы служат главным образом в следующих семи категориях:
\vs p027 0:4 \li{1.}Руководители поклонения.
\vs p027 0:5 \li{2.}Мастера философии.
\vs p027 0:6 \li{3.}Хранители знаний.
\vs p027 0:7 \li{4.}Руководители поведения.
\vs p027 0:8 \li{5.}Интерпретаторы этики.
\vs p027 0:9 \li{6.}Главы назначений.
\vs p027 0:10 \li{7.}Инициаторы покоя.
\vs p027 0:11 Только после того как восходящие пилигримы достигают статуса постоянных жителей Рая, они попадают под непосредственное влияние этих супернафимов, проходя затем обучение под руководством этих ангелов в последовательности, обратной вышеуказанной. То есть ты вступишь на свой Райский путь под опекой инициаторов покоя, и после ряда этапов, на каждом из которых тебя будет обучать представитель промежуточной категории, ты закончишь этот период подготовки с руководителями поклонения. И тогда ты будешь готов начать свой нескончаемый путь завершителя.
\usection{ИНИЦИАТОРЫ ПОКОЯ}
\vs p027 1:1 Инициаторы покоя~--- это инспекторы Рая, которые отправляются с центрального Острова к внутреннему контуру Хавоны, чтобы сотрудничать там со своими коллегами, дополнениями покоя из второй категории супернафимов. Одна из существенных составляющих для наслаждения Раем,~--- это покой, божественный отдых; и эти инициаторы покоя~--- последние наставники, которые готовят пилигримов времени к вступлению в вечность. Они начинают свою работу на последнем круге достижения центральной вселенной и продолжают её, когда пилигрим пробуждается от последнего сна перехода, дремоты, которая выпускает создание пространства в царство вечности.
\vs p027 1:2 \pc Покой имеет семичастную природу: есть отдых сна и досуга у низших форм жизни, отдых открытия~--- у более высоких существ, и отдых поклонения~--- у высшего типа личности духа. Есть также обычный отдых для получения энергии, перезарядка существ физической или духовной энергией. И ещё существует сон перехода, бессознательный сон в объятиях серафима при перемещении с одной сферы на другую. Полностью отличается от всех этих форм глубокий сон метаморфозы, отдых перехода от одной стадии бытия к другой, от одной жизни к другой, от одного состояния существования к другому, сон, который всегда сопутствует переходу к новому вселенскому \bibemph{статусу,} в отличие от эволюции через различные \bibemph{стадии} какого\hyp{}либо одного статуса.
\vs p027 1:3 Однако последний сон метаморфозы~--- нечто большее, чем те предыдущие сны перехода, которые отмечали последовательные достижения следующего статуса на пути восхождения; с его помощью создания времени и пространства пересекают внутренние границы временн\'ого и пространственного, чтобы достичь статуса постоянного жительства во вневременн\'ых и внепространственных обителях Рая. Инициаторы покоя и дополнения покоя точно так же необходимы для этой трансцендентной метаморфозы, как серафимы и связанные с ними существа для выживания смертного существа после смерти.
\vs p027 1:4 \pc Ты погрузишься в покой на последнем контуре Хавоны и воскреснешь в вечности на Рае. И после духовной реперсонализации ты мгновенно узнаешь в приветствующем тебя на вечных берегах инициаторе покоя того самого первичного супернафима, который погрузил тебя в последний сон на внутреннем контуре Хавоны; и вспомнишь своё последнее великое усилие веры, когда ты вновь был готов вверить своё существование в руки Всеобщего Отца.
\vs p027 1:5 Ты насладился последним покоем времени; испытал последний сон перехода; теперь ты пробуждаешься к жизни вечной на берегах вечной обители. <<И не будет больше сна. Присутствие Бога и его Сына стоит перед вами, и вы слуги его навечно; ты увидел его лицо, и имя его~--- твой дух. Там не будет ночи; и они не нуждаются в свете солнца, ибо Великий Источник и Центр даёт им свет; они будут жить во веки веков. И Бог отрёт все слёзы с их глаз; не будет больше смерти, ни печали, ни плача, и не будет больше боли, ибо прежнее прошло.>>\fnst{Ср. Откровение\,21:23: <<И город не имеет нужды ни в солнце, ни в луне для освещения своего, ибо слава Божия осветила его, и светильник его~--- Агнец.>>, Откровение\,21:4: <<И отрёт Бог всякую слезу с очей их, и смерти не будет уже; ни плача, ни вопля, ни болезни уже не будет, ибо прежнее прошло.>>}.
\usection{ГЛАВЫ НАЗНАЧЕНИЙ}
\vs p027 2:1 
\vs p027 2:2 
\vs p027 2:3 
\usection{ИНТЕРПРЕТАТОРЫ ЭТИКИ}
\vs p027 3:1 
\vs p027 3:2 
\vs p027 3:3 
\vs p027 3:4 
\usection{РУКОВОДИТЕЛИ ПОВЕДЕНИЯ}
\vs p027 4:1 
\vs p027 4:2 
\vs p027 4:3 
\vs p027 4:4 
\usection{ХРАНИТЕЛИ ЗНАНИЙ}
\vs p027 5:1 
\vs p027 5:2 
\vs p027 5:3 
\vs p027 5:4 
\vs p027 5:5 
\usection{МАСТЕРА ФИЛОСОФИИ}
\vs p027 6:1 
\vs p027 6:2 
\vs p027 6:3 
\vs p027 6:4 
\vs p027 6:5 
\vs p027 6:6 
\usection{РУКОВОДИТЕЛИ ПОКЛОНЕНИЯ}
\vs p027 7:1 
\vs p027 7:2 \pc 
\vs p027 7:3 
\vs p027 7:4 \pc 
\vs p027 7:5 
\vs p027 7:6 
\vs p027 7:7 
\vs p027 7:8 \pc 
\vs p027 7:9 \pc 
\vs p027 7:10 
\vsetoff
\vs p027 7:11 
\quizlink
