\upaper{195}{ПОСЛЕ ПЯТИДЕСЯТНИЦЫ}
\uminitoc{Influence of the Greeks}
\uminitoc{The Roman Influence}
\uminitoc{Under the Roman Empire}
\uminitoc{The European Dark Ages}
\uminitoc{The Modern Problem}
\uminitoc{Materialism}
\uminitoc{The Vulnerability of Materialism}
\uminitoc{Secular Totalitarianism}
\uminitoc{Christianity’s Problem}
\uminitoc{The Future}
\author{Промежуточные создания}
\vs p195 0:1 Результаты проповеди Петра в день Пятидесятницы были таковы, что они определили будущую политику и планы большинства апостолов в их усилиях по провозглашению евангелия царства. Пётр был подлинным основателем христианской церкви; Павел нёс христианскую весть язычникам, а верующие\hyp{}греки разнесли её по всей Римской империи.
\vs p195 0:2 Хотя связанные традициями и порабощённые священниками евреи, как народ, отказались принять как евангелие Иисуса отцовства Бога и братства людей, так и возвещение Петра и Павла о воскресении и вознесении Христа (последующее христианство), остальная часть Римской империи оказалась восприимчивой к развивающимся христианским учениям. Западная цивилизация в то время была интеллектуальной, уставшей от войн и крайне скептически относящейся ко всем существующим религиям и мировоззренческим философиям. Народы западного мира, испытавшие на себе благотворное влияние греческой культуры, чтили традиции своего великого прошлого. Они могли созерцать унаследованные великие достижения в философии, искусстве, литературе и политическом прогрессе. Но при всех этих достижениях у них не было религии, удовлетворяющей душу. Их духовные желания оставались неудовлетворёнными.
\vs p195 0:3 На такой сцене человеческого общества внезапно появились учения Иисуса, облачённые в христианскую весть. Таким образом, голодным сердцам этих западных народов был представлен новый образ жизни. Эта ситуация означала немедленный конфликт между старыми религиозными практиками и новой христианизированной версией послания Иисуса миру. Такой конфликт должен привести либо к решительной победе нового или старого, либо к некоторой степени \bibemph{компромисса}. История показывает, что борьба закончилась компромиссом. Христианство взяло на себя слишком много, чтобы в каком\hyp{}то одном народе оно могло ассимилироваться за одно или два поколения. Оно не было простым духовным призывом, с которым Иисус обращался к душам людей; с самого начала оно заняло решительную позицию по религиозным ритуалам, образованию, магии, медицине, искусству, литературе, закону, управлению, морали, взаимоотношениям полов, полигамии и, в какой\hyp{}то степени, даже рабству. Христианство появилось не просто как новая религия~--- нечто, ожидаемое всей Римской империей и всем Востоком,~--- но как \bibemph{новое устройство человеческого общества}. И, как такое притязание, оно быстро спровоцировало социально\hyp{}моральное столкновение эпох. Идеалы Иисуса, по\hyp{}новому истолкованные греческой философией и социализированные в христианстве, теперь смело бросили вызов традициям человеческого рода, воплощённым в этике, морали и религиях западной цивилизации.
\vs p195 0:4 \pc 
\vs p195 0:5 
\vs p195 0:6 
\vs p195 0:7 
\vs p195 0:8 
\vs p195 0:9 
\vs p195 0:10 
\vs p195 0:11 \pc 
\vs p195 0:12 
\vs p195 0:13 
\vs p195 0:14 
\vs p195 0:15 
\vs p195 0:16 
\vs p195 0:17 
\vs p195 0:18 \pc 
\usection{Influence of the Greeks}
\vs p195 1:1 
\vs p195 1:2 
\vs p195 1:3 
\vs p195 1:4 
\vs p195 1:5 \pc 
\vs p195 1:6 
\vs p195 1:7 \pc 
\vs p195 1:8 
\vs p195 1:9 
\vs p195 1:10 
\vs p195 1:11 \pc 
\usection{The Roman Influence}
\vs p195 2:1 
\vs p195 2:2 
\vs p195 2:3 
\vs p195 2:4 
\vs p195 2:5 
\vs p195 2:6 \pc 
\vs p195 2:7 
\vs p195 2:8 
\vs p195 2:9 
\usection{Under the Roman Empire}
\vs p195 3:1 
\vs p195 3:2 
\vs p195 3:3 
\vs p195 3:4 \pc 
\vs p195 3:5 \pc 
\vs p195 3:6 \pc 
\vs p195 3:7 \pc 
\vs p195 3:8 
\vs p195 3:9 
\vs p195 3:10 
\vs p195 3:11 
\usection{The European Dark Ages}
\vs p195 4:1 
\vs p195 4:2 
\vs p195 4:3 
\vs p195 4:4 \pc 
\vs p195 4:5 \pc 
\usection{The Modern Problem}
\vs p195 5:1 
\vs p195 5:2 
\vs p195 5:3 
\vs p195 5:4 
\vs p195 5:5 
\vs p195 5:6 
\vs p195 5:7 
\vs p195 5:8 \pc 
\vs p195 5:9 
\vs p195 5:10 
\vs p195 5:11 \pc 
\vs p195 5:12 \pc 
\vs p195 5:13 
\vs p195 5:14 \pc 
\usection{Materialism}
\vs p195 6:1 
\vs p195 6:2 
\vs p195 6:3 
\vs p195 6:4 
\vs p195 6:5 
\vs p195 6:6 
\vs p195 6:7 
\vs p195 6:8 
\vs p195 6:9 
\vs p195 6:10 
\vs p195 6:11 
\vs p195 6:12 
\vs p195 6:13 
\vs p195 6:14 \pc 
\vs p195 6:15 
\vs p195 6:16 
\vs p195 6:17 
\usection{The Vulnerability of Materialism}
\vs p195 7:1 
\vs p195 7:2 
\vs p195 7:3 
\vs p195 7:4 
\vs p195 7:5 
\vs p195 7:6 
\vs p195 7:7 
\vs p195 7:8 
\vs p195 7:9 
\vs p195 7:10 
\vs p195 7:11 
\vs p195 7:12 
\vs p195 7:13 \pc 
\vs p195 7:14 \pc 
\vs p195 7:15 
\vs p195 7:16 
\vs p195 7:17 \pc 
\vs p195 7:18 \pc 
\vs p195 7:19 \pc 
\vs p195 7:20 \pc 
\vs p195 7:21 
\vs p195 7:22 
\vs p195 7:23 
\usection{Secular Totalitarianism}
\vs p195 8:1 
\vs p195 8:2 
\vs p195 8:3 
\vs p195 8:4 
\vs p195 8:5 \pc 
\vs p195 8:6 \pc 
\vs p195 8:7 
\vs p195 8:8 
\vs p195 8:9 
\vs p195 8:10 
\vs p195 8:11 
\vs p195 8:12 
\vs p195 8:13 
\usection{Christianity’s Problem}
\vs p195 9:1 
\vs p195 9:2 
\vs p195 9:3 
\vs p195 9:4 \pc 
\vs p195 9:5 
\vs p195 9:6 \pc 
\vs p195 9:7 
\vs p195 9:8 
\vs p195 9:9 \pc 
\vs p195 9:10 
\vs p195 9:11 
\usection{The Future}
\vs p195 10:1 
\vs p195 10:2 
\vs p195 10:3 \pc 
\vs p195 10:4 
\vs p195 10:5 \pc 
\vs p195 10:6 
\vs p195 10:7 
\vs p195 10:8 \pc 
\vs p195 10:9 
\vs p195 10:10 
\vs p195 10:11 
\vs p195 10:12 
\vs p195 10:13 
\vs p195 10:14 
\vs p195 10:15 
\vs p195 10:16 
\vs p195 10:17 
\vs p195 10:18 
\vs p195 10:19 
\vs p195 10:20 
\vs p195 10:21 
\quizlink
