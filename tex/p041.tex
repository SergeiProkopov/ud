\upaper{41}{ФИЗИЧЕСКИЕ АСПЕКТЫ ЛОКАЛЬНОЙ ВСЕЛЕННОЙ}
\uminitoc{ЦЕНТРЫ МОЩИ НЕБАДОНА}
\uminitoc{ФИЗИЧЕСКИЕ РЕГУЛЯТОРЫ САТАНИИ}
\uminitoc{НАШИ ЗВЁЗДНЫЕ СОСЕДИ}
\uminitoc{ПЛОТНОСТЬ СОЛНЦА}
\uminitoc{СОЛНЕЧНОЕ ИЗЛУЧЕНИЕ}
\uminitoc{КАЛЬЦИЙ --- КОСМИЧЕСКИЙ СТРАННИК}
\uminitoc{ИСТОЧНИКИ СОЛНЕЧНОЙ ЭНЕРГИИ}
\uminitoc{СОЛНЕЧНО-ЭНЕРГЕТИЧЕСКИЕ РЕАКЦИИ}
\uminitoc{СТАБИЛЬНОСТЬ СОЛНЦА}
\uminitoc{ПРОИСХОЖДЕНИЕ ОБИТАЕМЫХ МИРОВ}
\author{Архангел}
\vs p041 0:1 Характерным пространственным феноменом, отделяющим каждое локальное творение от всех остальных, является присутствие Созидательного Духа. Весь Небадон совершенно пронизан пространственным присутствием Божественной Служительницы Спасограда, и это присутствие так же определённо исчезает у внешних границ нашей локальной вселенной. То, что пронизано Материнским Духом нашей локальной вселенной, \bibemph{есть} Небадон; то, что выходит за пределы его пространственного присутствия, находится за пределами Небадона, являясь вненебадонскими пространственными областями сверхвселенной Орвонтон~--- другими локальными вселенными.
\vs p041 0:2 \pc Хотя административная организация большой вселенной раскрывает чёткое разделение между правительствами центральной, сверх- и локальной вселенных, и хотя эти деления астрономически соответствуют пространственному разграничению Хавоны и семи сверхвселенных, между локальными творениями таких чётких линий физической демаркации не существует. Даже большие и малые сектора Орвонтона (для нас) чётко различимы, но определить физические границы локальных вселенных не так-то просто. Это связано с тем, что эти локальные творения административно организованы в соответствии с определёнными \bibemph{созидательными} принципами, определяющими сегментацию общего энергетического заряда сверхвселенной, тогда как их физические компоненты, сферы пространства~--- солнца, тёмные острова, планеты и т.\,д.~--- происходят преимущественно из туманностей, а те астрономически появляются в соответствии с определёнными \bibemph{предсозидательными} (трансцендентными) планами Зодчих Главной Вселенной.
\vs p041 0:3 Одна или несколько, и даже много, таких туманностей могут быть заключены в пределах одной локальной вселенной. Так, Небадон был физически образован из звёздного и планетарного потомства Андроновера и других туманностей. Сферы Небадона происходят от различных туманностей, однако все они обладали определённой минимальной общностью пространственного движения, которое было скорректировано разумными усилиями управляющих мощью таким образом, чтобы создать наше нынешнее скопление пространственных тел, движущихся как одно целое по орбитам сверхвселенной.
\vs p041 0:4 Таково строение локального звёздного облака Небадон, которое сегодня обращается по всё более устойчивой орбите вокруг находящегося в Стрельце центра того малого сектора Орвонтона, к которому принадлежит наше локальное творение.
\usection{ЦЕНТРЫ МОЩИ НЕБАДОНА}
\vs p041 1:1 Начало спиральным и другим туманностям~--- материнским дискам сфер пространства~--- кладут Райские организаторы силы; и вслед за эволюцией небулярной гравитационной реакции их функции передаются центрами мощи и физическим регуляторам, которые с этого момента берут на себя полную ответственность за управление физической эволюцией последующих поколений звёздного и планетарного потомства. Этот физический надзор за предвселенной Небадон после прибытия нашего Сына\hyp{}Создателя был сразу же скоординирован с его планом организации вселенной. Во владениях этого Райского Сына Бога Верховные Центры Мощи и Главные Физические Регуляторы сотрудничали с появившимися позже Управляющими Моронтийной Мощью и другими, с целью создания того обширного комплекса линий связи, энергетических контуров и магистралей мощи, которые прочно связывают разнообразные пространстшенные тела Небадона в единую интегрированную административную единицу.
\vs p041 1:2 
\vs p041 1:3 
\vs p041 1:4 
\vs p041 1:5 
\usection{ФИЗИЧЕСКИЕ РЕГУЛЯТОРЫ САТАНИИ}
\vs p041 2:1 
\vs p041 2:2 
\vs p041 2:3 \pc 
\vs p041 2:4 
\vs p041 2:5 
\vs p041 2:6 \pc 
\vs p041 2:7 \pc 
\vs p041 2:8 
\usection{НАШИ ЗВЁЗДНЫЕ СОСЕДИ}
\vs p041 3:1 
\vs p041 3:2 
\vs p041 3:3 \pc 
\vs p041 3:4 
\vs p041 3:5 \pc 
\vs p041 3:6 \pc 
\vs p041 3:7 
\vs p041 3:8 \pc 
\vs p041 3:9 
\vs p041 3:10 
\usection{ПЛОТНОСТЬ СОЛНЦА}
\vs p041 4:1 
\vs p041 4:2 \pc 
\vs p041 4:3 \pc 
\vs p041 4:4 
\vs p041 4:5 
\vs p041 4:6 
\vs p041 4:7 
\usection{СОЛНЕЧНОЕ ИЗЛУЧЕНИЕ}
\vs p041 5:1 
\vs p041 5:2 
\vs p041 5:3 
\vs p041 5:4 
\vs p041 5:5 
\vs p041 5:6 \pc 
\vs p041 5:7 \pc 
\vs p041 5:8 
\usection{КАЛЬЦИЙ --- КОСМИЧЕСКИЙ СТРАННИК}
\vs p041 6:1 
\vs p041 6:2 
\vs p041 6:3 \pc 
\vs p041 6:4 
\vs p041 6:5 
\vs p041 6:6 \pc 
\vs p041 6:7 \pc 
\usection{ИСТОЧНИКИ СОЛНЕЧНОЙ ЭНЕРГИИ}
\vs p041 7:1 
\vs p041 7:2 
\vs p041 7:3 \pc 
\vs p041 7:4 
\vs p041 7:5 
\vs p041 7:6 
\vs p041 7:7 
\vs p041 7:8 
\vs p041 7:9 
\vs p041 7:10 
\vs p041 7:11 \pc 
\vs p041 7:12 \pc 
\vs p041 7:13 
\vs p041 7:14 \pc 
\vs p041 7:15 
\usection{СОЛНЕЧНО-ЭНЕРГЕТИЧЕСКИЕ РЕАКЦИИ}
\vs p041 8:1 
\vs p041 8:2 \pc 
\vs p041 8:3 \pc 
\vs p041 8:4 
\usection{СТАБИЛЬНОСТЬ СОЛНЦА}
\vs p041 9:1 
\vs p041 9:2 \pc 
\vs p041 9:3 \pc 
\vs p041 9:4 
\vs p041 9:5 
\usection{ПРОИСХОЖДЕНИЕ ОБИТАЕМЫХ МИРОВ}
\vs p041 10:1 
\vs p041 10:2 
\vs p041 10:3 \pc 
\vs p041 10:4 
\vs p041 10:5 \pc 
\vsetoff
\vs p041 10:6 
\quizlink
\begin{thebibliography}{100}
\bibitem{Eddington1}
A.S.~Eddington
{``Stars and Atoms''}
{\em Oxford: Clarendon Press}, 1927.
\bibitem{Jeans1}
J.~Jeans.
{``Through Space and Time''}
{\em Cambridge: University Press}, 1934.
\end{thebibliography}
