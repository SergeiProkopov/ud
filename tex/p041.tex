\upaper{41}{ФИЗИЧЕСКИЕ АСПЕКТЫ ЛОКАЛЬНОЙ ВСЕЛЕННОЙ}
\uminitoc{ЦЕНТРЫ МОЩИ НЕБАДОНА}
\uminitoc{ФИЗИЧЕСКИЕ РЕГУЛЯТОРЫ САТАНИИ}
\uminitoc{НАШИ ЗВЁЗДНЫЕ СОСЕДИ}
\uminitoc{ПЛОТНОСТЬ СОЛНЦА}
\uminitoc{СОЛНЕЧНОЕ ИЗЛУЧЕНИЕ}
\uminitoc{КАЛЬЦИЙ --- КОСМИЧЕСКИЙ СТРАННИК}
\uminitoc{ИСТОЧНИКИ СОЛНЕЧНОЙ ЭНЕРГИИ}
\uminitoc{СОЛНЕЧНО-ЭНЕРГЕТИЧЕСКИЕ РЕАКЦИИ}
\uminitoc{СТАБИЛЬНОСТЬ СОЛНЦА}
\uminitoc{ПРОИСХОЖДЕНИЕ ОБИТАЕМЫХ МИРОВ}
\author{Архангел}
\vs p041 0:1 Характерным пространственным феноменом, отделяющим каждое локальное творение от всех остальных, является присутствие Созидательного Духа. Весь Небадон совершенно пронизан пространственным присутствием Божественной Служительницы Спасограда, и это присутствие так же определённо исчезает у внешних границ нашей локальной вселенной. То, что пронизано Материнским Духом нашей локальной вселенной, \bibemph{есть} Небадон; то, что выходит за пределы его пространственного присутствия, находится за пределами Небадона, являясь вненебадонскими пространственными областями сверхвселенной Орвонтон~--- другими локальными вселенными.
\vs p041 0:2 \pc Хотя административная организация большой вселенной раскрывает чёткое разделение между правительствами центральной, сверх- и локальной вселенных, и хотя эти деления астрономически соответствуют пространственному разграничению Хавоны и семи сверхвселенных, между локальными творениями таких чётких линий физической демаркации не существует. Даже большие и малые сектора Орвонтона (для нас) чётко различимы, но определить физические границы локальных вселенных не так-то просто. Это связано с тем, что эти локальные творения административно организованы в соответствии с определёнными \bibemph{созидательными} принципами, определяющими сегментацию общего энергетического заряда сверхвселенной, тогда как их физические компоненты, сферы пространства~--- солнца, тёмные острова, планеты и т.\,д.~--- происходят преимущественно из туманностей, а те астрономически появляются в соответствии с определёнными \bibemph{предсозидательными} (трансцендентными) планами Зодчих Главной Вселенной.
\vs p041 0:3 Одна или несколько, и даже много, таких туманностей могут быть заключены в пределах одной локальной вселенной. Так, Небадон был физически образован из звёздного и планетарного потомства Андроновера и других туманностей. Сферы Небадона происходят от различных туманностей, однако все они обладали определённой минимальной общностью пространственного движения, которое было скорректировано разумными усилиями управляющих мощью таким образом, чтобы создать наше нынешнее скопление пространственных тел, движущихся как одно целое по орбитам сверхвселенной.
\vs p041 0:4 Таково строение локального звёздного облака Небадон, которое сегодня обращается по всё более устойчивой орбите вокруг находящегося в Стрельце центра того малого сектора Орвонтона, к которому принадлежит наше локальное творение.
\usection{ЦЕНТРЫ МОЩИ НЕБАДОНА}
\vs p041 1:1 Начало спиральным и другим туманностям~--- материнским дискам сфер пространства~--- кладут Райские организаторы силы; и вслед за эволюцией небулярной гравитационной реакции их функции передаются центрам мощи и физическим регуляторам, которые с этого момента берут на себя полную ответственность за управление физической эволюцией последующих поколений звёздного и планетарного потомства. Этот физический надзор за предвселенной Небадон после прибытия нашего Сына Создателя был сразу же скоординирован с его планом организации вселенной. Во владениях этого Райского Сына Бога Верховные Центры Мощи и Главные Физические Регуляторы сотрудничали с появившимися позже Управляющими Моронтийной Мощью и другими, с целью создания того обширного комплекса линий связи, энергетических контуров и магистралей мощи, которые прочно связывают разнообразные пространственные тела Небадона в единую интегрированную административную единицу.
\vs p041 1:2 Сто Верховных Центров Мощи четвёртой категории постоянно закреплены за нашей локальной вселенной. Эти существа принимают входящие линии энергии из центров третьей категории Уверсы и передают пониженные и модифицированные контуры центрам мощи наших созвездий и систем. В совокупности эти центры мощи образуют живую систему контроля и выравнивания, которая работает для поддержания баланса и распределения энергий, которые иначе были бы флуктуирующими и переменными. Однако центры мощи не связаны с кратковременными и локальными энергетическими потрясениями, такими как солнечные пятна и системные электрические возмущения; свет и электричество не являются основными энергиями пространства; они являются вторичными и побочными проявлениями.
\vs p041 1:3 Сто центров локальной вселенной размещены на Спасограде, где они функционируют в самом энергетическом центре этой сферы. Архитектурные сферы, такие как Спасоград, Эденция и Иерусем, освещаются, обогреваются и снабжаются энергией методами, которые делают их совершенно независимыми от солнц пространства. Эти сферы были сконструированы~--- созданы по плану~--- центрами мощи и физическими регуляторами таким образом, чтобы оказывать мощное влияние на распределение энергии. Сосредоточивая свою деятельность на таких фокусных точках энергетического контроля, центры мощи своим живым присутствием направляют и перераспределяют физические энергии пространства. И эти энергетические контуры лежат в основе всех физико\hyp{}материальных и моронтийно\hyp{}духовных явлений.
\vs p041 1:4 Каждому из первичных подразделений Небадона~--- ста созвездий~--- назначено по десять Верховных Центров Мощи пятой категории. В вашем созвездии Норлатиадек они расположены не на столичной сфере, а в центре огромной звёздной системы, образующей физическое ядро созвездия. На Эденции расположено десять соответствующих механических регуляторов и десять франдаланков, которые находятся в идеальной и постоянной связи с ближайшими центрами мощи.
\vs p041 1:5 Точно в фокусе гравитации каждой локальной системы находится по одному Верховному Центру Мощи шестой категории. В системе Сатания назначенный центр мощи занимает тёмный остров пространства, расположенный в астрономическом центре системы. Многие из этих тёмных островов представляют собой огромные генераторы [dynamos], мобилизующие и направляющие определённые энергии пространства, и эти естественные обстоятельства эффективно используются Центром Мощи Сатании, чья живая масса функционирует в качестве связи с высшими центрами, направляя потоки более материализованной мощи Главным Физическим Регуляторам на эволюционных планетах пространства.
\usection{ФИЗИЧЕСКИЕ РЕГУЛЯТОРЫ САТАНИИ}
\vs p041 2:1 Хотя Главные Физические Регуляторы служат вместе с центрами мощи по всей большой вселенной, их функции в локальной системе, такой как Сатания, более просты для понимания. Сатания является одной из 100 локальных систем, составляющих административную организацию созвездия Норлатиадек, а её непосредственными соседями являются системы Сандматия, Ассунция, Порогия, Сортория, Рантулия и Глантония. Системы Норлатиадека во многих отношениях отличаются друг от друга, но все они являются эволюционными и развивающимися, в чём весьма схожи с Сатанией.
\vs p041 2:2 Сама Сатания состоит из более чем 7\,000 астрономических групп или физических систем, лишь немногие из которых имеют происхождение, подобное вашей Солнечной системе. Астрономический центр Сатании представляет собой огромный тёмный остров пространства, расположенный вместе с сопровождающими его сферами недалеко от центра правительства системы.
\vs p041 2:3 \pc Если не считать присутствия назначенного центра мощи, управление всей физико\hyp{}энергетической системой Сатании сосредоточено на Иерусеме. Главный Физический Регулятор, размещённый на этой столичной сфере, работает согласованно с центром мощи системы, выступая в качестве начальника связи инспекторов мощи, имеющих свой центр на Иерусеме и действующих по всей локальной системе.
\vs p041 2:4 Заключением в контуры и перераспределением [channelizing] энергии руководят 500\,000 живых и разумных манипуляторов энергии, разбросанных по всей Сатании. Благодаря действию таких физических регуляторов, управляющие центры мощи осуществляют полный и совершенный контроль над большинством основных энергий пространства, в том числе излучений сильно нагретых сфер и тёмных энерго\hyp{}заряженных сфер. Эта группа живых существ может мобилизовывать, преобразовывать, превращать, манипулировать и передавать почти все виды физической энергии организованного пространства.
\vs p041 2:5 Жизни присуща способность к мобилизации и трансмутации всеобщей энергии. Тебе знаком механизм растительной жизни, посредством которого материальная энергия света преобразуется в разнообразные проявления растительного царства. Ты также знаешь кое-что о методе, посредством которого эта растительная энергия может быть преобразована в явления животной деятельности, но ты практически ничего не знаешь о методе управляющих мощью и физических регуляторов, наделённых способностью мобилазиции, преобразования, направления и концентрации разнообразных видов энергии пространства.
\vs p041 2:6 \pc Эти существа энергетического царства не имеют прямого отношения к энергии как составляющему фактору живых созданий, даже к области физиологической химии. Иногда они занимаются физическими предпосылками жизни, разработкой тех энергетических систем, которые могут служить физическими проводниками жизненных энергий элементарных материальных организмов. В некотором смысле физические регуляторы связаны с предшествующими жизни проявлениями материальной энергии, так же как адъютанты разумо\hyp{}духи связаны с преддуховными функциями материального разума.
\vs p041 2:7 \pc Эти разумные создания, связанные с контролем мощи и управлением энергии, должны приспосабливать свои методы на каждой сфере в соответствии с физическим строением и архитектурой данной планеты. Они неизменно используют расчёты и выводы своего штата физиков и других технических консультантов относительно локального влияния сильно нагретых солнц и других типов сверхзаряженных звёзд. Необходимо также считаться с огромными холодными и тёмными гигантами пространства и роящимися облаками звёздной пыли; вся эта материя связана с практическими проблемами манипулирования энергией.
\vs p041 2:8 Управление мощью\hyp{}энергией эволюционных обитаемых миров является обязанностью Главных Физических Регуляторов, но эти существа не несут ответственности за все энергетические сбои на Урантии. Существует целый ряд причин для таких нарушений, некоторые из которых находятся вне области и контроля физических хранителей. Урантия находится на линиях колоссальных энергий,~--- маленькая планета в контуре чудовищных масс, и локальные регуляторы иногда привлекают огромное число существ своей категории в усилии уравновесить эти линии энергии. Они довольно хорошо справляются с физическими контурами Сатании, однако сталкиваются с трудностями, изолируя от мощных токов Норлатиадека.
\usection{НАШИ ЗВЁЗДНЫЕ СОСЕДИ}
\vs p041 3:1 Более 2\,000 сверкающих солнц\fnst{Сфера радиусом 65 световых лет с центром в Урантии содержит около 2\,000 звезд. Если предположить однородную плотность звёзд по всей Сатании, то можно сделать вывод, что размер Сатании не может значительно превышать 130 световых лет.} изливают свет и энергию в Сатании, и ваше собственное солнце представляет собой обычный пылающий шар. Из тридцати ближайших солнц только три ярче вашего\fnst{Тридцать ближайших звезд находятся в сфере радиусом 12,6 световых лет, а три звезды ярче нашей следующие: $\alpha$~Центавра (4,35 св. лет), Сириус (8,57 св. лет) и Процион (11,4 св. лет). Это те самые три звезды, которые упомянты в книге Дж.~Джинса \cite{Jeans1}, являющейся одним из человеческих источников данного документа Откровения.}. Вселенские Управляющие Мощью инициируют специальные потоки энергии, протекающие между отдельными звёздами и соответствующими системами. Эти солнечные печи вместе с тёмными гигантами пространства служат центрам мощи и физическим регуляторам в качестве промежуточных станций для эффективной концентрации и направления энергетических контуров материальных творений.
\vs p041 3:2 
\vs p041 3:3 \pc 
\vs p041 3:4 
\vs p041 3:5 \pc 
\vs p041 3:6 \pc 
\vs p041 3:7 
\vs p041 3:8 \pc 
\vs p041 3:9 
\vs p041 3:10 
\usection{ПЛОТНОСТЬ СОЛНЦА}
\vs p041 4:1 
\vs p041 4:2 \pc 
\vs p041 4:3 \pc 
\vs p041 4:4 
\vs p041 4:5 
\vs p041 4:6 
\vs p041 4:7 
\usection{СОЛНЕЧНОЕ ИЗЛУЧЕНИЕ}
\vs p041 5:1 
\vs p041 5:2 
\vs p041 5:3 
\vs p041 5:4 
\vs p041 5:5 
\vs p041 5:6 \pc 
\vs p041 5:7 \pc 
\vs p041 5:8 
\usection{КАЛЬЦИЙ --- КОСМИЧЕСКИЙ СТРАННИК}
\vs p041 6:1 
\vs p041 6:2 
\vs p041 6:3 \pc 
\vs p041 6:4 
\vs p041 6:5 
\vs p041 6:6 \pc 
\vs p041 6:7 \pc 
\usection{ИСТОЧНИКИ СОЛНЕЧНОЙ ЭНЕРГИИ}
\vs p041 7:1 
\vs p041 7:2 
\vs p041 7:3 \pc 
\vs p041 7:4 
\vs p041 7:5 
\vs p041 7:6 
\vs p041 7:7 
\vs p041 7:8 
\vs p041 7:9 
\vs p041 7:10 
\vs p041 7:11 \pc 
\vs p041 7:12 \pc 
\vs p041 7:13 
\vs p041 7:14 \pc 
\vs p041 7:15 
\usection{СОЛНЕЧНО-ЭНЕРГЕТИЧЕСКИЕ РЕАКЦИИ}
\vs p041 8:1 
\vs p041 8:2 \pc 
\vs p041 8:3 \pc 
\vs p041 8:4 
\usection{СТАБИЛЬНОСТЬ СОЛНЦА}
\vs p041 9:1 
\vs p041 9:2 \pc 
\vs p041 9:3 \pc 
\vs p041 9:4 
\vs p041 9:5 
\usection{ПРОИСХОЖДЕНИЕ ОБИТАЕМЫХ МИРОВ}
\vs p041 10:1 
\vs p041 10:2 
\vs p041 10:3 \pc 
\vs p041 10:4 
\vs p041 10:5 \pc 
\vsetoff
\vs p041 10:6 
\quizlink
\begin{thebibliography}{100}
\bibitem{Eddington1}
A.S.~Eddington
{``Stars and Atoms''}
{\em Oxford: Clarendon Press}, 1927.
\bibitem{Jeans1}
J.~Jeans.
{``Through Space and Time''}
{\em Cambridge: University Press}, 1934.
\end{thebibliography}
