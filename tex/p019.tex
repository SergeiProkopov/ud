\upaper{19}{РАВНЫЕ СУЩЕСТВА ТРОИЧНОГО ПРОИСХОЖДЕНИЯ}
\uminitoc{ТРОИЧНЫЕ СЫНЫ УЧИТЕЛЯ}
\uminitoc{СОВЕРШЕНСТВОВАТЕЛИ МУДРОСТИ}
\uminitoc{БОЖЕСТВЕННЫЕ СОВЕТНИКИ}
\uminitoc{ВСЕОБЩИЕ ЦЕНЗОРЫ}
\uminitoc{ВДОХНОВЛЁННЫЕ ТРОИЧНЫЕ ДУХИ}
\uminitoc{УРОЖЕНЦЫ ХАВОНЫ}
\uminitoc{ГРАЖДАНЕ РАЯ}
\author{Божественный Советник}
\vs p019 0:1 Данная Райская группа~--- Равные Существа Троичного происхождения~--- включает Троичных Сынов Учителей, также причисляемых к Райским Сынам Бога, три группы высоких администраторов сверхвселенных и до некоторой степени безличностную категорию Вдохновлённых Троичных Духов. Даже уроженцы Хавоны могут быть по праву включены в эту классификацию Троичных личностей вместе с многочисленными группами существ, постоянно пребывающих в Раю. В этом обсуждении будут рассмотрены следующие существа Троичного происхождения:
\vs p019 0:2 \li{1.} Троичные Сыны Учителя.
\vs p019 0:3 \li{2.} Совершенствователи Мудрости.
\vs p019 0:4 \li{3.} Божественные Советники.
\vs p019 0:5 \li{4.} Всеобщие Цензоры.
\vs p019 0:6 \li{5.} Вдохновлённые Троичные Духи.
\vs p019 0:7 \li{6.} Уроженцы Хавоны.
\vs p019 0:8 \li{7.} Граждане Рая.
\vs p019 0:9 \pc За исключением Троичных Сынов Учителей и, возможно, Вдохновлённых Троичных Духов, эти группы имеют определённое число членов; их создание~--- завершённое событие прошлого.
\usection{ТРОИЧНЫЕ СЫНЫ УЧИТЕЛЯ}
\vs p019 1:1 Из всех раскрытых тебе высоких категорий небесных личностей только Троичные Сыны Учителя действуют в двойном качестве. По происхождению будучи Троичной природы, они по своим функциям почти полностью посвящены служению божественного сыновства. Они являются связующими существами, которые как бы перекидывают мост через вселенскую пропасть между личностями Троичного и двойственного происхождения.
\vs p019 1:2 В то время как число Стационарных Сынов Троицы окончательно, число Сынов Учителей постоянно увеличивается. Каково будет окончательное число Сынов Учителей, я не знаю. Однако я могу констатировать, что, согласно Райским спискам, на момент последнего периодического отчёта, поступившего на Уверсу, на службе находилось 21\,001\,624\,821 этих Сынов.
\vs p019 1:3 Эти существа --- единственная раскрытая вам группа Сынов Бога, чьё происхождение --- от Райской Троицы. Их присутствие охватывает центральную и сверхвселенные, и каждой локальной вселенной назначается огромный корпус. Они также служат отдельным планетам, как и другие Райские Сыны Бога. Поскольку схема большой вселенной ещё не полностью разработана, большое число Сынов Учителей содержится в резервах на Рае, и они вызываются добровольцами на службу в чрезвычайных ситуациях и для необычного служения во всех подразделениях большой вселенной, на одиноких мирах космоса, в локальные и сверхвселенные, а также миры Хавоны. Они функционируют также на Рае, но будет полезнее отложить их подробное рассмотрение, пока мы не перейдём к обсуждению Райских Сынов Бога.
\vs p019 1:4 В этой связи, однако, можно отметить, что Сыны Учителя --- это верховные координирующие личности Троичного происхождения. В такой обширной вселенной вселенных всегда таится огромная опасность поддаться заблуждению из\hyp{}за ограниченности точки зрения, --- злу, присущему разрозненной концепции реальности и божественности.
\vs p019 1:5 Например: человеческий разум обычно жаждет приблизиться к космической философии, изображённой в этих откровениях, путём перехода от простого и конечного к сложному и бесконечному\fnst{Эта тенденция нередко проявляется в стремлении многих людей изучать Пятое Эпохальное Откровение в обратном порядке: начиная с жизни Иисуса на Урантии и кончая концепциями экзистенциального Божества на Рае.}, от человеческого происхождения к божественному предназначению. Но этот путь не ведёт к \bibemph{духовной мудрости}. Такая процедура --- самый лёгкий путь к определённой форме \bibemph{генетического знания,} которое в лучшем случае может раскрыть лишь происхождение человека, но мало или совсем ничего не раскрывает касательно его божественного предназначения.
\vs p019 1:6 Даже в изучении биологической эволюции человека на Урантии возникают серьёзные возражения против чисто исторического подхода к его нынешнему статусу и текущим проблемам. Истинная перспектива любой связанной с реальностью проблемы --- человеческой или божественной, земной или космической --- может быть достигнута только путём полного и беспристрастного изучения и корреляции трёх фаз вселенской реальности: происхождения, истории и предназначения. Правильное понимание этих трёх эмпирических реальностей даёт основу для мудрой оценки текущего статуса.
\vs p019 1:7 \pc Когда человеческий разум берётся следовать философскому методу, начиная с низшего, чтобы приблизиться к высшему, будь то в биологии или теологии, он всегда рискует совершить четыре логические ошибки:
\vs p019 1:8 \li{1.} Он может совершенно не понять конечную и окончательную эволюционную цель как личного достижения, так и космического предназначения.
\vs p019 1:9 \li{2.} Он может допустить грубейшую философскую ошибку, чрезмерно упрощая космическую эволюционную (эмпирическую) реальность, что приводит к искажению фактов, извращению истины и неправильному представлению о предназначениях.
\vs p019 1:10 \li{3.} Изучение причинности есть внимательное изучение истории. Но знание того, \bibemph{как} происходит становление существа, необязательно ведёт к разумному пониманию текущего статуса и истинного характера такого существа.
\vs p019 1:11 \li{4.} Одна лишь история не может адекватно раскрыть будущее развитие --- предназначение. Знание конечного происхождения полезно, но только божественные причины раскрывают окончательные следствия. Цели вечности не показаны в истоках времени. Настоящее можно истинно интерпретировать только в свете взаимосвязанных прошлого и будущего.
\vs p019 1:12 \pc В силу этих и других причин мы используем метод приближения к человеку и его планетарным проблемам, отправляясь во время\hyp{}пространственное путешествие из бесконечного, вечного и божественного Райского Источника и Центра всей личностной реальности и всего космического существования.
\usection{СОВЕРШЕНСТВОВАТЕЛИ МУДРОСТИ}
\vs p019 2:1 
\vs p019 2:2 
\vs p019 2:3 \pc 
\vs p019 2:4 
\vs p019 2:5 
\vs p019 2:6 \pc 
\usection{БОЖЕСТВЕННЫЕ СОВЕТНИКИ}
\vs p019 3:1 
\vs p019 3:2 
\vs p019 3:3 
\vs p019 3:4 \pc 
\vs p019 3:5 
\vs p019 3:6 \pc 
\vs p019 3:7 
\usection{ВСЕОБЩИЕ ЦЕНЗОРЫ}
\vs p019 4:1 
\vs p019 4:2 
\vs p019 4:3 
\vs p019 4:4 \pc 
\vs p019 4:5 \pc 
\vs p019 4:6 
\vs p019 4:7 
\vs p019 4:8 \pc 
\vs p019 4:9 
\usection{ВДОХНОВЛЁННЫЕ ТРОИЧНЫЕ ДУХИ}
\vs p019 5:1 
\vs p019 5:2 
\vs p019 5:3 
\vs p019 5:4 
\vs p019 5:5 \pc 
\vs p019 5:6 
\vs p019 5:7 
\vs p019 5:8 \pc 
\vs p019 5:9 
\vs p019 5:10 
\vs p019 5:11 
\vs p019 5:12 
\usection{УРОЖЕНЦЫ ХАВОНЫ}
\vs p019 6:1 
\vs p019 6:2 
\vs p019 6:3 \pc 
\vs p019 6:4 \pc 
\vs p019 6:5 
\vs p019 6:6 
\vs p019 6:7 
\vs p019 6:8 
\usection{ГРАЖДАНЕ РАЯ}
\vs p019 7:1 
\vs p019 7:2 \pc 
\vs p019 7:3 \pc 
\vs p019 7:4 
\vs p019 7:5 
\vsetoff
\vs p019 7:6 
\quizlink
\begin{thebibliography}{100}
\bibitem{Doresey1}
John Morris Dorsey,
{<<The Foundations of Human Nature: The Study of the Person>>.}
{\em New York: Longmans, Green and Co.}, 1935.
\end{thebibliography}
