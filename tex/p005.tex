\upaper{5}{СВЯЗЬ БОГА С ИНДИВИДУУМОМ}
\uminitoc{ПРИБЛИЖЕНИЕ К БОГУ}
\uminitoc{ПРИСУТСТВИЕ БОГА}
\uminitoc{ИСТИННОЕ ПОКЛОНЕНИЕ}
\uminitoc{БОГ В РЕЛИГИИ}
\uminitoc{БОГОСОЗНАНИЕ}
\uminitoc{БОГ ЛИЧНОСТИ}
\author{Божественный Советник}
\vs p005 0:1 Если конечному разуму человека не удаётся понять, как столь великий и столь возвышенный Бог, как Всеобщий Отец, может спуститься из своей вечной обители в бесконечном совершенстве, чтобы побрататься с индивидуальным человеческим созданием, тогда такому конечному интеллекту остаётся только опираться для уверенности в божественном общении на истинность того факта\fnst{Этот \bibemph{факт,} впрочем, постигается не иначе, как \bibemph{верой}. Возможность постижения \bibemph{факта} присутствия Бога независимо от \bibemph{веры} в Бога описывается также в предпоследнем параграфе всего Откровения --- \bibref[196:3.34]{p196 3:34}.}, что реальный фрагмент живого Бога живёт в интеллекте каждого смертного Урантии, обладающего здравым умом и моральным сознанием. Обитающие в людях Настройщики Мыслей являются частью вечного Божества Райского Отца. Человеку не нужно идти дальше своего внутреннего опыта созерцания душой этого присутствия духовной реальности, чтобы найти Бога и попытаться вступить с ним в общение.
\vs p005 0:2 Бог распределил бесконечность своей вечной природы по экзистенциальным реальностям равных себе шести абсолютов, но он может в любое время вступить в прямой личный контакт с любой частью, фазой или видом творения при помощи своих доличностных фрагментов. Причём вечный Бог также оставил за собой прерогативу наделения личностью божественных Создателей и живых созданий вселенной вселенных, и, кроме того, он сохранил за собой прерогативу поддерживать прямой и родительский контакт со всеми этими личностными существами по личностному контуру.
\usection{ПРИБЛИЖЕНИЕ К БОГУ}
\vs p005 1:1 Неспособность конечного создания приблизиться к бесконечному Отцу заложена не в отчуждённости Отца, а в конечности и материальных ограничениях сотворённых существ. Величина духовного различия между высшей личностью вселенского существования и низшими группами созданных разумных существ непостижима. Если бы низшие уровни разумных существ можно было мгновенно перенести в присутствие самого Отца, они бы даже не узнали, что находятся там. Они бы там точно так же не заметили присутствия Всеобщего Отца, как и там, где они находятся сейчас. Смертному человеку предстоит пройти долгий\hyp{}предолгий путь, прежде чем он сможет уверенно и в пределах возможностей просить о безопасном сопровождении в Райское присутствие Всеобщего Отца. В духовном плане человек должен быть преобразован много раз, прежде чем он сможет достичь уровня, дающего такое духовное зрение, которое позволит ему увидеть хотя бы одного из Семи Главных Духов.
\vs p005 1:2 Наш Отец не прячется; он не находится в капризном уединении. Он мобилизовал ресурсы божественной мудрости в нескончаемом усилии раскрыть себя детям своих вселенских владений. Бесконечное величие и невыразимая щедрость связаны с величием его любви, которая заставляет его тосковать по общению с каждым сотворённым существом, которое может понимать, любить или приближаться к нему; и поэтому именно ограничения, присущие тебе, неотделимые от твоей конечной личности и материального существования, определяют время, место и обстоятельства, в которых ты можешь достичь цели путешествия восхождения смертного и предстать в присутствии Отца в центре всех вещей.
\vs p005 1:3 \pc Хотя приближение к Райскому присутствию Отца должно дожидаться достижения тобой самых высоких конечных уровней прогресса духа, ты должен радоваться, осознавая постоянно присутствующую возможность непосредственного общения с дарованным духом Отца, так сокровенно связанным с твоей внутренней душой и твоим одухотворяющимся <<я>>.
\vs p005 1:4 Смертные сфер времени и пространства могут сильно отличаться по врождённым способностям и интеллектуальному дарованию; они могут иметь исключительно благоприятную окружающую среду для социального развития и нравственного прогресса, или же страдать от отсутствия почти всех видов человеческих инструментов развития культуры и так называемого прогресса в мире искусства цивилизации; но возможности духовного прогресса на пути восхождения равны для всех; возрастающие уровни духовной проницательности и космических смыслов достигаются совершенно независимо от всех подобных социоморальных различий разнообразных материальных окружающих сред в эволюционных мирах.
\vs p005 1:5 Как бы ни отличались смертные Урантии по своим интеллектуальным, социальным, экономическим и даже моральным возможностям и способностям, не забывай, что их духовное дарование одинаково и уникально. Все они обладают одинаковым божественным присутствием дара от Отца, и все они имеют равную привилегию стремиться к сокровенному личному общению с этим пребывающим в них духом божественного происхождения, в то время как все они могут в равной степени принять одинаковое духовное водительство этих Таинственных Наставников.
\vs p005 1:6 \pc Если смертный человек искренне духовно мотивирован, безоговорочно посвящён исполнению воли Отца, тогда, поскольку он столь определённо и эффективно духовно наделён внутренним и божественным Настройщиком, в опыте этого индивидуума не может не материализоваться возвышенное сознание Богопознания и неземная уверенность в выживании с целью отыскания Бога посредством постепенного опыта становления всё более и более подобным ему.
\vs p005 1:7 В человеке духовно обитает бессмертный Настройщик Мыслей. Если такой человеческий разум искренне и духовно мотивирован, если такая человеческая душа желает познать Бога и стать подобной ему, честно хочет исполнять волю Отца, то не существует ни отрицательного влияния смертных лишений, ни положительной силы возможного вмешательства, которые могли бы воспрепятствовать такой божественно мотивированной душе уверенно восходить к вратам Рая.
\vs p005 1:8 Отец желает, чтобы все его создания находились в личном общении с ним. У него на Рае хватит места, чтобы принять всех тех, чей статус выживания и духовная природа делают возможным такое достижение. Поэтому раз и навсегда определись в своей философии: каждому из вас и всем нам Бог доступен, Отец достижим, путь открыт; силы божественной любви и пути и средства божественного управления --- всё сопряжено в усилии способствовать продвижению каждого достойного разума каждой вселенной к Райскому присутствию Всеобщего Отца.
\vs p005 1:9 Тот факт, что на достижение Бога требуется огромное время, не делает присутствие и личность Бесконечного менее реальными. Твоё восхождение является частью контура семи сверхвселенных, и хотя ты проходишь по этому кругу несметное число раз, ты можешь ожидать, что по духу и по статусу ты постоянно движешься вовнутрь. Ты можешь рассчитывать на то, что тебя будут переводить со сферы на сферу, от внешних контуров всё ближе к внутреннему центру, и однажды, не сомневайся, ты встанешь в божественном и центральном присутствии и увидишь его, образно говоря, лицом к лицу. Это лишь вопрос достижения реальных и буквальных духовных уровней; и эти духовные уровни достижимы любым существом, в котором обитал Таинственный Наставник, и которое впоследствии навсегда слилось с этим Настройщиком Мыслей.
\vs p005 1:10 \pc Отец не находится в духовном укрытии, но очень многие из его созданий спрятались в тумане своих собственных волевых решений и на время отделили себя от общения с его духом и духом его Сына в результате выбора своих собственных извращённых путей и потаканию самоуверенности своих нетерпимых умов и недуховных характеров.
\vs p005 1:11 Смертный человек может приближаться к Богу и может многократно отрекаться от божественной воли, пока у него остаётся возможность выбора. Окончательная гибель человека не запечатана до тех пор, пока он не утратит способность выбирать волю Отца. Сердце Отца никогда не закрывается для нужд и просьб его детей. Только потомки его навсегда закрывают свои сердца для притягательной силы Отца, когда окончательно и навсегда теряют желание исполнять его божественную волю --- познавать его и быть похожими на него. Точно так же вечная судьба человека гарантирована, когда слияние с Настройщиком провозглашает вселенной, что восходящий сделал окончательный и бесповоротный выбор жить по воле Отца\fnst{Буквально <<жить волю Отца>> (англ. to live the Father's will).}.
\vs p005 1:12 Великий Бог вступает в прямой контакт со смертным человеком и даёт частицу своего бесконечного, вечного и непостижимого <<я>>, чтобы жить и обитать в нём. Бог пустился в вечное приключение с человеком. Если ты согласишься на водительство духовных сил в тебе и вокруг тебя, ты не можешь не достичь высокого предназначения, установленного любящим Богом в качестве вселенской цели его восходящих созданий из эволюционных миров пространства.
\usection{ПРИСУТСТВИЕ БОГА}
\vs p005 2:1 Физическое присутствие Бесконечного --- это реальность материальной вселенной. Присутствие разума Божества должно определяться глубиной индивидуального интеллектуального опыта и эволюционным уровнем личности. Духовное присутствие Божественности обязательно должно быть различным по вселенной. Оно определяется духовной способностью к восприятию и степенью посвящения воли создания исполнению божественной воли.
\vs p005 2:2 Бог живёт в каждом из его рождённых в духе сыновей. Райские Сыны всегда имеют доступ к присутствию Бога, <<деснице Отца>>, и все его создания\hyp{}личности имеют доступ к <<лону Отца>>. Это либо относится к личностному контуру --- где, когда и как ни осуществлялся бы контакт, --- либо подразумевает личный, самосознательный контакт и общение со Всеобщим Отцом, будь то в центральной обители или в каком\hyp{}либо другом предназначенном для этого месте, например, на одной из семи священных сфер Рая.
\vs p005 2:3 Однако божественное присутствие не может быть обнаружено где\hyp{}либо в природе или даже в жизни Богопознавших смертных так полно и с такой определённостью, как в твоей попытке общения с обитающим в тебе Таинственным Наставником, Райским Настройщиком Мыслей. Какая ошибка --- мечтать о далёком Боге в небесах, когда дух Всеобщего Отца живёт в твоём собственном разуме!
\vs p005 2:4 \pc Именно благодаря этому фрагменту Бога, который пребывает в тебе, ты можешь надеяться, по мере продвижения в гармонии с духовным руководством Настройщика, более полно различать присутствие и преобразующую силу тех других духовных влияний, которые окружают тебя и касаются тебя, но не функционируют как неотъемлемая часть тебя. Тот факт, что ты интеллектуально не осознаёшь тесного и сокровенного контакта с пребывающим в тебе Настройщиком, ни в коей мере не опровергает такой возвышенный опыт. Доказательство родства с божественным Настройщиком целиком и полностью состоит в характере и объёме плодов духа\fnst{См. \bibref[34:6.13]{p034 6:13}.}, приносимых в жизненном опыте индивидуального верующего. <<По плодам их узнаете их>>.
\vs p005 2:5 Слабо одухотворённому материальному разуму смертного человека чрезвычайно трудно ощутить явное сознание духовной деятельности таких божественных существ, как Райские Настройщики. По мере того как душа, созданная совместно разумом и Настройщиком, становится всё более существующей, также развивается новая фаза сознания души, которая способна ощущать присутствие и распознавать духовное водительство и другую сверхматериальную деятельность Таинственных Наставников.
\vs p005 2:6 Весь опыт общения с Настройщиками включает в себя нравственный статус, умственную мотивацию и духовный опыт. Самореализация такого достижения в основном, хотя и не исключительно, ограничивается областями сознания души, но доказательства не заставляют себя ждать и изобилуют в проявлении плодов духа в жизнях всех таких контактёров с внутренним духом.
\usection{ИСТИННОЕ ПОКЛОНЕНИЕ}
\vs p005 3:1 Хотя Райские Божества со вселенской точки зрения едины, в своих духовных отношениях с такими существами, какие населяют Урантию, они также являются тремя особыми и отдельными личностями. Между ипостасями Божества существует различие в смысле личных обращений, общения и других сокровенных отношений. В высшем смысле этого слова мы поклоняемся Всеобщему Отцу и только ему. Конечно, мы можем поклоняться и поклоняемся Отцу в том виде, в каком он проявляется в своих Сынах Создателях, но именно Отцу, прямо или косвенно, поклоняются и его обожают.
\vs p005 3:2 Мольбы всех видов относятся к сфере Вечного Сына и его духовной организации. Молитвы, все формальные сообщения, всё, кроме обожания и поклонения Всеобщему Отцу, --- являются вопросами, затрагивающими локальную вселенную; они обычно не выходят за пределы юрисдикции Сына Создателя. Но поклонение, несомненно, подхватывается личностным контуром и отсылается Создателю действием этого контура Отца. Кроме того, мы полагаем, что такая регистрация поклонения существ, наделённых Настройщиком, облегчается присутствием духа Отца. Существует великое множество доказательств, подтверждающих подобное мнение, и я знаю, что все категории фрагментов Отца наделены способностью регистрировать искреннее\fnst{В англ. тексте лат. bona fide.} обожание своих подопечных приемлемым образом в присутствии Всеобщего Отца. Настройщики, несомненно, также используют прямые доличностные каналы связи с Богом, и таким же образом они могут использовать контуры духовной гравитации Вечного Сына.
\vs p005 3:3 Поклонение существует ради самого поклонения; молитва же воплощает в себе элемент выгоды этой или другой\fnst{То есть той, за которую эта личность молится.} личности; в этом принципиальное отличие между поклонением и молитвой. В истинном поклонении нет абсолютно никаких личных запросов или других элементов личной выгоды; мы просто поклоняемся Богу за то, чем он в нашем понимании является. Поклонение ничего не просит и ничего не ожидает для поклоняющегося. Мы не поклоняемся Отцу из\hyp{}за того, что мы можем что\hyp{}то получить из такого поклонения; мы выказываем такую преданность и занимаемся таким поклонением, ибо это естественная и спонтанная реакция на признание несравненной личности Отца, а также из\hyp{}за его привлекательной природы и достойных обожания качеств.
\vs p005 3:4 Как только элемент личной выгоды вторгается в поклонение, в тот же миг это религиозное чувство трансформируется из поклонения в молитву, и более уместно её адресовать личности Вечного Сына или Сына Создателя. Но в практическом религиозном опыте нет причин, по которым молитва не может быть адресована Богу Отцу как часть истинного поклонения.
\vs p005 3:5 В практических вопросах своей повседневной жизни ты находишься в руках духовных личностей, происходящих от Третьего Источника и Центра; ты сотрудничаешь с силами Совместного Вершителя. Итак: ты поклоняешься Богу; молишься Сыну и общаешься с ним; и вырабатываешь детали твоего земного временного пребывания совместно с разумными существами Бесконечного Духа, действующими в твоём мире и по всей твоей вселенной.
\vs p005 3:6 \pc Сыны Создатели или Суверенные Сыны, которые руководят судьбами локальных вселенных, представляют как Всеобщего Отца, так и Вечного Сына Рая. Во имя Отца эти Вселенские Сыны принимают обожание поклонения и прислушиваются к мольбам своих просящих подданных по всем своим творениям. Для детей локальной вселенной Сын Михаил, в сущности, является Богом. Он --- олицетворение Всеобщего Отца и Вечного Сына в локальной вселенной. Бесконечный Дух поддерживает личный контакт с детьми этих сфер через Вселенских Духов, административных созидательных помощников\fnst{То есть партнёрш.} Райских Сынов Создателей.
\vs p005 3:7 \pc Искреннее поклонение означает мобилизацию всех сил человеческой личности под началом развивающейся души и с подчинением божественному направляющему влиянию соответствующего Настройщика Мыслей. Материально ограниченный разум никогда не сможет осознать в высокой степени настоящее значение истинного поклонения. Осознание человеком реальности опыта поклонения в основном определяется статусом развития его эволюционирующей бессмертной души. Духовный рост души происходит совершенно независимо от интеллектуального самосознания.
\vs p005 3:8 Опыт поклонения состоит в возвышенной попытке обручённого Настройщика передать божественному Отцу невыразимые желания и непередаваемые словами стремления человеческой души --- совместного творения ищущего Бога смертного разума и раскрывающего Бога бессмертного Настройщика. Таким образом, поклонение --- это акт согласия материального разума на попытку своего одухотворяющегося <<я>> под руководством ассоциированного духа общаться с Богом в качестве верующего сына Всеобщего Отца. Смертный разум соглашается поклоняться; бессмертная душа жаждет поклонения и инициирует его; присутствие божественного Настройщика руководит таким поклонением от имени смертного разума и развивающейся бессмертной души. В конечном счёте истинное поклонение становится опытом, реализуемым на четырёх космических уровнях: интеллектуальном, моронтийном, духовном и личностном --- сознания разума, души и духа и их объединения в личности.
\usection{БОГ В РЕЛИГИИ}
\vs p005 4:1 Мораль эволюционных религий движущей силой страха \bibemph{подталкивает} людей вперёд в поисках Бога. Религии откровения \bibemph{привлекают} людей к поиску Бога любви, потому что они жаждут стать подобными ему. Но религия --- это не просто пассивное чувство <<абсолютной зависимости>> и <<уверенности в выживании>>; это живой и динамичный опыт достижения божественности, основанный на служении человечеству.
\vs p005 4:2 Великое и непосредственное назначение истинной религии --- это установление прочного единства в человеческом опыте, устойчивого мира и глубокой уверенности. Для первобытного человека даже политеизм представляет собой относительное объединение эволюционирующего представления о Божестве; политеизм --- это монотеизм в процессе становления. Рано или поздно Бог неизбежно постигается как реальность ценностей, сущность смыслов и жизнь истины.
\vs p005 4:3 Бог не только является определяющим предназначение; он и \bibemph{есть} вечное предназначение человека. Все виды нерелигиозной человеческой деятельности стремятся подчинить вселенную искажающему служению себе; истинно религиозный индивидуум стремится отождествить своё <<я>> со вселенной, а затем посвятить деятельность этого объединённого <<я>> служению вселенской семье собратьев --- человеческих и сверхчеловеческих.
\vs p005 4:4 \pc Области философии и искусства лежат между нерелигиозной и религиозной деятельностью человеческого <<я>>. С помощью искусства и философии материально мыслящий человек вовлекается в созерцание духовных реальностей и вселенских ценностей вечных смыслов.
\vs p005 4:5 \pc Все религии учат поклонению Божеству и какой\hyp{}нибудь доктрине человеческого спасения. Буддистская религия обещает спасение от страданий, вечный покой; иудейская религия обещает спасение от трудностей, процветание, основанное на праведности; греческая религия обещала спасение от дисгармонии, уродства через осознание красоты; христианство обещает спасение от греха, святость; магометанство предоставляет избавление от строгих моральных норм иудаизма и христианства. Религия Иисуса \bibemph{является} спасением от <<я>>, избавлением от зла, связанного с изоляцией созданий во времени и в вечности.
\vs p005 4:6 Евреи основывали свою религию на доброте; греки --- на красоте; обе религии искали истину. Иисус явил Бога любви, а любовь содержит в себе истину, красоту и доброту.
\vs p005 4:7 У зороастрийцев была религия морали; у индусов --- религия метафизики; у конфуцианцев --- религия этики. Иисус жил религией \bibemph{служения}. Все эти религии ценны тем, что являются годными подходами к религии Иисуса. Религии суждено стать реальностью духовного объединения всего, что есть доброго, красивого и истинного в человеческом опыте.
\vs p005 4:8 У греческой религии был девиз: <<Познай себя>>; Евреи сосредоточили своё учение на словах: <<Познай своего Бога>>; христиане проповедуют евангелие, направленное на <<познание Господа Иисуса Христа>>; Иисус провозгласил благую весть <<познания Бога и себя как сына Бога>>. Эти различающиеся представления о цели религии определяют отношение человека к различным жизненным ситуациям и предопределяют глубину поклонения и характер его личных привычек к молитве. Духовный статус любой религии может быть определён по характеру её молитв.
\vs p005 4:9 \pc Представление о ревнивом Боге\hyp{}получеловеке является неизбежным переходом от политеизма к более высокому монотеизму. Возвышенный антропоморфизм --- вот высший уровень достижения чисто эволюционной религии. Христианство подняло концепцию антропоморфизма от идеала человека до трансцендентной и божественной концепции личности прославленного Христа. И это --- наивысший антропоморфизм, который только может придумать человек.
\vs p005 4:10 \pc Христианская концепция Бога является попыткой объединить три отдельных учения:
\vs p005 4:11 \li{1.}\bibemph{Еврейская концепция} --- Бог как защитник моральных ценностей, праведный Бог.
\vs p005 4:12 \li{2.}\bibemph{Греческая концепция} --- Бог как объединитель, Бог мудрости.
\vs p005 4:13 \li{3.}\bibemph{Концепция Иисуса} --- Бог как живой друг, любящий Отец, божественное присутствие.
\vs p005 4:14 \pc Поэтому должно быть очевидным, что составная христианская теология сталкивается с огромными трудностями в стремлении к последовательности. Эти трудности ещё больше усугубляются тем фактом, что доктрины раннего христианства в основном базировались на личном религиозном опыте трёх различных лиц: Филона Александрийского, Иисуса из Назарета и Павла из Тарса.
\vs p005 4:15 \pc Изучая религиозную жизнь Иисуса, рассматривайте его позитивно. Думайте не столько о его безгрешности, сколько о его праведности, его служении, полном любви. Иисус поднял пассивную любовь, раскрытую в еврейской концепции небесного Отца, до более высокого \bibemph{активного} и основанного на любви к созданию чувства Бога --- Отца каждого индивидуума, даже грешника.
\usection{БОГОСОЗНАНИЕ}
\vs p005 5:1 Мораль происходит от рассуждений самосознания; она является сверхживотной, но всецело эволюционной. Человеческая эволюция в своём развитии включает в себя все дары, предшествующие посвящению Настройщиков и излиянию Духа Истины. Но достижение уровней морали не избавляет человека от настоящих испытаний смертного существования. Физическая среда человека влечёт за собой борьбу за существование; социальное окружение делает необходимым этическое регулирование; моральные ситуации требуют выбора в высших сферах разума; духовный опыт (после осознания Бога) требует, чтобы человек нашёл его и искренне стремился быть подобным ему.
\vs p005 5:2 Религия не основана на фактах науки, обязательствах перед обществом, допущениях философии или подразумеваемом моральном долге. Религия --- это независимая область человеческой реакции на жизненные ситуации, которая неизменно проявляется на всех постморальных стадиях человеческого развития. Религия может пронизывать все четыре уровня реализации ценностей и воплощения вселенского братства: \begin{itemize}\item физический --- или материальный --- уровень самосохранения; \item социальный --- или эмоциональный --- уровень общения; \item моральный --- или основанный на чувстве долга --- уровень разума; \item духовный --- уровень осознания вселенского братства через божественное поклонение.\end{itemize}
\vs p005 5:3 Учёный, ищущий факты, представляет себе Бога как Первопричину, Бога силы. Эмоциональный художник видит в Боге идеал красоты, Бога эстетики. Рассуждающий философ иногда склонен постулировать Бога универсального единства, даже пантеистическое Божество. Религиозный верующий человек верит в Бога, способствующего выживанию, Отца Небесного, Бога любви.
\vs p005 5:4 \pc Нравственное поведение всегда предшествует эволюционной религии и является частью даже религии откровения, но никогда не является религиозным опытом в целом. Общественное служение --- это результат нравственного мышления и религиозной жизни. Нравственность биологически не ведёт к более высоким духовным уровням религиозного опыта. Поклонение абстрактному прекрасному --- это не поклонение Богу; ни возвышение природы, ни почитание единства не являются поклонением Богу.
\vs p005 5:5 Эволюционная религия --- мать науки, искусства и философии, которая подняла человека до уровня восприимчивости к религии откровения, включая дар Настройщиков и приход Духа Истины. Эволюционная картина человеческого существования начинается и заканчивается религией, хотя и очень разными качествами религии, одно эволюционное и биологическое, другое --- основанное на откровении и периодическое. И поэтому, хотя религия нормальна и естественна для человека, она вместе с тем необязательна. Человеку не обязательно быть религиозным против своей воли.
\vs p005 5:6 \pc Религиозный опыт, будучи по сути своей духовным, никогда не может быть полностью понят материальным разумом; отсюда и функция теологии --- психологии религии. Основная доктрина человеческого осознания Бога порождает парадокс в конечном понимании. Для человеческой логики и конечного разума почти невозможно привести к гармонии концепцию божественной имманентности --- Бог внутри и как часть каждого индивидуума --- с идеей трансцендентности Бога, божественного господства во вселенной вселенных. Эти две существенные концепции Божества должны быть объединены в понимании через веру концепции трансцендентности личностного Бога и в реализации\fnst{Или <<осознании>> (англ. realization).} внутреннего присутствия частицы этого Бога, чтобы оправдать разумное поклонение и подтвердить надежду на выживание личности. Трудности и парадоксы религии неотъемлемы от того факта, что религиозные реалии лежат совершенно за пределами способности интеллектуального понимания смертных.
\vs p005 5:7 \pc Смертный человек даже во дни своего временного пребывания на земле получает от религиозного опыта три великих удовлетворения:
\vs p005 5:8 \li{1.}\bibemph{Интеллектуально} он получает удовлетворение более объединённого человеческого сознания.
\vs p005 5:9 \li{2.}\bibemph{Философски} он получает удовольствие от обоснования своих идеалов нравственных ценностей.
\vs p005 5:10 \li{3.}\bibemph{Духовно} он преуспевает в переживании божественного общения, в духовных удовлетворениях истинного поклонения.
\vs p005 5:11 \pc Богосознание, как его переживает развивающийся смертный миров, должно состоять из трёх варьирующихся факторов, трёх различных уровней реализации\fnst{Или <<осознания>> (англ. realization).} реальности. Первым является сознание разума --- понимание \bibemph{идеи} Бога. Затем следует сознание души --- реализация \bibemph{идеала} Бога. Последним пробуждается сознание духа --- осознание \bibemph{реальности духа} Бога. Объединением этих факторов божественного осознания, каким бы неполным оно ни было, смертная личность в любое время покрывает все уровни сознания осознанием \bibemph{личности} Бога. В тех смертных, которые достигли Корпуса Завершения, всё это со временем приведёт к осознанию \bibemph{верховности} Бога и может впоследствии разрешиться в осознание \bibemph{предельности} Бога --- некоторой фазы абсонитного сверхсознания Райского Отца.
\vs p005 5:12 Опыт Богосознания остаётся неизменным из поколения в поколение, но с каждой новой эпохой в человеческом знании философская концепция и теологические определения Бога \bibemph{должны} меняться. Богопознание, религиозное сознание --- это вселенская реальность, но сколь бы действительным (реальным) ни был религиозный опыт, он должен быть готов подвергнуться разумной критике и обоснованной философской интерпретации; он не должен стремиться быть чем\hyp{}то обособленным в совокупности человеческого опыта.
\vs p005 5:13 \pc Вечное выживание личности полностью зависит от выбора смертного разума, чьи решения определяют потенциал выживания бессмертной души. Когда разум верит в Бога, а душа знает Бога, и когда вместе с помогающим Настройщиком все они \bibemph{жаждут} Бога, тогда выживание гарантировано. Ограниченность интеллекта, неполнота образования, отсутствие культуры, ухудшение социального статуса до нищеты, даже неполноценность человеческих стандартов морали из\hyp{}за прискорбного отсутствия образовательных, культурных и социальных преимуществ не могут свести на нет присутствие божественного духа в таких несчастных и по\hyp{}человечески ущербных, но верующих индивидуумах. Пребывание Таинственного Монитора составляет начало и гарантирует возможность потенциала роста и выживания бессмертной души.
\vs p005 5:14 Способность смертных родителей производить потомство не зависит от их образовательного, культурного, социального или экономического статуса. Объединения родительских факторов в естественных условиях вполне достаточно для зачатия потомства. Человеческий разум, различающий добро и зло\fnst{Англ. right and wrong.} и обладающий способностью поклоняться Богу в союзе с божественным Настройщиком, --- это всё, что требуется от этого смертного, чтобы инициировать и способствовать созданию его бессмертной души с качествами выживания, если такой наделённый духом человек ищет Бога и искренне желает стать подобным ему, честно выбирает творить волю Отца небесного.
\usection{БОГ ЛИЧНОСТИ}
\vs p005 6:1 Всеобщий Отец --- Бог личностей. Сфера вселенской личности, от низшего смертного и материального создания статуса личности до высших личностей достоинства создателя и божественного статуса, имеет свой центр и периферию во Всеобщем Отце. Бог Отец --- даритель и хранитель каждой личности. Райский Отец также является предназначением всех тех конечных личностей, которые всем сердцем решают исполнять божественную волю, тех, кто любит Бога и страстно желает быть подобными ему.
\vs p005 6:2 \pc Личность --- одна из неразгаданных тайн вселенных. Мы способны сформировать адекватные представления о факторах, входящих в состав различных категорий и уровней личности, но мы не полностью понимаем истинную природу самой личности. Мы ясно представляем себе многочисленные факторы, которые в совокупности составляют проводник человеческой личности, но мы не до конца понимаем природу и значение такой конечной личности.
\vs p005 6:3 Личность потенциальна во всех наделённых разумом созданиях: от минимального --- самосознания, до максимального --- Богосознания. Но одно лишь наделение не является личностью, как не является ею дух или физическая энергия. Личность --- это то качество и ценность в космической реальности, которыми только Бог Отец может наделить эти живые системы ассоциированных и координированных энергий материи, разума и духа. Не является личность и постепенным достижением. Личность может быть материальной или духовной, но личность либо есть, либо её нет. Неличностное никогда не достигает уровня личностного, кроме как прямым действием Райского Отца.
\vs p005 6:4 Наделение личностью --- исключительная функция Всеобщего Отца, --- персонализация живых энергетических систем, которые он наделяет атрибутами относительного творческого сознания и добровольным контролем над ним. Не существует личности отдельно от Бога Отца, и никакая личность не существует иначе как для Бога Отца. Фундаментальные атрибуты человеческой индивидуальности, так же как и абсолютное ядро человеческой личности --- Настройщик, --- это дары Всеобщего Отца, действующего в его исключительно личной сфере космического служения.
\vs p005 6:5 \pc Настройщики доличностного статуса пребывают в многочисленных типах смертных созданий, тем самым гарантируя, что эти же существа смогут пережить физическую смерть и персонализоваться как моронтийные существа с потенциалом предельного достижения духа. Ибо когда в разуме создания, наделённого личностью, пребывает фрагмент духа вечного Бога --- доличностный дар личностного Отца, --- тогда эта конечная личность действительно обладает потенциалом божественного и вечного и стремится к предназначению, близкому к Предельному, даже протягиваясь до осознания Абсолюта.
\vs p005 6:6 Способность превращения в божественную личность присуща доличностному Настройщику; способность к обладанию человеческой личностью потенциальна в наделении человеческого существа космическим разумом. Но эмпирическая личность смертного человека не может быть наблюдаема как активная и функциональная реальность до тех пор, пока материального жизненного проводника смертного создания не коснётся освобождающая божественность Всеобщего Отца, таким образом пускающая его в плавание по морям опыта как самосознающую и (относительно) самоопределяющуюся и самосозидательную личность. Материальное <<я>> поистине и \bibemph{безусловно личностно}.
\vs p005 6:7 \pc Материальное <<я>> имеет личность и индивидуальность, вр\'еменную индивидуальность; доличностный дух\hyp{}Настройщик также имеет индивидуальность --- индивидуальность вечную. Эта материальная личность и этот дух\hyp{}доличностность способны так объединить свои творческие атрибуты, чтобы создать выживающую индивидуальность бессмертной души.
\vs p005 6:8 Создав, таким образом, условия для роста бессмертной души и освободив внутреннее <<я>> человека от пут абсолютной зависимости от априорной причинности, Отец отходит в сторону. Теперь, когда человек освобождён от оков причинно\hyp{}следственной реакции, по крайней мере, в том, что касается вечного предназначения, и были созданы условия для роста бессмертного <<я>> --- души,--- человеку остаётся самому желать сотворения или воспрепятствовать сотворению этого выживающего и вечного <<я>>, которое станет его, если только он сделает этот выбор. Никакое другое существо, сила, создатель или фактор во всей обширной вселенной вселенных не может в какой\hyp{}либо степени вмешиваться в абсолютный суверенитет смертной свободной воли, когда она действует в сферах выбора относительно вечного предназначения личности выбирающего смертного. Что касается вечного выживания, Бог провозгласил суверенность материальной и смертной воли, и этот указ является абсолютным.
\vs p005 6:9 \pc Наделение создания личностью приносит относительное освобождение от рабской реакции на априорную причинность, и личности всех таких нравственных существ, эволюционных или иных, сосредоточены в личности Всеобщего Отца. Их всегда влечёт к его Райскому присутствию то родство, которое составляет обширный и всеобщий семейный круг и братский контур вечного Бога. Родственное свойство всех личностей --- божественная спонтанность.
\vs p005 6:10 \pc Личностный контур вселенной вселенных сосредоточен в личности Всеобщего Отца, и Райский Отец лично осознаёт и находится в личном контакте со всеми личностями всех уровней самосознательного существования. И это личностное сознание всего творения существует независимо от миссии Настройщиков Сознания.
\vs p005 6:11 \pc Как вся гравитация замкнута в контуре на Острове Рай, как весь разум замкнут в контуре в Совместном Вершителе, а весь дух --- в Вечном Сыне, так и всякая личность замыкается в контуре в личном присутствии Всеобщего Отца, и этот контур безошибочно передаёт поклонение всех личностей Изначальной и Вечной Личности.
\vs p005 6:12 \pc Касательно личностей, в которых не пребывает Настройщик: атрибут свободы выбора также дарован Всеобщим Отцом, и такие личности также включены в великий контур божественной любви, личностный контур Всеобщего Отца. Бог обеспечивает возможность суверенного выбора всех истинных личностей. Ни одно личностное существо не может быть принуждено к вечному приключению; портал вечности открывается только в ответ на свободный выбор свободных сыновей Бога свободной воли\fnst{Англ. freewill choice of the freewill sons of the God of free will.}.
\vs p005 6:13 \pc Изложенное выше представляет собой мои усилия показать отношение живого Бога к детям времени. В конечном счёте единственно полезное, что я могу сделать, --- это повторить, что Бог --- ваш вселенский Отец, и что все вы --- его планетарные дети.
\vsetoff
\vs p005 6:14 [Это пятое и последнее из серии, представляющей повествования Божественного Советника Уверсы о Всеобщем Отце.]
\quizlink
\begin{thebibliography}{100}
\bibitem{Strong1}
Rev.~Josiah Strong, D.D.
{<<The New World\hyp{}Religion.>>}
{\em Garden City, New York: Doubleday, Page \& Company}, 1915.
\bibitem{Matthews1}
W. R. Matthews, K.C.V.O., D.D., D.Lit.
{<<God in Christian Thought and Experience>>.}
{\em London: Nisbet \& Co. Ltd.}, 1930.
\bibitem{Illingworth1}
J.R. Illingworth, M.A.
{<<Personality Human and Divine>>.}
{\em London and New York: The Macmillan Company}, 1894.
\end{thebibliography}
